
\chapter{The Collectivity After the Abolition of the Universe and Time: Escaping from Social Science(1996)}
% CONTENTS
% A. Principles of Natural Sociography
% B. Community and Social Causation in Personhood Theory
% C. Inevitable Stages?
% D. Retroactive Signification
% E. Dissolution of Natural Society
% F. Recapitulation

\section{Principles of Natural Sociography}

Let me introduce a new term for the recounting of social phenomena: \textbf{sociography}. Then both Herodotus and Lefebre practiced sociography; but they practiced it in very different ways. We may call sociography before the time of David Hume (or whoever you want) \textbf{legendary sociography}. The modern sociography which Hume and his successors fought to establish we may call \textbf{natural sociography}.

Here I am forced to interject a qualification which gets ahead of my exposition. Ultimately I propose to dissolve temporally rectilinear natural society. For that reason, I cannot ask for the historical references in this manuscript to be taken literally. These references should be understood as headed by the phrase: "as conventional wisdom affirms."

The aim of this reflection is to escape from social science. But many modern intellectuals would say: "We never accepted social science as a real science to begin with. We already showed that social science cannot be a science." Such remarks are massively misdirective. The positivist and literary-irrationalist critics of social science in the universities are loyal to the social \emph{fact base}---and that fact base is one of modernity’s hardest-won achievements. Indeed, modernist relativists are more loyal to social naturalism than they are to physics, not less. As far as I know, there has never been a published challenge to natural sociography. (Aside from the implicit challenge by the rearguard, those who defend divine intervention, astrology, etc.)

Let me ask a heuristic question. What are the conceptual boundaries–in discourse on social affairs–between
\begin{itemize}
\item the common-sense notion of the world
\item theoretical conceptualization?
\end{itemize}

Social existence "already" involves conceptualization, even before philosophical thinking enters the picture. If we were to invoke ancient evidence (which I am not especially interested in doing), we would find striking examples of how social discourse is conceptualized over and above common sense. Ahistorically, notions of political legitimacy and of law transcend narrowly circumscribed common sense. Kinship. Family law. Property and contract as juridical concepts. Monetary measure of the value of goods. (Whether monetary capital is productive does not have a common-sense answer, as any student of capital theory will know.) The rest of this section will heed these considerations.

Natural sociography adheres to the following principles.
\begin{enumerate}[label=\alph*.]
\item The social collective excludes souls of deceased relatives, not to mention superhuman beings.

\item The social collective excludes animals. Animals enter sociography only as prey and as chattels.

\item All miracles in social records (occurrences precluded by modern scientific laws) must be repudiated.

\item All apocryphal occurrences in social records must be exposed and repudiated.

\item Supernatural causes of human events must be repudiated.

\item Reincarnation is not permissible as an explanation of individual "personality." That means that the Tibetan explanation of the Dalai Lama must be stripped from "real" sociography.

\item Careers in the afterlife (Egypt, Tibet) must be repudiated. Thus the most obvious productive activity of ancient Egypt (and China), the building of furnished tombs for royalty, must be judged a societal insanity.

\item All human needs and wants are posited as mundane, and involve opulence and pleasure. (Power and glory are also conceived as needs; but it is basically taboo to theorize about them.)

\item The concept of destiny, the future as cause of the present, is invoked by some authors–but it is not proper science.
\end{itemize}

All the while, there are certain immaterial ontologies which social science must embrace. Each of the multitude of individuals has a mind–notwithstanding that your mind is unobservable by me. Each individual engages in choice-making. (Various schools of psychology reject this as superstition; but to strip the subject-matter of mentation and choice-making would be intolerably reductionist.)

Humans exist in a culture-saturated realm; which means a realm filled with evidence which can only be appreciated via interpretation. (Inscriptions; pictures and images; etc. etc.) A key issue at every turn in social existence is legitimacy: why, after all, should one cooperate with (or submit to) governmental authority; why should one consent to offered terms of livelihood; etc. etc.? All of this utterly transcends physics and biology; the latter sciences have no basis to investigate these dimensions.

Sociology requires the crystallization and "evolution" of polities to be given a causal rationale which is not supernatural. Even though the individual is the atomic agent in society–a tenet at the foundation of bourgeois economics–his or her consciousness is causally insignificant in comparison to "conditions." Sociology is a phenomenology of political life which

i) treats states as units, as systems;

ii) attends to law;

iii) attends (to a lesser extent) to coalescing and motivating mythologies and rituals.

If sociology wanted to proceed like a natural science, it would have to abstract from concrete phenomena to obtain ideal elements which can have multiple instances. (That is what a scientific law presupposes.) However, history inherently finds its events to be individual. Abraham Lincoln was not an instance of an abstraction which can be repeated at will (as in a series of experiments). Nonactual possibility in history is extremely problematic. (Would Napoleon have commenced such-and-such a battle if he had not had a toothache?) Does sociology seek laws–on a timeline of unique events?

Sociologists indeed require the polity to be a natural system which obeys "socio-natural" laws. For some, the polity is referred to physics and biology as primary realities. At the same time, sociology rests on a modern common-sense notion of the human collective, consisting of an awareness of people one never meets, and some notion of one’s connectedness to them. Again, such information cannot be provided by physics and biology, which do not recognize the existence of individual minds, choice-making, culture, or legitimacy.

Some philosophers of science find a way for a unique human event to express laws. The event, although unique, belongs to a species for which an idealizing, quantifying, experimental science exists. If a great king dies from a heart attack, or from being thrown from a horse, heart attacks or injuries from mechanical shock are explicable in medical biology. [But to say that that is known to be the complete explanation is highly tendentious.]

If, on the other hand, one wants a law which says that "African socialism" could not possibly have worked–because it is not possible for a polity to skip the capitalist stage–then the law would be specific to social process: and the above solution would not help.

There is an opportunity to be far more trenchant here. Physics proposes to provide: an exhaustive account of what is not (nonactual possibility); combined with a voluntary "trifling with nature" called experiment. But concerning our lives and history, we do not want exhaustive accounts of what is not.–And the purpose of our choices is not to "trifle with nature" but to become this person. Actually, why wouldn’t existential self-actualization and scientific experiments have the same character–as choices?

Moreover, if human affairs comprise a unique actual career, then inanimate nature should also comprise a unique actual career.

How is it that modern thought conflates instrumental choice-making with The Order Of The Universe–and then turns around and segregates this package from existential choice-making? When it comes to summing the universe and human life–and to detecting nonactual possibility or not detecting it–modern thought is a shambles.

The prevailing culture tells us over and over: physical science is in good order; whereas social science is suspect (or has not yet proved itself). But nothing requires us to accept this separation of a tractable problem from an intractable one. Like a mantra, they keep repeating "all of it is nature." But then "the problem of knowledge" ought to be a single topic. When, in the name of "nature," modern thought gives us realms which are incommensurate and unsummable, it totally discredits itself.

• • •

B. Community and Social Causation in Personhood Theory

Our field of inquiry is posited to be collective human phenomena. These phenomena manifest intent; and involve meanings (in such a way that to overlook those meanings is intolerably reductionist). In other words, the phenomena involve intentions and interests and ideas. They involve how we conceive or apprehend, understand or appreciate. They involve the appearance of novelties in these respects.

Such (collective human) phenomena are correlative to argumentative discourse.

°

Let us reprise "Personhood II." In the realm of ordinary personhood, other people and culture are palpable to me. Other people and culture jointly constitute the interpersonal arena–or community.

Society is the aggregation which is hypothesized as subtending the (palpable) community. Society is the kingdom, the race, the nation. It is an abstraction, a matter of faith, to which allegiance is demanded by palpable specific people.

So society is a "grandiose Other." A grandiose Other is advanced as the ultimate source of meaning, the ultimate source of my emotional gratification and judgmental self-consciousness. At the same time, the grandiose Other is primarily speculative, and outside "my ostensible world."

The universe of physics (called Nature) must be mentioned in this connection as hypothetical, inferential, derived, and grandiose–as a modern god. The enshrining of Nature as a god is a precedent for the modern enshrining of society as a god. The physical universe is not claimed to be a source of meaning, however.

In modern culture, the grandiose Objectivity which has priority is (to repeat) society. Society's claim on us as persons (even when we are treated as pawns) is far broader and more important than the physical universe's claim on us. Typically, the primary avowed loyalties in modern culture are to society.

Extending from one's emotional involvement with other people, society becomes an object of one's passionate belief. The hypothesized abstraction seems to be a living presence: as when people march off to war for The Nation–or dramatically refuse to do so. "Attachment" makes society more compelling than the physical universe. Because society is an object of passionate belief, because it becomes a hallucinatory living presence, it cannot be sharply distinguished from community (which is palpable), even though it remains impalpable (a hypothesized abstraction). So society has a close and compelling connection to the palpable phenomena of other people and culture.

At certain points, personhood theory passes to a higher level of credulity and integrates its analysis with one of the preexisting hypotheses which it has discerned. This is what happens in the case of culture. Personhood theory pictures my cultural competences (e.g. English orthography) as deriving from society. Namely: culture is that palpable aspect of society which is interior to me and at the same time is an externality broader than other people as individuals.

Recognizing how close society is to community in belief, I propose to be flexible with regard to whether the person is conceived in a communal or a social context.

The community confronts me with symbols and offices which imply an organized collective, legitimation, manifestations of a group will, etc.

One cultural phase of community life includes the community's "tradition," symbolism, ritual, etc.–all of which are emotionally charged. This phase must be considered one source of my emotional sensitization or capacity.

The community may force upon me a significance, and an assortment of privileges and disadvantages–so much so that I am forced to carry out this "imposed social role" or to grapple with it. The role may place me in competition or conflict with other people. I can also be gratified by the celebration in ritual of my imposed status (although I do not earn this gratification).

I have a greater or lesser degree of autonomy, relative to the community, in respect to being supplied with pursuits and goals, and in respect to making judgments of every sort.

I can obscure my choice-making by becoming a vassal of "society"–of a legitimated organization or institution.

I may engage in a pursuit which I suspect to be dishonest or otherwise contemptible because the community approves of it. Of course I do so to gain tangible rewards, in analogy with knowingly deceiving another person to benefit myself. But something beyond my craftiness is involved here. I maintain a knowing self-deception and vassalage in which legitimacy means more to me than sincerity.

The interpersonal arena is a source of meanings to me. My connections to the interpersonal arena in regard to praxis, emotional sensitization, indoctrination, etc. have an effect on my sense of sanity, my personal identity, my level of fulfillment, etc. Thus, the interpersonal arena can be a source of skills worthy to be sustained and regenerated. It can also be a source of acute dilemmas and destructiveness impinging upon me. In either case, the interpersonal arena is a source of problems and missions.

Moreover, the problems and missions can appear in my consciousness as consequences of my skills. Having been indoctrinated with little choice in the matter, that indoctrination now surfaces in the guise of my skills, for one thing. (Examples at the level of the present discussion are language use, mathematics, music, profit-maximization.) If I do not consciously review my indoctrination, then I will carry it with me by default. Moreover, my private and idiosyncratic dilemmas with natural language, with mathematics, with art, with profit maximization, etc.–and my private and idiosyncratic ventures in these fields–can represent vital dilemmas and ventures for the interpersonal arena.

But the community's destructiveness or bankruptcy may consist precisely in its inability to embark upon vital ventures–and in its fostering of individual pursuits which disregard and exacerbate its dilemmas.

I can undertake a vital venture or address a vital problem; or I can avoid doing so. And I can belong to a community which wants such a task addressed; or to a community which discourages attention to such a task. The possible ramifications of the community attitude, for my judgment of myself, are complicated. Inner pride or lack of it can run counter to express community approval or contempt.

°

Social role can submerge a person. More accurately, the social role can be said to fixate the individual to mutilated perception. But I then say that the social role is a sort of ideology and skill which the individual is fixated to. The submergence of the person by a cumulating social role is an outcome in which the person is guaranteed to be traumatized, stigmatized, impaired, truncated.

Certainly, in some cultures or communities, socially acclaimed and validated roles can also allow intrinsic splendor. Even so, we must not allow the doctoring of sociography (at the level of renowned individuals in history, for example) to obscure the fact that these socially approved achievements had great difficulty coming to the surface in the first place–and that they were subsequently dishonored by deteriorating communities.

But to exist in fixation to a cumulating social role is always a depersonalized, mythified existence–even when it is producing useful output. Of course, being submerged in a social role is only one of a number of ways in which existence can be depersonalized and mythified.

My formulations give social role–or thematic identity–or imminent character–the guise of a self-caused cause or looped cause. The circuit of attachment through the person-world is not a linear causal phenomenon; it is a phenomenon of scrambled or turbulent causation. It is a dynamically balanced confined turbulence. What is awful about being submerged by a social role, in the cases known to me, is precisely that such submergence is self-reinforcing.

In "Personhood II," I had a reason for focusing on certain "ruinations" which individuals underwent. First, an ahistorical illustration. The culture may mutilate a child's faculties and inculcate him or her with debasement–without pushing the child to the point where he or she demands escape as a right or becomes a precocious social critic.

We again encounter the social doctoring of sociography–this time at the level of individual longitudinal records. Children do express distress, they do demand escape as a right, they are precocious social critics until they are subjugated. Higher and higher tolerances for anguish, or compensating rewards, have to be developed. In due course, the child begins to perpetuate the stigmas in him or herself. At the least, he or she acquiesces; at the most, he or she may become a well-rewarded advocate of the community.

A case at a different level is specific to America and the U.K. in the second half of the twentieth century. The Seventies saw the explosion of cults and ritualized degradation in America and the U.K. Then, in the Eighties, the American Establishment launched a campaign to win back the middle class; and it became fashionable to be a Yuppie. When the recession occurred at the end of the Eighties, the Yuppie role became tarnished. So social history is superficially changeable. These ebbs and flows are not the level I should address. What we should glean in this connection is that the cults and the ritualized degradation signal the long-term trend of techno-capitalist civilization.

I include these remarks on the doctoring of sociography, and on the civilization’s trend, to illustrate how early personhood theory arrived at hypothecations about society.

°

The person submerged by a social role emerges as a person who is "done to." On the other hand, in a tiny minority of cases, we have the emergence of person who "does" or "does to." Why, then, is a given person one way or the other?–and can he or she be switched from one type to the other?–and does a person who is always one type nevertheless have a potential for the other type?

Let us work through the notion of society’s imposition of the individual’s identity. Medieval serfs were illiterate and never saw money in their entire lives. Today their descendents in Western Europe all read, possess money, and spend money every day. The reason why serfs did not learn to read or to allocate money was that (in effect) they were not recruited and given cultivation to these ends.

There is a view which would say that the serfs, as a multitude which had been assigned the same fate, became aware that they were being taken advantage of in a common way, and fought for the cultivation (schools, etc.) which they subsequently received. This is not false (the French Revolution); but it is misleading in the extent to which it makes the serfs into autonomously rational protagonists. It does not take into account that the descendents of the serfs remained outside the controlling class–that the "toilers" have never commanded the system. The collapse of the workers' paradises makes this observation all the more decisive.

It is more realistic to say that advanced capitalism continually revolutionizes technology and continually erases and replaces social relationships. (Capitalism also spurs developments such as the dissolution of the nuclear family, and feminism, which the Establishment did not calculate.) So the aggregate displays a rationale which overrides the individual.

That is why the achievements and satisfactions which are possible to people are seen as results of how much cultivation the Establishment gives them.

But in personhood theory, the question of why people are what they are focuses in a different way. The topic was anticipated in sections of "Personhood II." The sociological perspective is tacitly dedicated to a doctrine of underprivilege and socially engineered redemption. Somehow that mind-set fails to engage our announced problem. Let me present a shock-question to clarify the issue.

Would a Nobel-prizewinning physicist agree

that he believes physics because his naivété

was exploited by malicious elders, because he

was crushed by his elders, because his elders

did not give him enough cultivation?	

The sociological perspective–in the name of recognizing that the serf's backwardness was imposed from without–treats the serf's effects on other people as if they were imaginary or didn't matter. It treats the serf's choices and life as if they were tuberculosis–a fatal disease which a few pennies' worth of medication could have cured.

Capitalist technology and centralization have created the possibility of imposing changed fates on entire populations. A member of the administrative class can regard all the choices and lives of a population as a reversible condition. Then people really are what the administrator chooses to make them by pushing this or that button. People are so thrilled by the prospect of human manipulation on this level–or by the prospect that the Establishment is due to give them cultivation–that they overlook that the sociological perspective makes all their choices and their lives chimerical (or revocable). "You did it because you were programmed improperly." How do you choose and act if you believe that your choices and actions have the ontological type of a disease, an error in past programming? And who says that the serf's life was "bad" or unnecessary? And yet people have learned to think in these terms–to want to be told that what their betters permit them is what they are.

A novelty may arise in how we conceive or apprehend, understand or appreciate. That led me to the notion of an unprecedented fate–of a person upholding an authentic identity-theme coming from the future. Such a person emerges as a person who "does" or "does to." Section D is devoted to this topic.

The ambition to transfer social engineering to seriousness and originality, by vaccinating people with seriousness and originality, is an ill-conceived ambition. Seriousness and originality are not "done to"; they "do (to)." They are not implanted. They appear unpredictably. (Of course, my attempt to assert my sincerity and to make the interpersonal arena conducive to it may reawaken seriousness and originality in another person.) I speculate about the authentic identity-theme which comes from the future: to show that one need not assume the social engineers' cause-and-effect. One does not even have to believe that "solutions" are fabricated from past to present.

Given my speculation about unprecedented fates, can average people be said to have routine fates? My considered answer is no. That is because to say that a person fulfills a routine fate cannot be distinguished from saying that that person is determined by the past, by circumstances.

Personhood theory refuses to acknowledge people as objectivities in a deterministic process. (Except to acknowledge that this conception itself is one of the characteristic nonsensical fantasies.) One who adopts the person-world outlook cannot consider his or her choices and life as a reversible mishap. Personhood theory cannot consider palpable choices and lives as chimeras or as revocable.

The demand for a calculus of society is, in the light of personhood theory, an ill-conceived demand. The notion of the authentic identity-theme coming from the future is introduced to show that one need not assume the social engineers' cause-and-effect. We do not even have to believe that "solutions" are fabricated from past to present.

Seriousness and originality cannot be thrust upon any given person by outside manipulation. Metaphorically, escape hatches are opened by the future, as coherent novelty, in conjunction with moments in which choice is forced–moments in which the arena of action might be reconceived, loyalty might be shifted, effectiveness and gratification might be reconceived, etc.

• • •

C. Inevitable Stages?

Marxism proves more decisively and relentlessly than any other ideology that we are robots. It then goes on to say that those of us who are in bondage should be freed. But at the level of the cogency of the ideology, if the slaves are robots, then why must they be freed? (So that there can be an exponential expansion of production? But to what end?) What difference does it make to a robot?

Before my turn to personhood theory, I indulged Marx’s historical materialism as a plausible explanation of the moral codes of past epoches. But this plausible contribution of Marxism has to be reconsidered. Perhaps the succession of stages in history (slavery, feudalism, capitalism) was necessary. But the person-world premise reconstitutes our understanding of what the stages comprised:

realized choice alongside external conditions of the moment;

realized choice and external conditions as equal constituents of a single "world."

It should also reconstitute our understanding of their necessity. The pivotal ingredient in the transition from one stage to another is an imagination and its embrace which have no sociological explanation.

The Marxist-Leftist tradition shares presuppositions of the modern Western culture of which it is a variant: blind faith in natural science; dogmatic materialism; the assumption that natural science and dogmatic materialism are allies of revolution; socio-idolatry.

Marx wanted "revolution" to transform the economic class structure while remaining relentlessly loyal to the scientific world-view. Ironically, this program may be self-frustrating. It may not be possible for a movement which preaches loyalty to the scientific world-view to gain support in late capitalist society for an insulated overturn of the economic class structure. (As I often mention, bourgeois economics has long since rooted itself in physical science.) Capitalism may be able to assimilate to its own fabric any scheme of economic liberation which proclaims the equality of people as robots and commodities.

• • •

D. Retroactive Signification

In rare cases, the individual may "steer" toward an identity which embodies coherent novelty–in that sense steering toward an authentic identity coming from the future. This depends on the earlier principle that the phenomena involve novelties in how we conceive or apprehend, understand or appreciate.

(In the present discussion, I am omitting the analysis which differentiates coherent novelty from the successful individual–from the rewarded celebrity. Today, the case has come to the foreground of a "creative submission" which is a compensatory experience of license, irresponsibility, puerile or malign misbehavior, etc. After all, criminals such as Manson become heroes; people live vicariously through them. These episodes are not what I mean by coherent novelty.

My psychology is inherently an introspective inquiry. Another principle is required which I do not expound here. The reader has to classify him or herself. The dishonest reader cheats him or herself, no more and no less. All these supporting principles are discussed in my depth psychology or in person-world analysis.)

The notion of steering toward an identity coming from the future belongs in a reconstituted discipline of psychology. All the more so because the problem of predicting individual outcomes has received so much attention in psychology–not only in the highly professionalized field of psychological testing, but in impromptu and unwritten appraisals made by psychoanalysts, etc. In turn, there are repercussions for the notion–so characteristic of sociology–that the individual’s identity is an imposition by society. There are repercussions for the notion that greatness is a gift which society gives to the individual; and there are repercussions for the interpretation of the metamorphosis of societies.

Retroactive signification means that a notion of deterministic evolution fails because of the emergence of coherent novelty. Even a liberal version of the scientific method, extrapolated to socio-psychology, would not be able to predict what certain people would become: because what they would become would in fact displace the reigning hermeneutic with an unprecedented hermeneutic. In other words, science would have to applaud its own death in order to predict the outcome.

To use the surprising outcome to upgrade the "laws" by which you analyze the "initial data" would deprive us of the lesson which the phenomenon affords. The earlier period’s "ignorance" is an essential feature of the realm being studied. It’s not scientific ignorance in the sense of lack of enough data-points to fit the curve. What the future brings are knowledges which blow up the scientist’s entire "personality." Faculties that the earlier scientist doesn’t have; successes that crush him as a person. The outcome exposes his life as a lie or sham. He is caught worshipping the wrong god.

This means that the locus of retroactive signification in the first instance is one person’s life: a course which is inherently individual, and which involves interests and ideas which fragment, conflict, and unite. The subject-matter is inherently about the antagonism of ideas and interests, about antagonisms in what anthropologists call culture. Retroactive signification is "psychological" and interpretative.

In the perspective of retroactive signification, only the future can teach the scientific observer what the past meant. He or she couldn’t have made an analysis of the past on past evidence which would have divined where it was going. It is impossible to know what the initial data mean when they appear. They are the germ of something incommensurate with his or her framework for appraisal.

The scientist of the later generation responds to coherent novelty by "growing" a new sort of "personality." He sees, in the past, what a past scientist could not have seen even if more "facts" had been provided.

That is not to say that the notion of the window to the future does not have risks–which I will note as I proceed. Why wouldn’t we blame Marx for Stalin, or Jesus for the Crusades? Here there is an answer: our interest is in the genuinely novel idea which arises. That this idea is put to use by selfish or psychopathic interests is important to the casualties–and to the historian–but does not prove that selfish or psychopathic aggression stems uniquely from the idea in question.

°

Can the notion of destiny, here called retroactive signification, be aggregated–applied to social totalities? I found retroactive signification to be almost vanishingly rare. It played the role of an exception to our much more usual apprehension that society shapes the individual.

I asked, earlier in this manuscript, whether average people could be said to have destinies which, since they are not awesomely surprising, should be called routine. My considered answer was no. To say that a person consummates a routine destiny cannot be distinguished from saying that the person is determined by the past, by circumstances.

History does not give us the same opportunity to contrast the rare outcome with the commonplace outcome that a consideration of individuals does. If one imagines that the rise of modernity in Europe was a rare outcome relative to societies which could be considered static or essentially repetitive, nevertheless modernity became prevalent and did not remain an individual possession. If modernity spreads and affects everybody, then it is indistinguishable from a Marxist "stage," or from a stage of civilization.

The rare individual’s "career" brings forth a coherent novelty which changes the basis of "knowing." Let me first clarify the connection I make between crystallization and "the future." The novel identity progressively focuses–in an individual life. That led me to say that it comes from "the future." Actually, there are cases in which the person focuses, but the public does not respond. John Philoponus. It took one thousand years for his work to be redone by successful men, and 1500 years for him to become famous.

I am suspicious of transferring the notion of an individual life-course to society so that we have the notion of a society’s career. Nations do not have selves; they are already chimeras.

The difficulty is not that there are not candidates for coherent novelty at the level of societies. [National cultures are such candidates.] Rather, a new liability appears. I was willing to recognize contributions from individuals which were diverse and relative. If we do that at the level of nations, we end up lionizing those myths which became successful. The subjective moment is lost, and all we are left with is a dominating myth. There is nothing wrong with it, except that it has lowered the discussion to the level of social history or history-of-ideas. Then we get involved with chauvinistic triumphalism.

Again: there have been many novelties at the level of national cultures which were underestimated by Establishments. But there are arguments against assigning destinies to national cultures:

1) Societies don’t have selves.

2) To recognize diverse and relative contributions at the level of societies can only mean taking successful myths as the topic.

Suppose we assume that European modernity is the fruition of humanity’s existence. To surround modernity with congratulation is dubious, since modernity brings terrible penalties from which we need to be rescued. Another profound difficulty: modernity’s judgment of the meaning of an earlier age is not necessarily worth more than that age’s own judgment of its meaning.

A major lesson emerges here. The perspective of retroactive signification is always discarding the past as merely anticipatory. But isn’t that too triumphalist? Past eras had their own values–which the future may not improve on.

When the future has the character of a regression to the status quo ante (as it often does)–possibly combined with a displacement of society’s preoccupations to other axes of antagonism–then the notion that this outcome consummates a destiny is disappointing. When the future can be conceived as a regression or mere displacement, then the future’s judgment of the past’s meaning can be a retreat into retardation. (Philosophy’s judgment of Hume, which regressed from his achievement.)

Is modern natural science the fruition of Greek natural philosophy? The trouble is that we may learn far more from Aristotle if we do not simply read him to see where he agreed with "us."

Ancient Judaism was underestimated by pagan élites, and so presaged coherent novelty. But what was its fruition? There is not a unique answer. To give an answer will almost automatically be invidious; unless we confine ourselves to commonplaces about the generic influence of Biblical religion. Again, that is a theme for history-of-ideas or social history.

Again, to apply the notion of destiny to society would only converge with Marx’s stages of history or with history-of-ideas. Surely retroactive signification’s liberating implications lie in a different direction.

°

Sociology has promulgated the cliche that society shapes the individual–or even that the individual’s identity is an imposition by society. Retroactive signification is credible in the individual life: that militates against sociological causation of the individual. A reconception of the way the individual is "inlaid" in society is demanded. Recall that one’s private conflicts over the skills with which one has been indoctrinated can evince vital dilemmas and vital ventures for the interpersonal arena.

Sociological causation of the individual is impressive only to the cynic. Personhood theory refuses to acknowledge people as objectivities in a deterministic process. One who adopts the person-world standpoint cannot consider his or her choices and life as a revocable mishap. Personhood theory cannot consider palpable choices and lives as chimeras–or as revocable.

In the rare case that one's authentic identity-theme comes from the future, guiding oneself toward it remains a matter of pronounced willfulness in a context of uncertainties. It is possible to drift rather than to push toward the distant identity-theme. And subjectively I often have to gamble–even if my purpose remains fixed. (By upholding or relinquishing the identity-theme from the future, one guarantees or nullifies it as a future?)

The scope of "choice" includes the possibility of shaping your loyalties. Such shaping of loyalties covers

–reconceiving effectiveness and gratification;

–reconceiving the purpose of life;

–reconceiving the arena of action.

In speaking of altering your loyalties, everything up to and including the determination of reality is open. When the individual is being "attracted by" an unprecedented fate, choice in a moment of crisis can be seen as a phenomenon in which the remote future contacts the present. The crisis gives one some choice over the way one's distant future shapes one's present.

The "career" which is interpreted as retroactive signification is correlative to seriousness and originality–and seriousness and originality cannot be instilled by outside manipulation. In turn, whether the individual will be cognitively protean, which is what I wanted to know, presumably depends on seriousness and originality.

Because we are talking about a novelty which depends on vital dilemmas for the collectivity which the collectivity doesn't acknowledge, the person who expresses the novelty refuses the depersonalization of social role.

• • •

E. Dissolution of Natural Society

Let me now consummate the dissolution of natural society and its rectilinear career as an ontological category. Drawing on previous writings, I sketch a hypothetical civilization outside the plane of natural society. [That means appealing even more urgently to personhood theory.]

In this hypothetical civilization, the collective can freely change the laws of nature. That presupposes claims, made previously and elsewhere, that scientific reality can be superseded. There is a dispelling of deceit and gullibility, concomitantly with the awakening of faculties, and with emotional sensitization: yielding intellectual techniques which supersede the compartmentation of faculties characterizing the present culture. Thereby, new mental abilities are invented. The community is open to avenues of metamorphosis of the life-world. The comprehensively assembled "meta-technology" would be self-conscious about the inherited view of factual reality, going beyond it in an operative way. Again, my perspective is that of a novel arena which outruns what was formerly considered factual reality. (My meta-technological writings, etc., are a prerequisite for understanding the terminology of the requirements to follow.)

The envisioned mode of life invokes dimensions of human potentiality which hitherto were supported only by different cultures. I'm seeking a unitary experience which transmits many dimensions of potentiality.

My interest here is with the ramifications of these claims for interpersonal life. If meta-technology could be implemented collectively, we would accede to an uncanny life-world. To express the matter from a present-day standpoint, the new mode of life would be a waking-dream reality or enchanted reality.

In order for a collective to be able freely to change the laws of nature, all persons would have to have parity of "station in life" and parity of authority in the culture. Moreover, the total of menial and routine labor would have to decrease to the vanishing point.

Let me consolidate here all the consequences of direct import for the present discussion.

×	One intellectual consequence is that the realism of history would be placed in suspension. The higher civilization would consign history to a lesser grade of realism. The supposed edifying effect of history is dispensable. Whereas today, we need to preserve traditional culture as a bulwark against dehumanization by the current culture, the higher civilization would mean a revival of personalistic and hallowed expression, on a new level: "soul" would not longer reside only in old languages, old buildings, old statues, old texts.

From another angle, the motive for people to keep score as to their ancestral status (or lack of it) would disappear. "Consciousness" could break free of its material antecedents (circumstances).

×	The higher civilization presupposes an intellectual defeat for physics; for Marx’s materialism; and for all the doctrines which hold that capitalism is necessitated by physico-biological nature itself. See (A) and (1) below.

×	The new mode of life is not compatible with a social order in which most people are consigned to material servitude. Not only would the sought-for inspiration not appear; the uncanny instrumental activity or meta-technology would not appear.

So it's not like Pakistan and the atomic bomb (or the priesthood in ancient Egypt)–advanced technology coexisting with a population of paupers or slaves. See (E) below.

°

The following principles are requirements–expressed as if from within the new mode of life, in the new terminology. Parenthesized numbers refer to comments on each statement, collected in the following section.

A. The life-world (lived experience) is understood as an integration of:

–substantial, operative interdependencies of awareness and objectivity;

–the conventionalistic grading of experiences (as to "realism");

–logically impossible situations (states of the world)–i.e. situations requiring simultaneous mutually exclusive descriptions in the medium of thought inherited from scientific civilization.

The principle of the personality's orientation in "reality" is: consciously to maneuver through the logically impossible world-states, manifesting instrumental mastery over objectivities inherited from the previous civilization. (I.e. scientific objectivities). (1)

B. The foregoing cannot be achieved merely by adopting a neutral, inert mental state, by positioning oneself mentally relative to propositions.* Sustainable inspiration (exalted centered activation and presence) and uncanny states of consciousness are required.

C. The principles of evaluational processing of experience (or grading of experience) which underlie a novel determination of reality are shared or collective. Only thus can novel determinations of reality be promulgated in the life-world.

D. The novel determinations of reality are linked to emotionally supportive intersubjectivity. Only thus can the novel determinations of reality appeal to a community.

E. The other persons have parity of "station in life" and parity of authority in the culture with "the self" ("this individual," myself). Only thus can they stimulate inspiration and uncanny states in "this individual."

F. The community from which people concretely originate and "learn to feel" becomes the same community that pursues mastery over scientific objectivities and gains an uncanny or ecstatic sense of the world. Inasmuch as the required shared principles of grading experience, and the required intersubjective emotional gratification, connect, a person-configuration freed from demeaned pragmatism is evinced. (2)

G. The individual experiences "desirables" as qualitatively specific.

H. The individual insists on the satisfaction of the qualitatively specific and unequal needs of self and peers for the material requisites of life. (To recognize inequality of individual needs does not mean endorsing different grades of reward. To resolve competing claims, a representative body is needed.) (3)

I. Production of the material requisites of life is planned by a representative body to shrink necessary labor time. (Automated collectivism.)

J. Individual and the collective entertain spontaneous "amusement" or "play" ("brend"), without seeking to displace or objectify it.

K. Sensuous-concrete vehicles for the collective expression of exalting values are encouraged.

L. Individual and collective are receptive to future novelty which is unpredictable and incomparable and yet is coherent or thematic. (4)

°

Next, the commentary, which is expressed in the old terminology.

(1) Self-subsistent objectivities, and affirmative consistent theories, would no longer be sought as foundations of reality. As far as the physical world is concerned, a fragment of what I envision is provided by my "Superseding Scientific Apprehension of the Inanimate World: The Phenomenological Basis of Physics" (1990).

(2) Here uncanniness and ecstasis are positioned as notions reactive to everyday banality. In the new mode of life such counterposition would no longer be necessary.

(3) This statement on satisfaction of needs is pertinent so long as a separate sphere of material requisites of life can be distinguished.

(4) To the present civilization, the new mode of life would seem a waking-dream-reality or enchanted reality.

\visbreak

F. Recapitulation

Let me try again to specify the object of social science. A world-wide aggregate of humans on a geological or biological time-line whose future is determined by efficient causation. No author but me would remark that this object is a phantom. Nonetheless, the urgency of rotating out of social science stems from considerations in lived experience; considerations which envision a novel existence whose preconditions have begun to be worked out in theory. These are concerns unique to me which have occupied me for many years.

The exposition is rambling and not yet sorted out. (That applies especially to the overhanging progressivist identitifcation of ‘futural’ with ‘superior’.) But never mind that. The "society" discerned by social science is a hypnotically instilled hallucination. The notion of causation which subtends it is humiliating and enslaving. I have exposed crucial junctures at which sociological causation–deeply plausible though it may be–is annulled. Beyond that, there comes a point in historical time at which the historical time-axis evaporates. The collectivity awakens from, outgrows, the imaginary order with which it had surrounded itself.