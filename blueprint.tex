\documentclass[10pt,twoside]{memoir}

\usepackage{salitter}
\usletterlayout

\usepackage{mwe}
\usepackage{csquotes}
\usepackage{bbold}
\usepackage{stix}

\coelfont

\newcommand{\speech}[1]{
	\textquote{\emph{#1}}}

\newcommand{\essaytitle}[1]{
	\emph{#1}}

\begin{document}

\graphicspath{{img/}}
\pagestyle{ruled}

{
\thispagestyle{empty}
\img{creep.png}

Henry Flynt presents "Creep" lecture in Adam Hovre upper common room, Harvard 
University, May 15, 1962 


(photo by Tony Conrad) 
\clearpage
}

\tableofcontents*

\mainmatter

\chapter{Introduction}


This essay is the third in a series on the rationale of my career. It 
summarizes the results of my activities, the consistent outlook on a whole 
range of questions which I have developed. The first essay, 
\essaytitle{On Social Recognition}, noted that the official social philosophy of practically every 
regime in the world says that the individual has a duty to serve society to the 
best of his abilities. Social recognition is supposed to be the reward which 
indicates that the individual is indeed serving society. Now it happens that 
the most important tasks the individual can undertake are tasks (intellectual, 
political, and otherwise) posed by society. However, when the individual 
undertakes such tasks, society's actual response is almost always persecution 
(Galileo) or indifference (Mendel). Thus, the doctrine that the'individual has 
a duty to serve society is a hypocritical fraud. I reject every social 
philosophy which contains this doctrine. The rational individual will obtain 
the means of subsistence by the most efficient swindle he can find. Beyond 
this, he will undertake the most important tasks posed by society for his 
own private gratification. He will not attempt to benefit society, or to gain 
the recognition which would necessarily result if society were to utilize his 
achievements. 

The second essay, \essaytitle{Creep}, discussed the practices of isolating oneself; 
carefully controlling one's intake of ideas and influences from outside; and 
playing as a child does. I originally saw these practices as the effects of 
certain personality problems. However, it now seems that they are actually 
needed for the intellectual approach which I have developed. They may be 
desirable in themselves, rather than being mere effects of personality 
problems. 

I chose fundamental philosophy as my primary subject of investigation. 
Society presses me to accept all sorts of beliefs. At one time it would have 
pressed me to believe that the earth was flat; then it reversed itself and 
demanded that I believe the earth is round. The majority of Americans still 
consider it "necessary" to believe in God; but the Soviet government has 
managed to function for decades with an atheistic philosophy. Thus, which 
beliefs should I accept? My analysis is presented in writings entitled 
\essaytitle{Philosophy Proper}, \essaytitle{The Flaws Underlying Beliefs}, and 
\essaytitle{Philosophical Aspects of Walking Through Walls}. 
The question of whether a given belief is valid 
depends on the issue of whether there is a realm beyond my "immediate 
experience." Does the Empire State Building continue to exist even when I 
am not looking at it? If such a question can be asked, there must indeed be 
a realm beyond my experience, because otherwise the phrase 'a realm 
beyond my experience' could not have any meaning. (Russell's theory of 
descriptions does not apply in this case.) But if the assertion that there is a 
realm beyond my experience is true merely because it is meaningful, it 
cannot be substantive; it must be a definitional trick. In general, beliefs 
depend on the assertion of the existence of a realm beyond my experience, 
an assertion which is nonsubstantive. Thus, beliefs are nonsubstantive or 
meaningless; they are definitional tricks. Psychologically, when I believe that 
the Empire State Building exists even though I am not looking at it, I 
imagine the Empire State Building, and I have the attitude toward this 
mental picture that it is a perception rather than a mental! picture. The 
attitude involved is a self-deceiving psychological trick which corresponds to 
the definitional trick in the belief assertion. The conclusion is that al! beliefs 
are inconsistent or self-deceiving. It would be beside the point to doubt 
beliefs, because whatever their connotations may be, logically beliefs are 
nonsense, and their negations are nonsense also. 

The important consequence of my philosophy is the rejection of truth 
as an intellectual modality. I conclude that an intellectual activity's claim to 
have objective value should not depend on whether it is true; and also that 
an activity may perfectly weil employ false statements and still have 
objective value. I have developed activities which use mental capabilities that 
are excluded by a truth-oriented approach: descriptions of imaginary 
phenomena, the deliberate adoption of false expectations, the thinking of 
contradictions, and meanings which are reversed by the reader's mental 
reactions; as well as illusions, the deliberate suspension of normal beliefs, and 
phrases whose meaning is stipulated to be the associations they evoke. It 
must be clear that these activities are not in any way whatever a return to 
pre-scientific trrationalism. My philosophy demolishes astrology even more 
than it does astronomy. The irrationalist is out to deceive you; he wants you 
to believe that his superstitions are truths. My activities, on the other hand, 
explicitly state that they are using non-true material. My intent is not to get 
you to believe that superstitions are truths, but to exploit non-true material 
for rational purposes. 

The other initial subject of investigation I chose was art. The art which 
claims to have cognitive value is already demolished by my philosophical 
results. However, art at its most distinctive does not need to claim cognitive 
value; its value is claimed to be entertainmental or amusemental. What about 
art whose justification is simply that people like it? Consider things which 
are just liked, or whose value is purely subjective. I point out that each 
individual already has experiences, prior to art, whose value is purely 
subjective. (Call these experiences "brend.") The difference between brend 
and art is that in art, the thing valued is separated from the valuing of it and 
turned into an object which is urged on other people. Individuals tend to 
overlook their brend, and they do so because of the same factors which 
perpetuate art. These factors include the relation between the socialization 
of the individual and the need for an escape from work. The conditioning 
which causes one to venerate "great art" is also a conditioning to dismiss 
one's own brend. If one can become aware of one's brend without the 
distortion produced by this conditioning, one finds that one's brend is 
superior to any art, because it has a level of personalization and originality 
which completely transcends art. 

Thus, I reject art as an intellectual or cultural modality. In rejecting 
truth, I advocated in its place intellectual activities which have an objective 
value independent of truth. In rejecting art, I do not propose that it be 
replaced with any objective activity at all. Rather, I advocate that the 
individual become aware of his just-likings for what they are, and allow them 
to come out. If I succeed in getting the individual to recognize his own 
just-likings, then I will have given him infinitely more than any artist ever 
can. 

We are not finished with art, however. Ever since art began to 
disintegrate as an institution, modern art has become more and more of a 
repository for activities which represent pure waste, but which counterfeit 
innovation and objective value. A two-way process is involved here. On the 
one hand, the modern artist, faced with the increasing gratuitousness of his 
profession, desperately incorporates superficial references to science in his 
products in the hope of intimidating his audience. On the other hand, art 
itself has become an institution which invests waste with legitimacy and even 
prestige; and it offers instant rewards to people who wish to play the game. 
What is innovation in modern art? You take a poem by Shelly, cut it up into 
little pieces, shake the pieces up in a box, then draw them out and write 
down whatever is on them in the order in which they are drawn. If you call 
the result a "modern poem," people will suddenly be awed by it, whereas 
they would not have been awed otherwise. This sort of innovation is utterly 
mechanical and superficial. When artists incorporate scientific references in 
their products, the process is similarly a mechanical, superficial 
amalgamation of routine artistic material with current gadgets. 

Now there may be some confusion as to what the difference is between 
the products which result from this attempt to "save" art, and activities in 
the intellectual modality which I favor. There may be a tendency to confuse 
activities which are neither science nor art, but have objective value, with art 
products which are claimed to be "scientific" and therefore objectively 
valuable. To dispel this confusion, the following questions may be asked 
about art products. 
\begin{enumerate}
\item If the product were not called art, would it immediately be seen to be 
worthless? Does the product rely on artistic institutions to "carry" it? 

\item Suppose that the artist claims that his product embodies major scientific 
discoveries, as in the case of a ballet dancer who claims to be working in the 
field of antigravity ballet. If the dancer really has an antigravity device, 
why can it only work in a ballet theater? Why can it 
only be used to make dancers jump higher? Why do you have to be able to 
perform "Swan Lake" in order to do antigravity experiments? 
\end{enumerate}
To use a phrase from medical research, I contend that a real scientist would seek to 
isolate the active principle---not to obscure it with non-functional mumbo-jumbo. 

Both of these sets of questions make the same point, from somewhat 
different perspectives. Given an individual with a product to offer, does he 
actively seek out the lady art reporters, the public relations contracts, the 
museum officials, or does he actively dissociate himself from them? Does he 
seek artistic legitimation of his product, or does he reject it? The objective 
activities which I have developed stand on their own feet. They are not art, 
and to construe them as art would make it impossible to comprehend them. 

A definition of the intellectual modality which I favor is now in order. 
Until now, this modality has involved the construction of ideas such that the 
very possibility of thinking these ideas is a significant phenomenon. In other 
words, the modality has consisted of the invention of mental abilities. The 
ideas involve physical language, that is, language which occurs in beliefs 
about the physical world. Such language is philosophically meaningless, but 
it has connotations provided by the psychological trick involved in believing. 
The connotations are what are utilized; factual truth is irrelevant. Then, the 
ideas cannot be reduced to the mechanical manipulation of marks or 
counters---unlike ordinary mathematics. Also, logical truth, which happens to 
be discredited by my philosophical results, is irrelevant to the ideas. 

But the defining requirement of the modality is that each activity in it 
must have objective value. The activity must provide one with something 
which is useful irrespective of whether one likes it; that is, which is useful 
independently of whether it produces emotional gratification. 

We can now consider the following principle. "spontaneously and 
without any prompting to sweep human culture aside and to carry out 
elaborate, completely self-justifying activities." Relative to the social context 
of the individual's activities, this principle is absurd. We have no reason to 
respect the eccentric hobbyist, or the person who engages in arbitrary 
antisocial acts. If an action is to have more than merely personal significance, 
it must have a social justification, as is explained in On Social Recognition. 
In the light of The Flaws Underlying Beliefs and the brend theory, however, 
the principle mentioned above does become valid when it is interpreted 
correctly, because it becomes necessary to invent ends as well as means. The 
activity must provide an objective value, but this value will no longer be 
standardized. 

The modality I favor is best exemplified by \essaytitle{Energy Cube Organism},
\essaytitle{Concept Art}, and the \essaytitle{Perception-Dissociator Model}. 
\essaytitle{Energy Cube Organism} is a perfect example of ideas such that the very 
possibility of thinking them is a significant phenomenon. It is also a perfect example of an 
activity which is useful irrespective of whether it provides emotional 
gratification. It combines the description of imaginary physical phenomena 
with the thinking of contradictions. It led to \essaytitle{Studies in Constructed 
Memories}, which in turn led to \essaytitle{The Logic of Admissible Contradictions}.
With this last writing, it becomes obvious that the activity has applications 
outside itself. 

\essaytitle{Concept Art}\footnote{published in An Anthology ed. LaMonte Young, 1963}
uses linguistic expressions which are changed by the reader's mental 
reactions. It led to \essaytitle{Post-Formalism in Constructed Memories}, and this led 
in turn to \essaytitle{Subjective Propositional Vibration}.

The \essaytitle{Perception-Dissociator Model}\footnote{published in I-KON, Vol. 1, No. 5} 
was intended to exploit the realization that humans are the most 
advanced machines (or technology) that we have. I wanted to build a model 
of a machine out of humans, using a minimum of non-human props. Further, 
the machine modelled was to have capabilities which are physically 
impossible according to present-day science. I still think that the task as I 
have defined it is an excellent one; but the model does not yet completely 
accomplish the objective. The present model uses the deliberate suspension 
of normal beliefs to produce its effects. 

\essaytitle{Post-Formalism in Constructed Memories} and \essaytitle{Studies in 
Constructed Memories} together make up \booktitle{Mathematical Studies} (1966). In 
this monograph, the emphasis was on extending the idea of mathematics as 
formalistic games to games involving subjectivity and contradiction. In two 
subsequent monographs, the material was developed so as to bring out its 
potential applications in conjunction with science. 
\essaytitle{Subjective Propositional Vibration} investigates the logical 
possibilities of expressions which are changed by the reader's mental responses.
\essaytitle{The Logic of Admissible Contradictions} starts with the experiences 
of the logically impossible which 
we have when we suffer certain perceptual illusions. These illusions enable us 
to imagine certain logical impossibilities just as clearly as we imagine the 
logically possible. The monograph models the content of these illusions to 
obtain a system of logic in which some (but not all) contradictions are 
"admissible." The theory investigates the implications of admitting some 
contradictions for the admissibility of other contradictions. A theory of 
many-valued numbers is also presented. 

The \essaytitle{Perception-Dissociator Model} led to 
\essaytitle{The Perception-Dissociation of Physics.} Again, here is an essay whose 
significance lies in the very possibility of thinking the ideas at all. The essay 
defines a change in the pattern of experience which would make it 
impossibie for physicists to "construct the object from experience." Finally, 
\essaytitle{Mock Risk Games} is the activity which involves the deliberate adoption of 
false expectations. It is on the borderline of the intellectual modality which I 
favor, because it seems to me to have objective value, and yet has not 
generated a series of applications as the other activities have. 

To summarize my general outlook, truth and art are discredited. They 
are replaced by an intellectual modality consisting of non-true activities 
having objective value, together with cach individual's brend. Consider the 
individual who wishes to go into my intellectual modality. What is the 
significance to him of the academic world, professional occupations, and the 
business of scholarships, fellowships, and grants? From the perspective of 
the most socially important tasks, these institutions have always rewarded 
the wrong things, as I argued in \essaytitle{On Social Recognition}. But in addition, the 
institutions as now organized are obstacles specifically to my intellectual 
modality. In fact, society in general has the effect of a vast conspiracy to 
prevent one from achieving the kind of consequential intellectual play which 
I advocate. The categories of thought which are obligatory in the official
intellectual world and the media are categories in which my outlook cannot 
be conceived. And here is where the creep practices mentioned at the 
beginning of this essay become important. Isolation from society is 
presumably not inherent in my intelectual modality; but under present 
social conditions isolation is a prerequisite for its existence. 


\part{PHILOSOPHY}


\chapter{The Flaws Underlying Beliefs}


We begin with the question of whether there is a realm beyond my 
"immediate experience." Does the Empire State Building continue to exist 
even when I am not looking at it? If either of these questions can be asked, 
then there must indeed be a realm beyond my experience. If I can ask 
whether there is a realm beyond my experience, then the answer must be 
yes. The reason is that there has to be a realm beyond my experience in 
order for the phrase "a realm beyond my experience" to have any meaning. 
Russell's theory of descriptions will not work here; it cannot jump the gap 
between my experience and the realm beyond my experience. The assertion 
\speech{There is a realm beyond my experience} is true if it is meaningful, and that 
is precisely what is wrong with it. There are rules implicit in the natura! 
language as to what is semantically legitimate. Without a rule that a 
statement and its negation cannot simultaneously be true, for example, the 
natural language would be in such chaos that nothing could be done with it. 
Aristotle's \booktitle{Organon} was the first attempt to explicate this structure formally, 
and Supplement D of Carnap's \booktitle{Meaning and Necessity} shows that hypotheses 
about the implicit rules of a natural language are well-defined and testable. 
An example of implicit semantics is the aphorism that \enquote{saying a thing is so 
doesn't make it so.} This aphorism has been carried over into the semantics 
of the physical sciences: its import is that there is no such thing as a 
substantive assertion which is true merely because it is meaningful. If a 
statement is true merely because it is meaningful, then it is too true. It must 
be some kind of definitional trick which doesn't say anything. And this is 
our conclusion about the assertion that there is a realm beyond my 
experience. Since it would be true if it were meaningful, it cannot be a 
substantive assertion. 

The methodology of this paper requires special comment. Because we 
are considering ultimate questions, it is pointless to try to support our 
argument on some more basic, generally accepted account of logic, language, 
and cognition. After all, such accounts are being called into question here. 
The only possible pproach for this paper is an internal critique of common 
sense and the natural language, one which judges them by reference to 
aspects of themselves. 

As an example of the application of our initial result to specific 
questions of belief, consider the question of whether the Empire State 
Building continues to exist when I am not looking at it. If this question is 
even meaningful, then there has to be a realm in which the nonexperienced 
Empire State Building does or does not exist. This realm is precisely the 
realm beyond my experience. The question of whether the Empire State 
Building continues to exist when I am not Jooking at it depends on the very 
assertion, about the existence of a realm beyond my experience, which we 
found to be nonsubstantive. Thus, the assertion that the Empire State 
Building continues to exist when I am not looking at it must also be 
considered as nonsubstantive or meaningless, as a special case of a 
definitional trick. 

We start by taking questions of belief seriously as substantive questions, 
which is the way they should be taken according to the semantics implicit in 
the natural language. The assertion that God exists, for example, has 
traditionally been taken as substantive; when American theists and Russian 
atheists disagree about its truth, they are not supposed to be disagreeing 
aboui nothing. We find, however, that by using the rules implicit in the 
natural language to criticize the natural language itself, we can show that 
belief-assertions are not substantive. 

Parallel to our analysis of belief-assertions or the realm beyond my 
experience, we can make an analysis of beliefs as mental acts. (We 
understand a belief to be an assertion referring to the realm beyond my 
experience, or to be the mental act of which the assertion is the verbal 
formulation.) Introspectively, what do I do when I believe that the Empire 
State Building exists even though I am not looking at it? I imagine the 
Empire State Building, and I have the attitude toward this mental picture 
that it is a perception rather than a mental picture. Let us bring out a 
distinction we are making here. Suppose I see a table. I have a so-called 
perception of a table, a visual table-experience. On the other hand, I may 
close my eyes and imagine a table. Independently of any consideration of 
"reality," two different types of experiences can be distinguished, 
non-mental experiences and mental experiences. A belief as a mental act 
consists of having the attitude toward a mental experience that it is a 
non-mental experience. The "attitude" which is involved is not a 
proposition. There are no words to describe it in greater detail; only 
introspection can provide examples of it. The attitude is a self-deceiving 
psychological trick which corresponds to the definitional trick in the 
belief-assertion. 

The entire analysis up until now can be carried a step farther. So far as 
the formal characteristics of the problem are concerned, we find that 
although the problem originally seems to center on "nonexperience," it 
turns out to center on "language." Philosophical problems exist only if there 
is language in which to formulate them. The flaw which we have found in 
belief-assertions has the following structure. A statement asserts the 
existence of something of a trans-experiential nature, and it turns out that 
the statement must be true if it is merely meaningful. The language which 
refers to nonexperience can be meaningful only if there is a realm beyond 
experience. The entire area of beliefs reduces to one question: are linguistic 
expressions which refer to nonexperience meaningful? We remark 
parenthetically that practically all language is supposed to refer to 
nonexperiences. Even the prosaic word 'table' is supposed to denote an 
object, a stable entity which continues to exist when I am not looking at it. 
Taking this into account, we can reformulate our fundamental question as 
follows. Is language meaningful? Is there a structure in which symbols that 
we experience (sounds or marks) are systematically connected to objects, to 
entities which extend beyond our experience, to nonexperiences? !n other 
words, is there language? (To say that there is language is to say that half of 
all belief-assertions are true. That is, given any belief-assertion, either it is 
true or its negation is true.) Thus, the only question we need to consider is 
whether language itself exists. But we see immediately, much more 
immediately than in the case of 'nonexperience,' that this question is 
caught in a trap of its own making. The question ought to be substantive. (Is 
there a systematic relation between marks and objects, between marks and 
nonexperiences? Is there an expression, 'Empire State Building,' which is 
related to an object outside one's experience, the Empire State Building, and 
which therefore has the same meaning whether one is looking at the Empire 
State Building or not? ) However, it is quite obvious that if one can even ask 
whether there is language, then the answer must be affirmative. Further, the 
distinction of language levels which is made in formal languages will not help 
here. Before you can construct formal languages, you have to know the 
natural language. The natural language is the infinite level, the container of 
the formal languages. If the container goes, everything goes. And this 
container, this infinite level language, must include its own semantics. There 
is no way to "go back before the natural language." As we mentioned 
before, the aphorism that 'saying a thing is so doesn't make it so" is an 
example of the natural language's semantics in the natural language. 

in summary, the crucial assertion is the assertion that there is language, 
made in the natural language. This assertion is true if it is meaningful. It is 
too true; it must be a definitional trick. Beliefs stand or fal! on the question 
of whether there is language. There is no way to get outside the definitional 
trick and ask this question in a way that would be substantive. The question 
simply collapses. 

\chapter{Philosophical Aspects of Walking Through Walls}


We read that in the Middle Ages, people found it impossible not to 
believe that they would be struck by lightning if they uttered a blasphemy. 
Yet I utterly disbelieve that I will be struck by lightning if I utter a 
blasphemy. Beliefs such as the one at issue here will be called fearful beliefs. 
Elsewhere, I have argued that all beliefs are self-deceiving. I have also 
observed that there are often non-cognitive motives for holding beliefs, so 
that a technical, analytical demonstration that a belief is self-deceiving wil! 
not necessarily provide a sufficient motive for renouncing it. The question 
then arises as to why people would hold fearful beliefs. It would seem that 
people would readily repudiate beliefs such as the one about blasphemy as 
soon as there was any reason to doubt them, even if the reason was abstract 
and technical. Yet fearful beliefs are held more tenaciously than any others. 
Further, when philosophers seek examples of beliefs which one cannot 
afford to give up, beliefs which are not mere social conventions, beliefs 
which are truly objective, they invariably choose fearful beliefs. 

Fearful beliefs raise some subtle questions about the character of beliefs 
as mental acts. If I contemplate blasphemy, experience a strong fear, and 
decide not to blaspheme, do I stand convicted of believing that I will be 
punished if I blaspheme, or may I claim that I was following an emotional 
preference which did not involve any belief? Is there a distinction between 
fearful avoidance and fearful belief? Can the emotion of fear be 
self-deceiving in and of itself? Must a belief have a verbal, propositional 
formulation, or is it possible to have a belief with no linguistic representation 
whatever? 

It is apparent that fearful beliefs suggest many topics for speculation. 
This essay, however, will concentrate exclusively on one topic, which is by 
far the most important. Given that people once held the belief about 
blasphemy, and that I do not, then I have succeeded in dispensing with a 
fearful belief. Two beliefs which are exactly analogous to the one about 
blasphemy are the belief that if I jump out of a tenth story window I will be 
hurt, and the belief that if I attempt to walk through a wali I wil! bruise 
myself. Given that I am able to dispense with the belief about blasphemy, it 
follows that, in effect, I am able to walk through walls relative to medieval 
people. That is, my ability to blaspheme without being struck by lightning 
would be as unimaginable to them as the ability to walk through walls is 
today. The topic of this essay is whether it is possible to transfer my 
achievement concerning blasphemy to other fearful beliefs. 

\visbreak

I am told that \enquote{if you jump out of a tenth story window you really will 
be hurt.} Yet the analogous exhortation concerning blasphemy is not 
convincing or compelling at all. Why not? I suggest that the nature of the 
"evidence" implied in the exhortation should be examined very closely to 
see if it does not represent an epistemological swindle. In the cases of both 
blasphemy and jumping out of the window, I am told that if I perform the 
action I will suffer injury. But do I concede that I have to blaspheme, in 
order to prove that I can get away with it? Actually, I do not blaspheme; I 
simply do not perform the action at all. Yet I do not have any belief 
whatever that it would be dangerous to do so. Why should anyone suppose 
that because I do not believe something, I have to run out in the street, 
shake my fist at the sky, and curse God in order to validate may disbelief? 
Why should the credulous person be able to put me in in the position of 
having to accept the dare that "you have to do it to prove you don't believe 
it's dangerous'? Could it not be that this dare is some sort of a swindle? 
The structure of the evidence for the supposedly unrelinquishable belief 
should be examined very closely to see if it is not so much legerdemain. 

The exhortation continues to the effect that if I did utter blasphemy I 
really would be struck by lightning. I stil! do not find this compelling. But 
suppose that I do see someone utter a blasphemy and get struck by lightning. 
Surely this must convert me. But with due apologies to the faithful, I must 
report that it does not. There is no reason why it should make me believe. I 
do not believe that blaspheming will cause me to be struck by lightning, and 
the evocation of frightful images---or for that matter, something that I 
see---would provide no reason whatever for sudden credulity. There is an 
immense difference between seeing a person blaspheme and get struck by 
lightning, and believing that if one blasphemes, one will get struck by 
lightning. This difference should be quite apparent to one who does not hold 
the belief.\footnote{In more conventional terms, the civilization in which I tive is so 
profoundly secular that its secularism cannot be demolished by one 
"sighting."}

In general, the so-called evidence doesn't work. There is a swindle 
somewhere in the evidence that is supposed to make me accept the fearful 
belief. Upon close scrutiny, each bit of evidence misses the target. Yet the 
whole conglomeration of "evidence" somehow overwhelmed medieval 
people. They had to believe something that I do not believe. I can get away 
with something that they could not get away with. 

It is not that I stand up in a society of the faithful and suddenly 
blaspheme. It is rather that the whole medieva! cognitive orientation had 
been completely reoriented by the time it was transmitted to me. Or in other 
words, the medieval cognitive orientation was restructured throughout 
during the modern era. In the process, the compelling conglomeration of 
evidence was disintegrated. Isolated from their niches in the old orientation, 
the bits of evidence no longer worked. Each bit missed the target. I do not 
have a head-on confrontation with the medieval impossibility of 
blaspheming. I slip by the impossibility, where they could not, because I 
structure the entire situation, and the evidence, differently. 

The analysis just presented, combined with analyses of beliefs which I 
have made elsewhere, assures me that the belief that 'if I try to walk 
through the wall I wil! fail and will bruise myself" is also discardable. I am 
sure that I can walk through walls just as successfully as I can blaspheme. 
But to do so will not be trivial. As I have shown, escaping the power of a 
fearful belief is not a matter of head-on confrontation, but of restructuring 
the entire situation, of restructuring evidence, so that the conglomeration of 
evidence is disintegrated into isolated bits which are separately powerless. 
Only then can one slip by the impossibility. I cannot exercise my freedom to 
walk through walls until the whole cognitive orientation of the modern era is 
restructured throughout. 

The project of restructuring the modern cognitive orientation is a vast 
one. The natural sciences must certainly be dismantled. In this connection it 
is appropriate to make a criticism about the logic of science as Carnap 
rationalized it. Carnap considered a proposition meaningful if it had any 
empirically verifiable proposition as an implication. But consider an 
appropriate ensemble of scientific propositions in good standing, and 
conceive of it as a conjunction of an infinite number of propositions about 
single events (what Carnap called protocol-sentences). Only a very small 
number of the latter propositions are indeed subject to verification. If we 
sever them from the entire conjunction, what remains is as effectively 
blocked from verification as the propositions which Carnap rejected as 
meaningless. This criticism of science is not a mere technical exercise. A 
scientific proposition is a fabrication which amalgamates a few trivially 
testable meanings with an infinite number of untestable meanings and 
inveigles us to accept the whole conglomeration at once. It is apparent at the 
very beginning of \booktitle{Philosophy and Logical Syntax} that Carnap recognized this 
quite clearly; but it did not occur to him to do anything about it. For us, 
however, it is essential to be assured that science can be dismantled just as 
the proof can be dismantled that I will be struck by lightning if I blaspheme. 

We can suggest some other approaches which may contribute to 
overcoming the modern cognitive orientation. The habitual correlation of 
the realm of sight and the realm of touch which occurs when we perceive 
"objects" is a likely candidate for dismantling.\footnote{The psychological jargon for 
this correlation is "the contribution of intermodal organization to the 
object Gestalt."}

From a different traditon, the critique of scientific fact and of 
measurable time which is suggested in Luk\'{a}cs' \booktitle{Reification and the 
Consciousness of the Proletariat} might be of value if it were developed.\footnote{Lulkacs also implied that scientific truth would disappear in a communist 
society---that is, a society without necessary labor, in which the right to 
subsistence was unconditional. He implied that scientific quantification and 
facticity are closely connected with the work discipline required by the 
capitalist mode of production; and that like the price system, they constitute 
a false objectivity which we accept because the social economic institutions 
deprive us of subsistence if we fail to submit to them. Quite aside from the 
historical unlikelihood of a communist society, this suggestion might be 
pursued as a thought experiment to obtain a more detailed characterization 
of the hypothetical post-scientific outlook.}

Finally, I may mention that most of my own writings are offered as 
fragmentary beginnings in the project of dismantling the modern cognitive 
orientation. 

Someday we will realize that we were always free to walk through 
walls. But we could not exercise this freedom because we structured the 
whole situation, and the evidence, in an enslaving way. 

\chapter{Philosophical Reflections I}

\begin{enumerate} % TODO letters, sub numbers
\item If language is nonsense, why do we seem to have it? How do these 
intricate pseudo-significant structures arise? If beliefs are self-deceiving, why 
are they there? Why are we so skilled in the self-deceptive reflex that I find 
in language and belief? Why are we so fluent in thinking in self-vitiating 
concepts? Granting that language and belief are mistakes, are mistakes of 
this degree of complexity made for nothing? Is not the very ability to 
concoct an apparently significant, self-vitiating and self-deceiving structure a 
transcendent ability, one that points to something non-immediate? Do not 
these conceptual gymnastics, even if self-vitiating, make us superior to the 
mindless animals? 

Such questions tempt one to engage in a sort of philosophical 
anthropology, using in part the method of introspection. Beliefs could be 
explained as arising in an attempt to deal with experienced frustrations by 
denying them in thought. The origin of Christian Science and magic would 
thereby be explained. Further, we could postulate a primal anxiety-reaction 
to raw experience. This anxiety would be lessened by mythologies and 
explanatory beliefs. The frustration and the anxiety-reaction would be 
primal non-cognitive needs for beliefs. 

Going even farther, we could suppose that a being which could 
apprehend the whole universe through direct experience would have no need 
of beliefs. Beliefs would be a rickety method of coping with the limited 
range of our perception, a method by which our imperfect brains cope with 
the world. There would be an analogy with the physicist's use of phantom 
models to make experimental observations easier to comprehend. 

However, there are two overwhelming objections to this philosophical 
anthropology. First, it purports to study the human mind as a derivative 
phenomenon, to study it from a God-like perspective. The philosophical 
anthropology thus consists of beliefs which are subject to the same 
objections as any other beliefs. It is on a par with any other beliefs; it has no 
privileged position. Specifically, it is in competition not only with my 
philosophy but with other accounts of the mind-reality relation, such as 
behaviorism, Platonism, and Thomism. And my philosophy provides me with 
no basis to defend my philosophical! anthropology against their philosophical 
anthropologies. My philosophy doesn't even provide me with a basis to 
defend my philosophical anthropology against its own negation. 

In short, the paradoxes which my philosophy uncovers must remain 
unexplained and unresolved. 

The other objection to my philosophical anthropology is that its 
implications are unnecessarily conservative. An explanation of why people 
do something wrong can become an assertion that it is necessary to do wrong 
and finally a justification for doing wrong. But just because I tend, for 
example, to construe my perceptions as confirmations of propositions about 
phenomena beyond my experience does not mean that I must think in this 
way. To explain the modern cognitive orientation by philosophical 
anthropology tends to absolutize it and to conceal its dispensability. 


\item There are more legitimate tasks for the introspective "anthropology" 
of beliefs than trying to find primal non-cognitive needs for beliefs. 
Presupposing the analysis of beliefs as mental acts and self-deception which I 
have made elsewhere, we need to examine closely the boundary line between 
beliefs and non-credulous mental activity. 

Is my fear of jumping out of the window a belief? Strictly speaking, 
no. In psychological terms, a conditioned reflex does not require 
propositional thought. 

Is my identification of an object in different spatial orientations 
(relative to my field of vision) as "the same object" a belief? Apparently, 
but this is very ambiguous. 

Is my identification of tactile and visual "pencil-perceptions" as aspects 
of a single object (identity of the object as it is experienced through 
different senses) a belief? Yes. 

It is possible to subjectively classify bodily movements according to 
whether they are intentional, because drunken awkwardness, adolescent 
awkwardness, and movements under ESB are clearly unintentional. Then 
does intentional movement of my hand require a belief that I can move my 
hand? Definitely not, although in rare cases some belief will accompany or 
precede the movement of my hand. But believing itself will not get the hand 
moved! 

Is there any belief involved in identifying my leg, but not the leg of the 
table at which I am sitting, as part of my body? Maybe---another ambiguous 
case.

Are my emotions of longing and dread beliefs in future time? Is my 
emotion of regret belief in past time? Philosophical anthropology: these 
temporal feelings precede and give rise to temporal beliefs. (?) 

How can I introspectively analyze my dread as dread of future injury if 
my belief in the existence of the future is invalid to begin with? Easily--- the 
object of the fear is a belief or has a belief associated with it. 

\plainbreak{2}

\item At one point Alten claimed that his dialectical approach does not 
take any evidence as being more immediate, more primary, than any other 
evidence. Our "immediate experience" is mediated; it is a derived 
phenomenon which only subsists in an objective reality that is outside our 
subjective standpoint. 

\begin{enumerate}

\item But Alten does not seriously defend the claim that he does not 
distinguish between immediate and non-immediate. The claim that there is 
no distinction would be regarded as demented in every human culture. Every 
culture supposes that I may be tricked or cheated: there is a realm, the 
non-immediate or non-experienced, which provides an arena for surreptitious 
hostility to me. Every culture supposes that it is easier for me to tell what I 
am thinking than what you are thinking. Every culture supposes that I will 
hear things which I should not accept before I go and see for myself. Alten is 
simply not iconoclastic enough to reject these commonplaces. What he 
apparently does is, like the perceptual psychologist, to accept the distinction 
between immediate and non-immediate, and to accept the former as the only 
way of confirming a model, but to construct a mode! of the relation between 
the two in which the former is analyzed as a derivative phenomenon. 

\item Alten proposes to analyze his own awareness as a derivative 
phenomenon, to take a stance outside all human awareness. But this is the 
pretense of the God-like perspective. He postulates both his own limitedness 
and his ability to step outside it! This is an overt contradiction. Indeed, it is 
the archetype of the overt self-deception in beliefs which my philosophy 
exposes. "I can tell the Empire State Building exists now even though I 
cannot now perceive it." 
\end{enumerate}

\item In my technical philosophical writings, I call attention to certain 
self-vitiating "nodes" in the logic of common sense. These nodes include the 
concept of non-experience and the assertion that there is language. I often 
find that others dismiss these examples as jokes that can be isolated from 
cognition or the logic of common sense, rather than acknowledging that they 
are self-vitiating nodes in the logic of common sense. As a result, I have 
concluded that it is probably futile to debate the abstract validity of my 
analysis of these nodes. It does indeed appear as if I am debating over an 
abstract joke, and it is not apparent why I would attribute such great 
importance to a joke. 

\essaytitle{Philosophical Aspects of Walking Through Walls} represents my 
present approach. The advantage of this approach is that it makes 
unmistakable the reason why ! attribute so much importance to these 
philosophical studies. I am not merely debating the abstract validity of a few 
isolated linguistic jokes; I seek to overthrow the life-world. The only 
significance of my technical philosophical writings is to offer an explanation 
of why the life---world is subject to being undermined. 

When I speak of walking through walls, the mistake is often made of 
trying to understand this reference within the framework of present-day 
scientific common sense. Walking through walls is understood as it would be 
pictured in a comic-book episode. But such an understanding is quite beside 
the point. What I am advocating---to skip over the intermediate details and go 
directly to the end result---is a restructuring of the whole modern cognitive 
orientation such that one doesn't even engage in scientific hypothesizing or 
have "object perceptions," and thus wouldn't know whether one was 
walking through a wail or not. 

At first this suggestion may seem like another joke, a triviality. But my 
genius consists in recognizing that it is not, that there is a residue of 
non-vacuity and non-triviality in this proposal. There may be only a 
hair's-breadth of difference between the state ! propose and mental 
incompetance or death---but still, there is all of a hair's-breadth. I magnify 
this hair's-breadth many times, and use it as a lever to overturn civilization. 

\item I am often asked in philosophical discussion how it is that we are 
now talking if language is vitiated. Let me comment that merely pointing 
over and over to one of the two circumstances which create a paradox does 
not resolve the paradox. Indeed, a paradox arises when there are two 
circumstances in conflict. The "fact" that we are talking is one of the two 
circumstances which conjoin in the paradox of language; the other 
circumstance being the self-vitiating "nodes" I have mentioned. To repeat 
over and over that we are now talking does not resolve any paradoxes. 

Contrary to what the question of how it is that we are now talking 
suggests, we do not "see" language. (That is, we do not experience an 
objective relation between words and things.) The !anguage we "see" is a 
shell whose 'transcendental reference" is provided by self-deception. 

\item Does the theory of amcons show that the contradiction exposed in 
\essaytitle{The Flaws Underlying Beliefs} is admissible and thus loses its philosophical 
force? No. An amcon is between two things that you see, e.g. stationary 
motion. It is between two sensed qualities, the simultaneous experiencing of 
contradictory qualities. (But "He left an hour ago" begins to be a borderline 
case. Here the point is the ease with which we swallow an expression which 
violates logical rules. Also expansion of an arc: a case even more difficult to 
classify.) The contradiction in "The Flaws Underlying Beliefs" has to do first 
with the logic of common sense, with the logical rules of language. It has to 
do, secondly, with the circumstance that you don't see something, yet act as 
if you do. Amcons should not be used to justify self-deception in the latter 
sense, to rescue every cheap superstition. 


{
5/15/1962 


Comments from the audience 
(photo by Tony Conrad) 


"Creep" lecture, May 15, 1962 
}

\clearpage

{
5/15/1962 


Comments from the audience 
(photo by Tony Conrad) 


"Creep" lecture, May 15, 1962 
}

\clearpage


\chapter{Instructions for the Flyntian Modality}


1. STOP ALL "GROSS BELIEVING," SUCH AS BELIEF IN OTHER 
MINDS, CAUSALITY, AND THE PHANTOM ENTITIES OF SCIENCE 
(ATOMS, ELECTRONS, ETC.). 


2. STOP THINKING IN PROPOSITIONAL LANGUAGE. 


3. STOP ALL SCIENTIFIC HYPOTHESIZING. DO NOT CONSIDER 
YOUR "SIGHTINGS" OF THE EMPIRE STATE BUILDING AS 
CONFIRMATIONS THAT IT IS THERE WHEN YOU ARE NOT LOOKING 
AT !T-OR FOR THAT MATTER, AS CONFIRMATIONS THAT IT IS 
THERE WHEN YOU ARE LOOKING AT IT. 


4. STOP ORGANIZING VISUAL EXPERIENCES AND TACTILE 
EXPERIENCES INTO OBJECT-GESTALTS. STOP ORGANIZING 
SO-CALLED "DIFFERENT SPATIAL ORIENTATIONS OR DIFFERENT 
TOUCHED SURFACES OF OBJECTS" INTO OBJECT-GESTALTS. THAT 
IS, STOP HAVING PERCEPTIONS OF OBJECTS. 


5. STOP BELIEVING IN PAST AND FUTURE TIME. THAT 15, LIVE 
OUT OF TIME. STOP FEELING LONGING, DREAD, OR REGRET. 


6. STOP BELIEVING THAT YOU CAN MOVE YOUR BODY. 


7. STOP BELIEVING THAT THESE INSTRUCTIONS HAVE ANY 
OBJECTIVE MEANING. 


8. YOU ARE NOW FREE TO WALK THROUGH WALLS (IF YOU CAN 
FIND THEM). 


25 


6. Some Objections to My Philosophy 


A. The predominant attitude toward philosophical questions in 
euucated circles today derives from the later Wittgenstein. Consider the 
philosopher's question of whether other people have minds. The 
Wittgensteinian attitude is that in ordinary usage, statements which imply 
that other people have minds are not problematic. Everybody knows that 
other people have minds. To doubt that other people have minds, as a 
philosopher might do, is simply to misuse ordinary language. (See 
Philosophical Investigations, $420.) Statements which imply that other 
people have minds works perfectly well in the context for which they were 
intended. When philosophers find these statements problematic, it is because 
they subject the statements to criticism by logical! standards which are 
irrelevant and extraneous to ordinary usage. (§ § 402, 412, 119, 116.) 

For Wittgenstein, the existence of God, immortal souls, other minds, 
and the Empire State Building (when I am not looking at it) are all things 
which everybody knows; things which it is impossible to doubt "in a real 
case.' (§303, Iliv. For Wittgenstein's theism, see Norman Malcolm's 
memoir.) The proper use of language admits of no alternative to belief in 
God; atheism is just a mistake in the use of language. 


Chapter 6 : Discussion of Some Basic Beliefs 


In the preceding chapters I have been concerned, in discrediting any 
given belief, to show what the right philosophical position is. In this chapter 
I will turn to particular beliefs, supposed knowledge, to make it clear just 
what, specifically, have been discredited. Now if the reader will consider the 
entire "history of world thought", the fantastic proliferation of activities at 
least partly "systems of knowledge" which constitute it, Platonism, 
psychoanalysis, Tibetian mysticism, physics, Bantu witchcraft, 
phenomenology, mathematical logic, Konko Kyo, Marxism, alchemy, 
comparative linguistics, Orgonomy, Thomism, and so on indefinitely, each 
with its own kind of conclusions, method of justifying them, applications, 
associated valuations, and the like, he will quickly realize that I could not 


26 


ee eR eT A ee OE eT Ee a 


hope to analyze even a fraction of them to show just how "non-experiential 
language', and beliefs, are involved in them. And I should say that it is not 
always obvious whether the concepts of non-experiential language, and 
belief, are relevant to them. Zen is an obvious example (although as a matter 
of fact is unquestionably does involve betiefs, is not for example an 
anticipation of my position). Further, many quasi-systems-of- knowledge are 
difficult to discuss because the expositions of them which are what one has 
to work with, are badly written, in particular, fail to state the insights behind 
what is presented, the real reasons why it can be taken seriously, and are 
incomplete and confused. 

What I will do, then, to specifically illustrate my results, is to discuss a 
few particular beliefs which are found in almost all systems of 'knowledge'; 
have been given especial attention in modern Western philosophy and are 
thus especially relevant to the immediate audience for this book; and are so 
"basic" (accounting for their ubiquity} that they are either just assumed, as 
too trivially factual to be worthy the attention of a profound thinker, or if 
they are explicit are said to be so basic that persons cannot do without them. 
The discussion will make it specifically clear that it is not necessary to have 
these beliefs, that not having them is not "inconsistent" with one's 
experience; and is thus important for the reader who is astonished at the idea 
of rejecting any given belief, the idea of any given belief's being wrong and 
of not having it. 

Consider beliefs to the effect "that the world is ordered', beliefs 
formulated in 'natural laws", beliefs "about substance', and the like. 
Rejection of them may seem to lead to a problem. After all, one's "perceived 
world" is not "chaotic", is it? The reader should observe that in rejecting 
beliefs "that the world is ordered" I do not say that his "perceived world" is 
("subjectively") chaotic (that is, extremely unfamiliar, strange). The 
non-strange character of one's 'perceived world" is associated with beliefs 
"about substance" and beliefs formulated in natural laws, but it is not "the 
world being ordered"; and taking note of the non-strange character of one's 
"perceived world" is not part of what is 'essential' in these beliefs. 

Rejection of "spatio-temporal" beliefs may seem to lead to a problem. 
After all, cannot one watch oneself wave one's hand towards and away from 
oneself? Of course one can "watch oneself wave one's hand" (in a non-strict 
sense---and if the reader uses the expression in this sense it will not be a 
formulation of a belief for him). However, that one can "watch oneself wave 
one's hand" (in the non-strict sense) does not imply 'that there are spatially 
distant, and past and future events"; and although experiences such as a 
visual - "moving" - hand experience are associated with spatio-temporal 
beliefs, taking note of them is not part of what is essential in those beliefs. 


27 


Rejection of beliefs "about the objectivity of linguistic referring' may 
seem to lead to a problem. After all, when one says that a table is a "table", 
doesn't one do so unhesitatingly, with a feeling of satisfaction, a feeling that 
things are less mysterious, strange, when one has done so, and without the 
slightest intention of saying that it is a "non-table"? The reader should 
observe that I do not deny this. These experiences are associated with beliefs 
"about the objectivity of referring', but they are not "objective referring'; 
and taking note of them is not part of what is essential in those beliefs. 

Rejection of the belief "that other humans (better, things) than oneself 
have minds" my seem to lead to a problem. After all, "perceived other 
humans" talk and so forth, do they not? The reader should observe that in 
rejecting the belief "that others have minds" I do not deny that "perceived 
other humans" talk and so forth. Other humans' talking and so forth is 
associated with the belief 'that others have minds', but it is not "other 
humans having minds"; and taking note of others talking and so forth is not 
part of what is essentia! in believing "that others have minds", points I 
anticipated in the second chapter. 

Finally, many philosophers will violently object to rejection of 
temporal beliefs of a certain kind, namely beliefs of the form 'If x, then y 
will follow in the future', especially if y is something one wants, and x is 
something one can do. {After all, doesn't it happen that one throws the 
switch, and the light goes on?) They object so strongly because they fear 
"that one cannot live unless one has and uses such knowledge'. They say, 
for example, "that one had better know that one must drink water to live, 
and drink water, or one won't live". Now "one's throwing the switch and the 
light's coming on" (in a non-strict sense) is like the experiences associated 
with other temporal beliefs; that one can do it (in the non-strict sense) does 
not imply "that there are past or future events", and taking note of it is not 
part of what is essential in the belief "that if one throws the switch, then the 
light will come on'. As for what the philosophers say, fear, believe "about 
the necessity of such knowledge for survivai", it is just more beliefs of the 
same kind, so that rejection of it is similarly unproblematic. If this abrupt 
dismissal of the fears as wrong is terrifying to the reader, then it just shows 
how badly he is in need of being straightened out philosophically. 
Incidentally, all this should make it clear that it is futile to try to "save" 
beliefs (render them justifiable) by construing them as predictions. 

By now the reader has probably observed that the beliefs, and their 
formulations, which I have been discussing, the ones he is presumably most 
suspicious of rejecting, are all strongly (but not essentially) associated with 
non-mental experiences of his. The reader may no longer seriously have the 
beliefs, but have problems in connection with them, get involved in 


28 


ee ee ee eR 


defending them, and be suspicious of rejecting them, merely because he 
continues to use the formulations of the beliefs, but to refer to the 
experiences associated with them (as there's no other way in English to do 
so), and confusedly supposes that to reject the beliefs and formulations is to 
deny that he has the experiences. Now {I am not denying that he has the 
experiences. As I said in the last chapter, I am not trying to convince the 
reader that he doesn't have experiences he has, but to point out to him the 
self-deception experiences involved in his beliefs. The reader should be wary 
of thinking, however, on reading this, that maybe he doesn't have any beliefs 
after all, just uses the belief language he does to refer to experiences. It 
sometimes happens that people who have beliefs and as a result use belief 
language excuse themselves on the basis that they are just using the language 
to refer to experiences, an hypocrisy. If one uses belief formulations, it's 
usually because one has beliefs. 

The point that the language which one may use to describe experiences 
is formulations of beliefs, is true generally. As I said in the third chapter, all 
English sentences are, traditionally anyway, formulations of beliefs. As a 
result, those who want to talk about experiences {my use) and still use 
English are forced to use formulations of beliefs to refer to strongly 
associated experiences, and this seems to be happening more and more; often 
among quasi-empiricists who naively suppose that the formulations have 
always been used that way, except by a few "metaphysicians". I have had to 
so use belief language throughout this book, the most notable example being 
the introduction of my use of 'experience' in the third chapter. Thus, some 
of what I say may imply belief formulations for the reader when it doesn't 
for me, and be philosophically problematic for him; he must understand the 
book to some extent in spite of the language, as I suggested in the third 
chapter. I have tried to make this relatively easy by choosing, to refer to 
experiences, languag2 with which they are very strongly associated and 
which is only weakly associated with beliefs, and, the important thing, by 
announcing when the language is used for that purpose. 

It is time, though, that I admit, so as not to be guilty of the hypocricy I 
was exposing earlier, that most of the sentences in this book will be 
understood as formulations of beliefs, that, in other words, I have presented 
my philosophy to the reader by getting him to have a series of beliefs. This 
does not invalidate my position, because the beliefs are not part of it. They 
are for the heuristic purpose of getting the reader to appreciate my position, 
which is not having beliefs {and realizing, for any belief one happens to think 
of, that it is wrong (which doesn't involve believing)); and they may well not 
be held when they have accomplished that purpose. I hope f will eventually 
get around to writing a version of this book which presents my position by 


29 


suggesting to the reader a series of imaginings (and no more), rather than 
beliefs; developing a new language to do so. The reason I stick with English 
in this book is of course (!) that readers are too "unmotivated" (lazy!) to 
learn a language of an entirely new kind to read a book, having 
unconventional conclusions, in philosophy proper. 


Chapter 7 : Summary 


The most important step in understanding my work is to realize that I 
am trying neither to get one to adopt a system of beliefs, nor to just ignore 
beliefs or the matter of whether they are right. Once the reader does so, he 
will find that my position is quite simple. The reader has probably tended to 
construe the body of the book, the second through the sixth chapters, as a 
formulation of a system of beliefs; or as a proposal that he ignore beliefs or 
the matter of whether they are right. Even if he has, a careful reading of 
them will, I hope, have prepared him for a statement of my position which is 
supposed to make it clear that the position is simple and right. This 
statement is a summary, and thus cannot be understood except in 
connection with the second through the sixth chapters. First, I reiterate that 
my position is not a system of beliefs, supported by a long, plausible 
argument. This means, incidentally, that it is absurd to "remain 
unconvinced" of the rightness of my position, or to 'doubt, question" it, or 
to take a long time to decide whether it is right: one can "question" (not 
believe) disbelief, but not unbelief. (Not to mention that it is a wrong belief 
to be "skeptical" of my position in the sense of believing "that although the 
position may subjectively seem right, there is always the possibility that it is 
objectively wrong".) I am trying, not to get one to adopt new beliefs but to 
reject those one already has, not to make one more credulous but less 
credulous. If one "questions my position" then one is misconstruing it as a 
belief for which I try to give a long, plausible argument, and is trying to 
decide which is more plausible, my argument that all beliefs are false, say, or 
the arguments that beliefs are true. It may well! take one a long time to 
understand my position, but if one is taking a jong time to decide whether it 
is right then one is wasting one's time thinking about a position I show to be 
wrong. Secondly, my position is not a proposal that one ignore beliefs or the 
matter of whether they are right. Thus, it is absurd to conclude that my 
position is irrefutable but trivial, that one who has beliefs can also be right. 

Now for the statement of the position. Imagine yourself without 
beliefs. One certainly is without beliefs when one is not thinking, for 


30 


example (although not only then). This being without beliefs is my position. 
Now this position can't be wrong inasmuch as you aren't doing anything to 
be "true or false', to be self-deceiving. Now imagine that someone asks you 
to believe something, for example, to believe 'that there is a table behind 
you". Then if you are going to do what he asks, and believe (as opposed to 
continuing not to think; or only imagining---for example, "visualizing 
yourself with your back to a table'), you are going to have to have the 
attitude that you are in effect perceiving what you don't perceive, that is, 
deceive yourself. (What else could he be asking you do do? ) You are going 
to have to be wrong. That's all there is to it. 

As for my language here, it is primarily intended to be suggestive, 
intended, at best, to suggest imaginings to you which will enable you to 
realize what the right philosophical position is (as in the last paragraph). The 
important thing is not whether the sentences in this book correspond to true 
statements in your language (although I expect the key ones will, the 
expressions in them being construed as referring to the experiences 
associated with them); it is for you to realize, observe what you do when 
you don't have beliefs and when you do. You are not so much to study my 
language as to begin to ask what one who asks you to believe wants you to 
do, anyway. The language isn't sufficiently flawless to absolutely force the 
complete realization of what the right position is on you {it doesn't have to 
be flawless to unquestionably discredit "non-experiential language'); if you 
don't want to realize where the self-deception is in believing you can just 
ignore the book, and "justify" your doing so on the basis of what I have said 
about language such as I have used. The point is that the book is not 
therefore valueless. 

So much for what the right philosophical position is. From having 
beliefs to not having them is not a trivial step; it is a complete 
transformation of one's cognitive orientation. Yet astonishing as the latter 
position is when first encountered, does it not become, in retrospect, 
"obvious"? What other position could be the resolution of the fantastic 
proliferation of conflicting beliefs, and of the "profound" philosophical 
problems (for example, 'Could an omnipotent god do the literally 
impossible? ', 'Are statements about what I did in the past while alone 
capable of intersubjective verification? ') arising from them? And again, one 
begins to ask, when one is asked to believe something, what it is that one is 
wanted to do, anyway; and one's reaction to the request comes to be 'Why 
bother? Cognitively, what is the value of doing so? I'd just be deceiving 
myself'. Also, how much simpler my position is than that of the believer. 
And although in a way the believer's position is the more natural, since one 
"naturally" tends to deceive oneself if there's any advantage in doing so 


31 


(that is, being right tends not to be valued), in another way my position is, 
since it is simple, and since the non-believer isn't worried by the doubts 
which arise for one who tries to keep himself deceived. 

In arguing against Wittgenstein, I will concentrate on the real reason 
why I oppose him, rather than on less fundamental technical issues. We read 
that in the Middle Ages, people found it impossible not to believe that they 
would be struck by lightning if they uttered a blasphemy; just as 
Wittgenstein finds the existence of God impossible to doubt "in a real case." 
Yet even Wittgenstein does not defend the former belief; while the Soviet 
Union has shown that a government can function which has repudiated the 
latter belief. There is a tremendous discovery here: that beliefs which were as 
inescapable---as impossible to doubt in a real case---as any belief we may have 
today, were subsequently discarded. How was this possible? My essay "The 
Flaws Underlying Beliefs" shows how. Further, it shows that the belief that 
the Empire State Building exists when I am not looking at it, or the belief 
that I would be killed if I jumped out of a tenth story window, are no 
different in principle from beliefs which we have already discarded. It Is 
perfectly possible to project a metaphysical outlook on experience which is 
totally different from the beliefs Wittgenstein inherited, and it is also 
possible not to project a metaphysical outlook on experience at all. Let us be 
absolutely clear: the point is not that we do not know with one hundred per 
cent certainty that the Empire State Building exists; the point is that we 
need not believe in the Empire State Building at all. "The Flaws Underlying 
Beliefs" shows that factual propositions, and the propositions of the natural 
sciences, involve outright self-deception. 

These discoveries have consequences far more important than the 
technical issues involved. It is by no means trivial that I do not have to pray, 
or to fast, or to accept the moral dictates of the clergy, or to give money to 
the Church. Because the Church prohibited the dissection of human 
cadavers, it took an atheist to originate the modern subject of anatomy. In 
analogy with this example, the rest of my writings are devoted to exploring 
the consequences of rejecting beliefs that Wittgenstein says are impossible to 
doubt in a real case, as in my essay "Philosophical Aspects of Walking 
Through Walls." I oppose Wittgenstein because he descended to extremes of 
intellectual dishonesty in order to prevent us from discovering these 
consequences. 

A reply to the Wittgensteinian attitude which is technically adequate 
can be provided in short order, for when Wittgenstein's central philosophical 
maneuver is identified, its dishonesty becomes transparent. It is not 
necessary to enumerate the fallacies in the Wittgensteinian claim that logical 
connections and logical standards are extrinsic to the natural language, or in 


32 


the aphorism that "the meaning is the use" (as an explication of the natural 
language). In other words, there is no reason why I should bandy descriptive 
linguistics with Wittgenstein. Wittgenstein was wrong at a level more basic 
than the level on which his philosophical discussions were conducted. 

Wittgenstein held that philosophical or metaphysical controversies 
literally would not arise if it were not for bad philosophers. They would not 
arise because there is nothing problematic about sentences, expressing 
Wittgenstein's inherited beliefs, in ordinary usage. This rhetorical maneuver 
is the inverse of what it seems to be. Wittgenstein doesn't prove that the 
paradoxes uncovered by "bad" philosophers result from a misuse of ordinary 
language; he defines the philosophers' discussions as a misuse of ordinary 
language because they uncover paradoxes is ordinary language propositions. 
Wittgenstein waits to see whether a philosopher uncovers problems in 
ordinary language propositions; and if the philosopher does so, then 
Wittgenstein defines his discussion as improper usage. Wittgenstein waits to 
see whether evidence is against his side, and if it is, he defines it as 
inadmissible. 

Consider the philosopher's question of how I know whether the Empire 
State Building continues to exist when I am not looking at it. The 
Wittgensteinian position on this question would be that it is problematic 
because it is a misuse of ordinary language; and because there is no 
behavioral context which constitutes a use for the question. According to 
this position, we would not encounter such problems if we would use 
ordinary language properly. But what does this position amount to? The 
philosopher's question has not been proved improper; it has been defined as 
improper because it leads to problems. The reason why "the proper use of 
ordinary language never leads to paradoxes" is that Wittgenstein has defined 
proper use as use in which no paradoxes are visible. Wittgenstein has not 
resolved or eliminated any problems; he has just refused to notice them. 
Wittgenstein attempts to pass off, as a discovery about philosophy and 
language, a gratuitous definition to the effect that certain portions of the 
natural language which embarrass him are inadmissible, a gratuitous ban on 
certain portions of the natural language which embarrass him. His purpose is 
to make criticism of his inherited beliefs impossible, to give them a spurious 
inescapability. Wittgenstein's maneuver is the last word in modish 
intellectual dishonesty. 


B. In philosophy, arguments which start from an immediate which 
cannot be doubted and attempt to prove the existence of an objective reality 
are called transcendental arguments. Typically, such an argument says that if 


33 


there is experience, there must be subject and object in experience; if there 
are subject and object, subject and object must be objectively real; and thus 
there must be objectively real mind and matter. Clearly, the belief which 
leaps the gap from the immediate to the objectively real is smuggled into the 
middle of the argument by a play on the words "subject" and "object." 

When the sophistry is cleared away, it becomes apparent that the 
attempt to attain the trans-experiential or extra-experiential within 
experience faces a dilemma of overkill. If the attempt could succeed, it 
would have only collapsed objective reality to my subjectivity. If it could be 
"proved" that I know the distant past, other minds, God, angels, archangels, 
etc. from immediate experience, then ail these phenomena would be 
trivialized. If other minds were given in my experience, they would only be 
my mind. The interest of the notion of objective reality is precisely its 
otherness and unreachability. If it could be reached from the immediate, it 
would be trivial. We ask how I know that the Empire State Building exists 
when I am not looking at it. If the answer is that I know through immediate 
experience, then objective reality has been collapsed to my subjectivity. The 
dilemma for transcendental arguments is that they propose to overcome the 
gap between the appearance of a thing and the thing itself, yet they do not 
want to conclude that appearances exhaust reality. 

There are two special assumptions which are smuggled into supposedly 
assumptionless transcendental arguments. First, there is the belief that there 
is an objective relationship between descriptive words and the things they 
describe, an objective criterion of the use of descriptive words. Secondly, 
there is the belief that correlations between the senses have an objective 
basis. (It is claimed that this belief cannot be doubted, but the claim is 
controverted by intersensory illusions such as the touching of a pencil with 


crossed fingers.) 
Transcendental arguments are secular theology, because they are 


addressed to a reader who wants only philosophical analyses that have 
conventional conclusions. A transcendental argument will contain a step 
such as the following, for example. We can have "real knowledge" of 
particular things only if there is an objective relationship between descriptive 
words and the things they describe; thus there must be such a relationship. 
This argument is plausible only if the reader can be trusted to overlook the 
alternative that we don't have this "real knowledge." 

In the way of supplementary remarks, we may mention that 
transcendental arguments typically commit the ontological fallacy: inferring 
the existence of a thing from the idea or name of the thing. Finally, 
transcendental arguments share a confusion which originates in the 
empiricism they are directed against: the confusion between doing 


34 


fundamental philosophy and doing the psychology of perception. Many 
transcendental arguments are similar to current doctrines in scientific 
psychology. But they fail as philosophy, because scientific psychology takes 
as presuppositions, and cannot prove, the very beliefs which transcendental 
arguments are supposed to prove. 


35 


7. Philosophy Proper ("Version 3," 1961) 
Chapter 1: Introduction (Revised, 1973) 


This monograph defines philosophy as such---philosophy proper---to be 
an inquiry as to which beliefs are 'true,' or right. The right beliefs are 
tentatively defined to be the beliefs one does not deceive oneself by holding. 
Although beliefs will be regarded as mental acts, they will be identified by 
their propositional formulations. Provisionally, beliefs may be taken as 
corresponding to non-tautologous propositions. 

Philosophy proper is an ultimate activity in the sense that no belief or 
supposed knowledge is conceded to be above philosophical examination. It is 
also an unavoidable activity in the sense that the notion of a belief, and the 
notion of judging the truth of a belief, are intrinsic to common sense and the 
natural language. Philosophers may not have achieved convincing results in 
philosophy proper; but the question of which beliefs are right is 
continuously posed for us even if we do not respect the way in which 
philosophers have dealt with it. 

All of the obstacles to philosophy proper arise because beliefs are 
normally held in order to satisfy non-cognitive needs. It will be heipful to 
examine this situation at some length. However, nothing can be done here 
beyond examining the situation. It is already clear that the interest of this 
monograph in beliefs is cognitive. It would be inappropriate to try to gain 
approval for philosophy proper by appealing to the values of those who hold 
beliefs in order to satisfy non-cognitive needs. 

it is implicit in beliefs that they correspond to cognitive claims, that 
they are subject to being judged true or false, and that their value rests on 
their truth. Nevertheless, beliefs can and do satisfy non-cognitive needs, 
quite apart from whether they are true. In order for a belief to satisfy some 
non-cognitive need, it is not necessary for the belief to be true; it merely has 
to be held. Concern with the ultimate philosophical validity of beliefs is rare. 
Concern with beliefs is normally concern with their ability to satisfy 
non-cognitive needs. 

To be specific, the literature of credulity contains remarks such as "! 
could not stand to live if I did not believe so-and-so," or "Even if so-and-so is 
true I don't want to know it." These remarks manifest the needs with which 


36 


we are concerned. To take note of these remarks is already to uncover a level 
of self-deception. It is important to realize that this self-deception is explicit 
and self-admitted. To recognize it has nothing to do with imputing 
subconscious motives to behavior, as is done in psychoanalysis. Further, to 
recognize it is by no means to advance a theory of the ultimate origin of 
beliefs, a theory which would presuppose a judgment as to the philosophical 
validity of the beliefs. To theorize that the ultimate origin of beliefs lies in 
the denial! of frustrating experiences, or in primal anxieties which are 
alleviated by mythological inventions, would be inappropriate when we have 
not even begun our properly philosophical inquiry. The only self- deceptions 
being considered here are admitted self-deceptions. 

A partial classification of the circumstances in which beliefs are held for 
non-cognitive reasons follows. 

1. Beliefs may be directly tied to one's morale. "I couldn't stand to 
live if 1 didn't believe in God." "If President Nixon is guilty I don't want to 
know it." 

2. One may believe for reasons of conformity. The conversion of Jews 
to Catholicism in late medieval Spain was an extreme example. 

3. The American philosopher Santayana said that he believed in 
Catholicism for esthetic reasons. 

4, Moral doctrines are sometimes justified on the grounds of their 
efficacy in maintaining public order, rather than their philosophical validity. 

5. A more complicated and more interesting situation arises when one 
who claims to be engaged in a cognitive inquiry somehow circumscribes the 
inquiry so as to ensure in advance that it will yield certain preferred results. 
Such a circumscribed inquiry wil! be called 'theologizing," in recognition of 
the archetypal activity in this category. 

When we raise the question of whether the natural sciences are 
instances of theologizing, it becomes apparent that the issue of non-cognitive 
motives for beliefs is no light matter. According to writers on the scientific 
method such as A. d'Abro, the scientist is compelled to operate as if he 
believed in the "real existence of a real absolute objective universe---a 
common objective world, one existing independently of the observer who 
discovers it bit by bit." The scientist holds this belief, even though it is a 
commonplace of college philosophy courses that it is unprovable, because he 
must do so in order to get on to the sort of results he considers desirable. 
The scientist claims to be engaged in a cognitive inquiry; yet the inquiry 
begins with an act of faith which it is impermissible to scrutinize. It follows 
that science is an instance of theologizing. If scientists cannot welcome a 
demonstration that their "metaphysical" presuppositions are invalid, then 
their interest in science cannot be cognitive. 


37 


The scientist's non-cognitive motive for believing differs from the 
non-cognitive motives described earlier in one notable respect. Each of the 
non-cognitive needs described earlier required a given belief, and could not 
be satisfied by that belief's negation. But inside a science's circumscribed 
area of inquiry, the scientist can welcome the establishment of either of two 
contradictory propositions; in other words, his non-cognitive need can be 
satisfied by either proposition. It is in this sense that he can impartially test 
or decide between two propositions, or make new discoveries. On the other 
hand, with regard to the metaphysical presuppositions of science, only a 
single alternative is welcome. 

6. Academicians will readily acknowledge that they are not interested 
in scholarly work by unknown persons with no academic credentials. To 
academic mathematicians and biologists, whether Galois and Mendel had 
made vatid discoveries was irrelevant. Thus, academicians as academicians 
circumscribe their purported interest in the cognitive in two ways---once as 
scientists; and once for reasons of personal gain and prestige. 

7. The strangest instance of a non-cognitive need for a belief is 
provided by the person who holds a fearful! belief which is widely considered 
to be superstitious, such as belief in Hell. As always, the test of whether the 
motive for the belief is cognitive is the question of whether the person would 
welcome a demonstration that the belief is invalid. There is reason to suspect 
that persons who cling to fearful beliefs would not welcome such a 
demonstration, perverse as their attitude may seem. After all, they take no 
comfort in the widespread rejection of the belief as superstitious. Thus, it 
seems that a masochistic need for fearful beliefs must be recognized. 

This examination of non-cognitive motives for beliefs is, to repeat, 
limited to circumstances in which there is explicit self- deception, or 
self-deception that can be demonstrated directly from internal evidence. The 
examination cannot be carried further unless we become able to judge 
whether the beliefs referred to are, after all, valid. Thus, we will now turn to 
our properly philosophical inquiry, which will occupy the remainder of this 
monograph. 


(Note: Chapters 2-7 were written in 1961, at a time when I used 
unconventional syntax and punctuation. They are printed here without 
change.) 


38 


Part I : The Linguistic Solution of Properly Philosophical Problems 
Chapter 2 : Preliminary Concepts 


In this part of the book I will be concerned to solve the problem of 
philosophy proper, the problem of which beliefs are right, by discussing 
language, certain linguistic expressions. To motivate what follows I might 
tentatively say that I will consider beliefs as represented by statements, 
formulations of them (for example, 'Other persons have minds' as 
representing the belief that other persons have minds), so that the problem 
will be which statements are true. Actually, to solve this problem we will be 
driven far beyond answers to the effect that given statements are true (or 
false). 

To make this book as engaging as possible, I would like to start right 
into the solution of the problem, to begin with the material in the next 
chapter. However, it effects, I think, a considerable clarification and 
simplification of the presentation of the solution if I first introduce certain 
concepts in an extended discussion. Then, when they enter into the solution 
they won't have to be just suggested in a condensed explanation which has 
to be repeated over and over. Thus, this chapter will be a properly 
philosophically neutral introduction of the concepts, an introduction which 
doesn't in itself say anything about the rightness of given beliefs (or the 
truth of given statements). The chapter is as a result not so interesting as the 
others, but I hope the reader will bear with me through it. 

The first concept is a new one, that of 'explication'. Explication of a 
familiar linguistic expression is what might traditionally be said to be finding 
a definition of the expression; it amounts partly to determining what it is 
wanted that the expression 'mean'. To explain: I will be discussing 
philosophically important expressions, familiar to the reader, such that their 
"meaning" needs clarifying, such that it is not clear to him how he wants to 
use them. I will be concerned with the suggestion of expressions, of which 
the "meanings", uses, are clear, which will be acceptable to the reader as 
replacements for the expressions of which the uses are obscure; that is, 
which have the uses that, it will turn out, the expressions of which the uses 
are obscure are supposed to have. Since the expressions which are to be 
replacements can be equivalent as expressions (sounds, bodies of marks) to 
the expressions they are to replace, it can also be said that ! will be 
concerned with the suggestion of clear uses, of the expressions of which the 
uses are obscure, which are, it will turn out, the uses the reader wants the 
expressions to have. To be more specific about the conditions of 
acceptability of such replacements, if the familiar expressions {expressions of 


39 


which the uses were obscure) were supposed to be names, have referents 
(and non-referents), then the new: expressions must clearly have referents. 
Further, the new expressions must deserve (by having appropriate referents 
in the case of names) the principal connotations of the familiar expressions, 
especially the distinctive, honorific connotations of the familiar expressions. 
(1 will not say here just how I use 'connotation'. What the connotations of 
an expression are will be suggested by giving sentences about, in the case of a 
supposed name for example, what the referents of the expression are 
supposed to be like.) 'Finding', or constructing, an expression (with its use) 
supposed to be acceptable to oneself as.a replacement, of the kind described, 
for an expression familiar to oneself, will be said to be "explicating" the 
expression familiar to oneself. The expression to be replaced wil! be said to 
be the "explicandum", and the suggested replacement, the 'explication'. 
Incidentally, if clarification shows that the desired use of the explicandum is 
inconsistent, then it can't have an explication at all acceptable, or what is the 
same thing, any explication will be as good as any other. 

I should mention that my use of "explication" is different from that of 
Rudolph Carnap, from whom I have taken the word rather than use the very 
problematic 'definition'. For him, explication is a scientist's, or philosopher 
of science's, devising a new precise concept, useful in natural science, 
suggested by a vague, unclear common concept (for example, that of 
"work"); whereas for me it is in effect constructing (if possible) that precise, 
clear concept which is the nearest equivalent to an unclear common concept. 

Here is an example in the acceptability of explications. Suppose that an 
expression is suggested, as an explication for 'thing having a mind' (if 
supposed to be a name, have referents), which has as referents precisely the 
things which have certain facial expressions, or talk, or have certain other 
"overt" behavior, or even certain brain electricity. Then I expect that this 
expression will not be acceptable to the reader as an explication for 'thing 
having a mind', since 'thing having a mind' presumably has the connotations 
for the reader "that having a mind is not the same as, is very different from, 
higher than, having certain facial expressions, talking, certain other overt 
behaving, or having certain brain electricity---the mind is observable only by 
the thing having it", and the explication doesn't deserve these connotations: 
the connotations of the explicandum are exclusive of the referents of the 
proposed explication. It doesn't make any difference if there's a causual 
connection between having a mind and the other things, because the 
expression 'thing having a mind' itself, and not the supposed effects of 
having a mind, is what is under discussion. 

As the reader can tell from the example, I will, in evaluating 
expressions, have to speak of what I assume the connotations of words are 


40 


for the reader. If any of my assumptions are incorrect, the book will be 
slightly less relevant to the reader's philosophical problems than it would be 
otherwise. Even so, the reader should get from this part the method of 
finding good explications, and its use in solving properly philosophical 
problems. 

Especially important in deciding whether an explication for a supposed 
name is good is the check of the referents of the explication against the 
connotations of the explicandum. Traditional philosophers, in the rare cases 
when they have suggested explications for expressions in dealing with 
philosophical problems, have suggested absurdly bad ones, which can quickly 
be shown up by such a check. Examples which are typically horrible are the 
explications for 'thing having a mind' mentioned above. 

The second concept I will discuss is that of true statement. As I will be 
discussing the "truth" of formulations of beliefs, statements, in the next two 
chapters, and as the concept of true statement is quite obscure (making it a 
good example of one needing explication), it will be helpful for me to clarify 
the concept beforehand, to give a partial explication for 'true statement'. 
(Partial because the explication, although much clearer than the 
explicandum, will itself have an unclear word in it.) 

Well, what is a "statement"? How do what are usually said to be 
"statements" state? Take a book and look through it, a book in a language 
you don't read, so you won't assume that it's obvious what it means. What 
does the book, the object, do? How does it work? Note that talking just 
about the marks in the book, or what seem (!) to be the rules of their 
arrangement, or the like, won't answer these questions. In fact, I expect that 
when the reader really thinks about them, the questions won't seem easy 
ones to answer. Now to begin answering them, one of the most important 
connotations of 'true statement', and, more generally, of 'statement', as 
traditionally and commonly used, is that a "statement" is an "assertion 
which has truth value" (is true or false) (or "has content', as it is sometimes 
said, rather misleadingly). That is, the "verbai" part of a statement is 
supposed to be related in a certain way to something "non-verbal", or at 
least not in the language the verbal part of the statement is in. Further, a 
statement is supposed to be "true" or not because of something having to do 
with the non-verbal thing to which the verbal part of the statement is 
related. {The exceptions are the "statements" of formalist logic and 
mathematics, which are not supposed to be assertions; they are thus 
irrelevant to statements of the kind ordinary persons and philosophers are 
interested in.) Thus, if 'true statement' is to be explicated, 'assertion having 
truth value' and 'is true' (and 'has content' in a misleading use) have to be 
explicated, as they are obscure, and as it must be clear that the explication 


41 


for 'true statement' deserves the connotations which were suggested with 
'assertion having truth value' and 'is true'. One important conclusion from 
these observations is that although "sentences" (the bodies of sound or 
bodes of marks such as 'The man talks') are often said to be "statements", 
would not be sufficient (to say the least) to explicate 'statement' by simply 
identifying it with 'sentence' (in my sense); something must be said about 
such matters as that of being an assertion having truth value. For the same 
reason, it is not sufficient (to say the least) to simply identify 'statement' 
with 'sentence', the latter being explicated in terms of the ('formal') rules 
for the formation of (grammatical) sentences, as these rules have no 
reference to such matters as that of being an assertion having truth value. 

In explicating 'true statement' I wil! use the most elegant approach, one 
relevant to the interest in such matters as that of being an assertion having 
truth value. This is to begin by describing a simple, if not the simplest, way 
to make an assertion. As an example, I will describe the simplest way to 
make the assertion that a thing is a table. The way is to "apply" 'table' to 
the thing. It is supposed that 'table' has been "interpreted", that is, that it is 
"determinate" to which, of ail things, applications of 'table' are (to be said 
to be) "true". (It is good to realize that it is also supposed that it is 
'determinate' which, of all things (events), are "occurrences of the word 
'table", are expressions "equivalent to" 'table'.) The word 'determinate' is 
the intentionally ambiguous one in this explication; I don't want to commit 
myself yet on how an expression becomes interpreted. As for 'apply', one 
can "apply" the word to the thing by pointing out "first" the word and 
"then" the thing. 'point out' is restricted to refer to "ostension", pointing 
out things in one's presence, things one is perceiving, and not to "directing 
attention to things not in one's presence" as well. The assertion is 'true', of 
course, if and only if the thing to which 'table' is applied is one of the things 
to which it is determinate that the application of 'table' is (to be said to be) 
"true", otherwise "false". It should be clear that such a pointing out of a 
"first" thing and a "second", the first being an interpreted expression, is an 
assertion of a simple kind, does have truth value and so forth. Let me further 
suggest 'interpreted expression' as an explication for 'name'; with respect to 
this explication, the things to which equivalent names ("occurances of a 
name") may be truthfully applied are the referents of the equivalent names, 
other things being non-referents. (Incidentally, I could have started with the 
concept of a name and its referents, and then said how to make a simple 
assertion using a name.) Then what I have intentionally left ambiguous is 
how a name has referents; I have not said, for example, whether the relation 
between name and referents is an 'objective, metaphysical entity", which 
would be getting into philosophy proper. 


42 


The point of describing this simple way of making an assertion is that 
what one wants to say are "statements", namely sentences used in the 
context of certain conventions, can be regarded as assertions of the "simple" 
kind; thus an explication for 'true statement' can be found. To do so, first 
let us say that the "complex name" gotten by replacing a sentence's "main 
verb" with the corresponding participle is the "associated name" of the 
sentence. For example, the associated name of 'Boston is in Massachusetts' is 
'Boston being in Massachusetts'. In the case of a sentence with coordinate 
clauses there may be a choice with respect to what is to be taken as the main 
verb, but this presents no significant difficulty. Example: sentence: 'The 
table in the room will have been black only if it had been pushed by one 
man while the other man talked'; main verb: 'will have been' or 'had been 
pushed'. Also, English may not have a participle to correspond to every verb, 
but this is in theory no difficulty; the lacking participle could obviously be 
invented. Now what we would like to say one does, in using a sentence to 
make <a statement, is to so to speak "assert" its associated name; this 
"asserted name' being "true" if and only if it has a referent. However, one 
doesn't assert names; names just have referents—-it is statements that one 


makes, "asserts", and that are "true" or "false". How, then, do we explicate 
this "asserting" of a name? By construing it as that assertion, of the simple 
kind, which is the application of 'having a referent' to the name. tn other 
words, from our theoretical point of view, to use a sentence to make a 
statement, one begins with a name (the sentence's associated name), and 
puts it into the sentence form, an act equivalent by convention to applying 
'having a referent' to it. For example, the sentence 'Boston is in 
Massachusetts' should be regarded as the simple assertion which is the 
application of 'having a referent' to 'Boston being in Massachusetts'. 

Now this approach may seem "unnatural" or incomplete to the reader 
for several reasons. First there is the syntactical oddity: the sentence is 
replaced by a statement "about" it (or to be precise its associated name). 
Well, all 1 can say is that this oddity is the inevitable result of trying to 
describe explicitly all that happens when one uses a sentence to make a 
statement; I can assure the reader that the alternate approaches are even 
more unnatural. Secondly, it may seem natural enough to speak of 
interpreting "simple names" (Fries' Class 1 words), but not so natural to 
speak of interpreting complex names (what could their referents be?). Of 
course, this is because complex names are to be regarded as formed from 
simpler names by specified methods; that is, their interpretations (and thus 
referents) are in specified relations to those of the simple names from which 
they are formed. The relations are indicated by the words, in the complex 
names, which are not names, and by the order of the words in the complex 


43 


names. An example worth a comment is associated names containing such 
words as 'the'; in making statements, these names have to be in the context 
of additional conventions, understandings, to have significance. It will be 
clear that what these relations (and referents) are, the explication of these 
relations, is not important for my purposes. Thirdly, I have not said anything 
about what the "meaning" (intension), as opposed to the referents {and 
non-referents), of a name is. {I might say that a thing can't have an intension 
unless it has referents or non-referents.) This matter is also not important for 
my purposes (and gets into philosophy proper). Finally, my approach tells 
the reader no more than he already knew about whether a given statement is 
true. Quite so, and I said that the discussion would be properly 
philosophically neutral. In fact, it is so precisely because of the ambiguous 
word 'determinate', because I haven't said anything about how names get 
referents. Even so, we have come a long way from blank wonder about how 
one (sounds, marks) could ever state anything, a long way towards 
explicating how asserting works. (And to the philosopher of language with 
formalist prejudices, the discussion has been a needed reminder that if 
language is to be assertional, say something, then names and referring in 
some form must have the central role in it.) 

"Statements", then, can be regarded as assertions of the 'simple' kind 
which are made in the special, conventional way, involving sentences, I have 
described. I could thus explicate 'true statement' as referring to those true 
"simple" assertions made in the special way, and it should be clear that this 
would be a good explication. However, as the connotations of 'true 
statement' having to do with the method of apptying the first member to the 
second are, I expect, of secondary importance compared to those having to 
do with such matters as being an assertion having truth value, it ts more 
elegant to explicate 'true statement' as referring to all true assertions of the 
"simple" kind. For the purposes of this book it is not important which of 
the two explications the reader prefers. 

So much for the preliminaries. 


Chapter 3 : "Experience" 


1 will introduce in this chapter some basic terminology, as the main step 
in taking the reader from ordinary English and traditional philosophical 
language to a language with which my philosophy can be exposited. This 
terminology is important because one of the main difficulties in expositing 
my philosophy (or any new philosophy) is that current language is based on 


44 


precisely some of the assumptions, beliefs, I intend to question. It will, I 
think, be immediately clear to the reader at all familiar with modern 
philosophy that the problems of terminology I am going to discuss are 
relevant to the problem of which beliefs are right. 

First, consider the term 'non-experience'. Although the concept of a 
non-experience is intrinsically far more "difficult" than the concept of 
"experience" which I will be discussing presently, it is, I suppose, 
presupposed in all "natural languages" and throughout philosophy, is so 
taken for granted that it is rarely discussed in itself. Thus, the reader should 
have no difficulty understanding it. Examples of non-experiences are 
perceivable objects---for example, a table (as opposed to one's perceptions of 
it), existing external to oneself, persisting when one is not perceiving it; the 
future (future events); the past; space {or better, the distantness of objects 
from oneself); minds other than one's own; causal relationships as ordinarily 
understood; referental relationships (the relationships between names and 
their referents as ordinarily understood; what I avoided discussing in the 
second chapter); unperceivable "things" (microscopic objects (of course, 
viewing them through microscopes does not count as perceiving them), 
essences, Being); in short, most of the things one is normally concerned with, 
normally thinks about, as well as the objects of uncommon knowledge. (To 
simplify the explanation of the concept, make it easier on the reader, I am 
speaking as if I believed that there are non-experiences, that is, introducing 
the concept in the context of the beliefs usually associated with it.) 
Non-experiences are precisely what one has beliefs about. One believes that 
there are microscopic living organisms, or that there are none (or that one 
can not know whether there are any---this is not a non-belief but a complex 
belief about the relation of the realm where non-experiences could be to the 
mind). Incidentally, that other minds, for example, are non-experiences is 
presumably a connotation of 'other minds' for the reader, as explained in the 
second chapter. 

In the history of philosophy, the concept of non-experience comes first. 
Then philosophers begin to develop theories of how one knows about 
non-experiences (epistemological theories). The concept of a perception, or 
experience of something, is introduced into philosophy. The theory is that 
one knows about non-experiences by perceiving, having experiences of, some 
of them. For example, one knows that there is a table before one's eyes 
(assuming that there is) by having a visual perception or experience of it, by 
having a "visual-table-experience'. The theory goes on to say that these 
perceptions are in the mind. Then, if one has a visual-table-experience in 
one's mind when there is no table, one is hallucinated. And so forth. Now 
there are two sources of confusion in ail this for the naive reader. First, 


45 


saying that perceptions of objects are in one's mind is not saying that they 
are, for example, visualizations, imaginings, such as one's visualization of a 
table with one's eyes closed. Perceptions of objects do not seem "mental". 
The theory that they are in the mind is a belief. This point leads directly to 
the second source of confusion. Does the English word 'table', as ordinarily 
used to refer to a table when one is looking at it, refer to the table, an entity 
external to one's perceptions which persists when not perceived, or to one's 
perception of it, to the visual-table-experience? If distinguishing between 
the two, and the notion that the table-experience is in his mind, seem silly to 
the reader, then he probably uses 'table', 'perceived table', and 
'table-experience' as equivalent some of the time. The distinction, however, 
is not just silly; anyone who believes that there are tables when he is not 
perceiving them must accept it to be consistent. At any rate there is this 
confusion, that it is not always clear whether English object-names are being 
used to refer to perceived non-experiences or to experiences, the 
perceptions. 

Now let us ignore for a moment the connotations that experiences are 
experiences, perceptions, of non-experiences, and are in the mind. The term 
'experience' is important here because with it philosophers finally made a 
start at inventing a term for the things one knows directly, unquestionabiy 
knows, or, better, which one just has, or are just there (whether they are 
experiences, perceptions, of non-experiences or not). A_ traditional 
philosopher would say that if one is having a table-experience, one may not 
know whether it's a true perception of a table, whether there's an objective 
table there; or whether it's an hallucination; but one unquestionably knows, 
has, the table-experience. And of course, with respect to one's experiences 
not supposed to be perceptions of anything, such as visualizations, one 
unquestionably knows, has them too. A better way of putting it is that there 
is no question as to whether one has one's experiences or what they are like. 
One doesn't believe (that one has) one's experiences; to try to do so would 
be rather like trying to polish air. In fact, "thinking" that one doesn't have 
one's experiences, if this is possible, is a belief, a wrong one (as will be 
shown, although it should already be obvious if the reader has the slightest 
idea of what I am talking about), and in fact a perfectly insane one. Now the 
reader must not think that because I say experiences are unquestionably 
known {I am talking about tautologies, or about beliefs which some 
philosophers say can be known by intuition even though unprovable, or say 
cannot really be doubted without losing one's sanity (for example, some 
philosophers say this about the belief that other persons have minds). In 
speaking of experiences I am not trying to trick the reader into accepting a 
lot of beliefs I am not prepared to justify, as many philosophers do by 


46 


appealing to intuition or sanity or what not, a reprehensible hyprocrisy 
which shows that they are not the least interested in philosophy proper. One 
does not have other-persons'-having-minds-experiences {nor are the objective 
tables one supposedly perceives table-experiences); one believes that other 
persons have minds (or that there is an objective table corresponding to one's 
table-experience), and this belief could very well be wrong (in fact, it is, as 
will be shown). 

I have explained the current use of the term 'experience'. Now I want 
to propose a new use for the term, which, except where otherwise noted, 
will be that of the rest of this book. (Thus whereas in discussing 
'non-experience' I was merely explaining and accepting the current use of 
the term, in the case of 'experience' I am going to suggest a new use for the 
term.) As I explained, the concept of non-experience preceded that of 
experience, and the latter was developed to explain how one knows the 
former. What I am interested in, however, is not 'experience' as it implies. 
'perceptions, of non-experiences, and in the mind', but as it refers to that 
which one unquestionably knows, is immediate, is just there, is not 
something one believes exists. I am going to use 'experience' to refer, as it 
already does, to that immediate "world", but without the implication that 
experience is perception of non-experience, and in the mind: the same 
referents but without the old connotations. In other words, in my use 
'experiei.ce' is completely neutral with respect to relationships to 
non-experiences, is not an antonym for 'non-experience' as conventionally 
used, does not presuppose a metaphysic. The reader is being asked to take a 
leap of understanding here, because there is all the difference in philosophy 
between 'experience' as implying, connoting, relatedness to non-experiences 
or in particular the realm where they could be, and 'experience' without 
these connotations. 

Viewing this discussion of terminology in retrospect, it should be 
obvious that although my term 'experience' was introduced last, it is 
intrinsically, logically, the simplest, most immediate, most inevitable of the 
terms, and should be the easiest to understand. In contrast, the notions I 
discussed in reaching it may seem a little arbitrary. As a matter of fact, I 
have used the perspective of the Western philsophical tradition to explain my 
term, but this doesn't mean that it is relevant only to that tradition or, 
especially, the theory of knowing about non-experiences. Even if the reader's 
conceptual background does not involve the concept of non-experience, and 
especially the modern Western theory of knowing about non-experiences, he 
ought to be able to understand, and realize the "orimacy" of, my term 
'experience'. The term should be supra-cultural. 

I have gone to some length to explain my use of the term 'experience'. 


47 


As I have said, it is "intrinsically" the simplest term, but I can not define it 
by just equating it to some English expression because all English, including 
the traditional term 'experience', the antonym of 'non-experience', is based 
on metaphysical assumptions, does have implications about non-experience, 
in short, is formulations of beliefs. These implications are different for 
different philosophers according as their metaphysics (or, as is sometimes 
(incorrectly) said, "ontologies") differ. Even such a sentence as 'The table is 
black' implies the formulation 'Material objects are real' (to the materialist), 
or 'So-called objects are ideas in the mind' (to the idealist), or 'Substances 
and attributes are real', and so forth, traditionally. As a result, in order to 
explain the new term I have had to use English in a very special way, 
ultimately turning it against itself, so as to enable the reader to guess how I 
use the term. That is, although there is nothing problematic about my use of 
'experience', about its referents, there is about my English, for example 
when I say that the connotation of relatedness to non-experience is to be 
dropped from 'experience'. There can be this new term, the philosopher is 
not irrevocably tied to English or other natural language and its implied 
philosophy, as some philosophers claim; because a term is able to be a name, 
to be used to make assertions, not by being a part of conventional English or 
other natural language, but by having referents. 

As I suggested at the beginning of this chapter, I need to introduce my 
'experience' because without it I cannot question all beliefs, everything 
about non-experiences, since in English there is always the implication that 
there could be non-experiences. The term is a radical innovation; one of the 
most important in this book. The fact that although it is the 'simplest' and 
least questionable term, it is a radical innovation and is difficult to explain 
using English, shows how philosophically inadequate English and the 
philosophies it implies are. Now if the reader has not understood my 
'experience' he is likely to precisely mis-understand the rest of the book as 
an attempt to show that there are no non-experiences. (It's good that this 
isn't what I'm trying to show, because it is self-contradictory: for there to be 
no non-experiences there would have to be a realm empty of them, and this 
realm would have to be a non-experience.) If he is lucky he will just find the 
book incomprehensible, or possibly even come to understand the term from 
the rest of what I say, using it. But if he does understand the term, then he is 
past the greatest difficulty in understanding the book; in fact, he may 
already realize what I'm going to say. 


48 


Chapter 4 : The Linguistic Solution 


Now that I have explained the key terminology for this part of the 
book, I can give the solution to properly philosophical problems, the 
problems of which beliefs are right, in the form of conclusions about the 
language in which the beliefs are formulated. My concern here is to present 
the solution as soon as possible, so as to make it clear to the reader that my 
work contains important results, is an important contribution to philosophy, 
and not just admirable sentiments or the formulation of an attitude or a 
philosophically neutral analysis of concepts or the like. For this reason I will 
not be too concerned to make the solution seem natural, or intuitive, or to 
explore all its implications; that will come later. 

However, in the hope that it will make the main "argument" of this 
chapter easier to understand, I will precede it with a short, non-rigorous 
version of it, which should give the "intuitive insight' behind the main 
argument. Consider the question of whether one can know if a given belief is 
true. Now a given belief is cognitively arbitrary in that it cannot be justified 
from the standpoint of having no beliefs, cannot be justified without 
appealing to other beliefs. Thus the answer must be skepticism: one cannot 
know if a given belief is true. However, this skepticism is a belief---a 
contradiction. The ultimate conclusion is that to escape inconsistency, to be 
right, one must, at the linguistic level, reject all talk of beliefs, of knowing if 
they are true, reject all formulations of beliefs. The "necessity", but 
inconsistency, of skepticism "shows" my conclusion in an intuitively 
understandable way. : 

To get on to the definitive version of my "argument". I will say that 
one name "depends" on another if and only if it has the logical relation to 
that other that 'black table' has to 'table': a referent of the former is 
necessarily a referent of the latter (one of the relations between names 
mentioned in the second chapter). Now the associated name of any 
statement, or formulation, of a _ belief of necessity depends on 
'non-experience', since non-experiences are what beliefs are about. For 
example, 'Other persons having minds', the associated name of the 
formulation 'Other persons have minds', certainly depends on 
'non-experience'. Thus, anything true of 'non-experience' will be true of the 
associated name of any formulation of a belief. 

In the last chapter I introduced, explained the concepts of 
non-experience and experience (in the traditional sense, as the antonym of 
'non-experience'), showed the connotations of the expressions 
'non-experience' and 'experience' (traditional). What ! did not go on to 


49 


show, left for this chapter, is that if one continues to analyze these concepts, 
one comes on crucial implications which result in contradictions. What 
follows is perhaps the most concentrated passage in this book, so that the 
reader must be willing to read it slowly and thoughtfully. Consider one's 
experience (used in my, "neutral", sense unless I say otherwise). Could there 
be something in one's experience, a part of one's experience, which was 
awareness of whether it's experience (traditional), of whether it's related to 
non-experience, of whether there is non-experience, awareness of 
non-experience? No, as should be obvious from the connotations shown in 
the last chapter. (Compare this with the point that one cannot (cognitively) 
justify a belief from the standpoint of having no beliefs, cannot justify it 
without appealing to other beliefs). If there could be, if such awareness were 
just an experience, the distinctness of experience from experience 
(traditional) and so forth would disappear. The concepts of experience 
(traditional) and so forth would be superfluous, in fact, one couldn't have 
them: experience (traditional) and so forth would just be absorbed into 
experience. One concludes that there cannot be anything in one's experience 
which is awareness of whether it's experience (traditional), of whether there 
is non-experience. But then this awareness, which is in part about experience 
(traditional) and non-experience and thus involves awareness of them, is in 
one's experience---a contradiction. In fact, the same holds for the awareness 
which is "understanding the concepts" of non-experience and the rest as 
they are supposed to be understood. And for 'understanding' 
'non-experience' {and the rest) as it is supposed to be, being aware of its 
referents (and non-referents); since to name non-experience, it must be an 
experience (traditional). And even for being aware of the referents (and 
non-referents) of "non-experience", which to name an_ experience 
{traditional) must be one. One mustn't assume that one understands 
'non-experience' --- and "non-experience" --- and "non-experience"; but here 
one is, using "non-experience" and "non-experience" to say so (which 
certainly implies that one assumes one understands them). It is impossible 
for there to be non-experiences. When one begins to examine closely the 
concept of non-experience, it collapses. (A final point for the expert. This 
tangle of contradictions is intrinsic in the concept of non-experience; it does 
not result because I have introduced a violation of the law that names cannot 
name themselves. This should be absolutely clear from the two sentences 
about names, which show contradictions --- that one must not assume that 
one understands certain expressions, but that one uses the expressions to say 
so (does assume it) --- with explicit stratification.) : 
My exposition has broken down in a tangle of contradictions. Now 
what is important is that it has done so precisely because ! have talked about 


50 


experience (traditional), non-experience, and the rest, because I have spoken 
as if there could be non-experiences, because I have used 'experience' 
(traditional), 'non-experience', and the rest. Thus, even though what I have 
said is a tangle of contradictions, it is not by any means valueless. Since it is 
a tangle of contradictions precisely because it involves 'experience' 
(traditional), 'non-experience', and the rest, it shows that one who "accepts" 
the expressions, supposes that they are valid language, has inconsistent 
desires with respect to how they are to be used. The expressions can have no 
explications at all acceptable to him. He cannot consistently use the 
expressions (the way they're supposed to be). The expressions, and, 
remembering the paragraph before last, any formulation of a belief, are 
completely discredited. (What is not discredited is language referring to 
experiences (my use). If it happens that an expression I have said is a 
formulation of a belief does have a good explication for the reader, then it is 
not a formulation of a belief for him but refers to experiences.) Now there is 
an important point about method which should be brought out. If all 
"non-experiential language', 'belief language", is inconsistent, how can I 
show this and yet avoid falling into contradiction when I say it? The answer 
is that 1 don't have to avoid falling into contradiction; that I fall into 
contradiction precisely because I use formulations of beliefs shows what I 
want to show. This, then, is the linguistic solution; as 1 said we would, we 
have been driven far beyond any such conclusion as 'all formulations of 
beliefs are false'. 

Now what do these conclusions about formulations of beliefs, about 
belief language, say about beliefs themselves, about whether a given belief is 
right? Well, to the extent that a belief is tied up with its formulation, since 
the formulation is discredited, the belief is, must be wrong. After all, if a 
belief were right, its formulation would necessarily have an acceptable 
explication which was true; in short, the belief would have a true 
formulation (to see this, note that the contrary assertion is itself a 
formulation of a belief---leading to a contradiction). Incidentally, this point 
answers those who would say, that the inconsistency of their statements of 
belief taken literally does not discredit their beliefs, as the statements are not 
to be taken literally, are metaphorical or symbolic truths. To continue, one 
who because of having a belief took its formulation seriously, expected that 
it could have an acceptable explication for him, could not turn out to be an 
expression he could not properly use, must be deceiving himself in some 
way. Now there is another important point about "method" to be made. 
The question will probably continually recur to the critical reader how one 
can "know", be aware that any given belief is wrong, without having beliefs. 
The answer is that one way one can be aware of it is simply to be aware of 


51 


the inconsistency of belief language, which awareness is not a belief. 
(Whether belief language is inconsistent is not a matter of belief but of the 
way one wants expressions used; being aware of the inconsistency is like 
being aware with respect to a table, "that in my language, this is to be said to 


be a "table".) Incidentally, to wrap things up, the common belief as to how 
a name has referents is that there is a relation between the name and its 
referents which is an objective, metaphysical entity, a non-experience; this 
belief is wrong. How, in what sense a name can have referents will not be 
discussed here. 

The unsophisticated reader may react to all of this with a lot of 'Yes, 
but...' thoughts. !f he doesn't more or less identify beliefs with their 
formulations, and doesn't have an intuitive appreciation of the force of 
linguistic arguments, he my tend to regard my result as a mere (if 
embarrassing) curiosity. (Of course, it isn't, but 1 am concerned with how 
well the reader understands that.) And there does remain a lot to be said 
about beliefs themselves (as mental acts), and where the self-deception is in 
them; it is not even clear yet just what the relation of a belief to its 
formulation is. Then the reader might ask whether there aren't beliefs whose 
rejection as wrong would conflict with experience, or which it would be 
impossible or dangerous not to have. I now turn to the discussion of these 
matters. 


52 


2/22/1963 


Tony Conrad and Henry Flynt demonstrate 


1963 
(photo by Jack Smith} 


53 


against Lincoln Center, February 22, 


Part 11 : Completion of the Treatment of 
Properly Philosophical Problems 


Chapter 5 : Beliefs as Mental Acts 


In this chapter I will solve the problems of philosophy proper by 
discussing believing itself, as a ("conscious") mental act. Although I will be 
talking about mental acts and experience, it must be clear that this part of 
the book, like the fast part, is not epistemology or phenomenology. I will 
not try to talk about "perception" or the like, in a mere attempt to justify 
"common-sense" beliefs or what not. Of course, both parts are incidentally 
relevant to epistemology and phenomenology, since in discussing beliefs I 
discuss the beliefs which constitute those subjects. ; 

i should say immediately that 'belief', in its traditional use as supposed 
to refer to "mental acts, often unconscious, connected with the realm of 
non-experience", has no explication at all satisfactory, has been discredited. 
This point is important, as it means that one does not want to say that one 
does or does not "have beliefs", in the sense important to those having 
beliefs, that beliefs {in my sense) will not do as referents for 'belief' in the 
use important to those having beliefs; helping to fill out the conclusion of 
the last part. Now when I speak of a "belief" I will be speaking of an 
experience, what might be said to be "an act of consciously believing, of 
consciously having a belief', of what is "in one's head" when one says that 
one "believes a certain thing'. Further, I will, for convenience in 
distinguishing beliefs, speak of belief 'that others have minds', for example, 
or in general of belief "that there are non-experiences" (with quotation 
marks), but I must not be taken as implying that beliefs manage to be 
"about non-experiences". (Thus, what I say about beliefs will be entirely 
about experiences; I! will not be trying to talk "about the realm of 
non-experience, or the relation of beliefs to it".) I expect that it is already 
fairly clear to the reader what his acts of consciously believing are (if he has 
any); I will be more concerned with pointing out to him some features of his 
"beliefs" (believing) than with the explication of 'act of consciously 
believing', although {I will need to make a few comments about that too. 
What I am trying to do is to get the reader to accept a useful, possibly new, 
use of a word ('belief') salvaged from the unexplicatible use of the word, 
rather than rejecting the word altogether. 

There is a further point about terminology. The reader should 
remember from the third chapter that quite apart from the theory "that 
perceptions are in the mind', one can make a distinction between mental 
and non-mental experiences, between, for example, visualizing a table with 


54 


one's eyes closed, and a "seen" table, a visual-table-experience. Now ! am 
going to say that visualizations and the like are "imagined-experiences". For 
example, a _ visualization of a table will be said to be an 
"imagined-visual-table-experience". The reader should not suppose that by 
'imagined' I mean that the experiences are "hallucinations", are "unreal". I 
use 'imagined' because saying 'mental-table-experience' is too much like 
saying 'table in the mind' and because just using 'visualization' leaves no way 
of speaking of mental experiences which are not visualizations. Speaking of 
an "imagined-table-experience" seems to be the best way of saying that it is 
a mental experience, and then distinguishing it from other mental 
experiences by the conventional method of saying that it is an imagining "of 
a (non-mental) table-experience" (better thought of as meaning an imagining 
like a  (non-mental)  table-experience). in other words, an 
imagined-x-experience (to generalize) is a "valid" experience, all right, but it 
is not a non-mental x-experience; it is a mental experience which is like a 
(non-mental) x-experience in a certain way. Incidentally, an "imagined- 
imagined-experience" is impossible by definition; or is no different from an 
imagined-experience, whichever way you want to look at it. If this 
terminology is a little confusing, it is not my fault but that of the 
conventional method of distinguishing different mental experiences by 
saying that they are imaginings "of one or another non-mental experiences". 

I can at last ask what one does when one believes "that there is a table, 
not perceived by oneself, behind one now', or anything else. Well, in the 
first place, one takes note of, gives one's attention to, an 
imagined-experience, such as an imagined-table-experience or a visualization 
of oneself with one's back to a table; or to a linguistic expression, a supposed 
statement, such as 'There is a table behind me'. This is not all one does, 
however; if it were, what one does would not in the least deserve to be said 
to be a "belief" (a point about the explication of my 'belief'). The 
additional, "essential" component of a belief is a self-deceiving "attitude" 
toward the experience. What this attitude is will be described below. Observe 
that one does not want to say that the additional component is a belief 
about the experience because of the logical absurdity of doing so, or, in 
other words, because it suggests that there is an infinite regress of mental 
action. Now the claim that the attitude is "self-deceiving" is not, could not 
be, at all like the claim 'that a belief as a whole, or its formulation, fails to 
correspond in a certain way to non-experience, to reality, or is false". The 
question of "what is going on in the realm of non-experience" does not arise 
here. Rather, my claim is entirely about an experience; it is that the attitude, 
the experience not itself a belief but part of the experience of believing, is 
"consciously, deliberately' self-deceiving, is a "self-deception experience". I 


55 


don't have to "prove that the attitude is self-deceiving by reference to what 
is going on in the realm of non-experience"; when I have described the 
attitude and the reader is aware of it, he wil! presumably find it a good 
explication, unhesitatingly want, to say that it is "self-deceiving". 

I will now say, as well as can be, what the attitude is. In believing, one 
is attentive primarily to the imagined-experience or linguistic expression as 
mentioned above. The attitude is 'peripheral', is a matter of the way one is 
atttentive. Saying that the attitude is 'conscious, deliberate', is a little 
strong if it seems to imply that it is cynical self-brain washing; what I am 
trying to say is that it is not an "objective" or "subconscious" self-deception 
such as traditional philosophers speak of, one impossible to be aware of. This 
is about as much as I can say about the attitude directly, because of the 
inadequacy of the English descriptive vocabulary for mental experiences; 
with respect to English the attitude is a 'vague, elusive" thing, very difficult 
to describe. I will be able to say more about what it is only by suggestion, by 
saying that it is the attitude "that such and such" (the reader must not think 
I mean the belief "that such and such"). If the experience to which the 
attention is primarily given in believing is an imagined-x-experience, then the 
self-deceiving attitude is the attitude "that the imagined-x-experience is a 
(non-menta!) x-experience". As an example, consider the belief 'that there is 
a table behind one". If one's attention in believing is not on a linguistic 
expression, it will be on an _ imagined-experience such as an 
imagined-table-experience or a visualization of a person representing oneself 
(to be accurate) with his back to a table, and one will have the self-deceiving 
attitude "that the imagined-experience is a table or oneself with one's back 
to a table". Of course, if one is asked whether one's imagined-x-experience is 
a (non-menta!) x-experience, one will say that it is not, that it is admittedly 
an imagined-experience but "corresponds to a non-experience". This is not 
inconsistent with what I have said: first, I don't say that one believes "that 
one's imagined-x-experience is an x-experience"; secondly, when one is asked 
the question, one stops believing 'that there is a table behind one" and starts 
believing "that one's imagined-experience corresponds in a certain way to a 
non-experience", a different matter (different belief). 

lf one's attention in believing is primarily on a linguistic expression 
(which if a sentence, will be pretty much regarded as its associated name), 
the self-deceiving attitude is the attitude "that the expression has a 
referent'. With respect to the belief "that there is a table behind one", one's 
attention in believing would be primarily on the expression 'There is a table 
behind me', pretty much regarded as 'There being a table behind me', and 
one would have the self-deceiving attitude "that this name has a referent'. 
Unexplicatible expressions, then, function as principal components of 


56 


beliefs. 

(This paragraph is complicated and inessential; if it begins to confuse 
the reader it can be skipped.) I will now describe the relation between the 
version, of a belief, involving language and the version not involving 
language. In the version not involving language, the attention is on an 
imagined-x-experience which is "regarded" as an x-experience, whereas in 
the version involving language, the attention is on something which is 
"regarded" as having as referent "something" (the attitude is vague here). 
For the latter version, the idea is "that the reality is at one remove', and 
correspondingly, one whose "language" consists of formulations of beliefs 
doesn't desire to have as experiences, or perceive, or even be able to imagine, 
referents of expressions---which, for the more critical person, may make 
believing easier. Thus, just as one takes note of the imagined-x-experience in 
the version of the belief not involving language, has something which 
functions as the thing the belief is about, so in the version involving language 
one has the attitude that the expression has a referent. Further, just as one 
has the attitude that the imagined-x-experience is an x-experience in the 
version not involving language, does not recognize that what functions as the 
thing believed in is a mere imagined-experience, so in the version involving 
"Yanguage" one takes note of an 'expression' not having a referent, since a 
referent could only be a (mere) experience. One who expects an expression, 
which is the principal component of a belief, to have a good explication does 
so on the basis of the self-deceiving attitude one has towards it in having the 
belief. In trying to explicate the expression, one finds inconsistent desires 
with respect to what its referents must be. These desires correspond to the 
way the expression functions in the belief: the desire that it be possible for 
awareness of the referent to be part of one's experience corresponds to the 
attitude, in believing, that the expression has a referent; and the desire that it 
not be possible for awareness of the referent to be (merely) part of one's 
experience corresponds to the expression's not having a referent in believing. 
Pointing out that the expression is unexplicable discredits the belief of which 
it is the principal component, just as pointing out that a belief not involving 
language consists of being attentive to an imagined-experience and having the 
attitude that it is not an imagined-experience, discredits that belief. 

Such, then, is what one does when one believes. If the reader is rather 
unconvinced by my description, especially because of my speaking of 
"attitudes", then let him consider the following summary: there must be 
something more to a mental act than just taking note of an experience for it 
to be a "belief"; this something is "peripheral and elusive', so that I am 
calling the something an "attitude", the most appropriate way in English to 
speak of it; the attitude, an experience not itself a belief but part of the 


57 


experience which is the belief, is thus isolated; the attitude is 
"self-deceiving', is a "(conscious) self-deception experience', because when 
aware of it the reader will presumably want to say that it is. The attitude just 
about has to be a ("conscious") self-deception experience to transform mere 
taking note of an experience into something remotely deserving to be said to 
be a 'belief'. The decision as to whether the attitude is to be said to be 
"self-deceiving" is to be made without trying to think "about the relation of 
the belief as a whole to the realm of non-experience", to do which would be 
to slip into having beliefs, other than the one under consideration, which 
would be irrelevant to our concern here. Ultimately, the important thing is 
to observe what one does in believing, and particularly the attitude, more 
than to say that the attitude is "self-deceiving". 

In order for my description of believing to be complete, I must mention 
some things often associated with believing but not "essential" to it. First, 
one may take note of non-mental and imagined-experiences other than the 
one to which attention is primarily given. If one has a table-experience and 
believes "that it is a table-perception corresponding to an objectively existing 
table', one may give much of his attention to the table-experience in so 
believing, associate the table-experience strongly with the belief. One may in 
believing give attention to non-mental experiences supposed to be 'evidence 
for, confirmation of, one's belief" (more will be said about confirmation 
shortly). If one's attention in believing is primarily on the linguistic 
expression 'x', one may give attention to a_ referent of 
'imagined-x{-experience)', an "imagined-referent" of 'x'; or to 
imagined-y-experiences such that y-experiences are supposed, said, to be 
"analogous to the referent of 'x". In the latter case the y-experiences will be 
mutually exclusive, and less importance will be given to them than would be 
to imagined-referents. An example of imagined-referents in believing is 
visualizing oneself with one's back to a table, as the imagined-referent of 
'There being a table behind one'. An example of imagined-y-experiences 
(such that y-experiences are mutually exclusive) which are said to be 
"analogous to referents", in believing, is the visualizations associated with 
beliefs "about entities wholly other than, transcending, experience, such as 
Being'. 

Secondly, there are associated with beliefs logical "justifications", 
"arguments", for them, "defenses" of them. I will not bother to explicate 
the different kinds of justifications because it is so easy to say what is wrong 
with all of them. There are two points to be made. First, explication would 
show that the matter of justifications for beliefs is just a matter of language 
and beliefs of the kind already discussed. Secondly, as I have suggested 
before, whether a statement or belief is right is not dependent on what the 


58 


t 
i 
$ 
} 
} 
ig 


justifications, arguments for it are. (If this seems to fail for inductive 
justification, the kind invoiving the citing of experience supposed to be 
evidence for, confirmation of, the belief, it is because the metaphysical 
assumptions on which induction is based are rarely stated. Without them 
inductive justifications are just non sequiturs. An example: this table has 
four legs; therefore ("it is more probable that') any other table has four 
legs.) Justification of a statement or belief does nothing but conjoin to it 
superfluous statements or beliefs, if anything. The claim that a justification, 
argument can show that a belief is not arbitrary, gratuitous, in that it can 
show that to be consistent, one must have the belief if one has a Sesser, 
weaker belief, is simply self-contradictory. If a justification induces one to 
believe what one apparently did not believe before hearing the justification, 
then one already had the belief 'implicitly' (it was a conjunct of a belief 
one already had), or one has accepted superfluous beliefs conjoined with it. 

f will conclude this chapter first with a list of philosophical positions 
my position is not. Although I have already suggested some of this material, 
I repeat it because it is so important that the reader not misconstrue my 
position as some position which is no more like mine than its negation is, 
and which I show to be wrong. My position is not disbelief. (Incidentally, it 
is ironic that 'disbeliever', without qualification, has been used by believers 
as a term of abuse, since, as disbelief is belief which is the negation of some 
belief, any belief is disbelief.) In particular, I am not concerned to deny "the 
existence of non-experience", to "cause non-experiences to vanish", so to 
speak, to change or cause to vanish some of the reader's non-mental 
experiences, "perceived objects". My position is not skepticism of any kind, 
is not, for example, the belief "that there is a realm where there could either 
be or not be certain entities not experiences, but our means of knowing are 
inadequate for finding which is the case." My position is not a mere 
"decision to ignore non-experiences, or beliefs". The philosopher who denies 
"the existence of non-experiences", or denies any belief, or who is skeptical 
of any belief, or who merely "decides to ignore non-experiences, or beliefs", 
has some of the very beliefs 1 am concerned to discredit. 

What I have been concerned to do is to discredit formulations of 
beliefs, and beliefs as mental acts, by pointing out some features of them. In 
the first part of the book I showed the inconsistency of linguistic expressions 
dependent on 'non-experience', and pointed out that those who expect them 
to have explications at all acceptable are deceiving themselves; discrediting 
the beliefs of which the expressions are formulations. In this chapter, I have 
described the mental act of believing, calling the reader's attention to the 
self-deception experience involved in it, and thus showing that it is wrong. 
To conclude, in discrediting beliefs I have shown what the right 


59 


d realizing, for any belief 
sn't involve having 


it is not having beliefs (an 


philosophical position is: 
it is wrong (which doe 


one happens to think of, that 
beliefs)). 


60 


ESTHETICS 


8. Down With Art 


1; 
ART or BREND? by Henry Flynt 


1. Perhaps the most diseased justification the artist can give of his profession 
is to say that it is somehow scientific. L7-Monte Young, Milton Babbitt, and 
Stockhausen are exponents of this sort of justification. 

The flaw which reiates the mass of a body to its velocity has predictive value 
and is an outstanding scientific law. Is the work of art such a law? The 
experiment which shows that the speed of light is independent of the motion 
of its source is a measurement of a phenomenon crucial to the confirmation of 
a scientific hypothesis; it is an outstanding scientific experiment. Is the work 
of art such a measurement? The invention of the vacuum tube was an 
outstanding technological advance. Is the work of art such a technological 
advance? Differential geometry is a deductive analysis of abstract relations 
and an outstanding mathematical theory. ts the work of art such an 
analysis? 

The motives behind the "scientific" justification of art are utterly sinister. 
Perhaps LaMonte Young is merely rationalizing because he wants an 
academic job. But Babbitt is out to reduce music to a_ pedantic 
pseudo-science. And Stockhausen, with his "scientific music', intends 
nothing less than the suppression of the culture of 'lower classes" and 
"ower races." 

It is the creative personality himself who has the most reason to object to 
the "scientific" justification of art. Again and again, the decisive step in 
artistic development has come when an artist produces a work that shatters 
all existing 'scientific' laws of art, and yet is more important to the 
audience than all the works that "obey" the laws. 

2. The artist o: entertainer cannot exist without urging his product on other 
people. In fact, after developing his product, the artist goes out and tries to 
win public acceptance for it, to advertise and promote it, to sell it, to force it 
on people. If the public doesn't accept it at first, he is disappointed. He 
doesn't drop it, but repeatedly urges the product on them. 

People have every reason, then, to ask the artist: Is your product good for 


63 


me even if I don't like or enjoy it? This question really lays art open. One of 
the distinguishing features of art has always been that it is very difficult to 
defend art without referring to people's liking or enjoying it. (Functions of 
art such as making money or glorifying the social order are real enough, but 
they are rarely cited in defense of art. Let us put them aside.) When one 
artist shows his latest production to another, all he can usually ask is 'Do 
you like it?" Once the "scientific" justification of art is discredited, the 
artist usually has to admit: If you don't like or enjoy my product, there's no 
reason why you should "consume" it. 

There are exceptions. Art sometimes becomes the sole channel for political 
dissent, the sole arena in which oppressive social relations can be 
transcended. Even so, subjectivity of value remains a feature which 
distinguishes art and entertainment from other activities. Thus art is 
historically a leisure activity. 

3. But there is a fundamental contradiction here. Consider the object which 
one person produces for the liking, the enjoyment of another. The value of 
the object is supposed to be that you just like it. It supposedly has a value 
which is entirely subjective and entirely within you, is a part of you. Yet---the 
object can exist without you, is completely outside you, is not you or your 
valuing, and has no inherent connection with you or your valuing. The 
product is not personal to you. 

Such is the contradiction in much art and entertainment. it is unfortunate 
that it has to be stated so abstractly, but the discussion is about something 
so personal that there can be no interpersonal examples of it. Perhaps it will 
help to say that in appreciating or consuming art, you are always aware that 
it is not you, your valuing---yet your liking it, your valuing it is usually the 
only thing that can justify it. 

In art and entertainment, objects are produced having no inherent 
connection with people's liking, yet the artist expects the objects to find 
their value in people's liking them. To be totally successful, the object would 
have to give you an experience in which the object is as personal to you as 
your valuing of it. Yet you remain aware that the object is another's 
product, separable from your liking of it. The artist tries to "be oneself' for 
other people, to "express oneself" for them. 
4. There are experiences for each person which accomplish what art and 
entertainment fail to. The purpose of this essay is to make you aware of t 
these experiences, by comparing and contrasting them with art. I have 
coined the term "brend" for these experiences. 

Consider all of your doings, what you already do. Exclude the gratifying of 
physiological needs, physically harmful activities, and competitive activites. 
Concentrate on spontaneous self-amusement or play. That is, concentrate on 


64 


everything you do just because you like it, because you just like it as you do 
it. 

Actually, these doings should be referred to as your just-likings. In saying 
that somebody likes an art exhibit, it is appropriate to distinguish the art 
exhibit from his liking of it. But in the case of your just-likings, it is not 
appropriate to distinguish the objects valued from your valuings, and the 
single term that covers both should be used. When you write with a pencil, 
you are rarely attentive to the fact that the pencil! was produced by 
somebody other than yourself. You can use something produced by 
somebody else without thinking about it. In your just-likings, you never 
notice that things are not produced by you. The essence of a just-liking is 
that in it, you are not aware that the object you value is less personal to you 
than your very valuing. 

These just-likings are your "brend." Some of your dreams are brend; and 
some children's play is brend (but formal children's games aren't). In a sense, 
though, the attempt to give interpersonal examples of brend is futile, 
because the end result is neutral things or actions, cut off from the valuing 
which gives them their only significance; and because the end result suggests 
that brend is a deliberate activity like carrying out orders. The only examples 
for you are your just-likings, and you have to guess them by directly 
applying the abstract definition. 

Even though brend is defined exclusively in terms of what you like, it is not 
necessarily solitary. The definition simply recognizes that valuing is an act of 
individuals; that to counterpose the likes of the community to the likes of 
the individuals who make it up is an ideological deception. 

5. It is now possible to say that much art and entertainment are 
pseudo-brend; that your brend is the total originality beyond art; that your 
brend is the absolute self-expression and the absolute enjoyment beyond art. 
Can brend, then, replace art, can it expand to fill the space now occupied by 
art and entertainment? To ask this question is to ask when utopia will 
arrive, when the barrier between work and leisure will be broken down, 
when work will be abolished. Rather than holding out utopian promises, it is 
better to give whoever can grasp it the realization that the experience 
beyond art already occurs in his life---but is totally suppressed by the general 
repressiveness of society. 


Note: The avant-garde artist may 'raise a final question. Can't art or 
entertainment compensate for its impersonality by having sheer newness as a 


65 


value? Can't the very foreignness of the impersonal object be entertaining? 
Doesn't this happen with Mock Risk Games, for example? The answer is 
that entertainmenta! newness is also subjective. What is entertainingly 
strange to one person is incomprehensible, annoying, oF irrelevant to 
another. The only difference between foreignness and other entertainment 
values is that brend does not have more foreignness than conventional 
entertainment does. 

As for objective newness, or the objective value of Mock Risk Games, these 
issues are so difficult that I have been unable to reach final conclusions 
about them. 


66 


2. 


Letter from Terry Riley, Paris, to Henry Flynt, Cambridge, 
Mass., dated 11/8/62 


One day a little boy got up and looked at his toys, appraised them and 
decided they were of no value to him so he did them in. Seeing that others 
were blindly and blissfully enjoying theirs he offered them a long and 
"radical new theory" of "pure recreation" for their enjoyment but before he 
let them in for this highly secret and "revolutionary theory' they should 
follow his example and partake of a little 20th C. iconoclasm. From those 
that balked he removed the label "avant-garde" and attached the label 
traditionalist' or if they were already labeled "traditionalist" he added one 
more star. If they accepted they got a "hip" rating with gold cluster and if 
they comprehended the worth of his theory well enough to destroy their 
own art they would be awarded assignments to destroy those works whose 
designers were no longer around to speak out in their behalf. 
Now about this hip radical new theory of pure recreation.---Well---alor! its 
simply what people do anyway but don't realize it but it seems that what 
people "do anyway and don't realize it" will not be fully appreciated until 
"what people do in the name of art" is eliminated. If art can be relegated to 
obscurity, if some one can get John Coltrane to stop blowing, if someone 
can smash up all the old Art tatum records as well as all the existing pianos, 
if someone can get all that stuff out of those museums, If someone can only 
burn down all those concert halls, movie houses, small galleries as well as 
rooms in private houses that contain signs of art, If someone can do in all the 
cathedrals and monuments bridges etc, If someone can get rid of the sun, 
moon, stars, ocean, desert trees birds, bushes mountains, rivers, joy, sadness 
inspiration or any other natural phenomenon that reminds us of the ugly 
scourge art that has preoccupied and plagued man since he can remember 
then yes then at last Henry Flynt, sorry! 

sites tere tase 


> Henry Flynt 


v 
er 


. 
TaySs 


will show us how to really enjoy ourselves. Whooopeeee 
[Terry Riley's spelling etc. carefully preserved] 


67 


3. 


letter from Bob Morris to Henry Flynt, dated 8/13/62 


Dear Henry, 

perhaps the desirability of certain kinds of experience in art is not 
important. The problem has been for some time one of ideas---those most 
admired are the ones with the biggest, most incisive ideas (e.g. Cage & 
Duchamp). The mere exertion in the direction of finding "new" ideas has 
not shown too much more than that it has become established as a 
traditional method; not much fruit has appeared on this vine. Also it can't be 
avoided that this is an academic approach which presupposes a history to 
react against---what I mean here is the kind of continuity one is aware of 
when involved in this activity: it just seems academic (if the term can 
somehow be used without so much emotion attached to it). The difficulty 
with new ideas is that they are too hard to manufacture. Even the best have 
only had a few good ones. {I suppose none of this is very clear and I can't 
seem to get in the mood to do any more than put it down in an off-hand 
way---but what I mean by "new ideas" is not only what you might call! 
"Concept Art" but rather effecting changes in the structures of art forms 
more than any specific content or forms) Once one is committed to attempt 
these efforts---and tries it for a while---one becomes aware that if one wants 
"experience" one must repeat himself until other new things occur: a 
position difficult if not impossible to accept with large "idea" ambitions. So 
one remains idle, repeats things, or finds some form of concentration and 
duration outside the art---jazz, chess, whatever. I think that today art is a 
form of art history. 

I don't think entertainment solves the problem presented by avant gard art 
since entertainment has mostly to do with replacing that part of art which is 
now hard to get---i.e. experience. It seems to me that to be concerned with 
"just liked" things as you present it is to avoid such things as tradition in art 
(some body of stuff to react against---to be thought of as opponent or 
memory or however}. As I said before, I for one am not so self-sufficient and 
when avoiding "given" structures, e.g. art, or even the most tedious and 
decorous forms of social intercourse, I am bored. {f I need concentration, 
which I do, I can't think of anything on my own as good as chess. 

One accepts language, one accepts logic. 

Best regards, 

Bob Morris 


68 


> 
i 
fe 
Ff 


4, 


FROM "CULTURE" TO VERAMUSEMENT 
Boston-New York 
PRESS RELEASE: for March-April, 1963 


Henry Flynt, Tony Conrad, and Jack Smith braved the cold to demonstrate 
against Serious Culture (and art) on Wednesday, February 27. They began at 
the Museum of Modern Art at 1:30 p.m., picketing with signs bearing the 
slogans DEMOLISH SERIOUS CULTURE! /DESTROY ART! ; DEMOLISH 
ART MUSEUMS! / NO MORE ART! ; DEMOLISH CONCERT HALLS! / 
DEMOLISH LINCOLN CENTER! ; and handing out announcements of 
Flynt's lecture the next evening. Benjamin Patterson came up to give 
encouragement. There was much spontaneous interest among people around 
and in the Museum. At about 1:50, a corpulent, richly dressed Museum 
official came out and imperiously told the pickets that he was going to 
straighten them out, that the Museum had never been picketed, that it could 
not be picketed without its permission, that it owned the sidewalk, and that 
the pickets would have to go elsewhere. The picket who had obtained police 
permission for the demonstration was immediately dispatched to call the 
police about the matter, while the other two stood aside. !t was found that 
the Museum official had not told the truth; and the picketing was resumed. 
People who care about the rights of pickets generally should recognize the 
viciousness of, and oppose, the notion that picketing can only be at the 
permission of the establishment being picketed. (As for previous picketing of 
the Museum, it is a matter of record.) Interest in the demonstration 
increased; people stopped to ask questions and talk. There was a much 
greater demand for announcements than could be supplied. Some people 
indicated their sympathy with the demonstrators. The demonstrators then 
went on to the Metropolitan Museum of Art. Because of the unexpected 
requirement of a permit to picket on a park street, they had to picket on 
Lexington Avenue, crossing 82nd Street. As a result they were far from the 
fools lined up to worship the Mona Lisa, but there was still interest. Finally, 
they went to Philharmonic Hall. Because of the time, not many people were 
there, but still there was interest; people stopped to talk and wanted more 
announcements than were available. The demonstrations ended at 3:45 p.m. 
Photos of the pickets were taken at all three places. 

On Thursday evening, February 28, at Walter DeMaria's loft, Henry Flynt 
gave a long lecture expositing the doctrine the Wednesday demonstrations 
were based on. On entering the lecture room, the visitor found himself 
stepping in the face of a Mona Lisa print placed as the doormat. To one side 


69 


was an exhibition of demonstration photos and so forth. Behind the lecturer 
was 2 large picture of Viadimir Mayakovsky, while on either side were the 
signs used in the demonstrations, together with one saying 
VERAMUSEMENT---NOT CULTURE. About 20 people came to the lecture. 
The lecturer showed first the suffering caused by Serious-Cultural snobbery, 
by its attempts to force individuals in line with things supposed to have 
objective validity, but actually representing only alien subjective tastes 
sanctioned by tradition. He then showed that artistic categories have 
disintegrated, and that their retention has become obscurantist. (He showed 
that the purpose of didactic art is better served by documentaries.) Finally, 
in the most intellectually sophisticated part of the lecture, he showed the 
superiority of each individual's veramusement (partially defined on the 
lecture announcement) to institutionalized amusement activities (which 
impose foreign tastes on the individual) and indeed to all "culture" the 
lecture was concerned with. After the lecture, Flynt told how his doctrine 
was anticipated by little known ideas of Mayakovsky, Dziga Vertov, and 
their group, as related in Ilya Ehrenburg's memoirs and elsewhere. He 
touched on the Wednesday demonstrations. He spoke of George Maciunas' 
FLUXUS, with which all this is connected. Several people at the lecture 
congratulated Flynt on the clarity of the presentation and logicality of the 
arguments. Photos were taken. 


5. 
Statement of November 1963 


Back in March 1963, I sent the tirst FCTB PRESS RELEASE, about FCTB's 
February picketing and lecture, to all the communications media, including 
the New Yorker. It is so good that the New Yorker wanted to use it, but 
they didn't want to give FCTB any free publicity; so they finally published 
an inept parody of it, in the October 12, 1963 issue, pp. 49-51. They 
changed my last name to Mackie, changed February 27 to September 25, the 
Museum of Modern Art to a church, changed our slogans to particularly 
idiotic ones {although they got in 'NO MORE ART/CULTURE?' later on), 
and added incidents; but the general outlines, and the phrases lifted verbatim 
from the FCTB RELEASE, make the relationship clear.---Henry Fiynt 


70 


pee 


Henry, 3/6/63 


Received your note this morning. I had written down a few things about the 
lecture the very night I got home but decided they were not very clear so I 
didn't send them. Don't know if I can make it any clearer...actually I keep 
thinking that I must have overlooked something because the objection I have 
to make seems too obvious. You spend much time and effort locating 
Veramusement, stating clearly wnat it is not, and stating that it is, if I get it, 
of the essence of an awareness, rather memory, of an experience which 
cannot be predicted and therefore cannot be located or focused by external 
activities. And, in fact, as you said, may cut across, or "intersect" one or 
another or several activities. You have discredited activities---like art, 
competitive games---as pseudo work or unsatisfactory recreation by employing 
arguments which are external to "experiencing" these activities (e.g. chess is 
bad because why agree to some arbitrary standard of performance which 
doesn't fit you)...weil it seems to me that Veramusement could never replace 
any cultural form because it has no external "edges" but rather by definition 
can occur anywhere anytime anyplace (By the way I want to say here that 
its existence as a past tense or memory I find objectionable---but I can't at the 
moment really say why.) It seems that you have these two things going: 
Veramusement, that has to do with experience, and art, work, 
entertainment, that have to do with society and I don't think that the 
exposition of how the two things are related has been very clear. George 
Herbert Mead, an early Pragmatist (don't shudder at that word, but I can see 
you throwing up your hands in despair) talked about this relation as a kind 
of double aspect of the personality (which he called the "me" and the "I" 
..can't remember his book, something like Mind, Self, and Society). 

I thought you presented the lecture very weil, but towards the end I was 
getting too tired to listen very carefully and I am sorry because this was the 
newest writing. I would like very much to read this part, i.e. that which dealt 
with the evolution of work, automation and the liberation from 
drudgery---send me a copy if you can. 

Best regards, 

Bob Morris 


71 


Henry 3/12/1963 


(anti-art? } 
Jgfz Cage "Folk Music" Communism ____...----.----- 
(communism) 
I've been along this road too. 
Yes I certainly do see the harmfullness of serious culture. My favorite movies 
are plain documentaries. 


"Veramusement" 
questions: the way you set it up it sound like veramusement is IT. Some 


kind of Absolute good state or activity. --ie) ATHLETICS are out. 
-now my brother is a healthy athelete--he enjoys nothing so much as 
swimming or playing tennis all! day (he likes to use his body--and he likes the 
form--competition) 

Is this "wrong" 

Should he stop.-- 


or wouldn't your "creep theory" which lets each person be himself and 
relish in himself--by extention from this--shouldn't the atheletic person be 


alowed to be himself? --too. 
I think you were opening up the world to the people at the lecture-- 


making them move free-- 
sd "ready to be themselves 


I think you were right in not giving examples! 


however 

your absolute--statements and 'come on"--and blend with the communist 
ideas--(My mind was pretty tired by then and I didn't follow how the 
veramusement--was tied to communism)--this IT kind of taik.--can only shoo 
people off-and let them wait for the next revision or explication. 

people off--and let them wait for the next revision or explication. 


Walter DeMaria 


72 


8. 


Dear Henry, March 18, 1963 


As I said before, my main reactions to yr lecture & ideas is that I'm for 
Henry Flynt but not for his ideas. I think the spirit you show in carrying on 
yr crusade is admirable and exciting. However, I am not against art and think 
that any artist who would say that he is or think that he is would be 
masochistic enough to need psychiatric care. Since you make no claims to 
being an artist this does not refer to you. However, I do call myself a poet 
and do think of myself as one. I like art, culture, etc. and do not yet feel 
that I am being screwed by it. Until I do, I will not need to turn to anti-art 
movements. 

All best wishes. 

Yours, 

Diane Wakoski 


"Dear Mr. Flynt...Since I may be depending on o-ganized culture for my 
loot & livelihood I can wish you only a limited success in your movement... 
Cornelius Cardew" [froma postcard of June 7, 1963] 


73 


2/22/1963 


Jack Smith and Henry Flynt demonstrate against the 
February 22, 1963 


(photo by Tony Conrad) 


74 


Museum of Modern Art, 


PARA—SCIENCE 


> 9, The Perception-Dissociation of Physics 


From the physicist's point of view, the human dichotomy of sight and 
touch is a coincidence. It does not correspond to any dichotomy in the 
objective physical world. Light exerts pressure, and substances hot to the 
touch emit infrared light. It is just that the range of human receptors is too 
limited for them to register the tactile effect of light or the visual effect of 
moderate temperatures. 

Our problem is to determine what observations or experiences would 
cause the physicist to say that the objective physical world had split along 
the humen sight-touch boundary, to say that the human sight-touch 
dichotomy was an unavoidable model of objective physical reality. Our 
discussion is not about perfectly transparent matter, or light retlection and 
emission in the absence of matter, or the dissociation of electromagnetic and 
inertial phenomena, or the fact that human sight registers light, while touch 
registers inertia, bulk modulus, thermal conduction, friction, adhesion, and 
so on. (However, these concepts may have to be introduced to complete our 
discussion.) Our discussion is about a change in the physicist's observations 
or experiences, such that the anomalous state of affairs would be an 
experimental analogue to the sight-touch dichotomy of philosophical 
subjectivism. Of course, philosophical subjectivism itself will not enter the 
discussion. 

Because of the topic, our discussion will often seem psychological and 
even philosophical. However, the psychology involved always has to do with 
experimentally demonstrable aspects of perception. The philosophy involved 
is always scientific concept formation, the relating of concepts to 
experiments. Sooner or later it will be clear that our only concern is with 
experiences that would cause a physicist to modify physics. 

Throughout much of the discussion, we have to assume that the human 
physicist exists before the sight-touch split occurs, that he continues to exist 
after it occurs, and that he functions as a physicist after it occurs. Therefore, 
we begin as follows. A healthy human has a realm of sights, and a realm of 
touches: and there is a correlation between the two which receives its highest 
expression in the concept of the object. (In psychological jargon, intermodal 
organization contributes to the object Gestalt. Incidentally, for us "touch" 
includes just about every sense except sight, hearing, smel!.) Suppose there is 


77 


a change in which the tactile realm remains coherent, if not exactly the same 
as before, and the visual realm also remains coherent; but the correlation 
between the two becomes completely chaotic. A totally blind person does 
not directly experience any incomprehensible dislocation, nor does a person 
with psychogenic tactile anesthesia (actually observed in hysteria patients). 
Let us define such a change. Consider the sight-touch correlation identified 
with closing one's eyes. The point is that there is a whole realm of sights 
which do not occur when one can feel that one's eyes are closed. 

Let T indicate tactile and V indicate visual. Let the tactiie sensation of 
open eyes be T, and of closed eyes be To. Now anything that can be seen 
with closed eyes--from total blackness, to the multicolored patterns produced 
by waving the spread fingers of both hands between closed eyes and direct 
sunlight--can no doubt be duplicated for open eyes. Closed-eye sights are a 
subset of open-eye sights. Thus, let sights seen only with open eyes be V1, 
and sights seen with either open or closed eyes be V>: If there are sights seen 
only with closed eyes, they will be V3; we want disjoint classes. We are 
interested in the temporal concurrence of sensations. Combining our 
definitions with information about our present world, we find there are no 
intrasensory concurrences (eyes open and closed at the same time). Further, 
our change will not produce intrasensory concurrences, because each realm 
will remain coherent. Thus, we will drop them from our discussion. There 
remain the intersensory concurrences, and four can be imagined; let us 
denote them by the ordered pairs (T,, Vj), (17, V9), (To, V4), (Tp, V9). In 
reality, some concurrences are permitted and others are forbidden, Let us 
designate each ordered pair as permitted or forbidden, using the following 
notation. Consider a rectangular array of "places" such that the place in the 
ith row and jth column corresponds to (T;, Vj), and assign a p or f (as 
appropriate) to each place. Then the following state array is a description of 
regularities in our present world. ¢ 3 


fp 


So far as temporal successions of concurrences (within the présent 
world) are concerned, any permitted concurrence may succeed any other 
permitted concurrence. The succession of a concurrence by itself is 
excluded, meaning that at the moment, a Vv, is defined as lasting from the 
time the eyes open until the time they next close. 

We have said that our topic is a certain change; we can now indicate 
more precisely what this change is. As long as we have a 2x2 array, there are 
16 ways it can be filled with p's and f's. That is, there are 16 imaginable 
states. The changes we are interested in, then, are specific changes from the 


present state'p p\to another state such ap fI\ However, 
tp pp/ 

we want to exclude some changes. The change that changes nothing is 
excluded. We aren't interested in changing to a state having only f's, which 
amounts to blindness. A change to a state with a row or column of f's leaves 
one sight or touch completely forbidden {a person becomes blind to 
open-eye sights); such an "impairment" is of little interest. Of the remaining 
changes, one merely leaves a formerly permitted concurrence forbidden: 
closed-eye sights can no longer be seen with open eyes. The rest of the 
changes are the ones most relevant to perception-dissociation. They are 
changes in the place of the one f ; the change to the state having only p's; 


and finally / 
PP) > fp 


\f p pf 


In general, we speak of a partition of a sensory realm into disjoint 
classes of perceptions, so that the two partitions are [Tj] and [Vj]. The 
number of classes in a partition, m for touch and n for sight, is its 
detailedness. The detailedness of the product partition [T;] X [V;] is written 
m x n. This detailedness virtually determines the (mn)? imaginable states, 
although it doesn't determine their qualitative content. Now suppose one 
change is followed by another, so that we can speak of a change series. It is 
important to realize that by our definitions so far, a change series is not a 
conposition of functions; it is a temporal phenomenon in which each state 
lasts for a finite time. (A function would be a genera! rule for rewriting 
states. A 2X2 rule might say, rotate the state clockwise one place, fromja b 
to/ca\. cd 

ce 


But a composition of rules would not be a temporal series; it would be a new 
rule.) Returning to the sorting of changes, we always exclude the no-change 
changes, and states having only f's. We are unenthusiastic about 'impairing' 
changes, changes to states with rows or columns of f's. Of the remaining 
changes, some merely forbid, repiacing p's with f's. The rest of the changes 
are the most perception-dissociating ones. 

As for changes in the succession state in the eye case, either they leave 
the forbidden concurrence permitted; or else they merely leave permitted 
successions forbidden--for example, in order to open your eyes in the dark 
you might have to open them in the light and then turn the light off. These 
secondary changes are of secondary interest. 

If we simply continue with the material we already have, two lines of 
investigation are possible. The first investigation is mathematical, and 


79 


s 


apparently amounts to combinatorial algebra. The second investigation 
concerns the relation between concurrences and commands of the will 
(observable as electrochemica! impulses along efferent neurons). If a change 
occurs, and the perceptual feedback from a willed command consists of a 
formerly forbidden concurrence, is it T or V that conflicts with the 
command? Is it that you tried to close your eyes but couldn't get the sight 
to go away, or that you were trying to look at something but felt your eyes 
close anyway? 

Before we carry out these investigations, however, we must return to 
our qualitative theory. If one of our eye changes happens to a physicist, he 
may immediately conclude that the cause of the anomaly is in himself, that 
the anomaly is psychological. But suppose that starting with a state for an 
extremely detailed product partition describing the present world, a whole 
change series occurs. Let p's be black dots and f's be white dots, and imagine 
a continuously shaded gray rectangle whose shading suddenly changes from 
time to time. We evoke this image to impress on the reader the 
extraordinary qualities of our concept, which can't be conveyed in ordinary 
English. Suppose also that to the extent that communication between 
scientists is still possible, perhaps in Braille, everybody is subjected to the 
same changes. !f the physicist turns to his instruments, he finds that the 
anomalies have spread to his attempts to use them. The changes affect 
everything-- everything, that is, except the intrasensory coherence of each 
sensory realm. Intrasensory coherence becomes the only stable reference 
point in the "world." The question of "whether the anomaties are really 
outside or only in the mind" comes to have less and less scientific meaning. 
If physics survived, it would have to recognize the touch-sight dichotomy as 
a physical one! This scenario helps answer a question the reader may have 
had: what is the methodological status of our states? They don't seem to be 


either physics or psychology, yet it is quite clear how we would know if the , 


asserted regularities had changed; in fact, that is the whole point of the 
states. The answer is that the states are perfectly good assertions (of 
observed regularities) which would acquire primary importance if the 
changes actually occurred. In fact, the changes would among other things 
shift the boundaries of physics and psychology; but we insist that our 
interest is in the physicist's side of the boundary. To complete the 
investigation we have outlined, the relation between what the states say and 
what existing physics says should be established, so that we will know what 
has to be done to the photons and electrons to produce the changes. It is the 
same as with time travel: the hard part is deciding what it is and the even 
harder part is making it happen. 


* * * 


80 


However, the foundations of our qualitative theory are not yet 
satisfactory, We have assumed that the physicist will be able to identify the 
subjective concurrences of perceptions, and will be able to identify his 
perceptions themselves, even if sense correlation becomes completely 
chaotic. We have assumed that the physicist will be able to say "I see a book 
in my hand but I concurrently feel a pencil.' These assumptions may not be 
justified at all. It is quite likely that the physicist will say, 'I don't even 
know whether the sight and the touch seem concurrent; I don't even know 
whether I think I see a book; I don't even know whether this sensation is 
visual." In fact, the anomalies may cause the physicist to decide that books 
never looied like books in the first place. In this case, the occurrence of the 
changes would render meaningless the terms in which the changes are 
defined. Alternately, if the changes produce a localized chaos, so that 
everything fits together except the book seen in the hand, the physicist may 
literally force himself to re-see that-book as a pencil, and in time this 
compensation may become habitual and "pre-conscious." In this case, if the 
physicist remembers the changes, he will be convinced that they were a 
temporary psychological malfunction. 

These criticisms are based on the fact that our simple perceptions are 
actually learned, "unconscious" interpretations of raw data which by 
themselves don't look like anything. This fact is demonstrated by a vast 
number of standard experiments in which the raw data are distorted, the 
subject perceptually adapts to the distorted data, and then the subject is 
confronted with normal sensations again. The subject finds that the old 
familiar sensation of a table looks quite wrong, and that he has to make an 
effort to see the table which he knows is there. 

Consider a modification of the clock-bell simultaneity experiment. The 
subject sits facing a large clock with a second-hand. His hearing is blocked in 
some way. Behind him, completely unseen, is a device which can give hima 
quick tap, a tactile sensation. There is also an unseen movie camera which 
photographs both the tactile contact and the clock face. The subject is 
tapped, and must call out the second-hand reading at the time of the tap. We 
expect a discrepancy between what the subject says and what the film says; 
but even if there is none, the experiment can proceed. Teli the subject that 
he always placed the tap earlier than it actually occurred, and that he will be 
given a reward if he learns to perceive more accurately. The purpose of the 
experiment is to demonstrate to the subject that even his perception of 
subjective simultaneity can be consciously modified. In the course of 
modification, he may not even know whether two perceptions seem 
simultaneous. 

This criticism of the changes defined earlier is important, but it may 


81 


not be insurmountable. Although Stratton became used to his trick 
eyeglasses, the image continued to seem distorted. There is some stability to 
our identification of our perceptions. Also, the physicist in our earlier 
scenario might ultimately adapt to the changes. He might realize that it is 
possible separately to identify sights and touches. Only the sight-touch 
correlation is unidentifiable; and the concept of such a correlation might 
become an abstract concept of physics just as the concept of particle 
resonance is today. 

Time is inescapably involved in our discussion; so we must decide what 
happens to time as a distinct physical category, and as a sense, in 
perception-dissociation. Here, we will simply distinguish three sorts of time. 
First, there is subjective concurrence, which we have already begun to 
discuss. Secondly, there is the physicist's operational definition of time. 
There must be two repeating processes, which to the best of our knowledge 
are causally independent, so that irregularities in one process aren't 
automatically introduced in the other. !f the ratio of the repetitions of the 
two processes is constant, we assume that the repetitions divide time into 
equal intervals. Eventually the physicist arrives at a concept of time as a real 
line along which movement can be both forward and backward (Feynman). 
One effect of perception-dissociation relating to this sort of time would be 
to disrupt the ratios of visual clocks (such as electric wall clocks) to tactile 
clocks (such as the pulse). The third idea of time comes from an unpublished 
manuscript by John Alten, a Harvard classmate of mine. According to Alten, 
our most intimate sensation of futurity is associated with our acts of will. 
"The future" is simply the time of willing. In comparison with volitional 
futurity, the physicist's linear, reversible time is a mere spatial concept. The 
empirical importance of Alten's idea is thet it raises the question of what the 
perceptual frustration of the will (as we defined it) would do to the sense of 
futurity. 


We now come to some considerations which will help us develop the 
state descriptions, and which also show that from one point of view, the 
states are actually necessary for the operational definition of physical 
language. Let parallel but separated sheets of clear plastic and colored plastic 
be mounted in lighting conditions so that the subject can't see the clear 
plastic. He touches the clear plastic, but from what he sees, he believes he is 
touching the colored plastic. The lighting is then changed and his error is 
exposed. In some sense, the sight-touch concurrence identifying an object 
was a mere coincidence. Next, we produce another colored sheet for the 


82 


subject to touch, and we are able to convince him that this time the 
object-identifying concurrence is more than a coincidence. 

The physicist interprets this latter case by saying that the matter which 
resists the pressure of the subject's finger also reflects the light into his eyes. 
To the extent that the physicist's interpretation is causal, it employs the 
concept of "matter," a concept which is not really either visual or tactile. 
The physicist explains a sight and a touch with a reference beyond both sight 
and touch. It is important, then, to know the operational definition of the 
physicist's statement, the testing procedures which give the statement its 
immediate meaning. What is significant is that the testing procedures cannot 
be reduced to purely visual procedures or purely tactile procedures. 
Affecting the world requires tactile operations; and the visual "reading" of 
the world is so woven into physics that it can't be given up. Yet our 
experiment showed that the subject can be fooled by object-identifying 
concurrences, and the physicist is supposed to te!l us how to avoid being 
fooled. 

We find, then, that there is nothing the physicist can appeal to, in 
testing object-identifying concurrences, that doesn't immediately rely on 
other object-identifying concurrences, the very concurrences which are 
suspect. It is as if the physicist proposed to prove that clicks come from a 
certain metronome by manipulating a detecting device that outputs its data 
as sounds. But suppose the physicist proves that the clicks come from the 
metronome by showing (1) that the metronome has to be stopped or 
removed to stop the clicks, and (2) that the clicks stop if the metronome is 
stopped or removed. The physicist proves that the object-identifying 
concurrence is not a coincidence by demonstrating that certain related 
concurrences are forbidden. We suggest that the physicist ultimately handles 
touch-sight concurrences in just this way. The operational basis of the 
physicist's activity comes down to our states. (But note that the physicist 
has tests, which do not rely directly on his hearing, to determine whether the 
clicks come from the metronome! ) One way to develop our states, then, 
may be to develop substates which express the differences between those 
object-identifying concurrences that are coincidental and those that 
aren't--the differences illustrated by the plastic sheet experiment. 


83 


2/22/1963 


Henry Flynt and Jack Smith demonstra 
February 22, 1963 
(foto by Tony Conrad) 


te against the Metropolitan Museum of Art, 


84 


10. 1966 Mathematical Studies 


QO. Introduction 


Pure mathematics is the one activity which is intrinsically formalistic. It 
is the one activity which brings out the practical value of formal 
manipulations. Abstract games fit in perfectly with the tradition and 
rationale of pure mathematics; whereas they would not be appropriate in 
any other discipline. Pure mathematics is the one activity which can 
appropriately develop through innovations of a formalistic character. 

Precisely because pure mathematics does not have to be immediately 
practical, there is no intrinsic reason why it should adhere to the normal 
concept of logical truth. No harm is done if the mathematician chooses to 
play a game which is indeterminate by normal logical standards. All that 
matters is that the mathematician clearly specify the rules of his game, and 
that he not make claims for his results which are inconsistent with his rules. 

Actually, my pure philosophical writings discredit the concept of 
logical truth by showing that there are flaws inherent in all non-trivial 
language. Thus, no mathematics has the logical validity which was once 
claimed for mathematics. From the ultimate philosophical standpoint, all 
mathematics is as "indeterminate" as the mathematics in this monograph. 
All the more reason, then, not to limit mathematics to the normal concept 
of logical truth. 

Once it is realized that mathematics is intrinsically formalistic, and need 
not adhere to the norma! concept of logical truth, why hold back from 
exploring the possibilities which are available? There is every reason to 
search out the possibilities and present them. Such is the purpose of this 
monograph. 

The ultimate test of the non-triviality of pure mathematics is whether it 
has practical applications. I believe that the approaches presented on a very 
abstract level in this monograph will turn out to have such applications. In 
order to be applied, the principles which are presented here have to be 
developed intensively on a level which is compatible with applications. The 
results will be found in my two subsequent essays, 'Subjective Propositional 
Vibration" and "The Logic of Admissible Contradictions." 


85 


1. Post-Formalism in Constructed Memories 
1.1 Post-Formalist Mathematics 

Over the last hundred years, a philosophy of pure mathematics has 
grown up which I prefer to call "formalism." As Willard Quine says in the 
fourth section of his essay "Carnap and Logical Truth,' formalism was 
inspired by a series of developments which began with non-Euclidian 
geometry. Quine himself is opposed to formalism, but the formalists have 
found encouragement in Quine's own book, Mathematical Logic. The best 
presentation of the formalist position can be found in Rudolph Carnap's The 
Logical Syntax of Language. As a motivation to the reader, and 
as a heuristic aid, I will relate my study to these two standard books. (It will 
heip if the reader is thoroughly familiar with them.) it is not important 
whether Carnap, or Quine, or formalism--or my interpretation of them--is 
"correct," for this essay is neither history nor philosophy. I am using history 
as a bridge, to give the reader access to some extreme mathematical 
innovations. 

The formalist position goes as follows. Pure mathematics is the 
manipulation of the meaningless and arbitrary, but typographically 
well-defined ink-shapes on paper 'w,' 'x,' 'y,' 'z,' %,? "7 *),° fy and 'e.' 
These shapes are manipulated according to arbitrary but well-detined 
mechanical rules. Actually, the rules mimic the structure of primitive 
systems such as Euclid's geometry. There are formation rules, mechanical 
definitions of which concatenations of shapes are "sentences." One sentence 
is '{((x) (xex}) I (x) (xex)).' There are transformation rules, rules for the 
mechanical derivation of sentences from other sentences. The best known 
trasformation rule is the rule that may be concluded from yand"™y> w" ; 
where '>' is the truth-furctional conditional. For later convenience, I will 
say that y and "y D yw are "impliors," and that y is the "implicand." 
Some sentences are designated as "axioms." A 'proof' is a series of 
sentences such that each is an axiom or an implicand of preceding sentences. 
The last sentence in a proof is a "theorem." 

This account is ultrasimplified and non-rigorous, but it is adequate for 
my purposes. (The reader may have noticed a terminological issue here. For 
Quine, an implication is merely a logically true conditional. The rules which 
are used to go from some statements to others, and to assemble proofs, are 
rules of inference. The relevant rule of inference is the modus ponens; wW is 
tie ponential of pand "yD W7. What I am doing is to use a terminology of 
implication to talk about rules of inference and ponentials. The reason is 
that the use of Quine's terminology would result in extremely awkward 
formulations. What I will be doing is sufficiently transparent that it can be 
translated into Quine's terminology if necessary. My results will be 


86 


unaffected.) The decisive feature of the arbitrary game called "mathematics" 
is as follows. A sentence-series can be mechanically checked to determine 
whether it is a proof. But there is no mechanical method for deciding 
whether a sentence is a theorem. Theorems, or rather their proofs, have to be 
puzzled out, to be discovered. in this feature lies the dynamism, the 
excitement of traditional mathematics. Traditional mathematical ability is 
the ability to make inferential discoveries. 


A variety of branches of mathematics can be specialized out from the 
basic system. Depending on the choices of axioms, systems can be 
constructed which are internally consistent, but conflict with each other. A 
system can be "interpreted," or given a meaning within the language of a 
science such as physics. So interpreted, it may have scientific value, or it may 
not. But as pure mathematics, all the systems have the same arbitrary status. 

By "formalist mathematics' I will mean the present mathematical 
systems which are presented along the above lines. Actually, as many authors 
have observed, the success of the non-Euclidian "tmaginary' geometries 
made recognition of the game-like character of mathematics inevitable. 
Formalism is potentially the greatest break with tradition in the history of 
mathematics. In the Foreward to The Logical Syntax of Language, Carnap 
brilliantly points out that mathematical innovation is still hindered by the 
Widespread opinion that deviations from mathematical tradition must be 
justified--that is, proved to be "correct" and to be a faithful rendering of 
"the true logic." According to Carnap, we are free to choose the rules of a 
mathematical system arbitrarily. The striving after correctness must cease, so 
that mathematics will no longer be hindered. 'Before us lies the boundless 
ocean of unlimited possibilities." In other words, Carnap, the most reputable 
of academicians, says you can do anything in mathematics. Do not worry 
whether whether your arbitrary game corresponds to truth, tradition, or 
reality: it is still legitimate mathematics. Despite this wonderful Principle of 
Tolerance in mathematics, Carnap never ventured beyond the old 
ink-on-paper, axiomatic-deductive structures. I, however, have taken Carnap 
at his word. The result is my "post-formalist mathematics." I want to stress 
that my innovations have been legitimized in advance by one of the most 
reputable academic figures of the twentieth century. 

Early in 1961, I constructed some systems which went beyond 
formalist mathematics in two respects. 1. My sentential elements are 
physically different from the little ink-shapes on paper used in all formalist 
systems. My sentences are physically different from concatenations of 
ink-shapes. My transformation rules have nothing to do with operations on 
ink-shapes. 2. My systems do not necessarily follow the axiomatic-deductive, 
sentence-implication-axiom-proof-theorem structure. Both of these 


87 


possibilities, by the way, are mentioned by Carnap in "Languages as 

Calculi." A "post-formalist system," then, is a formalist system which differs 

physically from  an_ ink-on-paper system, or which lacks the 

axiomatic-deductive structure. 

As a basis for the analysis of post-formalist systems, a list of structural 
properties of formalist systems is desirable. Here is such a list. By 
"Implication" I will mean simple, direct implication, unless I say otherwise. 

1. Asentence can be repeated at will. 

2. The rule of implication refers to elements of sentences: sentences 
are structurally composite. 

A sentence can imply itself. 

4. The repeat of an implior can imply the repeat of an implicand: an 

implication can be repeated. 

Different impliors can imply different implicands. 

6. Given two or three sentences, it is possible to recognize 

mechanically whether one or two directly imply the third. 

No axiom is implied by other, different axioms. 

8. The definition of 'proof' is the standard definition, in terms of 
implication, given early in this essay. 

9. Given the axioms and some other sentence, it is not possible to 
recognize mechanically whether the sentence is a theorem, 
Compound indirect implication is a puzzle. 

Now for the first post-formalist system. 


wo 


a 


~ 


"TY Hlusions" 


A "sentence" is the following page (with the figure on it) so long as the 
apparent, perceived ratio of the length of the vertical line to that 
of the horizontal line (the statement's "associated ratio") does not 
change. (Two sentences are the "same" if end only if their 
associated ratios are the same.) 

A sentence Y is "implied by" a sentence X if and only if Y is the same as X, 
or else Y is, of all the sentences one ever sees, the sentence having 
the associated ratio next smaller than that of X. 

Take as the axiom the first sentence one sees. 

Explanation: The figure is an optical illusion such that the vertical line 
normally appears longer than the horizontal line, even though their 
lengths are equal. One can correct one's perception, come to see 
the vertical line as shorter relative to the horizontal line, decrease 
the associated ratio, by measuring the lines with a ruler to convince 
oneself that the vertical line is not longer than the other, and then 


88 


trying to see the lines as equal in length; constructing similar 
figures with a variety of real (measured) ratios and practicing 
judging these ratios; and so forth. 


"IIlusions" has Properties 1, 3-5, and 7-8. Purely to clarify this fact, the 
following sequence of integers is presented as a model of the order in which 
associated ratios might appear in reality. (The sequence is otherwise totally 
inadequate as a model of "Iilusions.") 4 2 1; 4 2;5421;43 1. The 
implication structure would then be 


4<> 
s—-TXp>s SZ Lo ZN 
4<> 4 oo yD der 1 


The axiom would be 4, and 5 could not appear in a proof. "IIlusions" has 
Property 1 on the basis that one can contro! the associated ratio. Turning to 
Property 4, it is normally the case that when an implication is repeated, a 
given occurrence of one of the sentences involved is unique to a specific 
occurrence of the implication. In "Illusions," however, if two equal 
sentences are next smaller than X, the occurrence of X does not uniquely 
belong to either of the two occurrences of the implication. Compare 'the', 


e 


89 


where the occurrence of 't' is not unique to either occurrence of 'the'. 
Subject to this explanation, "lilusions" has Property 4. "Illusions" has 
Property 8, but it goes without saying thut the type of implication is not 
modus ponens. Properties 3, 5, and 7 need no comment. As for Property 2, 
the rule of implication refers to a property of sentences, rather than to 
elements of sentences. The interesting feature of "IIlusions" is that it 
reverses the situation defined by Properties 6 and 9. Compound indirect 
implication is about the same as simple implication. The only difference is 
the difference between being smaller and being next smalier. And there is 
only one axiom (per person). 

Simple direct implication, however, is subjective and illusive. It 
essentially involves changing one's perceptions of an illusion. The change of 
associated ratios is subjective, elusive, and certainly not numerically 
measurable. Then, the order in which one sees sentences won't always be 
their order in the implications and proofs. And even though one is exposed 
to all the sentences, one may have difficulty distinguishing and remembering 
them in consciousness. If I see the normal illusion, then manage to get 
myself to see the lines as being of equal length, I know I have seen a 
theorem. What is difficult is grasping the steps in between, the simple direct 
implications. If the brain contains a permanent impression of every sensation 
it has received, then the implications objectively exist; but they may not be 
thinkable without neurological techniques for getting at the impressions. In 
any case, "proof" is well-defined in some sense--but proofs may not be 
thinkable. "I!lusions" is, after all, not so much shakier in this respect than 
even simple arithmetic, which contains undecidable sentences and 
indefinable terms. 

In The Logical Syntax of Language, Carnap distinguishes pure syntax 
and descriptive syntax; and says that pure syntax should be independent of 
notation, and that every system should be isomorphic to some ink-on-paper 
system. In so doing, Carnap violates his ov'n Principle of Tolerance. Consider 
the following trivial formalist system. 


"Order" 


A"sentence" is a member of a finite set of integers. 

Sentence Y is "implied by" sentence X it and only if Y=X, or else of all the 
sentences, Y is the one next smaller than X. 

Take as the axiom the largest sentence. 


js the pure syntax of "Iilusions' insomorphic to "Order"? The preceding 
paragraph proved that it is not. The implication structure of "Order" is 


90 


mechanical to the point of idiocy, while the implication structure of 
"Illusions" is, as I pointed out, elusive. The figure 


Axlom 6 eles gt abe eae eek Theorem 


where loops indicate multiple occurances of the same sentence, could 
adequately represent a proof in "Order," but could not remotely represent 
one in "Illusions." The essence of 'Illusions' is that it is coupled to the 
reader's subjectivity. For an ink-on-paper system even to be comparable to 
"IIlusions," the subjectivity would have to be moved out of the reader and 
onto the paper. This is utterly impossible. 

Here is the next system. 


"I nnperseqs" 


Explanation: Consider the rainbow halo which appears to surround a small 
bright light when one looks at it through fogged glass (such as 
eyeglasses which have been breathed on). The halo consists of 
concentric circular bands of color. As the fog evaporates, the halo 
uniformly contracts toward the light. The halo has a vague outer 
ring, which contracts as the halo does. Of concern here is what 
happens on one contracting radius of the halo, and specifically 
what happens on the segment of that radius lying in the vague 
outer ring: the outer segment. 

A "sentence" {or halopoint) is the changing halo color at a fixed point, in 
space, in the halo; until the halo contracts past the point. 

Several sentences "imply" another sentence if and only if, at some instant, 
the several sentences are on an outer segment, and the other 
sentence is the inner endpoint of that outer segment. 

An "axiom" is a sentence which is in the initial vague outer ring (before it 
contracts), and which is not an inner endpoint. 

An "innperseq" is a sequence of sequences of sentences on one radius 
satisfying the following conditions. 1. The members of the first 
sequence are axioms, 2. For each of the other sequences, the first 
member is implied by the non-first members of the preceding 
sequence; and the remaining inembers (if any) are axioms or first 
members of preceding sequences. 3. All first members, of 
sequences other than the last two, appear as non-first members. 4. 
No sentence appears as a non-first member more than once. 5. The 
last sequence has one member. 

In the diagram on the following page, different positions of the vague outer 


91 


Successive bands represent the vague outer ring at successive times as it fades in toward the small bright light. 


ring at different times are suggested by different shadings. The 
outer segment moves "down the page." The figure is by no means 
an innperseq, but is supposed to help explain the definition. 
Innperseqs Diagram 
"Sentences" at 


I time: a1 8 a3 aq ap ag a7 b 
44,89 > bh 


timeg: a9 a3 a4 a5 ag a7 be 
ag —— eee ( 


eS 
SS Sos 
Saas 
SSS 


WS 


U4 yy (VY) Mi Wy i 
Mey ae 
AAA AL ELH i wae aan Se tl 
peeling tatatsegZee 4,45 

EL DAMA ATLL 

Lita, YAP 

VALE RELAY AL 

LAA B68 94622 

VP AO 

WAL ALLL 

RINE, SALA ALIIY 

LAA 6 LAs 

eines: Yihks 


PT I SSS . m 

ar ee oe —— timeg: ag a7 bede 
Feo, Weak rds gis 
ROBES I) Ronee SSS times: ayb ede f 
KT g . 
Snes eaiiens Pa PASE, SS a 

PPT PS SL RISE Os a7,c +f 
CER I PRR 
FS re pa oon a 
ST RS EXER WS gat en gs 
SAE VE er 


"Axioms" ay a9 a3 a4 a5 ag a7 


Innperseq 
(a3,49,a4) 
(b, a3) 

(c, a5, aq) 
(d, b, ag) 
(e,c,a7) 
(f,e, d) 

(g) 


small bright light 


92 


In "Innperseqs," a conventional proof would be redundant unless al! 
the statements were on the same radius. And even if the weakest axiom were 
chosen (the initial outer endpoint), this axiom would imply the initial inner 
endpoint, and from there the theorem could be reached immediately. In 
other words, to use the standard definition of "proof" in "Innperseqs" 
would result in an uninteresting derivation structure. Thus, a more 
interesting derivation structure is defined, the "innperseq." The interest of 
an "innperseq" is to be as elaborate as the many restrictions in its definition 
will allow. Proofs are either disregarded in "Innperseqs"; or else they are 
identified with innpersegs, and lack Property 8. "Innperseqs" makes the 
break with the proof-theorem structure of formalist mathematics. 

Turning to simple implication, an implicand can have many impliors; 
and there is an infinity of axioms, specified by a general condition. The 
system has Property 1 in the sense that a sentence can exist at different 
times and be a member of different implications. It has Property 4 in the 
sense that the sentences in a specific implication can exist at different times, 
and the implication holds as long as the sentences exist. It has Property 3 in 
that an inner endpoint implies itself. The system also has Properties 5 and 7; 
and lacks Property 2. But, as before, Properties 6 and 9 are another matter. 
Given several sentences, it is certainly possible to tell mechanically whether 
one is implied by the others. But when are you given sentences? If one can 
think the sentences, then relating them is easy--but it is difficult to think the 
sentences in the first place, even though they objectively exist. The diagram 
suggests what to look for, but the actual thinking, the actual sentences are 
another matter. As for Property 9, when "theorems" are identified with last 
members of innperseqs, I hesitate to say whether a derivation of a given 
sentence can be constructed mechanically. If a sentence is nearer the center 
than the axioms are, an innperseq can be constructed for it. Or can it? The 
answer is contingent. "Innperseqs" is indeterminate because of the difficulty 
of thinking the sentences, a difficulty which is defined into the system. It is 
the mathematician's capabilities at a particular instant which delimit the 
indeterminacies. Precisely because of the difficulty of thinking sentences, I 
will give several subvariants of the system. 


Indeterminacy 


A "totally determinate innperseq" is an innperseq in which one thinks all the 
sentences. 

An "implior-indeterminate innperseq" is an innperseq in which one thinks 
only each implicand and the outer segment it terminates. 

A "sententially indeterminate innperseq" is an innperseq in which one thinks 


93 


only the outer segment, and its inner endpoint, as it progresses 
inward. 


Let us return to the matter of pure and descriptive syntax. The interest 
of "Illusions" and "Innperseqs" is precisely that their abstract structure 
cannot be separated from their physical and psychological character, and 
thus that they are not isomorphic to any conventional ink-on-paper system. I 
am trying to break through to unheard of, and hopefully significant, modes 
of implication; to define implication structures (and derivation structures) 
beyond the reach of past mathematics. 


1.2 Constructed Memory Systems 

In order to understand this section, it is necessary to be thoroughly 
familiar with "Studies in Constructed Memories," the essay following this 
one. {I have not combined the two essays because their approaches are too 
different.) I will define post-formalist systems in constructed memories, 
beginning with a system in an M*-Memory. 


"Dream Amalgams" 


A "sentence" is a possible method, an Ag. with respect to an M*-Memory. 
I 


The sentence A, "implies" the sentence A, if and only if the agth 


M*-assertion is actually thought; and either A, = Ag.» or else there is 
q p 


cross-method contact of a mental state in Mag with a state in Pa 

The axioms must be chosen from sentences which satisfy two conditions. 
The mental states in the sentences must have cross-method contact 
with mental states in other sentences. And the M*-assertions 
corresponding to the sentences must not be thought. 

Explanation: As "Studies in Constructed Memories" says, there can be 
cross-method contact of states, because a normal dream can 
combine totally different episodes in the dreamer's life into an 
amalgam. 

"Dream Amalgams" has Properties 1-5. For the first time, sentences are 
structurally composite, with mental states being the relevant sentential 
elements. Implication has an unusual character. The traditional type of 
implication, modus ponens, is "directed," because the conditional is 
directed. Even if "yDwW" is true "YDy" may not be. Now implication is also 
directed in 'Dream Amalgams," but for a very different reason. 


94 


Cross-method contact, unlike the conditional, has a symmetric character. 
What prevents implication from being necessarily symmetrical is that the 
implicand's M*-assertion actually has to be thought, while the implior's 
M*-assertion does not. Thus, implication is both subjective and mechanical, 
it is subjective, in that it is a matter of volition which method is remembered 
to have actually: been used. It is mechanical, in that when one is 
remembering, one is automatically aware of the cross-method contacts of 
states in Ag . The conditions on the axioms ensure that they will have 


implications without losing Property 7. 


As for compound implication in "Dream Amalgams," the organism 
with the M*-Memory can't be aware of it at all; because it can't be aware 
that at different times it remembered different methods to be the one 
actually used. (In fact, the organism cannot be aware that the system has 
Property 5, for the same reason.) On the other hand, to an outside observer 
of the M*-Memory, indirect implication is not only thinkable but 
mechanical. It is not superfluous because cross-method contact of mental 
states is not necessarily transitive. The outside observer can decide whether a 
sentence is a theorem by the following mechanical procedure. Check 
whether the sentence's M*-assertion has acually been thought; if so, check ail 
sentences which imply it to see if any are axioms; if not, check all the 
sentences which imply the sentences which imply it to see if any are axioms; 
etc. The number of possible methods is given as finite, so the procedure is 
certain to terminate. Again, an unprecedented mode of implication has been 
defined. 

When a post-formalist system is defined in a constructed memory, the 
discussion and analysis of the system become a consequence of constructed 
memory theory and an extension of it. Constructed memory theory, which 
is quite unusual but still more or less employs deductive inference, is used to 
study post-formalist modes of inference which are anything but deductive. 

To aid in understanding the next system, which involves infalls in a 
D-Memory, here is an 


mn 


"Exercise to be Read Aloud" 
(Read according to a timer, reading the first word at O' O", and prolonging 
and spacing words so that each sentence ends at the time in parentheses after 
it. Do not pause netween sentences.) 


(event) Ail men are mortal. (17°) 

(Sentence; =eventy) The first utterance tasted 17" and ended at 17"; and 
lasted 15" and ended 1" ago. (59") 

(Sp=event3) The second utterance lasted 42" and ended at 59": and 
lasted 50" and ended 2" ago. (1' 31") 


95 


(S3=eventy) The third utterance lasted 32" and ended at 1' 31"; and 
lasted 40" and ended 1" ago. (2' 16") 

Since '32' in $3 is greater than '2' in S9, S9 must say that S4 (=eventg) 

ended 30" after Sy began, or something equally unclear. The duration of Sy 

is greater than the distance into the past to which it refers. This situation is 

not a real infall, but it should give the reader some intuitive notion of an 

infall. 


"Infails" 


A "sentence" is a D-sentence, in a D-Memory such that event) + 4 is the first 
thinking of the jth D-sentence, for all j. 

Two sentences "imply" another if and only if all three are the same; or else 
the three are adjacent {and can be written Sit: S;, Si-1 ), and are such 
that 6 5 = xj44-Xj raat Sy is the implicand. (The function of Sj+4 is to 
give the duration 6,= +1 -%; of Sj. Sj states that event;, the first 
dae' of s? "4, ended ata aitence: Zj inte the past, where zj is smaller 
than s $s own vduretian The diagram indicates the relations.) 


G2: evenby obi: event 3 
occurred in [X5-40° x5 I occurred in ia, Xa 


shia and in IN-25-Y5) N-z.; and inI N- "Ared ya Needl oP 2 


event itd 


events 42 
xs 544] t 
Bi ese *y+4 A542 


"evenby ended 25 ago" "evenly 44 inI 


In this variety of D-Memory, the organism continuously thinks successive 
D-sentences, which are all different, just as the reader of the above exercise 
continuously reads successive and different sentences. Thus, the possibility 
of repeating a sentence depends on the possibility of thinking it while one is 
thinking another sentence--a possibility which may be far-fetched, but which 


96 


is not explicitly excluded by the definition of a "D-Memory." If the 
possibility is granted, then "Infalls" has Properties 1-5. Direct implication is 
completely mechanical; it is subjective only in that the involuntary 
determination of the z; and other aspects of the memory is a 'subjective' 
process of the organism. Compound implication is also mechanical to an 
outside observer of the memory, but if the organism itself is to be aware of 
it, it has to perform fantastic feats of multiple thinking. 

"Dream Amaigams" and "Infalls" are systems constructed with 
imaginary elements, systems whose "notation" is drawn from an imaginary 
object or system. Such systems have no descriptive syntax. Imaginary objects 
were introduced into mathematics, or at least into geometry, by Nicholas 
Lobachevski, and now I am using them as a notation. For these systems to 
be nonisomorphic to any ink-on-paper systems, the mathematician must be 
the organism with the M*-Memory or the D-*Memory. But this means that 
in this case, the mathematics which is nonisomorphic to any ink-on-paper 
system can be performed only in an imaginary mind. 

Now for a different approach. Carnap said that we are free to choose 
the rules of a system arbitrarily. Let us take Carnap literally. I want to 
construct more systems in constructed memories--so why not construct the 
system by a procedure which ensures that constructed memories are 
involved, but which is otherwise arbitrary? Why not suspend the striving 
after "interesting" systems, that last vestige of the striving after 
"correctness," and see what happens? Why not construct the rules of a 
system by a chance procedure? 

To construct a system, we have to fill in the blanks in the following rule 
schema in such a way that grammatically correct sentences result. 


Rule Schema 

A"sentence" isa(n)_ 

Two sentences "imply" a third if and only if the two sentences 
the third. 


I now spread the pages of 'Studies in Constructed Memories" on the floor. 
With eyes closed, I hold a penny over them and drop it. I open my eyes and 
copy down the expressions the penny covers. By repeating this routine, I 
obtain a haphazard series of expressions concerning constructed memories. It 
is with this series that I will fill in the blanks in the rule schema. In the next 
stage, I fill the first (second, third) blank with the ceries of expressions 
preceding the-first (second, third) period in the entire series. 


"Haphazard System" 

A "sentence" is a the duration D-sentences A (@") conclude these 
"*-Reflection," or the future Assumption voluntarily force of 
conviction for conclusion the Situation or by ongoing that this 
system? be given telling between the Situation 1. 

Two sentences "imply" a third if and only if the two sentences is/ was 
contained not have to the acceptance that a certain and malleable 
study what an event involves material specifically mathematics: 
construct accompanies the rest, extra-linguistically image organism 
can fantasy not remembering ® *-Memory, the future interval defined 
in dream the third. 

An "axiom" is a sentence that internally D-sentences, just as the 


"}*-Memory" sentences Ay is A,.. 
1 2 


In the final stage, I cancel the smallest number of words I have to in 
order to make the rules grammatical. 


"Fantasied Amnesia" 

A "sentence" is a duration or the future force of conviction for the Situation 
or this system given Situation 1. 

Two sentences "imply" a third if and only if the two sentences have the 
acceptance that a certain and malleable study extra-linguistically can 
fantasy not remembering the future interval defined in the third. 

An "axiom" is a sentence that internally just sentences ay: 

It becomes clear in thinking about "Fantasied Amnesia' that its 
metametamathematics is dual. Describing the construction of the rules, the 
metamathematics, by a systematic performance, is one thing. Taking the 
finished metamathematics at face value, independently of its origin, and 
studying it in the usual manner, is another. Let us take "Fantasied Amnesia" 
at face value. As one becomes used to its rules, they become somewhat more 
meaningful. I will say that an "interpretation" of a haphazard system is an 
explanation of its rules that makes some sense out of what may seem 
senseless. 'Interpreting' is somewhat like finding the conditions for the 
existence of a constructed memory which seemingly cannot exist. The first 
rule of "Fantasied Amnesia" is a disjunction of three substantives. The 

"Situation" referred to in the second substantive expression is either 

Situation 1 or else an unspecified situation. The third substantive expression 

apparently means 'this system, assuming Situation 1,' and refers to 

"Eantasied Amnesia" itself. The definition of 'sentence' is thus meaningful, 

but very bizarre. The second rule speaks of "the acceptance" as if it were a 

written assent. The rule then speaks of a "malleable study" as "fantasying" 


98 


something. This construction is quite weird, but let us try to accept it. The 
third rule speaks of a sentence that "sentences" (in the legal sense) a possible 
method. So much for the meaning of the rules. 


Turning to the nine properties of formalist systems, the reference to 
"the future interval' in the implication rule of "Fantasied Amnesia" 
indicates that the system has Property 2; and the system can perfectly well 
have Property 8. It does not have Property 6 in any known sense. Certainly 
it does have Property 9. it just might have Property. 1. But as for the other 
four properties, it seems out of the question to decide whether "Fantasied 
Amnesia' has them. For whatever it is worth, "Fantasied Amnesia' is on 
balance incomparable to formalist systems. 

My transformation rule schema has the form of a biconditional, in 
which the right clause is the operative one. If a transformation rule were to 
vary, in such a way that it could be replaced by a constant rule whose right 
clause was the disjunction of the various right clauses for the variable rule, 
then the latter would vary "trivially." 1 will say that a system whose 
transformation rule can vary non-trivially is a "heterodeterminate" system. 
Since 1 have constructed a haphazard metamathematics, why not a 
heterodeterminate metamathematics? Consider a mathematician with an 
M-Memory, such that each Ag. is the consistent use of a different 


transformation rule, a different definition of "imply," for the mathematics 
in which the mathematician is discovering theorems. The consistent use of a 
transformation rule is after all a method--a method for finding the 
commitments premisses make, and for basing conclusions in premisses. When 
the mathematician goes to remember which rule of inference he has actually 
been using, he "chooses" which of the possible methods is remembered to 
have actually been used. This situation amounts to a heterodeterminate 
system. tn fact, the metamathematics cannot even be written out this time; I 
can only describe it metametamathematically in terms of an imaginary 
memory. 

We are now in the realm of mathematical systems which cannot be 
written out, but can only be described metametamathematically. I will 
present a final system of this sort. It is entitled "System Such That No One 
Knows What's Going On." One just has to guess whether this system exists, 
and if it does what it is like. The preceding remark is the 
metametamathematical description, or definition, of the system. 


99 


1.3 Epilogue 

Ever since Carnap's Principle of Tolerance opened the floodgates to 
arbitrariness in mathematics, we have been faced with the prospect of a 
mathematics which is  indistinguishable from  art-for-art's-sake, or 
amusement-for-amusement's-sake. But there is one characteristic which saves 
mathematics from this fate. Mathematics originated by abstraction from 
primitive technology, and is indispensable to science and technology--in 
short, mathematics has scientific applications. The experience of group 
theory has proved, I hope once and for all, the bankruptcy of that narrow 
practicality which would limit mathematics to what can currently be applied 
in science. But now that mathematics is wide open, and anything goes, we 
should be aware more than ever that scientific applicability is the only 
objective value that mathematics has. I would not have set down constructed 
memory theory and the post-formalist systems if I did not believe that they 
could be applied. When and how they will be is another matter. 

And what about the "validity" of formalism? The rise of the formalist 
position is certainly understandable. The formalists had a commendable, 
rationalistic desire to eliminate the metaphysical! problems associated with 
mathematics. Moreover, formalism helped stimulate the development of the 
logic needed in computer technology (and also to stimulate this paper). In 
spite of the productiveness of the formalist position, however, it now seems 
beyond dispute that formalism has failed to achieve its original goals. (My 
pure philosophical writings are the last word on this issue.) Perhaps the main 
lesson to be learned from the history of formalism is that an idea does not 
have to be "true" to be productive. 


Note 
Early versions of "tllusions" and "Innperseqs" appeared in my essay 
"Concept Art," published in An Anthology, ed. La Monte Young, New 
York, 1963. An early, July 1961 version of "System Such That No One 
Knows What's Going On" appeared in dimension 14, Ann Arbor, 1963, 
published by the University of Michigan College of Architecture and Design. 


100 


2. Studies in Constructed Memories 


2.1 Introduction 


The memory of a conscious organism is a phenomenon in which 
interrelations of mind, language, and the rest of reality are especially evident. 
In these studies, I will define some conscious memory-systems, and 
investigate them. The investigation will be mathematical. In fact, the nearest 
precedent for it is perhaps the geometry of Nicholas Lobachevski. 
Non-Euclidian geometry had many founders, but Lobachevski in particular 
spoke of his system as an 'imaginary geometry." Lobachevski's system was, 
so to speak, the physical geometry of an "imaginary," or constructed, space. 
By analogy, my investigation could be called a psychological algebra of 
constructed minds. It is too early to characterize the investigation more 
exactly. Let us just remember Rudoiph Carnap's Principle of Tolerance in 
mathematics: the mathematician is free to construct his system in any way 
he chooses. 

I will begin by introducing a repertory of concepts informally, 
becoming more formal as I go along. Consider ongoing actions, which by 
definition extend through past, present, and future. For example, "1! am 
making the trip from New York to chicago." Consider also past actions 
which have probable consequences in the present. "I have been heating this 
water' (entailing that it isn't frozen now). I will be concerned with such 
actions as these. 

Our language provides for the following assertion: "I am off to the 
country today; I could have been off to the beach; I could not possibly have 
been going to the center of the sun". We distinguish an actual action from a 
possible action; and distinguish both from an action which is materially 
impossible. People insist that there are things they could do, even though 
they don't choose to do them (as opposed to things they couldn't do). What 
distinguishes these possible actions from impossible ones? Rather than 
trying to analyze such everyday notions in terms of the logic of 
counterfactual conditionals, or of modalities, or of probability, I choose to 
take the notions at their face value. My concern is not to philosophize, but 
to assemble concepts with which to define an interesting memory system. 

What is the introspective psychological difference between a thought 
that has the force of a memory, and a thought that has the force of a 
fantasied past, a merely possible past? I am not asking how I know that a 
verbalized memory is true; I! am asking what quality a naive thought has that 
marks it as a memory. Let Alternative E be that I went to an East Side 
restaurant yesterday, and Alternative W be that I went to a West Side one. 
By the "thought of E" I mean mainly the visualization of going into the East 


101 


Side restaurant. My thought of E has the force of memory. It actually 
happened. W is something I could have done. I can imagine I did do W. There 
is nothing present which indicates whether I did E or W. Yet W merely has 
the force of possibility, of fantasy. How do the two thoughts differ? Is the 
thought of E involuntarily more vivid? Is there perhaps an "attitude of 
assertion" involuntarily present in the thought of E? 

Consider the memory that I was almost run down by a truck yesterday: 
! could have been run down, but wasn't. In such a case, the possibility that I 
could have been run down would be more vivid than the actuality that I 
wasn't. (Is it not insanity, when a person is overwhelmed by the fear of a 
merely possible past event? ) My hold on sanity here would be the awareness 
that I am alive and well today. 

In dreams, do we not wholeheartedly "remember" that a misfortune 
has befallen us, and begin to adjust emotionally to it? Then we awake, and 
wholeheartedly remember that the misfortune has not befallen us. The 
thought that had the force of memory in the dream ceases to have that force 
as we awake. We remember the dream, and conclude that it was a fantasy. 
Even more characteristic of dreams, do I not to al! intents and purposes go 
to far places and carry out all sorts of actions in a dream, only to awaken in 
bed? We say that the dream falsifies my present environment, my 
sensations, my actions, memories, the past, my whole world, in a totally 
convincing way. Can a hypnotist produce artificial dreams, that is, can he 
control their content? Can the hypnotist give his subject one false memory 
one moment, and replace it with a contradictory memory the next 
moment? 

I will now = specify a_ situation involving possible actions and 
remembering. 

Situation 7. "! could have been accomplishing G by doing Aa, or by 


doing Aayy ..., or by doing A, ; but I have actually been accomplishing G by 
n 
doing Aas" Here the ongoing actions Age i= 1, ..., 9, a; * a, if ixh, are 


the possible methods of accomplishing G. (The subscripts are supposed to 
indicate that the methods are distinct and countable, but not ordered.) The 
possible methods cannot be combined, let us assume. 

In such a situation, perhaps the thought that I have been doing Aa, 


3 by the 

. . us - > n 

presence of the "attitude of assertion'. Since the possible methods are 

ongoing actions, the thought that I have been doing A,. has logical or 
i 


would be distinguished from similar thoughts about Aan! wy A 


probabie consequences I can check against the present. 
Now Aa, is actual and Aao is not, so that Aa, simply cannot have 


102 


material contact with ay' An actual liquid in Aay could not require a 
a, could have 
1 

with A, would be verbal and gratuitous. Therefore, in order to be possible 
methods, Aan' . 


not require a jar in Aas to contain it. If it did, Aan couldn't be actualized 


possible jar in Pao to contain it. The only "connection" A 
.., A, must be materially separable. A liquid in Aan must 
n 


while Aj, remained only a possibility. 


Enough concepts are now at hand for the studies to begin in earnest. 


2.2 M- Memories 
Definition. Given the sentences 'I have actually been doing A,.', where 
i 


the A,. are non-combinable possible methods as in Situation 1, an 
"M-Memory" is a memory of a conscious organism such that the organism 
can think precisely one of the sentences at a time, and any of the sentences 
has the force of memory. 

This definition refers to language, mind, and the rest of reality in their 
interrelations, but the crucial reference is to a property of certain sentences. 
I have chosen this formulation precisely because of what I want to 
investigate. I want to find the minimal, elegant, extra-linguistic conditions, 
whatever they may be, for the existence of an M-Memory (which is defined 
by a linguistic property). I can say at once that the conditions must enable 
the organism to think the sentences at will, and they must provide that the 
memory is consistent with the organism's present awareness. 

Definition. The "*P-Memory" of a conscious organism is its conscious 
memory of what it did and what happened to it, the past events of its life. I 
want to distinguish here the "personal" memory from the preconscious. 

Definition. An "L-Memory" is a linguistic P-Memory having no 
extra-linguistic component. Of course, the linguistic component has 
extra-linguistic mental associations which give it "meaning"--otherwise the 
memory wouldn't be conscious. But these associations lack the force of a 
mental reliving of the past independent of language. An L-Memory amounts 
to extra-linguistic amnesia. 

Assumption 1.1. With respect to normal human memory, when I forget 
whether I did x, I can't voluntarily give either the thought that I did x, or 
the thought that I didn't do x, the force of memory. I know that I either did 
or didn't do x, but I can create no conviction for either alternative. (An 
introspective observation.) 

Conclusion 1.2. An L-Memory is not sufficient for an M-Memory, even 
in the trivial case that the Aa. are beyond perception (as internal bodily 


103 


processes are). True, there would be no present perceptions to check the 
sentences '! have actually been doing A,." against. True, the L-Memory 
i 


precludes any extra-linguistic memory-"feelings" which would conflict with 
the sentences. But the L-Memory is otherwise normal. And Assumption 1.1 
indicates that normally, either precisely one of a number of mutually 
exclusive possibilities has the force of memory; or else the organism can give 
none of them the force of memory. 

Assumption 1.3.1 cannot, from within a natural dream, choose to swith 
to another dream. {An introspective observation. A "natural" dream is a 
dream involuntarily produced internally during sleep.) 

Conclusion 1.4. An M-Memory could not be produced by natural 
dreaming. It is true that in one dream one sentence could have the force of 
memory, and in another dream a different sentence could. But an M-Memory 
is such that the organism can choose one sentence-memory one moment and 
another the next. See Assumption 1.3. 

Assumption 1.5. Returning to the example of the restaurants, I find 
that months after the event, my thought of E no longer has the force of 
memory. All I remember now is that I used to remember that I did E. I 
remember that I did E indirectly, by remembering that I remembered that ! 
did E. (My memory that I did E is becoming an L-Memory.) The assumption 
is that a memory of one's remembering can indicate, if not imply, that the 
event originally remembered occurred. 

Conclusion 1.6. The following are adequate conditions for the existence 
of an M-Memory. 1. The sentences are the organism's only memory of which 
method he has been using. 2. When the organism thinks 'I have actually been 
doing A,.'. then (he artificially dreams that) he has been doing Ag,-and is 
now doing it. 3. When the dream ends, he does not remember that he 
remembered that "he has been doing A,.," That is, he does not remember 
the dream; and he does not remember that he thought the sentence. These 
conditions would permit the existence of an M-Memory or else a memory 
indistinguishable to all intents and purposes from an M-Memory. 

What I have in mind in Conclusion 1.6 is dreams which are produced 
artificially but otherwise have all the remarkable qualities of natural dreams. 
There would have to be a state of affairs such that the sentence would 
instantly start the dream going. 

So much for the conditions for the existence of an M-Memory. 
Consider now what it is like as a mental experience to have an M-Memory. 
What present or ongoing awareness accompanies an M-Memory? Conclusion 
1.6.2 already told what the remembering is like. For the rest, I will 
informally sketch some conclusions. The organism can extra-linguistically 
image the Aa: The organism can think 'l could have been doing Aa; When 


104 


not remembering, the organism doesn't have to do any Ag., or he can do any 


one of them. The organism must not do anything which would liquidate a 
possble method, render the action no longer possible for him. 

Assumption 2.1. A normal dream can combine two totally different 
past episodes in my life into a fused episode, or amalgam; so that I "relive" it 
without doubts as.a single episode, and yet remain vaguely aware that 
different episodes are present in it. Dreams have the capacity not only to 
falsify my world, but to make the impossible believable. (An introspective 
observation.) 

Conclusion 2.2. The conditions for the existence of an M-Memory 
further permit material contact between the possible methods, the very 
contact which is out of the question in a normal Situation 1. The dream is so 
flexible that the organism can dream that an (actual) fiquid is/was contained 
by a jar in a possible method. See Assumption 2.1. Thus, the A,, do not have 
to be separable to be possible methods. 

I will now introduce further concepts pertaining to the mind. 

Definition. A "mental state" is a mental "stage" or "space" or "mood" 
in which visualizing, remembering, and all imaging can be carried on. 


Some human mental states are stupor, general anxiety, empathy with 
another person, dizziness, general euphoria, clearheadedness (the normal 
state in which work is performed), and dreaming. In all but the last state, 
some simple visualization routine could be carried out voluntarily. Even ina 
dream, I can have visualizations, although here I can't have them at will. The 
states are not defined by the imaging or activities carried on while in them, 
but are "spaces" in which such imaging or activities are carried on. 

By definition. 

Conclusion 3.2. An M-Memory has to occur within the time which the 
possible methods require, the time required to accomplich G. By definition. 

Definition. An "M*-Memory" is an M-Memory satisfying these 
conditions. 1. Agi: for the entire time it requires, involves the voluntary 


assuming of mental states. i = 1, ..., n. 2. The material contact between the 
possible methods, the cross-method contact, is specifically some sort of 
contact between states. 

Conclusion 3.3. For an M*-Memory, to remember is to choose the 
mental state in which the remembering is required to occur (by the 
memory). After ail, for any M-Memory, to remember is to choose all the 


A,.-required things you are doing while you remember. 
i 
By now, the character of this investigation should be clearer. I seek to 


stretch our concepts, rather that to find the "true" ones. The investigation 
may appear similar to the old discipline of philosophical psychology, but its 


105 


thrust is rather toward the modern axiomatic systems. The reasoning is 
loose, but not arbitrary. And the investigation will become increasingly 
mathematical. 


2.3 D-Memories 


Definition. A "D-Memory" is a memory such that measured past time 


appears in it only in the following sentences: "Event; occurred in the interval 


of time which is xX] long and ended at Xj AF, and is Yj long and ended 2; 


ago," where Xj, 
and 'AF' means "after a fixed beginning time." XQ = 'O; XjPX5A and at any 


Yje and zj are positive numbers of time units (such as hours) 


one fixed time, the intervals IZj. zjtyjl nowhere overlap. Vit ZS%- For an 


integer m, the mth sentence acquires the force of memory, is added to the 
memory, at the fixed time x,,.j =1, ..., f(t), where the number of sentences 
f(t) is written as a function of time AF. Then f(t) = m when x,,<t<x,, 44. 
The sentences have the force of memory involuntarily. The organism does 
not make them up at will. ; : : 

Let me explain what the D-Memory involves. Event; is assigned to an 
abnormal "interval," a dual interval defined in two unrelated ways. The 
intervals defined by the Yj and z; are tied to the present instant rather than to 
a fixed time, and could be written IN-2;-Yj, N-zjl, where 'N' means "the time 
of the present instant relative to the fixed beginning time." 

Conclusion 4. The intervals IN-2)-Yj, N-Z;I nowhere overlap. Proof: By 
definition, the intervals IZj, zi+y;I nowhere overlap. If j #k, IZj, Ztyillz,, 
Zz. +y¥pl = 0. This fact implies that e.g. ZjZjtVjZKS ZK +YK- Then 
N-2-¥_SN-2<N-2)-9; <N-Zj. Then IN-2p-y,, N-z, 1 N-2jy 7, N-z)I = ¢. At 
any one time, the organism can think of all the sliding intervals, and they 
partly cover the time up to now without overlapping. 

Suppose you find the deck of n cards 


event j 
Zz i oa" 
(jj = 1, .., n and z, is a positive number of days), and you have no 


J 
information to date them other than what they themselves say. If you 


believe the cards, your mental experience will be a little like having a 
D-Memory. Then, the definition does not require that Yj = Xt Again, it is 


106 


not that two concepts of "length" are involved, but that the "interval" is 
abnormal. Of course this is ali inconsistent, but I want to study the 
conditions under which a mind will accept inconsistency. 

Assumption 5.1. With respect to normal human memory, it is possible 
to forget what day it is, even though one remembers a past date. (An 
empirical observation.) 

Assumption 5.2. This assumption is based on the fact that the sign 
'CLOSED FOR VACATION. BACK IN TWO WEEKS' was in the window of 
a nearby store for at least a month this summer; and the fact that a 
filmmaker wrote in a newspaper, "When an actor asks me when the film will 
be finished, I say 'In two months," and two months later I give the same 
answer, and I'm always right.' Even in normal circumstances, humans can 
maintain a dual and outright inconsistent awareness of measured time. [n 
general, inconsistency is a normal aspect of human thinking and even has 
practical value. 

Imagine a child who has been told to date events by saying, for 
example, x happened two days ago, and a day later saying again, x happened 
two days ago--and who has not been told that this is inconsistent. What 
conditions are required for the acceptance of this dating system? It is 
precisely because of Assumptions 5.1 and 5.2 that a certain answer cannot 
be given to this question. The human mind is so flexible and malleable that 
there is no telling how much inconsistency it can absorb. I can only study 
what flaws might lead the child to reject the system. The child might "fee!" 
that an event recedes into the past, something the memory doesn't express. 
An event might be placed by the memory no later than another, and yet 
"feel" more recent than the other. I speculate that if anything will discredit 
the system, it will be its conflict with naive, "felt," extra-linguistic memory. 

Conclusion 5.3. The above dating system would be acceptable to an 
organism with an L—Memory. 

Conclusion 5.4. The existence of an L-Memory is an adequate condition 
for the existence of a D-Memory. With extra-linguistic amnesia, the 
structure of the language would be the structure of the past in any case. The 
past would have no form independent of language. Anyway, time is gone for 
good, leaving nothing that can be checked directly. Without an 
extra-linguistic memory to fall back on, and considering Assumptions 5.1 
and 5.2, the dual temporal memory shouldn't be too much to absorb. 

As I said, the real difficulty with this line of investigation is putting 
limits on anything so flexible as the mind's capacity to absorb inconsistency. 

Now the thinking of a sentence in a D-Memory itself takes time. Let 
'tS; be the minimum number of time units it takes to think the jth 


D-sentence. This function, abbreviated '8y), is the duration function of the 


107 


D-sentences. 

Conclusion 6.1. If 5j>Z), the memory of the interval defined by Yj and 
Zj places the end of the interval after the beginning of the memory of it, or 
does something else equally unclear. If bj>yjtzj. the entire interval is placed 
after the beginning of the memory of it. When 5;>z;, let us say that the end 
of the remembered interval falis within the interval for the memory of it, or 
that the situation is an "infall." (Compare 'The light went out a half-second 
ago'.) 5 

Conclusion 6.2. If 6}>xj4,-xj, then Sj, is added to the preconscious 
before s can be thought once. The earliest interval during which the jth 
sentence can be thought "passes over" the (j+k)th interval. Let us say that 
the situation is a "passover." (Something of the sort is true of humans, 
whose brains contain permanent impressions of far more sensations than can 
be thought, remembered in consciousness.) 

Conclusion 6.3. If there are passovers in a D-Memory, the organism 
cannot both think the sentences during the earliest intervals possible and be 
aware of the passovers. Proof: The only way the organism can be aware of 6 
(S}) is for event j+h (h a positive integer) to be the thinking of Sj. If the 
thinking of Sj takes piace as the (j+1)th event, then the organism gets two 
values for 5(S)), namely 4h Xj and Yj+1- Assume that only Xj4I%y is 
allowed as a measure of 5(Sj). Since 5(S)) = X44%j, there is no passover. If 
the thinking of S; takes place as the (j+2)th event, then xj4.9-x j44 = 5(S)) 
could be greater than xj1%- But since Sj goes into the preconscious at x;, 
S: is not actually thought in the earliest interval during which it could be 
thought. See the diagram. 


So 4 St Sz+d Sz+2 


event+ I sven? ~ Pee even bs +2 I 
aoa ee aaa ee SS 


a "je "542 


Conclusion 6.4. Let there be an infall in the case where event) is the 
thinking of Sj- 5(S) = X45 and 5(Sj)>z;. Si+1 gives 5(S)), so that the 
organism can be aware of it. It is greater than z;. Thus, the organism can be 


aware of the infall. However, the infall will certainly be no more difficult to 
accept than the other features of the D-Memory. And the thinking of Sj has 


108 


to be one of the events for the organism to be aware of the infall. 


2.7 &-Memories 
I will conclude these studies with two complex constructions. 
Definition. A "&-Memory" is a memory which includes an M*-Memory 
and a D-Memory, with the following conditions. 1. The goal G, for the 
M*-Memory, is to move from one point to another. 2. For the D-Memory, 


"event," becomes a numerical term, the decrease in the organism's distance 


from the destination point during the temporal interval. "A 3-inch move 


toward the destination" is the sort of thing that 'event;' here refers to. 3. 


The number of Aa, equals the number of D-sentences factorial. The number 
of D-sentences, of course, increases. 

Consider the consecutive thinking of each D-sentence precisely once, in 
minimum time, while the number of sentences remains constant. Such a 
"D-paragraph" is a permutation of the D-sentences. Let H™ be a 
D-paragraph when the number of sentances equals the integer m. There are 
m! SA" s. When f(t) = m = 3, one of the sixH" sis sais}, thought in 
minimum time. Assume that the duration A of a D-paragraph depends only 
on the number of D-sentences and the bi. We can write 


The permutations of the D-sentences, as well as the D-paragraphs, can be 
indexed with the a;, just as the possible methods are. 

Definition. A "b*-Memory" is a ®-Memory in which the order of the 
sentences in the ajth Ti" has the meaning of 'I have actually been doing Aa. 


assigned to it. The order is the indication that A,. has actually been used; it 
i 
is the ajth M*-assertion. '! have actually been doing A,.' is merely an English 
i 


translation, and does not appear in the ®*- Memory. 

Conclusion 7. Given a $*-Memory, if one D-sentence is forgotten, not 
only will there be a gap in the awareness of when what events occurred; it 
will be forgotten which method has actually been used. 

This conclusion points toward a study in which deformations of the 
memory language are related to deformations of general consciousness. 

Definition. A "*-Reflection," or reflection in the present of a 
@*-Memory, is a collection of assertions about the future, derived from a 


&*-Memory, as follows. 1. There are the sentences 'Event; will occur in the 


109 


interval of time which is xxi long, and begins at twice the present time 


AF, minus Xj AF; and which is y; long and begins zj from now'. If event; was 
a 3-inch move toward the destination in the ®*-Memory, the sentence in the 
®*-Reflection says that a 3-inch move will be made in the future temporal 
interval. 2. The ajth permutation of the sentences defined in (1) is an 


assertion which has the meaning of 'I will do A,.'; and the organism can 
i 
think precisely one permutation at a time. The A,_, Xj Vir Bye and the rest are 
. . - I . . . 
defined as before (so that in particular the permutations can be indexed with 
the aj). 

Conclusion 8. Given that the @*-Memory's temporal! intervals x54, xj! 
are reflected as I2N-x;, 2N-x; 41, the reflection preserves the intervals' 
absolute distances from the present. Proof: The least distance of X74, xj 
from N is N-x;; the greatest distance is N-Xi 4. Adding the least distance, and 
then the greatest distance, to N, gives I2N-x;, 2N-xj 41. 

I will end with two problems. If a ©*-Memory exists, under what 
conditions will a ®*-Reflection be a precognition? Under what conditions 
will every assertion be prescience or foreknowledge? By a "precognition" I 
don't mean a prediction about the future implied by deterministic laws; I 
mean a direct "memory" of the future unconnected with general principles. 

Finally, what would a precognitive ®*-Reflection be like as a mental 
experience? What present or ongoing awareness would accompany a 
precognitive ®*-Reflection? 


110 


THE NEW MODALITY 


SE eS 
ESE BORO A SILGS 
Sie DLS 

Baie es 


) 
: 
: 
: 
; 


11. Representation of the Memory of an Energy Cube Organism 
1966 VERSION 


The energy cube organism is a conscious organism which is nothing but 
energy confined to a cubical space. It rests on a rectangular energy slab, in a 
stationary, colorless liquid, separated from the slab by a thin film of liquid. 
It has been on the slab for an indefinitely long time. There are in fact two 
infinite bodies of the liquid, alternating with two infinite empty spaces; the 
four volumes are outlined by two intersecting planes which just miss being 
perpendicular. The slab is poised, at a slant, on the faces of the upper body 
of liquid, near where they meet. There are no other objects in the bodies of 
liquid. The slab, liquid, and spaces are the energy cube organism's entire 
cosmology. (See the illustration.) 


Pay 
'€ 
= 
£ 
GS 
= 
£ 
Pag 
® 
Oe aa eae 
Sek Vn a ee a a , of 
\ / : 
liquid / 
\ / Lo 
/ 
vee: fof} ae 
\y oy 'i \/ empty space 
yy 
empty space /\ / 
voy \ 
/ / 

; . i \ 
/ aa / \ 
liquid 
/ Be eee ee 
i -7~ ae 

te pe 


Ajluljul 02 spuarxa 


ILLUSTRATION 


113 


The energy cube organism can continuously change position, 
continuously and instantly moving the liquid from its path into its wake so 
as to make no current in the liquid. For almost as long as it has been on the 
slab, the organism has devoted itself to crossing the slab, from the slab's edge 
on one face of the liquid to its edge on the other. 

The energy cube organism has a conscious memory (by which I mean 
strictly a memory of what it did and what happened to it, the past events of 
its existence). The memory consists of symbols which are given "meaning" 
by their extra-linguistic mental! associations--in human terms, it consists of 
language. The complete memory contains tens of thousands of partial 
memories, which the organism can only have one at a time. Going through 
the partials--which it does as if they were the phonemes of one long 
word--constitutes its one complete memory. Each partial is a memory of the 
difference in the organism's minimum distances from the destination edge, at 
the beginning, and at the end, of some interval of time. Call the difference its 
"progress." The total of time intervals in all the partials completely covers 
the interval from the earliest remembered event to the most recent 
remembered event. As time passes, more partials are added to the complete 
memory. The production of partial memories is an involuntary process of 
the organism. 

The memory is temporally dual. The interval for each partial is an 
interval of fixed time, defined by its duration, and the distance from the 
fixed time when the energy cube organism appeared on the slab up to the 
interval's end. But it is also a sliding interval, defined by its duration, and a 
constant distance from the present instant back to the interval's end. When 
partials are added to the memory, each of the former intervals exactly covers 
the tire not already covered, up to the absolute time when the partial is 
added. But the latter intervals, while they never overlap, can have gaps 
between them. The intervals generally are of different durations. The energy 
cube organism lacks any independent extra-linguistic memory, any mental 
reliving of the past, which could conflict with the dual temporal memory. 
There is no form to the past other than that of the memory's language. (See 
the graph.) 

The order of the partials in the complete memory is a linguistic 
phenomenon which indicates the method the organism has been using to 
move itself--and thus the order (with its extra-linguistic associations) is the 
memory of the method. A single method" is everything to be done by the 
energy cube organism to move itself, throughout the entire time it takes to 
reach the destination edge. There are different possible methods, and each 
could get the organism across; but the methods cannot be combined in any 
way. Every order of all partials signifies a different possible method. These 


114 


ao) 
14 
2 
oS 
- 
— 
c 

Qa 
£ 
2 
£ 
= 
- 
© 
@ 
£ 
- 
@ 
2 
2 
° 
nn 
O 
o 


a 
®D 
ao) 
oo 
ros) 
_ 
2 
© 
> 
< 
@ 
~ 
= 
nn 
a] 
£ 
rw) 


used to show intervals, 1st temporal memory omenrasmemiia 
used to show "intervals," 2nd temporal memory ............ 
used to show tracks of "intervals," 2nd memory 

usedtoshowrelationships eae ae 


4th partial 


I 
I 


I 

3rd partial 
, 2nd partial 
: : 

I I 
I I 


n 
l i 
I 
I I 


absolute times, covered by intervals 
{absolute times covered by) intervals, 
Ist temporal memory 


GRAPH showing a possible relationship 
in the dual temporal memory 


115 


possible methods are in no special order. When a partial is added to the 
memory, the number of possible methods is increased by a factor equal to 
the new number of partials. 


Now the complete memory is obtained by going through the partials--in 
any order! Any order gives the memory. This feature, which can be 
precisely characterized in terms of the memory language, is perhaps the most 
remarkable feature of the whole cosmology. An approach to this feature in 
human terms is to say that when the organism goes through the partials, (it 
dreams that) it has been using the method indicated--and is presently using 
it. It (does not remember the dream, and) does not remember going through 
the partials. It has no other memory of which method it has been using. 

The organism moves itself by mental exertion, teleports itself. The 
"possible methods" are mental routines. These routines draw on the 
following standard mental resources. The organism can assume at will many 
"mental states." By 'mental state' I refer to a mental "stage" or "space" or 
"mood" in which visualizing, remembering, and all imaging can be carried 
on. Some human mental states are general euphoria, stupor, general anxiety, 
dreaming, dizziness, empathy with another person, and clearheadedness, the 
normal state in which work is performed. These states are not defined by 
specific imagings, but are "spaces" in which imaging is carried on. The 
organism changes its state by changing from one form of energy to another, 
gravity, magnetism, electric energy, radiated heat, or light. In these states, 
the organism has an unlimited capacity to image; in human terms, to 
visualize. There are visualized regions of colored liquids. Call them "fluid 
colors." There are visualized glowing surfaces, and there are black regions or 
"holes." There are visualized "covers," "lattices," and "shells," which are all 
formed from transparent planes, spherical surfaces and the like. Call them 
"orojected surfaces." The fluid colors can be stationary or flowing. There are 
"channels," which are strung-out series of fluid colors. There are 
"reservoirs," which are clusters of fluid colors. A channel can be closed or 
Open. Two channels can cross each other. There are pairs of channels such 
that earlier members of each channel flow into later members of the 
other--calied "screw-connected" channels. Fluid colors often occur on or 
within projected surfaces. Projected surfaces can be growing or held. A 
visualization can be at the forefront of attention, or in the back of the mind. 
That is, states have depth, and visualizations can be at different depths. The 
state as a whole can be "frozen" or "melted." A human approach is to say 
that a "frozen" state is set or fixed; while a "melted" state is fluid--the state 
itself flows. A state can be projected into "superstate," gaining an abnormal 
amount of mental! energy and becoming superdizziness or superanxiety, for 
instance. 


116 


Most interesting, states in different possible methods can have contact 
with each other. A human approach is to say that dreams are so flexible that 
the organism can dream that an actual! state is/was in contact with a state in 
a possible method. One sort of cross-method contact is for states to be 
'Snterfrozen" --more easily frozen because they are somehow mixed. They 
can also be "intermelted." 

I will describe a method, as the organism would be conscious of it in 
remembering. For concreteness, I will refer to the different states with the 
names of human states rather than with letters. Channels are generated in a 
frozen stupor, and become fixed at the forefront of attention of euphoria 
intermelted with a possible state. The screw-crossed channels erode crevices 
in a held lattice, which breaks into growing sheets (a variety of covers). The 
sheets are stacked, and held in a frozen dream thawed at intervals for 
reshuffling of the stack. The dream becomes melted, and proceeds in a 
trajectory which shears, and closes, open channels. If no violation of the 
channels cross-mars the melt, the stack meshes with the sharp-open channels. 
The dream becomes interfrozen, and mixed clear-headed states compress the 
closed channels which were not fixed at the dream's surface. A fused 
exterior double-flash (a certain maximally 'glowing surface") is 
expand-enveloped by euphoria, which becomes dizziness; and oblique 
lattices are projected from the paralinear deviation of guided open channels 
in it. Growing shells are dreamed into violet sound-slices (certain synesthetic 
"fluid colors') by the needed jumped drag (a generic state}, a crossfrozen 
dream. Channels in a growing anxiety enspiral concentric shells having 
intermixed reservoirs between them, during cyclic intersection of the anxiety 
in superstate. And on and on. Time is here the time it takes to carry out the 
successive steps of the routine. 

The energy cube organism language, the symbols constituting the 
partials, are themselves mental entities. A partial is a rectangular plane 
glowing surface, which has two stationary plane reservoirs on it, and has a 
triangular hole in it. As a mental entity, in other words, a partial is a 
visualization like those which are part of the methods. The perimeter of the 
triangular hole equals the organism's progress in the corresponding time 
interval. Absence of the hole indicates zero progress. 

The fluid colors in each of the reservoirs on each partial memory are 
primary colors, and are mixed together. Speaking as accurately as possible in 
human terms, in each reservoir there is precisely one point of "maximum 
mixture" of the primary colors. The primary colors are mentally mixed in 
any way until the right amount of mixture is reached. There is a scale of 
measurement for amounts of mixture of the colors. There is a scale for 
vertical distances on the surface--for how far one point is below another. The 


difference in amounts of mixture at the two points of maximum mixture 
corresponds to the lengti; of the first temporal interval; and the difference 
between the>maximum possible amount of mixture and the lesser of the two 
amounts of maximum mixture on the surface corresponds to the distance 
from the fixed beginning time to the interval's and. The vertical distance 
between the two points of maximum mixture corresponds to the length of 
the second temporal interval; and the vertical distance from the middle of 
the surface to the point nearer it corresponds to the constant distance from 
the present instant back to the interval's enc. The middle of the surface 
represents the present, and the upper half represents the future; the 
reservoirs are all in the lower half. For each partial it is necessary to 
determine (1) the number of units of duration per unit difference in 
amounts of mixture; and (2) the number of units of duration per unit 
difference in vertical distances. The average glow per unit area of each 
glowing surface (excepting the hole) is correlated with a pair of numbers 
constituting this information. 

Finally, turning all the partial memories upside down--and reflecting the 
first temporal memory in the present instant, so that the intervals' absolute 
distances from the present are preserved--gives the precognition of the 
organism's future course of action, tells what progress will be made when 
and by which method. 


The Representation 

This essay accompanies a representation of the energy cube organism's 
memory--hence its title. The way to picture the memory, naturally, is to 
make something that looks like the partials. I have represented the partials 
by rectangular sheets of paper of different translucencies with mixtures of 
inks of primary colors on them and holes cut in them; together in an 
envelope, which bears the injunction not to have more than one sheet out at 
a time. Three of the tens of thousands of partials are represented. 


118 


ORIGINAL 1961 VERSION 


Foreward 

I have refrained from editing the Original Version except where 
absolutely necessary. It is full of inconsistencies and inadequate 
explanations, but I have flagged only two major ones, by placing them 
between the signs X and lX Part of the fourth paragraph is flagged because a 
sequence of units is not analogous to a sequence of inflected words; it is 
rather more like permutations of letters which form words ('rat', 'tar', 'art'). 
Most of the seventh paragraph is flagged because I promise to define intervals 
by their lengths and ends, but instead give their beginnings and ends. 

In the fourth paragraph, there are two different versions of the 
correspondence between possible methods and sequences of units, and of 
why any sequence is acceptable. Passages belonging exclusively to the 
"multiplex" version are set off by the sign #. Passages which belong 
exclusively to the "style" version and which should be deleted if the 
"multiplex" version is used are placed between slashes. The "style" version is 
the main version. In the fifth paragraph, a notion appears which is 
interesting, but unconvincingly explained. It is not clear whether this notion 
relates only to the "multiplex" version, or whether it would relate to the 
"style" version if the word 'multiplex' were omitted. The passages suggesting 
this notion are placed in brackets. 


1. Energy cube organisms are conscious organisms which are cubical 
spaces containing only energy. The particular energy cube organism of 
concern here has, for an indefinitely long time, been in a body of liquid, 
"resting on' a rectangular energy slab also in the body of liquid; the 
organism's "bottom" face is separated from the slab by only a very thin film 
of the liquid. The "universe" the organism and slab are in is made up of four 
infinite triangular right prisms, prismatic spaces, as defined geometrically by 
two intersecting planes almost perpendicular to each other. The prismatic 
spaces defined by the vertical obtuse dihedral angles are empty. The other 
spaces, defined by the vertical acute dihedral! angles, are infinite bodies of a 
stationary, colorless lfiquid--the "upper" body of liquid being what the 
organism and slab are in. The two opposite shorter edges of the slab are at 
the faces of the body of liquid, the planes, near their intersection; the slab is 
"slanted," so that the edges are at slightly different distances from the line 
of intersection. The organism and slab are the only "objects" in the bodies 
of liquid. (See the illustration.) The organism can move (the energy cube can 


119 


continuously change position) without creating currents in the liquid. For 
almost as 'ong as it has been in the liquid, the organism has devoted all its 
"intelligence," all its "energies," to moving across the slab, from one of the 
shorter edges to (any point on) the other. 

Z The organism's conscious, distinct memory is entirely concerned 
with, is entirely cf, its efforts to cross the slab. (1 am using 'memory' 
narrowly to refer to an organism's memory of its past. I am counting its 
"general information," for example Knowing a language, not as part of its 
memory but as imagings not memories. Thinking the sequence 1, 2, 1, 2 is 
not in itself remembering.) The total memory consists of a large number of 
units (tens of thousands), of which the organism can be attentive to precisely 
one at a time. 'Total recall," the total memory, involves considering, having, 
all units in any succession, which the organism can do very rapidly. Now 
from one point of view, the memory consists of its content; from another, it 
consists of symbols, just as human memories often consist of language. In 
describing the memory, I will go from considering primarily the content, 
what the memory is of; to considering the specific character of the units, 
specific symbolism used in the memory, and specific content. Each unit is 
first a memory of the amount of progress made toward the destination edge 
in a particular interval of time. The amount of progress is the difference 
between the minimum distance of the organism from the destination edge at 
the beginning of the interval, and the minimum distance at the end of the 
interval. The total of intervals, in the total of units, cover the "absolute" 
interval of time from the earliest to the most recent remembered event; as 
time passes, more units are added to the memory. 

3. Now the memory is temporally dual: the interval of time for each 
unit is first, an interval of 'absolute' time; defined by its duration, and the 
"absolute" time of its end (stated with respect to an "absolute event" such 
as the appearance of the organism on the slab); and secondly, an interval 
defined by its duration, and how far from the present instant its end is. It is 
like remembering that so much progress was made during one year which 
ended at January 1, 1000 A.D.; as well as remembering that it was made 
during one year which ended 1,000 years ago. In the second temporal 
memory, the absolute time of the end of the interval to which the progress is 
assigned changes according as the absolute time of the present instant 
changes. For example, it is like remembering "that so much progress was 
made during one year ending 1,000 years ago," and, 100 years later, 
remembering--'that so much progress was made during one year ending 
1,000 years ago"; and in general, always remembering "that so much 
progress was made during one year ending 1,000 years ago.' Both temporal 
memories are in their own ways "natural," the first being anchored at an 


120 


"absolute beginning," the second at the present instant. When a unit is added 
to the memory, the interval of time of the first temporal memory is added at 
the end, exactly covers the time not already covered, up to the absolute time 
when the unit is added; so that the total of intervals of the first temporal 
memory exactly cover, without overlap, the absolute total time. In contrast, 
although the intervals of the second temporal memory do not overlap at any 
time, there can be gaps between them; so that when a unit is added to the 
memory, the interval for the second temporal memory may be placed 
between existing intervals and not have to cover an absolute time which they 
have left behind, that is, not have to be placed farther back than all of them. 
Intervals of both temporal memories are of different sizes, a "natural 
complexity." (See the graph.) Incidentally, the condition for coincidence of 
the two temporal intervals of a unit is: if the two intervals are of the same 
duration, they will coincide at the absolute time which is the sum of the 
absolute time of the end of the first interval, and the distance from the 
present instant of the end of the second interval. The two temporal 
memories complement each other; aside from this comment I will not be 
concerned to "explain" the duality with respect to when the amounts of 
progress were made, whether when they were "really" made stayed the same 
and changed, or whether the memory is inconsistent about it, or what. 

4. I will now turn to the aspect of the memory concerned with the 
method the organism has used to move itself. # Methodologically, the 
memory is a multiplex symbol.# A "single method" is everything to be done 
by the organism, to move itself, throughout the total time it takes to reach 
the destination edge; so that the organism could not use two different 
"single methods," must, after it chooses its method, continue with it alone 
throughout. The organism has available different (single) methods, has 
different methods it could try. The different sequences, of all units, are 
assigned to the different (single) methods available to the organism to signify 
them; are symbols for them. (Thus, the number of available methods 
increases as units are added to the memory.) /Now ail this only approximates 
what is the case, because contrary to what I may have implied, which 
method is used is not a matter of "fact" as are the temporal intervals and 
amounts of progress. As I have said, having all units in any succession 
constitutes the total memory, total recall ('factually")--different sequences 
of all units are each the total memory, total recall, << but, as language, the 
total memory in different styles (like words in different orders in a highly 
inflected language); and the matter of method (which might better be said to 
be "manner") corresponds to the matter of style, rather than factual 
content, of language. Different styles exclude each other, but not what is 
said in each other's being true. Thus it is that the number of available 


121 


methods can increase; and that any sequence of all units can constitute the 
total memory, total recall ("factually"), although different sequences signify 
different methods used./ #As an indicator of the method used, the whole 
memory is a multiplex symbol. Names for each of the methods are combined 
in a single symbol, the totality of units. In remembering, the organism 
separates any single name by going through ail the units in succession, and 
that name is the complete reading of the multiplex symbol, the complete 
information about the method used. I will not be concerned to "explain" 
the matter of the increasing number of available methods; or the matter of 
any sequence of all units' constituting the complete reading, the total 
memory, total recall, but different sequences' signifying different methods 
used.# 

5. I will give just an indication of what the available methods [and 
their relations through the multiplex memory] are like. Throughout this 
description, there has been the difficulty that English lacks a vocabulary 
appropriate for describing the "universe" I am concerned with, but the 
difficulty is particularly great here, in the case of the methods [and their 
relations through the multiplex memory]; so that I will just have to 
approximate a vocabulary with present English as best as I can. The 
methods, instruments of autokinesis, are all mental, teleportation, resu!t in 
teleportation. The "consciousnesses" available to the organism to be 
combined into methods are infinitely many. It has available many states of 
mind (as humans have non-consciousness, autohypnotic trance, dizziness, 
dreaming, clear-headed calculation, and so forth), corresponding to different 
forms its energy can assume. To give this description more content I will 
differentiate its states of mind by referring to them with the names of the 
human states of mind (rather than just with letters). It has available an 
indefinite variety of contents, as humans have particular imagings, in its 
conscious states of mind. I will outline the principal contents. There are 
"visualized" fluid regions of color (like colored liquids), first-order contents. 
There are 'visualized' radient surfaces, and non-radient surfaces or regions 
("holes"}, the intermediate contents. The second-order contents are 
"projective" constructs of imaged geometric surfaces, "covers," "lattices," 
and "shells." Fluid colors can be stationary or flowing. They can occur in 
certain series, "channels"; and in certain arrays, "reservoirs." A channel can 
be "closed" or "open"; two channels can be "crossed," or 
"screw-connected" (earlier members of each channel flowing into later 
members of the other). First-order contents (fluid colors) often occur on or 
within second-order ones (projective surfaces). Second-order contents can be 
"held" or "growing." States of mind have depth, 'deeper' being 'farther from 
the forefront of attention'; and contents can be at different depths. A state 


122 


of mind as a unity can be "frozen," which is more than just unchanging (in 
particular having its contents stationary or held). It can be projected into 
"superstate," remaining a state of mind but being superenergized. [Most 
interesting, states of mind, in different methods signified by different 
symbols combined in the multiplex methodological memory, can have 
contact with each other, for example be "interfrozen."I] A partial description 
of a method will give an idea of the complexity of the methods. Channels are 
generated by a frozen non-conscious state, and become fixed in the surface 
layer of an [inter] melted trance. The screw-crossed channels erode crevices 
in a held shell, which breaks into growing sheets (certain covers). The sheets 
are stacked, and held in a frozen dream thawed at intervals for reshuffling. 
The dream becomes melted, and proceeds in a trajectory which shears, and 
closes, open channels. If no violation of the channels cross-mars the melt, the 
stack meshes with the sharp-open channels. The dream becomes [inter] 
frozen, and mixed calculation states compress the closed channels which 
were not surface-fixed in it. A fused exterior double-flash {a certain 
maximally radient surface) is expand-enveloped by a trance, which becomes 
dizziness; and oblique lattices are projected from the paralinear deviation of 
guided open channels in it. Growing shells are dreamed into violet 
sound-slices (certain fluid colors) by the needed jumped drag (a certain 
consciousness), a [cross] frozen dream. Channels in a growing trance enspiral 
concentric shells having intermixed reservoirs between them, during cyclic 
intersection of the trance in superstate. I will not say more about the 
available methods, because in a sense the memory does not: a sequence of 
units is a marker arbitrarily assigned to a method to signify it, like an 
arbitrary letter, say 'q', assigned to a certain table to signify it; it no more 
gives characteristics of the method than 'q' does of the table. In fact, the 
available methods and sequences do not have any particular order; one 
cannot speak of the "first" method, the "second," or the like. 


6. I will now concentrate on the character of the memory as a mental 
entity, and the rest of the symbolism used in it and specific content. A unit 
is a rectangular plane ("visualized") radient surface (! --the terminology is 
that introduced in the last paragraph), which has two stationary plane 
reservoirs {! ) on it, and has a triangular hole (! ) in it. The triangular hole is 
a simple symboi not yet explained: its perimeter equals the amount of the 
organism's progress, the difference in its minimum distances from the 
destination edge, in the interval the unit is concerned with. Absence of the 
hole indicates zero perimeter and no progress. 

7. As for the symbols for the temporal interval. The colors in each of 
the two reservoirs on each unit are primary, and are mixed together. 
Speaking as accurately as possible in English, in each reservoir there is 


123 


precisely one point of "maximum mixture' of the primary colors. (The rest 
of the reservoirs are not significant: the primary colors are mentally mixed in 
any way to get the right amount of mixture, as pigments are mixed on a 
palette.) X_ For the first temporal memory, these points are two points on a 
scale of amounts of color mixture. For the second memory, the points are 
two points on a scale of vertical distances from the imaginary horizontal! line 
which bisects the rectangular surface, divides it into lower and upper halves. 
The units are marked in their lower halves only; because for the second 
memory the imaginary dividing line represents the present instant, distances 
below it represent distances into the past, and distances above it distances 
into the future (lower and upper edges representing equal distances from the 
present). Now a scale is required so that it can be told what temporal 
intervals the interval on the amount of mixture scale and the interval on the 
distance scale represent. The parts of the scale which may vary from unit to 
unit and have to be specified in each unit are the "absolute" time 
corresponding to the maximum possible color mixture, the number of units 
of absolute duration per unit difference in amounts of mixture, and the 
number of units of absolute duration per unit difference in distances from 
the imaginary dividing line. The markers arbitrarily assigned to the triples of 
information giving these parts of the scale are average radiences per unit 
areas of the units (excepting the holes). —X 

8. A final aspect of interest. Not too surprisingly, the transformation 
which is inverting all units gives, if one considers not the first temporal 
memory but its reflection in the present instant, the organism's precognized 
course of action in the future, specifically, what progress will be made when. 


The Representation 

With this background, it is not surprising that the method of 
representation I have chosen is visual representation of the units, the 
"visualizations." Units are represented by rectangular sheets of paper of 
different translucencies with mixtures of inks of primary colors on them and 
holes cut in them, together in an envelope. Only one sheet should be out of 
the envelope at a time. A sheet should be viewed while placed before a white 
light in front of a black background, so that the light illuminates the whole 
sheet as evenly as possible without being seen through the hole, only the 
black being seen at the hole. The ultimate in fidelity would be to learn to 
visualize these sheets as they look when viewed properly; then one could 
have the memory as nearly as possible as the organism does. I have 
represented eleven of the tens of thousands of units in the total memory. 


Concept Art 
Copyright 1961 by Henry A.Flynt, Jr. 


Concept art is first of all an art of which the material is concepts, as the 
material of e.g. music is sound. Since concepts are closely bound up with 
language, concept art is a kind of art of which the material is language. That 
is, unlike e.g. a work of music, in which the music proper (as opposed to 
notation, analysis, etc.) is just sound, concept art proper will involve 
language. From the philosophy of language, we learn that a concept may as 
well be thought of as the intension of a name; this is the relation between 
concepts and language.* The notion of a concept is a vestige of the notion of 
a platonic form (the thing which e.g. all tables have in common: tableness), 
which notion is replaced by the notion of a name objectively, metaphysically 
related to its intension (so that all tables now have in common their 
objective relation to table). Now the claim that there can be an objective 
relation between a name and its intension is wrong, and (the word) concept, 
as commonly used now, can be discredited (see my book, Philosophy 
Proper). If, however, it is enough for one that there be a subjective relation 
between a name and its intension, namely the unhesitant decision as to the 
way one wants to use the name, the unhesitant decisions to affirm the names 
of some things but not others, then concept is valid language, and concept 
art has a philosophically valid basis. 

Now what is artistic, aesthetic, about a work which is a body of 
concepts? This question can best be answered by telling where concept art 
came from; I developed it in an attempt to straighten out certain traditional 
activities generally regarded as aesthetic. The first of these is structure art, 
music, visual art, etc., in which the important thing is "structure." My 
definitive discussion of structure art is in my unpublished essay Structure 
Art and Pure Mathematics; here I will just summarize that discussion. Much 
structure art is a vestige of the time when e.g. music was believed to be 
knowledge, a science, which had important things to say in astronomy etc. 
Contemporary structure artists, on the other hand, tend to claim the kind of 
cognitive value for their art that conventional contemporary mathematicians 


* The extension of the word 'table' is all existing tables; the intension of 
'table' is all possible instances of a table. 


125 


claim for mathematics. Modern examples of structure art are the fugue and 
total serial music. These examples illustrate the important division of 
structure art into two kinds according to how the structure is appreciated. In 
the case of a fugue, one is aware of its structure in listening to it; one 
imposes relationships, a categorization (hopefully that intended by the 
composer)on the sounds while listening to them, that is, has an (associated) 
artistic structure experience. In the case of total serial music, the structure is 
such that this cannot be done; one just has to read an analysis of the 
music, definition of the relationships. Now there are two things wrong with 
structure art. First, its cognitive pretensions are utterly wrong. Secondly, by 
trying to be music or whatever (which has nothing to do with knowledge), 
and knowledge represented by structure, structure art both fails, is 
completely boring, as music, and doesn't begin to explore the aesthetic 
possibilities structure can have when freed from trying to be music or 
whatever.The first step in straightening out e.g. structure music is to stop 
calling it music, and start saying that the sound is used only to carry the 
structure and that the real point is the structure--and then you will see how 
limited, impoverished, the structure is. Incidentally, anyone who says that 
works of structure music do occasionally have musical value just doesn't 
know how good real music (the Goli Dance of the Baoule; Cans on Windows 
by La Monte Young; the contemporary American hit song Sweets for My 
Sweets, by the Drifters) can get. When you make the change, then since 
structures are concepts, you have concept art. Incidentally, there is another, 
less important kind of art which when straightened out becomes concept art: 
art involving play with the concepts of the art such as, in music, the score, 
performer-vs. listener, playing a work. The second criticism of structure art 
applies, with the necessary changes, to this art. 

The second main antecedent of structure art is mathematics. This is the 
result of my revolution in mathematics, presented in my 1966 Mathematical 
Studies; here I will only summarize. The revolution occured first because for 
reasons of taste I wanted to deemphasize discovery in mathematics, 
mathematics as discovering theorems and proofs. I wasn't good at such 
discovery, and it bored me. The first way I thought of to de-emphasize 
discovery came not later than Summer, 1960; it was that since the value of 
pure mathematics is now regarded as aesthetic rather than cognitive, why not 
try to make up aesthetic theorems, without considering whether they are 
true. The second way, which came at about the same time, was to find, as a 
philosopher, that the conventional claim that theorems and proofs are 
discovered is wrong, for the same reason I have already given that 'concept' 
can be discredited. The third way, which came in the fall-winter of 1960, 
was to work in unexplored regions of formalist mathematics. The resulting 


126 


mathematics still had statements, theorems, proofs, but the latter weren't 
discovered in the way they traditionally were. Now exploration of the wider 
possibilities of mathematics as revolutionized by me tends to lead beyond 
what it makes sense to call mathematics; the category of mathematics, a 
vestige of Platonism, is an unnatural, bad one. My work in mathematics leads 
to the new category of concept art, of which straightened out traditional 
mathematics (mathematics as discovery) is an untypical, smal! but 
intensively developed part. 

I can now return to the question of why concept art is art. Why isn't it an 
absolutely new, or at least a non-artistic, non-aesthetic activity? The answer 
is that the antecedents of concept art are commonly regarded as artistic, 
aesthetic activities; on a deeper level, interesting concepts, concepts 
enjoyable in themselves, especially as they occur in mathematics, are 
commonly said to have beauty. By calling my activity art, therefore, I am 
simply recognizing this common usage, and the origin of the activity in 
structure art and mathematics. However: it is confusing to call things as 
irrelevant as the emotional enjoyment of (rea!) music, and the intellectual 
enjoyment of concepts, the same kind of enjoyment. Since concept art 
includes almost everything ever said to be music, at least, which is not music 
for the emotions, perhaps it would be better to restrict art to apply to art for 
the emotions, and recognize my activity as an independent, new activity, 
irrelevant to art (and knowledge). 


Concept Art Version of Mathematics System 3/26/61 (6/19/61) 

An element is the adjacent area (with the figure in it) so long as the 
apparent, perceived, ratio of the length of the vertical line to that of the 
horizontal line (the element's associated ratio) does not change. 

A selection sequence is a sequence of elements of which the first is the one 
having the greatest associated ratio, and each of the others has the associated 
ratio next smaller than that of the preceding one. (To decrease the ratio, 
come to see the vertical line as shorter, relative to the horizontal line, one 
might try measuring the lines with a ruler to convince oneself that the 
vertical one is not longer than the other, and then trying to see the lines as 
equal in length; constructing similar figures with a variety of real (measured) 
ratios and practicing judging these ratios; and so forth.) 

[Observe that the order of elements in a selection sequence may not be the 
order in which one sees them. ] 


127 


Implications--Concept Art Version of Colored Sheet Music No.1 3/14/61 
(10/11/61) 

[This is a mathematical system without general concepts of statement, 
implication, axiom, and proof. Instead, you make the object, and stipulate 
by ostension that it is an axiom, theorem, or whatever. My thesis is that 
since there is no objective relation between name and intension, all 
mathematics is this arbitrary. Originally, the successive statements, or sheets, 
were to be played on an optical audiorecorder. I 

The axiom: a sheet of cheap, thin white typewriter paper 

The axiom implies statement 2: soak the axiom in inflammable liquid which 
does not leave solid residue when burned; then burn it on horizontal 
rectangular white fireproof surface--statement 2 is ashes (on surface) 
Statement 2 implies s.3: make black and white photograph of s.2 in white 
light (image of ashes' rectangle with respect to white surface (that is, of the 
region (of surface, with the ashes on it) with bounding edges parallel to the 
edges of the surface and intersecting the four points in the ashes nearest the 
four edges of the surface) must exactly cover the film); develop film-- s. 3 is 
the negative 

$.2 and s.3 imply s.4: melt s.3 and cool in mold to form plastic doubly 
convex lens with small curvature; take color photograph of ashes' rectangle 
in yellow light using this lens; develop film-- s. 4 is color negative 

$.2 and s.4 imply s.5: repeat last step with s.4 (instead of 3), using red 
light-- s. 5 is second color negative 

S.2 and s.5 imply s.6: repeat last step with s.5, using blue light-- s. 6 is third 
color negative 

$.2 and s.6 imply s.7: make lens from s.6 mixed with the ashes which have 
been being photographed; make black and white photograph, in white fight, 
of that part of the white surface where the ashes' rectangle was; develop film 


128 


- s.7 is second black and white negative 

S.2, s.6, and s.7 imply the theorem: melt, mold, and cool lens used in last 
step to form negative, and make lens from s.7; using negative and lens in an 
enlarger, make two prints, an enlargement and a reduction--enlargement and 
reduction together constitute the theorem. 


Concept Art: Innpersegs (May - July 1961) 

A "halpoint" iff whatever is at any point in space, in the fading rainbow halo 
which appears to surround a small bright light when one looks at it through 
glasses fogged by having been breathed on, for as long as the point is in the 


halo. 
An "init'point" iff a halpoint in the initial vague outer ring of its halo. 


An "inn'perseq" iff a sequence of sequences of halpoints such that all the 
halpoints are on one (initial) radius of a halo; the members of the first 
sequence are initpoints; for each of the other sequences, the first member (a 
consequent) is got from the non-first members of the preceding sequence 
{the antecedents) by being the inner endpoint of the radial segment in the 
vague outer ring when they are on the segment, and the other members (if 
any) are initpoints or first members of preceding sequences; all first members 
of sequences other than the last [two] appear as non-first members, and 
halpoints appear only once as non-first members; and the last sequence has 
one member. 


Indeterminacy 

A ftotaliy determinate innperseq' iff an innperseq: in which one is aware of 
(specifies) all halpoints. 

An fantecedentally indeterminate innperseq' iff an innperseq in which one is 
aware of (specifies) only each consequent and the radial seqment beyond it. 
A 'thalpointally indeterminate innperseq' iff an innperseq'in which one is 
aware of (specifies) only the radial segment in the vague outer ring, and its 
inner endpoint, as it progresses inward. 

Innpersegs Diagram 

In the diagram, different positions of the vague outer ring at different times 
are suggested by different shadings. The radia! segment in the vague outer 
ring moves down the page. The figure is by no means an innperseq, but is 
supposed to help explain the definition. 


129 


Successive bands represent the vague outer ring at successive times as it fades in toward the small bright tight. 


INNPERSEOQS DIAGRAM ; 
Halpoints at 


! time): ay a9 a3 aq a5 ag a7 b 
4.49 + p 
if time: a9 43 aq a5 ag a7 be 


a3 — Se 


SSS 
Se 


wa 


SSS SS 


TSO 3 
V3 


KS SS 


time3: a4 @5agaz7 DC d 


24,45 > d 


WSS SSS 
BWBABWVAAs 
WIS SAR SRS SSS ORS SS OSS SS SS 


A 
Y 


a 
NY 


timeg: ag a7 bede 


ag,h ——>e 


CR 
=e 
iy 
be 
i 


PaaS 


s760%, 
é ue 
mone 


times: a7bc def 


a7,€ oe f 


timeg: cdefg 


de—>g 
SO ESOS Pee Oe 
ROSES EE 
SQ 
SSE REARS ES Initpoint a9 a9 aq ae apa 
SERA ic cama Nae "te Sat a 
RENEURE SEAEELES 
SSR RRA 
ALENRAS QCaRBen 
WRALECE ACRES 
AAEALET RANECAN 
TaN 
SOR RSS Sei 
WARERENTE aN lnnperseq 
SOS eo 
WR B (43,49, a] 
WO aes 
WS N AQ (c,a5,a4) 
(d,b, ag) 
(e, C, a7) 
\\ (f, e, d) 
x (g)- 


small bright light 


130 


13. Exhibit of a Working Model of a Perception-Dissociator 


STATEMENT OF OBJECTIVES 


To construct a model of a machine a thousand years before the machine 
itself is technologically feasible--to model a technological breakthrough a 
thousand years before it occurs 


(Analogies: constructing a model of an atomic power plant in ancient 
Rome; chess-playing-machine hoaxes of 19th-century Europe as 
models of computers; Soviet Cosmos Hall at Expo 67 as model 
of anti-gravity machine) 

To construct the mode! almost entirely from the visitors coming to see it, so 

that each visitor regards the others as the model! 


What the hypothetical perception-dissociator will do that is not 
possible now: 


Physically alter the world (relative to you): sound disappears; sights and 
touches are dissociated; other people unconsciously signal you. 
Physically, "psychoelectronically" induce conditioned reflexes in your 
nervous system. Physically break ddwn your sense of time. 


[INVITATION] 


Because of your interest in technology and science, you are invited to visit 
EXHIBIT OF AWORKING MODEL OFA 
PERCEPTION-DISSOCIATOR 
Sponsored by (legitimate sponsor) Open continuously from (date) 
to {date) At (lunar colony or space station) 

"The perception-dissociator is a machine which is the product of a 
technology far superior to that of humans. With it, a conscious organism can 
drastically transform its psychophysical relation to objects and to other 
conscious organisms... The exhibit spotlights the technical interest of the 
perception-dissociator, giving the visitor a working model of the machine 
which he can use to 'transform' himself." —from the Guidebook 


it isn't possible for this exhibit to be open or public, because of the nature of 
the model. You have been invited in the belief that you will be a cooperative 
visitor. Come alone. Don't discuss the exhibit at all before you see it; and 
don't discuss it afterwards except with other ex-visitors. Come prepared to 


131 


spend several hours without a break. There will be absolutely no risk or 
danger to you if you follow instructions. 


TO THE DIRECTOR 


Exhibit requires two adjacent rooms, on moon or other low-gravity 
location, so that humans can easily jump over each other and fall without 
being hurt. First room, the anteroom, has "normal" entrance door leading in 
from "normal" human world. Is filled with chairs or school desks. At far 
corner from normal door is two-step lock, built in anteroom, connecting 
rooms. Normai door on hinges leads from anteroom into first step of lock. 
Sliding panel door leads into second step; and smooth curtain with slit in 
middle leads into the exhibit hali. Another sliding door leads from lock's 
first step directly back out to normal human world, bypassing anteroom. 
Shelf required in first lock to check watches and shoes. 

Exhibit hall large and empty with very high ceiling (Fuller dome? ). I 
Room must be strongly lighted, so that objects in front of closed eyes will 
cast highly visible shadows on eyelids. Room's inner surfaces must be 
sound-absorbing, and moderate noise must be played into room to mask 
accidental sounds; thus humans will cease to notice sound. Floor must be of 
hard rubber or other material that will not splinter, and will not be too hard 
to fall and crawl on. 

Exhibit open continuously for days. Invite people who will seriously 
try to play along--preferably engineers; and invite many of them, because 
is better to have many in exhibit. Sample invitation enclosed. Attendants 
working in shifts must be at two posts throughout. Try to keep surprising 
features of exhibit secret from those who have not been through it. 

Procedure. Visitor arrives and enters anteroom. Entrance attendant 
gives him a Guidebook and sends him to sit down and start reading. Then 
visitor goes to lock. Lock attendant must try hard to see that no more than 


normal 
Entrance 


Anteroom 


Exhibit Hal! 


chairs or 
chooldesks 


: 


Exit 


@: attendant 


132 


one visitor is in lock at a time. If lock is empty of visitors, attendant lets 
entering visitor into first step, checks his watch and shoes, and sends him 
alone into second step and on to exhibit room. When visitor comes out of 
exhibit hall for any reason, he must be gotten into first step, and then 
attendant sends him out the exit. When a visitor comes out, he just goes out 
and doesn't go back in. 


133 


EXHIBIT OF A WORKING MODEL OF A PERCEPTION—DISSOCIATOR 
(CONCEIVED BY HENRY FLYNT) 


GUIDEBOOK 


READ THIS GUIDEBOOK AS DIRECTED-STRAIGHT THROUGH OR AS 
OTHERWISE DIRECTED. DON'T LEAF AROUND. 


READ PAGES 2-3 BEFORE YOU GO IN TO SEE THE EXHIBIT. 


134 


Introduction. The perception-dissociator is a machine which is the 
product of a technology far superior to that of humans. With it, a conscious 
organism can drastically transform its psychophysical relation to objects and 
to other conscious organisms. When the organism has transformed itself, 
sound disappears, time is immeasurable; and the relation between seeing and 
touching becomes a random one. That is, the organism never knows whether 
it will be able to touch or fee! what it sees, and never knows whether it will 
be able to see what it touches or what touches it. The world ceases to be a 
collection of objects (relative to the physically altered organism). Further, 
the machine induces a pattern of communication in the organism's nervous 
system, an involuntary pattern of responses to certain events, to help the 
organism cope with the invisible tactile phenomena. A dimension is added of 
involuntarily relating to other organisms as unconscious signalling devices. 
The transformation induced by the machine is permanent unless the 
organism subsequently uses the machine to undo it. 


The perception-dissociator is not conscious or alive in any human sense. 
The components of the machine that the user is aware of are: (1) Optical 
phenomena that are seen--"sights." (2) Solid or massive phenomena that are 
felt cutaneously--"touches." If the user tries to touch a sight, he may not be 
able to feel anything there. If he looks for a component that touches him, he 
may not be able to see it. 


(Keep reading) 


135 


In other words, from the beginning the machine has properties that the 
entire world comes to have to the transformed organism. 


The exhibit spotlights the technical interest of the 
perception-dissociator, giving the visitor a working model of the machine 
which he can use to "transform" himself. Nothing is said about the purpose 
of the perception-dissociator in the society that can make one. The model is 
sophisticated enough that it can run independently of the visitor's will, and 
can affect him. In fact, the visitor may be hurt if he doesn't follow the 
instructions for using the machine. 


When you have absorbed the above, go to the entrance and be admitted 
to the exhibit. You must check your shoes, and your watch (if you have 
one), with the attendant. As you enter, turn this page and begin reading Page 
4. 


136 


DO NOT TALK OR MAKE ANY OTHER UNCALLED-FOR NOISE. 


Be prepared for the touch of pulling your feet out from under you 
from behind. Don't resist; just fall forward, break your fali with your arms 
(and retrieve this Guidebook). The floor is not hard and the gravity is weak, 
so the fall should leave you absolutely unhurt. 


AVOID ALL TOUCHES (EXCEPT FLOOR AND YOUSELF) UNLESS 
DIRECTED OTHERWISE. (You have been directed not to resist having your 
feet pulled out from under you.) INEFFECT, IF YOU BUMP INTOASOLID 
OBJECT OR STEP ON ONE, DRAW BACK. REMEMBER THAT YOU 
AVOID TOUCHES BY YOUR TACTILE SENSES ALONE. Whether your 
eyes are open or closed makes no difference. It is not necessary to avoid 
sights unless you touch something. 


There may be the touch of being pushed forward at your shoulder 
blades. Don't resist; just move forward. 


As for the sights in this model, it happens that they will be humanoid. 
All the human appearances other than you in the exhibit hall are sights from 
the machine. This is just the way the model is; don't give it a thought. Sights 
may appear or disappear (for example, at the curtain) while you are looking. 


I am referring to the components of the model with the names of the 
components of the perception-dissociator. 


As soon as you understand the above and are prepared to remember 
and follow the instructions, go immediately to Page 6. 


137 


(81% {(sy0ua)I ) s It all uvk (d, 


Ss 
? ay u3/4e9s8 uk[syv<ds,A<I 


ux Alt > (8, fS2kv) taydu 


vas] Sid6> \solu 
89183 $7] $3 
ca 
*S;v \S> ne 
§1V s Si\> tiiad I 
S24 83 - fl s_ 3A ($482) 
(ae, 
S23 83 S<Vv 3 


u [s3<} ( {ual $V 


S v<]S_v< 
2 1 


u_{s,I 
Soe ful u/s 
ty toJus I Sus s 8) > 3 


138 


You will now begin the first phase of perception- dissociation by the 
machine. Throughout this phase, you walk erect. 


Instructions for operating the machine and for protecting yourself from 
it will be given both in English and in an abbreviated symbolism. It is 
important to master the symbolism, because later instructions car.'t be 
expressed without it. 


uemeans you 

S, $4, Sp, $3 mean different sights from the machine 

t, ty, tg, tg mean different touches from the machine 

aAmeans a's eyes are Open or a opens its eyes 

av means a's eyes are shut or a shuts its eyes 

a=b means a blows on b's hand 

aDb means a pushes b, typically from behind 
{a holds Guidebook under arm or elsewhere) 

albImeans a jumps over b, crossing completely above b (weak gravity 
should make this easy) 

a'b means a rapidly waves both hands in front of and near b's eyes so that 
moving shadows are cast on b's eyes (a "shadows" b) 

a means a pulls b's ankles back and up and immediately lets them go, so 
that b falls forward (a "tackles" b) 

afb means a jumps and falls on b, or a steps on b 

a.J means a rapidly moves aside 

{} parentheses around the symbol for an action mean the action will 
probably happen 

A line of action symbols constitutes an instruction. The order of symbols 

indicates the order of events. !f one symbol is right above another, the 

actions are simultaneous. 


YOU MAY ALWAYS TURN BACK TO THESE EXPLANATIONS !F 
YOU FORGET THEM. 


{Keep reading) 


Instructions 1-3 apply WHEN YOUR EYES ARE OPEN. 


1. If you see a sight close its eyes, a heavy touch from the machine 
may be falling toward you. You must instantly jump aside. s4A S4V ud 
uA (th) 


YOU MUST FOLLOW THIS AND SUCCEEDING INSTRUCTIONS AS 
LONG AS YOU STAY IN THE EXHIBIT. STAY WITH EACH 
INSTRUCTION UNTIL YOU HAVE IT THOROUGHLY IN MEMORY; 
AND CHECK OUT THE SYMBOLIC VERSION SO YOU LEARN TO 
READ THE SYMBOLS. 


2. tf a sight in front of you jumps over you, a touch may be about to 
tackle you. You must instantly jump to one side. 


uA Sia] ul 
(t> 


3. If a sight waves its hands in front of your open eyes, a touch may 
be about to shove from behind. Jump to one side. 


as (120) ut 
IF THERE ARE ANY SIGHTS, TRY STANDING AROUND AND 
FOLLOWING THESE INSTRUCTIONS FOR A SHORT WHILE. 


4. if you close your eyes, you must keep them closed until a touch 
tackles you, a touch shoves you, or you can't keep your mind on the exhibit 
(which you should also consider to be an effect of the machine). Then you 
immediately open your eyes. 


cls {A horizontal line between 
ag Clu eK laa symbols means "or." 
With it, instr. can be combined. 


y inattentive 


THE NEXT THREE INSTRUCTIONS TELL YOU WHAT TO DO 
WHEN YOUR EYES ARE CLOSED. LEARN THEM WELL. 


5. !f you feel a breath blowing on one of your hands, a touch may be 
falling on you. You must instantly jump to the side away from the breath. 


UV (efi) u (Tern page and convinue) 


6. If your closed eyes are shadowed, a touch may be about to tackle 
you. You must instantly jump aside. 


Saco : 
Uv + ul 
(¢ a>) 


7. If you sense a massive touch going above your head, another touch 
may be about to shove you from behind. Jump aside. 


orm 
C, fui 


Ae ee u-4 
UY (€,4u) 


8. If you have any time left over from following other instructions, 
close your eyes and go around with your hands in front of you, shoving 
touches whenever you fee! them. 


uy ud 


NOW TRY INSTR. 8, REMEMBERING AND FOLLOWING THE 
OTHER INSTRUCTIONS ABOUT CLOSED EYES (INSTR. 4-7). WHEN 
YOU HAVE TO OPEN YOUR EYES AGAIN, AS PER INSTR. 4, CHECK 
ANYTHING YOU FORGOT: AND THEN GO TO THE SUCCEEDING 
INSTRUCTIONS. NOW-CLOSE YOUR EYES. 

THE NEXT THREE INSTRUCTIONS APPLY WHEN YOUR EYES 
ARE OPEN. 


9. If you see a sight falling toward or about to step on another sight 
whose eyes are open, run until you face the sight on the ground and close 
your eyes. BEFORE YOU FOLLOW THIS INSTRUCTION YOU MUST 
HAVE MASTERED THE PRECEEDING INSTRUCTIONS ABOUT 
CLOSED EYES. 


/ 
/ 


f 
uy S24 (si /Sz) UY 


(Keep going) 


141 


10. If you see a sight about to tackle another whose eyes are open, run 
until you face the sight about to be tackled and jump over both sights. If the 
sight about to be tackled has closed eyes, you must immediately shadow 


them. 
S2/ (s, \e>) u SS 
S2V (s.(S) uP Sz 


11. If you see a sight about to push another with open eyes from 
behind, you must shadow the sight about to be pushed. But if the sight 
about to be pushed has closed eyes, you must immediately jump over both 
sights. 


UA 


A s,a ($12) U oe S2 
S2V¥0 (6,252) u [s, S21 


You must now put ali the instructions into practice until you have 
learned them thoroughly by doing as they say. In other words, carry out 
Instr. 8, and the other instructions as they apply. 


If you can't practice the instructions because you still have not seen a 
sight or felt a touch, skip directly to Page 18. 


Learning the instructions in practice should take a good while. When 


you have mastered them, the first phase is over. Turn to Page 10 and begin 
the second phase. 


142 


You are now in the second phase of transforming yourself with the 
perception-dissociator. Throughout this phase, you must stoop or crouch 
somewhat. That is, you must keep yourself below the height of your neck 
when you stand straight-- except when you jump over a sight. The symbol is 
uz u rad means that you crouch and close your eyes. Now crouch. 


The numbered instructions for this phase are so similar to those in the 
preceeding phase that they will be given in symbols only. Changes are noted 
parenthetically. You may turn back if you forget symbols. 


4. SA SV 4] 2. oa Tal. ud 
uZA (E(u) 2 4% (ec) 


t=Hu 'Chan @e. component Blows on you) 
Z, us A (t,Ju) ul \ cee of shadowing you. 


Jee. 
4. u2dv CIty yn 
uv mabtentive 


Seo Uu 
3 t.=4 yl 6 2 V u 
5. ou 4" (Cy fu 4S (¢ &) 
7, uy py. "AE a 4° Be 
(€271u) 


The big change comes next. 


(Keep going) 


143 


9 u3r SA (Sie) uv and abo 


uar S2V (s,/S2) Us S2 
4 


That is, if you see a sight falling or stepping on another sight with closed 
eyes, you must immediately blow on the sight on the ground. This is an 


addition. 
40. 
* S.Vv (S: >) u eax Oz 
44 > 
a, S24 (2.152) us Se 


3. 
* sv ($48) ule 
(Change: you blow on Sz) 


So far there have been only three changes in the instructions. Memorize 
them. Then go on to Instr. 12, which is new, and carry it out along with the 
other eleven instructions. 


AS SOON AS YOU HAVE PUT ANY CHANGED INSTRUCTION (3, 
9, OR 11) INTO PRACTICE, THE SECOND PHASE IS OVER. TURN TO 
PAGE 12 AND THE THIRD PHASE. 


If you can't practice the instructions because all the components have 
vanished, skip to Page 18. 


12. Adding to Instruction 8, if you have time left over from following 
other instructions, you may also keep your eyes open and jump over, blow 
on, or shadow sights. 


u fs] 


usa 4Uros 
ux9 


Throughout the third phase, you must squat or move on your hands 
and knees. That is, you must always keep yourself below the height of your 
waist when you stand straight--unless you are able to jump over a sight from 
your low position. The symbol is ut. Now get down. 

Instr. 1-7 from the last phase apply here without change. They are thus 


stated in the most abbreviated form. 


Sv 7 
" Ct) TE ae 
Uu 4" S rol ul . I ul uy A 
(clip : us 4 inattentive 
t,= u C,=u 
(t;u) (t, fa) 
Seo W ut =f 
ee) 
Jot 'u 
(tga u) 


The biggest change comes next. 

8. If you have any time left over, close your eyes and go around with 
your hands in front of you. If you encounter touches standing higher than 
you, tackle them. If you encounter touches as near the ground as you, shove 
them. You must be sensitive and judge heights with eyes closed. 


phy LoD 
2 Cao UIT 


C> MEANS ** (FE STANDS HIGH RELATIVE TO You 
tc MEANS jFE IGNEAR GROUND RELATIVE TO You 


9 No change. 
Ga S2N (5,2) uv 
- -$2V (S12) uz % 


10. The previous Instr. 10 applies if sy is near the ground, that is, it 
applies unless sz is too high for you to jump or shadow it. 


SAS (s, 3) ulS,5;] 
rag Cs are 
(Keep going) 


145 


44. uta S2a ( S, 1s.) U= S2 
The second half of the previous Instr. 11 is dropped. 


Except for the instruction to tackle touches, the changes are simply 
limitations to make the instructions feasible for u > They should be easy 
to remember. 


You will next go on to Instr. 12, and carry it out along with the other 
instructions. As soon as you encounter an actual! situation where you cannot 
act because u+., the third phase will be over. AT THAT POINT YOU 
MUST TURN TO PAGE 14 AND THE FOURTH PHASE. 


If you can't carry out the instructions because all the components have 
vanished, the third phase is over. Turn to Page 14 and the fourth phase. 


12. Adding to Instr. 8, if you have time left over, you may also keep 
your eyes open and blow on sights. You may also shadow or jump over 
sights unless they are too high. 


You are in the tourth phase of perception-dissociation. Throughout this 
phase, you must crawl on your stomach (keep below knee height). The 
symbol is u +.. Now get on the floor. 

You can no longer be tackled, nor can you jump. Thus, the numbered 
instructions are greatly limited, and they will be restated fully. 

THE FIRST TWO INSTRUCTIONS APPLY WHEN YOUR EYES ARE 


OPEN. 
1. If you see a sight close its eyes, a touch may be falling or stepping 
on you, and you must immediately scramble aside. 


SA SV mal 
ugar (Tia) 


THE NEXT THREE INSTRUCTIONS TELL YOU WHAT TO DO 
WHEN YOUR EYES ARE CLOSED. 
3. When to reopen your eyes. 


j Cou 
udu ene UA 
4+ u mMattentlive 


4. if your closed eyes are shadowed, a touch may be falling or 
stepping on you. Scramble aside. 


e. 4 Aa 'a' Al sf 
UZ V (tru) 
6. PM 
7 Av. E> ui b> 
4 cs ute 


TRY INSTR. 6, REMEMBERING AND FOLLOWING INSTR. 3-5. 
WHEN YOU HAVE TO REOPEN YOUR EYES AS PER INSTR. 3, CHECK 
ON ANYTHING YOU FORGOT. THEN GO TO PAGE 15. NOW--CLOSE 
YOUR EYES. 


The rest of the instructions apply when your eyes are open. 


ya —224 (6152) uv' 
4 $2VvE (1/2) Ur Sz 


\f $9's eyes are closed, you must shadow them unless they are too high. 


& y AA Sag (S13s2) us S, 


You blow on $9'S hand unless it is too high. 


9. Adding to Instr. 6, if you have time left over from following 
instructions, you may also shadow or blow on sights if they aren't too high. 


U a A sc Uso S 
u =S 
You must now put these nine instructions into practice unti] you have 


learned them thoroughly in practice; and even continue after that until you 
have difficulty keeping your mind on the exhibit. 


IF YOU CAN'T PRACTICE THE INSTRUCTIONS BECAUSE ALL 
THE COMPONENTS HAVE VANISHED, SKIP TO PAGE 18. 


Otherwise, stay with this phase until you have difficulty keeping your 
mind on it. Then turn to Page 16 and the final phase of 
perception-dissociation. 


You are now in the final phase of transforming yourself with the 
perception-dissociator. When you finish transforming yourself, you will have 
lost track of time, and will have ceased to notice sound. You will be dealing 
with sights and touches as unrelated phenomena; and you will be responding 
by reflex action to unconscious signals from "other people." 

For this last phase, you will turn to Page 5. You will go through the 
symbols there in any order you like as if they were one long instruction, 
carrying out that instruction. You are to "use" each symbol once. There 
have been enough precedents in the interpretation of the symbols that you 
should now be able to interpret any combination of them. Continue to 
follow the previous numbered instructions as they apply, depending on 
whether you are 1, 3/4, 1/2, or 1/4. (But forget the instructions for time left 
over; you won't have any extra time.) REMEMBER THE INSTRUCTIONS 
ABOUT WHEN TO REOPEN YOUR EYES IF YOU CLOSE THEM. 


When you are through, you will be transformed. NOW TURN TO 
PAGE 5 AND BEGIN. 


149 


If you have found these words and are reading them in desperation 
because you are completely confused; or because you have lost interest in 
the exhibit; or because you have finished; then you are transformed. 


If you want to use the model to simulate the reversal of your 
transformation before you leave the exhibit, do the following. Spend 50 
seconds erect, with open eyes, walking up to sights and pushing 
them--assuming that you will find touches where you see sights. Count the 
seconds "one-thousand-and-one," "one-thousand-and-two," etc. 


Then you will close your eyes. If you are blown on or pushed before 
250 seconds have passed, you will open your eyes and--assuming that you 
will find a sight where you were touched--you wil! shadow it. Otherwise you 
will open your eyes when the 250 seconds have passed. Now close your eyes 
and do as instructed. 


It is now suggested that you leave the exhibit. Go out through the 
curtain. 


150 


Stay in the exhibit and follow every instruction that is relevant, unti! 
you become thirsty. 


if you begin to encounter components, return to the page you were on 
before you turned to this one. 


lf you still don't encounter components, the mode! must be broken. 
Leave the exhibit by the same passage through which you entered. 


151 


2/22/1963 


Henry Flynt and Tony Conrad demonstrate against the Metropolitan Museum of Art, 
February 22, 1963 


(foto by Jack Smith} 


152 


14. Mock Risk Games 


Suppose you stand in front of a swinging door with a nail sticking out of it 
pointing at your face; and suppose you are prepared to jump back if the 
door suddenly opens in your face. You are deliberately taking a risk on the 
assumption that you can protect yourself. Let us call such a situation a "risk 
game." Then a mock risk game is a risk game such that the misfortune which 
you risk is contrary to the course of nature, a freak misfortune; and thus 
your preparation to evade it is correspondingly superficial. 

If the direction of gravity reverses and you fall on the ceiling, that is a 
freak misfortune. If you don't want to risk this misfortune, then you will 
anchor yourself to the floor in some way. But if you stand free so that you 
can fall, and yet try to prepare so that if you do fall, you will fall in such a 
way that you won't be hurt, then that is a mock risk game. if technicians 
could actually effect or simulate gravity reversal in the room, then the risk 
game would be a real one. But I am not concerned with real risk games. I am 
interested in dealing with gravity reversal in an everyday environment, where 
everything tells you it can't possibly happen. Your 'preparation' for the fall 
is thus superficial, because you still have the involuntary conviction that it 
can't possibly happen. 

Mock risk games constitute a new area of human behavior, because they 
aren't something people have done before, you don't know what they will be 
like until you try them, and it took a very special effort to devise them. 
They have a tremendous advantage over other activities of comparable 
significance, because they can be produced in the privacy of your own room 
without special equipment. Let us explore this new psychological effect; and 
let us not ask what use it has until we are more familiar with it. 

Instructions for a variety of mock risk games follow. (I have played 
each game many times in developing it, to ensure that the experience of 
playing it will be compelling.) For each game, there is a physical action to be 
performed in a physical setting. Then there is a list of freak misfortunes 
which you risk by performing the action, and which you must be prepared 
to evade. The point is not to hallucinate the misfortunes, or even to fear 
them, but rather to be prepared to evade them. First you work with each 
misfortune separately. For example, you walk across a room, prepared to 
react self-protectingly if you are suddenly upside down, resting on the top of 


153 


your head on the floor. In preparing for this risk, you should clear the path 
of objects that might hurt you if you fell on them; you should wear clothes 
suitable for falling; and you should try standing on your head, taking your 
hands off the floor and falling, to get a feeling for how to fail without 
getting hurt. After you have mastered the preparation for each misfortune 
separately, you perform the action prepared to evade the first misfortune 
and the second (but not both at once). You must prepare to determine 
instantly which of the two misfortunes befalls you, and to react 
appropriately. After you have mastered pairs of misfortunes, you go on to 
triples of misfortunes, and so forth. 

The principal games are for a large room with no animals or distracting 
sounds present. 

A. Walk across the lighted room from one corner to the diagonally 
opposite one, breathing normally, with your eyes open. 

1. You are suddenly upside down, resting on the top of your head on the 
floor. You must get down without breaking your neck. 

2. Although the floor looks unbroken and solid, beyond a certain point 
nothing is there. !f you step onto that area, you will take a fatal fall. Thus, as 
you walk, you must not shift your weight to your forward foot until you are 
sure it will hold. Put the ball of the forward foot down before the heel. 

3. Something happens to the cohesive forces in your neck so that if your 
head tips in any direction, it will come right off your body, killing you 
immediately. Otherwise everything remains normal. Thus, as you walk, you 
must "balance" your head on your neck. When you reach the other side of 
the room, your neck will be restored to normal. (Prepare beforehand by 
walking with a book balanced on your head.) 

4. Invisible conical weights fall around you with their points down, each 
whistling as it falls. You must evade them by ear in order not to be stabbed. 
Walk softly and fast. 

5. The room is suddenly filled with water. You have to contro! your lungs 
and swim to the top. Wear clothes suitable for swimming. 

A. Play game A while on a long walk on an uncrowded street. The floor 
is replaced by the sidewalk. The fifth misfortune becomes for space suddenly 
to be filled with water to a height of fifteen feet above the street. 

B. Lie on your back on a pallet in the dimly lit room, hands at your 

sides, with a pillow on your face so that it is slightly difficult to breathe, for 
thirty seconds at a time. 
1. The pillow suddenly hardens and becomes hundreds of pounds heavier. !t 
remains suspended on your face for a split second and then "falls," bears 
down with full weight. You must jerk your head out from under it in that 
split second. 


154 


2. The pillow adheres to your skin with a force greater than your skin's 
cohesion, and begins to rise. You must rise with it in such a way that your 
skin is not torn. 

C. Lie on your back on the pallet in the dimly lit room. 

1. Gravity suddenly disappears completely, so that nothing is held down by 
it; and the ceiling becomes red-hot. You must avoid drifting up against the 
ceiling. 

2. The surface you are lying on becomes a vast lighted open plane. From the 
distance, giant steel spheres come rolling in your direction. You must evade 
them. 

3. Your body is split in half just above the waist by an indefinitely long, 
rather high, foot-thick wall. Your legs and lower torso are on one side, and 
your upper torso, arms, and head are on the other side. Matter normally 
exchanged between the two halves of your body continues to be exchanged 
through the. wall by telekinesis. It is as if you are a foot longer above the 
waist. In order to reunite your body, you must first roll over and get up, 
bent way forward. There are depressions in the wall on the same side as your 
feet. You have to climb the wall, putting your feet in the depressions and 
balancing yourself. You will be reunited when you reach the top and your 
waist passes above the wall. 

D. Sit in a plain, small, straight chair, on the edge of the seat, hands 
hanging at the sides of the seat, feet together in front of the chair, in the 
lighted room, for about thirty seconds at a time. 

1. The chair is suddenly out from under you and sitting on you with Its legs 
straddling your lap and legs. You have to get your weight over your feet so 
you won't take a hard fall. 

2. The direction of gravity reverses and the chair remains anchored to the 
floor. You have to grab the seat and hold on in order not to fall on the 
ceiling. 

3. You are suddenly in a contra-terrene universe, in which the atmosphere is 
unbreathable and prolonged contact with either the atmosphere or the 
ground will disintegrate you. The seat and back of the chair become a 
penetrable hyperspatial sheet between the alien universe and your own. As 
soon as you feel the alien atmosphere, you must jerk your feet off the 
ground and deliberately sink or p!unge through the seat and back of the chair 
in the best way that you can. You will end up on the floor under the chair in 
your universe. 

4. You are suddenly in dark empty space in a three-dimensional lattice of 
gleaming wires. Segments of the lattice alternately burst into flame and cool 
off. You adhere to the chair as if it were part of you. With your hands 
holding onto the seat, you can move yourself and the chair forward by 


155 


from blundering into a radiation beam, you have to communicate 
pre-verbally to the other mind by every means from vocal cries to 
pantomine, and get your-body/his-mind out of range of the radiation. When 
the body is out, you will both be restored to normal. (The first thing to 
anticipate is the basic shift in viewpoint by which you will be looking at 
your own body from the other's position. There is no point in tensing your 
muscles in preparatiton for the misfortune, because if it occurs, you will be 
working with a strange set of muscles anyway. The next thing to prepare to 
do is to spot the radiation beams; and then to yell, gesture, or 
whatever--anything to get the "other" to avoid the radiation. Note finally 
that neither player prepares for the possibility that he will be surrounded by 
radiation. Each player prepares for the same role in an asymmetrical pas de 
deux.) 

Asymmetry: The two of you play a given duo game, but each prepares 
to evade a different misfortune. 

AB. Stay awake with eyes closed for an agreed upon time between one 
and fifteen minutes. Use a timer with an alarm. 

1. Each suddenly has the other's entire present consciousness in addition to 
his own, from perceptions to memories, ideologies, ambitions, and 
everything else--threatening both with psychological shock. 

The couple must take up positions such that their sensory perceptions 

are as nearly identical as possible. Beforehand, each must discuss with the 
other the aspects of the other's attitude to the world which each must fears 
having impused on his consciousness. During the game, each must think 
about these aspects and try to prepare for them. 
2. Each suddenly relives the other's most intense past feelings of depression 
and suicidal impulses. In other words, if five years ago the other attempted 
suicide because he failed out of college, you suddenly have the consciousness 
that "you" have just failed out of college, are totally worthless, and should 
destroy yourself. Presumably the other has since learned to live with his past 
disasters, but you do not have the defenses he has built up. You are 
overwhelmed with a despair which the other felt in the past, and which is 
incongruous with the rest of your consciousness. In summary, both of you 
risk shock and suicidal impulses. Beforehand, of course, each must tell the 
other of his worst past suicidal or depressed episode; and discuss anything 
else that may minimize the risk of shock. 


158 


Intrusions in Duo Games 

As before, distractions and modulations can be openly studied by 
consent of the players. As for bogies, it is possible in duo games for one 
player to create a bogy without warning, in effect acting as a saboteur. As 
soon as a game is sabotaged, though, confidence is lost, and each player just 
watches out for the other's bogies. Here are some sample intrusions. 


DISTRACTION BOGY MODULATION 


shout in other's each 
face take 
2, talk and Jaugh stamp hard 2a 
get out of step Ly different 


} 


ALB cough gasp 


talk and laugh silently pass palm back I 
& forth in front of 
other's face 


15. The Dream Reality 


A. Memo on the Dream Project 


Original aim: To recreate the effect of e.g. Pran Nath's singing--transcendent 
inner escape--in direct life rather than art. I needed material which could 
function as an alien civilization (since the source of Pran Nath's expression is 
an alien civilization relative to me); yet which was encultured in me and not 
an affectation or pretense. I decided to use dreams as the material, assuming 
that my dreams would take me to alien worlds. But mostly they did not. 
Mostly my dreams consist of long periods of tawdry, familiar life interrupted 
occasionally by senseless, unmotivated anomalies. In contrast, my original 
aim required alluring, psychically gratifying material. 


The emphasis shifted to redefining reality so that dreams were on the same 
level as waking life; so that they were apprehended as what they seem to be: 
literal reality {and not memory, precognition, or symbolism). The project 
was still arcane, but in a drastically different way. I was getting into an 
alternate reality which was extremely bizarre but not psychically gratifying. 
It was boringly frightful and sometimes obscene. I became concerned with 
analytical study of the natural order of the dream world, a para-scientific 
investigation. As I grappled with the rational arguments against treating 
dreams as literal reality, the project became a difficult analytical exercise in 
the philosophy of science. The original sensuous-esthetic purpose was lost. 


Now I would like to return to the original aim, but how to do it? Obtain 
other people's dreams--see if they are more suitable? Work only with my 
very rare dreams which do take me to alien worlds? Try to alter the content 
of my raw dreams? Attempt to affect content of dreams by experiment in 
which many people sleep in same room and try to communicate in their 
sleep? The most uncertain approach to a solution: set up a transformation 
on my banal dreams, so that to the first-order activity of raw dreaming is 
added a second-order activity. The transformation procedure to somehow 
combine conscious ideational direction--coding of the banal dreams--with 
alteration of my experience, my esthesia, my lived experience. 


160 


B. Dreams and Reality--An Experimental Essay 


Excerpts from my dream diary which are referred-to in the essay that 
follows. 
12/11/1973 

I notice a state between waking and dreaming: a waking dream. I have 
been asleep; I wake up; I close my eyes to sleep again. While not yet asleep, I 
experience isolated objects before me as in a dream, but with no 
background, only a dark void. !n this case, there are two pocket combs, both 
with teeth broken. In the waking world, I threw away one of my two pocket 
combs because I broke it; the other comb is still in good condition. 


12/30/1973 

I am chased by the police for one block west on West Market Street in 
Greensboro. I reach the intersection with Eugene Street, and in the north 
direction there is a steep hill rather than the street. The surface of the hill is 
bare ground and grass. I run up the hill, sensing that if ! can get over the hill 
I will find Friendly Road and the general neighborhood of my mother's 
houses on the other side. The police start shooting. If I can get a few yards 
farther on the top of the hill I will be past the line of fire. I take a headlong 
dive and awaken in the middle of the dive to find myself diving forward on 
my mattress in the front room of my apartment. The action is carried on 
continuously through waking up and through the associated change of 
setting. 


1/12/1974 

Just before ! go to sleep for the night, I am lying in bed drowsy. I think 
of being, and suddenly am, at the south edge of the Courant Institute plaza, 
which is several feet above the sidewalk. The edge of the plaza and the drop 
are all I see. It is night; and there is only a void where the peripheral 
environment should be. (Comment: It is of great theoretical importance that 
while most of the internal reality cues were present in this experience, some, 
like the peripheral environment, were not. In my dream experiences, all 
reality cues are present.) The drop expands to twenty or thirty feet, and I 
start to fall off. Fright jolts me completely awake. I have had something like 
a waking nightmare and have awakened from being awake. I thought of the 
scene, was suddenly in it (except for peripheral reality cues), lost control and 
became endangered by it, and then snapped back to my bedroom. 


1/1-/1974 
One or two nights after 1/12/74 I was lying in bed just before going to 


161 


sleep. I could see women standing on a sidewalk. The scene was real, but I 
was not in it; I was a disembodied spectator. Also, the peripheral 
environment was absent. The reality was between that of a waking 
visualization and that of the Courant Institute incident of 1/12/74. 
Comment: The differences between this experience and a _ waking 
visualization are that the latter is less vivid than seeing and is accompanied 
by waking reality cues such as cues of bodily location. 


1/16/1974 

1. I am in an apartment vaguely like the first place in which I lived, at 
1025 Madison Avenue in Greensboro. I am a spy. I am teen-aged and short; 
and I am in the apartment with several enemy men, who are middle-aged and 
adult-sized. My code sheets look like the sheets of Yiddish I have been 
copying out in waking life. Eventually the men discover me in the front 
room with the code sheets on a fold-up desk. They chase me out the front 
door and onto the west side of the lawn, and shoot me with a needle gun. At 
that moment my consciousness jumps from my body and becomes that of a 
disembodied spectator watching from an eastward location, as if I were 
watching a film. 

2. I am living in a dormitory in a rural setting with other males. At one 
point I walking barefoot in weeds outside the dormitory, and Supt. Toro 
tells me I am walking in poison ivy. My feet begin to show the rash, but I 
recognize that I am in a dream and think that the rash will not carry over to 
the waking state. I then begin to will away the rash in the dream, and I 
succeed, 


1/20/1974 

For some reason the dream associates Simone Forti with flute-like 
music. It is shortly before midnight. In the dream I believe that Simone lives 
in a loft on the east side of Wooster Street. The blocks in SOHO are very 
small. If I walk through the streets and whistle, she will hear me. I start to 
whistle but can only whistle a single high note. I half awaken but continue 
whistling, or trying to; the dream action continues into waking. But I cannot 
change pitch or whistle clearly because my mouth is taped. As I realize this, I 
awaken fully. 
Comments: I tape my mouth at night so I will sleep with my mouth closed. I 
experimented at trying to whistle with the tape on while fully awake. The 
breath just hisses against the tape. The pitch of the hiss can be varied. 


2/1/1974 

1. I try to assist a man in counterfeiting ten dollar bills by taking half 
of a ten, scotch taping it to half of a one, and then coloring over the one 
until it looks like the other half of the ten. The method fails because I bring 
old crumpled tens rather than new tens, and the one doilar bills are new. 


Comments: There are no natural anomalies in this dream at ali. What is 
anomalous is that this counterfeiting method seems perfectly sensible, and I 
only begin to question it when we try to fit the crumpled half-bill to the 
crisp half-bill. Why am I so foolish in this dream? I retain my identity as 
Henry Flynt, and yet my outlook, my sense of what is rational, is so 
different that it is that of a different person. More generally, the person I am 
in my dreams is much more limited in certain ways that I am in waking life. 
My waking preoccupations are totally absent from my dreams. Instead there 
is bland material about my early life which could apply to any child or 
teen-ager. Thus, I must warn readers who know me only from this diary not 
to try to make the image of me here fit my waking life. 


2/3/1974 

3. I have had several dreams that I am taking the last courses of my 
student career. (In waking life I have completed all course work.) I am 
usually failing them. Tonight I dream that I have gone all semester without 
studying (in a course in English? ). Now I am in the final exam and sinking. I 
will have to repeat these courses. Subsequently, I am sitting in a school 
office (of a professor or psychologist? ), giving him a long list (of words, a 
foreign vocabulary? ). {I mention this episode because I remember that while 
I retained my nominal identity as Henry Flynt, I had the mind of a different 
person. I experienced another person's existence instead of mine. Professor 
Nell also appeared somewhere in this dream; as he has in several school 
dreams I have had recently. 


2/3/1974 (This is the date I recorded, but it seems that it would have to be 
later.) 

} get up in the morning and decide to have a self-indulgent breakfast 
because of the unpleasantness of working on my income tax the day before. 
So I put two slices of pizza in the oven, and also eat two bakery sweets, 
possibly éclairs. Then I think that a Mexican TV dinner would have been 
better all around, but it is too late; I have to eat what I am already preparing. 
Subsequently, I go with John Alten to a Shoreham Cafeteria at Houston and 
Mercer Streets. The cafeteria chain is a good one, but this cafeteria is dark 
and extremely dingy upstairs where the serving line is. John coinplains that 
there is no ventilation and that he is suffocating, and he stalks out. 


163 


Comment: When I awoke, my first thought was that the pizza in the oven 
would be burning. {I assumed that I had arisen, put the pizza in the oven, 
and gone back to sleep.) But then I realized that the breakfast was a dream. I 
got up and prepared the Mexican dinner which I had decided was best in the 
dream, but I also ate one éclair. 


7/8/1974 

I am caught out in a theft of money, and I feel that the rest of my life 
will be ruined. Comments: The quality of the episode depended on my 
strong belief in the reality of the social future and in my ability to form 
accurate expectations about it. When I awakened, the whole misadventure 
vanished. 


End of excerpts from my dream diary. / 


".. It is correct to say that the objective world is a synthesis of private views 
or perceptions... But ... inasmuch as it is the common objective world that 
renders ... general knowledge possible, it will be this world that the scientist 
will identify with the world of reality. Henceforth the private views, though 
just as real, will be treated as its perspectives. ... the common objective 
world, whether such a thing exists or is a mere convenient fiction, is 
indispensable to science ... ." 

A. d'Abro, The Evolution of Scientific Thought (New York, Dover, 1950), 
pp. 176-7 


A. We wish to postulate that dreams are exactly what they seem to be 
while we are dreaming, namely, literal reality. Naively, we want to get closer 
to literal empiricism than natural science is. But science has worked out a 
very comfortable world-view on the assumption that both dreams and 
semi-conscious quasi-dreams are mere subjective phenomena of individual 
consciousness. If we wish to carry through the postulate that dreams are 
literal reality, then we will have to adopt a cognitive model quite different 
from that of natural science. It is of crucial importance that we are not 
interested in superstition. We do not wish to adopt a cognitive model which 
would simply be defeated in competition with science. We wish to be at least 
as rationa!, as empirical, and as cognitively parsimonious as science is. We 
want our cognitive model to be compelling, and not to be a plaything which 
is easily taken up and easily discarded. 

The question is whether there can be a rational empiricism which 
differs from science in placing dreamed episodes on the same level as waking 


164 


episodes, but which stops short of the "nihilistic empiricism' of my 
philosophical essay entitled "The Flaws Underlying Beliefs." (In effect, the 
latter essay rejects other minds, causality, persistent objective entities, past 
time, the possibility of objective categories and significant language, and so 
forth, ending up with ungraded immediate experience.) 

As an example of our problem, the waking scientific outlook assumes 
that a typewriter continues to exist even when we turn our backs on it 
(persistence of objective entities). In many of our dreams we make the same 
sort of assumption. In other words, in some of our dreams the natural order 
is not noticeably different from that of the waking world; and in many 
dreams our conscious world-view has much in common with waking 
common sense or scientific pragmatism. On 2/3/1974 I had a dream in which 
a typewriter was featured. I certainly assumed that the typewriter continued 
to exist when my back was turned to it. On 7/8/1974 I dreamed that ! was 
caught out in a theft of money, and I felt my life would be ruined because of 
it. I certainly assumed the reality of the social future, and my ability to form 
accurate expectations about it. These examples illustrate that we are not 
nihilistic empiricists in our dreams. The question is whether acceptance of 
the pragmatic outlook which we have in dreams is consistent with not 
regarding the dream-world as a subjective phenomenon of individual 
consciousness. Can we accept dreams as "literal reality"; or must we reject 
the very concept of "reality" on order to defend the placing of the dream 
world on the same level as the waking world? 

In summary, the question is whether we can place dreams on the same 
fevel as the waking world while stopping short of nihilistic empiricism. A 
further difficulty in accomplishing this aim is that neurological science might 
succeed in gaining complete experimental control of dreams. Scientists might 
become able to produce dreams at will and to monitor them. The whole 
phenomenon of dreaming would then tend to be totally assimilated to the 
outlook of scientists. Their decision to treat dreams as subjective phenomena 
of individual consciousness would be greatly supported by these 
developments. Would we have to go all the way to nihilistic empiricism in 
order to have a basis for rejecting the neurologists' accomplishments? 

Still another difficulty is presented for us by semi-conscious 
quasi-dreams such as the ones described in my diary. Semi-conscious 
quasi-dreams exhibit some reality cues, but lack other important internal 
reality cues. Science handles these experiences easily, by dismissing them 
along with dreams as subjective phenomena of individual consciousness. 
Suppose we accept that the semi-conscious quasi-dreams are illusory reality. 
But tf they can be illusory reality, how can we exclude the possibility that 
dreams might be aiso? !f, on the other hand, we accept the quasi-dreams as 


165 


literal reality, what about the missing reality cues? Can we justify different 
treatment for dreams and quasi-dreams by saying that all reality cues have to 
be present before an experience is accepted as non-illusory? If we propose 
to do so, the question then becomes whether we should accept the weight 
which common sense places on reality cues. 


Why do we wish to stop short of nihilistic empiricism? Because we do 
wish to assert that dreams can be remembered; that they can be described in 
permanent records; that they can be compared and studied rationally. We do 
wa..t to cite the past as evidence; we do want to distinguish between actual 
dream experience and waking fabrications, waking lies about what we have 
dreamed; and we do want to describe what we experience in intersubjective 
language. " 

As easy way out which would offend nobody would be to treat dreams 
as simulations of alternate universes. But this approach is a cowardly evasion 
for several reasons. It excludes the phenomenon of the semi-conscious 
quasi-dream, which poses the problem of internal reality cues in the sharpest 
way. Further, we cannot give up the notion that our project is nearer to 
literal empiricism than natural science is. We cannot accept the notion that 
we must dismiss some of our experiences as mere illusions, but not all of 
them. We do not see dreams as simulations of anything. Some of the most 
interesting observations I have made about connections between adjacent 
dreamed and waking episodes in my own experience are noticeable only 
because I take both dreamed and waking experience literally. 


B. Before we continue our attempt to resolve our methodological 
problem, we will provide more detail on topics which we have mentioned in 
passing. We begin with the purported empiricism of natural science. The 
philosopher Hume postulated that experience was the only raw material of 
reality or cognition. However, he did not content himself with ungraded 
experience. He insisted on draping the experiential raw material on an 
intellectual framework in such a way that experience was used to simulate 
the inherited conception of. reality, a conception which we will call 
Aristotelian realism. Similarly for the purported empiricism of natural 
science. In fact, the working scientist learns to think of the framework or 
model as primary, and of experiences and verification procedures as ancillary 
to it. The quotation by d'Abro which heads this essay concedes as much. 

What we are investigating is whether experiences can be draped on a 
different intellectual framework in which dreamed and waking life come out 
as equally real. Some examples of alternate verification conventions follow. 
1. Accept intersubjective confirmation of my experience of the dream world 
which occurs within the dream as confirmation of the reality of the dream 


166 


world. 

2. Accept intersubjective confirmation of the past of the dream world which 
occurs in the dream itself as confirmation of the reality of the dreamed past. 
3. Recognize that there is no infallible way to tell whether other people are 
lying about their dreamed expefience or their waking experience. 

4. Develop sophisticated interrogation techniques as a limited test of 
whether people are telling the truth about their dreams. 


5. Accept that a certain category of anomalies occurs in dreams only when 
several people have reported experiences in that category. 

The principal characteristic of the approach which these conventions 
represent is that each dream is treated as a separate world. There is no 
attempt to arrive at an account, for a given "objective" time period, which is 
consistent with more than one dream or with both dreamed and waking 
periods. Thus, many parallel worlds could be confirmed as real. As our 
discussion proceeds, we will move away from this approach, probably out of 
a sense that it is pointless to maintain a strong notion of reality and yet to 
forego the notion of the consistency of all portions of reality. 

C. Something that I have learned from a study of my dream records is 
that while dreams are not chaotic, while they can be compared and 
classified, it is not possibie to apply the method of natural science to them in 
the sense of discerning a consistent, impersonal natural order in the dream 
world. It is not that the natural order is different in dreams from what it is in 
the waking world; it is that the dream worlds are incommensurate with the 
discernment of a natural order in the scientific sense. Here are some specific 
observations which relate to this whole question. 

1. Some dreams are not noticeably anomalous. The laws of science are not 
violated in them. This observation is important in giving us a normal base for 
our investigation. Dreams are not all crazy and chaotic. 

2. In some dreams, it is impossible to abstract an impersonal natural order 
from personal experiences and anecdotes. There are no impersonal events. 
There is no nature whose order can be defined impersonally. The dreams are 
full of personal magic which cannot be generalized to a characteristic of an 
impersonal natural order. 

3. As a special case of (2), in some dreams, we jump back in time and move 
discontinuously in time and space. Chronological personal magic. 

4. In dreams, the distinction between myself and other people is blurred in 
many different ways. Also, ! sometimes become a_ disembodied 
consciousness. 

5. As a generalization of (4), sometimes it becomes impossible to distinguish 
objects from our sensing and perceiving function. The mediating sensory 
function becomes obtrusively anomalous. Stable object gestalts cannot be 


167 


identified. 

6. Sometimes we experience the logically impossible in dreams. My father 
was both dead and buried, and alive and walking around, in one dream. 

7. The possibility of identifying causal relationships is sometimes lacking in 
dreams. /t is not just that actions have unexpected effects. It is that events 
are strung together like beads on a string. There is no sense of willful acting 
on the world or manipulation of the world which can be objectified as a 
causal relation between impersonal! events. 

The possibility arises of using dreams as philosophical experiments in 
worlds in which one or more of the preconditions for application of the 
scientific method is absent. (But in the one case in which Alten and I tried 
this, we reached opposite conclusions. Alten said that dreams in which one 
can jump around in time proved that the irreversibility of time is the basis 
for distinguishing between time and space; I said that the dreams proved that 
time and space can be distinguished even when the irreversibility of time is 
lacking.) 

Observation (2) above can lead us to an insight about the waking world. 
Perhaps science insists on the elimination of personal anecdotes from the 
natural order which it recognizes because the scientist wants results which 
can be transferred from one life to another and which will give one person 
power over another. At any rate, science excludes anecdotal anomalies which 
cannot be made somehow into "objective" events. As an example, I may be 
walking down the street and suddenly find myself on the other side of the 
street with no awareness of any act of crossing the street. 

What dreams provide us with is worlds in which anecdotal anomalies 
cannot be relegated to limbo as they are in waking science. They are so 
prominent in dreams that we can become accustomed to identifying them 
there. We may then learn to recognize analogous anomalies in the waking 
world, where we had overlooked them before because of our scientific 
indoctrination. 

Of course, we run the risk that superstitious people will misuse our 
theory to justify their folly. But the difference between our theory and 
superstition is clear. When the superstitious person says that he 
communicates with spirits, he either lies outright; or alse he misinterprets his 
experiences--embedding them in an extraneous pre-scientific belief system, 
or treating them as controversions of scientific propositions. We, on the 
other hand, maintain more literally than science does that the only raw 
material of cognition is experience. We differ from science in draping 
experiences on a different organizational framework. The "reality" we arrive 
at is incommensurate with science; it does not falsify any scientific 
proposition. As for science and superstition, we headed this essay with the 


168 


quotation by d'Abro to emphasize that the scientist himself is superstitious: 
he is determined to believe in the common objective world, even though it is 
a fiction, because it is necessa~y to science. The superstitious person wants 
you to believe that his communication with spirits is intersubjectively 
consequential. Thus our theory, which tends toward the attitude that 
nothing is intersubjectively consequential, offers him even less comfort than 
science does. 

D. We next turn to semi-conscious quasi-dreams. Referring to my 
experience on the morning of 1/12/1974, I describe the experience by saying 
that I was on the Courant Institute plaza. But I cannot conclude that I was 
on the Courant Institute plaza. The reason is that important internal reality 
cues are missing in the experience. For one thing, the peripheral environment 
is missing; in its place is a void. Referring to my experience on 1/1-/1974, 
still other cues are missing. I am awake, and the scene is unstable and 
momentary. The slightest attention shift will cause the scene to vanish. 

When we recognize that we have disallowed falling asleep, awaking, and 
anomalous phenomena in dreams as evidence of unreality, a careful analysis 
yields only two types of reality cues. 

1. Presence of the peripheral environment. 

2. "Single consciousness." This cue is missing when we see a 
three-dimensional scene and move about in it, and yet have a background 
awareness that we are awake in bed; and lose the scene through a mere shift 
of attention. Its absence is even more marked if the scene is a momentary 
one between two waking periods. 

Let us recall our earlier discussion of the empiricism of science. Science 
does not content itself with ungraded experience. it drapes experience on an 
intellectual framework in such a way as to simulate Aristotelian realism. It 
feeds experience into a maze of verification procedures in order to confirm a 
model which is not explicit in ungraded experience. It short, science grades 
experience as to its reality on the basis of standards which are 
"intellectually" supplied. Internal reality cues are thus characteristics of 
experience which are given special weight by the grading procedure. The 
immediate problem for us is that ordinary descriptive language implicitly 
recognizes these reality cues; one would never say without qualification that 
one was on the Courant Institute plaza if the peripheral environment was 
missing and if one was also aware of being awake in bed at the time. (In 
contrast, it is fair to use ordinary descriptive language with respect to 
dreamed episodes when our consciousness is singulary, that is, when 
everything seems real and unqualified.) - 

For purposes of further comparison !«may mention an experience I 
have had on rare occasions while lying on my back in bed fully awake. It is 


169 


as if colored spheres whosé centers are located a few feet or yards in front of 
my chest expand until they press against me, one after the other. I use the 
phrase "as if' because reality cues are missing in this experience, and thus I 
cannot use the language of stable object gestalts without qualification in 
describing it. The colors are not vivid as real colors are. They are like 
visualized colors. The spheres pass through each other, and through me--with 
only a moderate sensation of pressure. I can turn the experience off by 
getting out of bed. The point, again, is that it is inherent in ordinary 
language not to use unqualified object descriptions in these circumstances. 
Yet the only language I have for such sensory configurations is the language 
of stable object gestalts-this is particularly obvious in the example of the 
Courant Institute plaza. (Is "ringing in the ears' in the same class of 
phenomena? } 

An insight that is crucial in elucidating this problem is that when I 
describe episodes, the descriptions implicitly convey not only sensations but 
beliefs, as when I speak of a typewriter in a dream on the assumption that it 
persisted while I was not looking at it. The peculiar quality of a quasi-dream 
comes about not only because it is an anomaly in my sensations but because 
it is an anomaly in the scientific-pragmatic cognitive model which underlies 
ordinary language. If I discard this cognitive model and then report the 
event, it will not be the same event: the beliefs implicit in ordinary language 
helped give the event its quality. As a further example, now that I have 
recognized experiences such as that of 1/12/1974, I am willing to entertain 
the possibility that they are the basis for claims by superstitious persons to 
have projected astrally. But to use the phrase "astral projection" is to embed 
the experiences in a_ pre-scientific belief system extraneous to the 
experiences themselves. !f we learn to report such experiences by using 
idioms like "ringing in the ears" and blocking any comparison with notions 
of objective reality or intersubjective import, we will have flattened out 
experience and will have moved in the direction of ungraded experience and 
nihilistic empiricism. 

E. We next take up connections between adjacent dreamed and waking 
periods. As a preliminary, we reject conventional notions that dreams are 
fabricated from memories of waking reality; or that dreams are precognitions 
of waking reality; or that dreams are mental phenomena which symbolize 
waking reality. We reject these notions because they conflict with the placing 
of the dream world on the same level as the waking world. 

Connections between dream and waking periods are important in this 
study because we may wish to create such connections deliberately, and even 
to attribute causal significance to them. Initially, we define the concept of 
dream control: it is to conduct one's waking life so that it is supportive of 


170 


one's dreamed life in some sense. We also define controlled dreaming: it is to 
manipulate a person "from outside" before sleep {or during sleep) so as to 
influence the content of that person's dreams. (An example would be to give 
somebody a psychoactive sleeping pill.) 

A careful analysis of connections between dream and waking periods 
yields the following classification of such connections. 

1. I walk around the kitchen in a dream, then awaken and walk around the 
kitchen. Voluntary continued action. 

2. Given a_ project with causally separate components, voluntarily 
assembled, I can carry out the project entirely while awake, entirely in 
dreams, or partly while awake and partly in dreams. 

3. I walk around the kitchen while awake, then sleep. I may then walk 
around the kitchen in a dream. Also, I draw a glass of water while awake. I 
may have the glass of water to use in the dream. We could postulate that 
such connections are not mere coincidences, if they occur. However, we 
certainly cannot produce such connections at will. We call these connections 
echoes of waking actions in dreams. Note the case in which I taped my 
mouth shut before sleeping, and could not whistle in the subsequent dream. 
4. We next have connections from dreamed to waking periods which can be 
postulated to have causal significance. First, misfortune or danger in dreams 
is regularly followed by immediate awaking. Secondly, I! have had 
experiences in which a headlong dive or an attempt to whistle continued 
from dream to waking, right through waking up. These experiences are 
causally continuous actions. However, I cannot bring them about at will. 

5. We can manipulate a person "from outside" before sleep (or during sleep) 
so as to influence the content of that person's dreams. The dream is not an 
echo of the waking action; the causal relationship is manipulative. Examples 
are to give someone a psychoactive sleeping drug or to create a special 
environment for sleep. The case in which I taped my mouth shut before 
sleeping was a remarkable borderline case between an echo and a 
manipulation. 

in conclusion, dream control is any of the connections described in 
(1)-(4). Controlled dreaming is (5). We have analyzed these concepts 
meticulously because we want to exclude all attempts at magic, all 
superstition from the project of placing dreamed and waking life on the same 
level. There must be no rain dancing, no false causality, in this project. 

F. Until now, we have analyzed our experience episode by episode. We 
could make this approach into a principle by assuming that each episode is a 
separate and complete world, which has its reality confirmed internally. In 
particular, the notion of objective location in space and time would be 
maintained if it appeared in a dream and was intersubjectively confirmed in 


171 


the dream, but the notion would be purely internal to each episode. The 
objection to these assumptions, as we mentioned at the end of (B), is that 
they propose to maintain the notion of objective location, and yet they 
forego the notion of the consistency of all portions of reality. if we adopt 
these assumptions and then compare all the reports of our dreamed and 
waking periods, we may find that we have experienced different events 
attributed to the same location--and indeed, that is exactly what we do 
experience. 

One of the main discoveries of this essay has been that dreamed and 
waking periods are more symmetrical than our scientific-pragmatic 
indoctrination would have us suppose. The reality of the dream world is 
intersubjectively confirmed--within the dream. Anecdotal anomalies can be 
found in waking periods as well as in dreams. Entities which resemble 
common object gestalts but which lack some of the reality cues of object 
gestalts can be encountered whicle we are fully awake. Now we can 
recognize a further symmetry between dreamed and waking life. A dreamed 
misfortune is usually "lost" when we awaken, and its disappearance is taken 
as evidence of the unreality of the dream (the nightmare). But we can also 
"lose" a waking misfortune by going to sleep and dreaming. Further, just as 
a waking misfortune can persist from one waking period to another, a 
dreamed misfortune can persist from one dream to another (recurrent 
nightmares). Thus, we conclude that in regard to the consistency of episodes 
with each other, there is no basis for preferring any one episode, dreamed or 
waking, as the standard by which the reality of other episodes will be judged. 
Of course, rather than maintaining the reality of each episode as a separate 
world, we can block all attributions of events to objective locations. This 
approach would alter the quality of the events and bring us closer to 
nihilistic empiricism. 

A further problem arises if we take the dream reports of other people as 
reports of reality. Suppose I am awake in my apartment at 3 AM on 
2/6/1974, but that someone dreams at that time that I am out of my 
apartment. Multiple existences which I do not even experience are now being 
attributed to me. (My own episodes also pose a problem of whether 
"multiple existences" are being attributed to me, but that problem concerns 
events I experience myself.) What we should recognize is that the problem of 
"multiple existences" is not as unique to our investigation as may at first 
appear. Natural science has an analogous problem in disposing of the notion 
of other minds. The notion of the existence of many minds, none of which 
can experience any other, is difficult to assimilate to the cognitive model of 
science. On the other hand, to deny the existence of any mind, as 
behaviorists do, is to repudiate the scientist's observations of his own mental 


172 


life. And if the scientist's observations of his own mental life are repudiated, 
then there is no good reason not to repudiate the scientist's observations of 
his budily sensations and of external phenomena also; that is, to repudiate 
the very possibility of scientific observation. Further, when behaviorists try 
to convince people that they have no awareness, whom (or what) are they 
trying to convince? And what is the behaviorist explanation of the origin of 
the fiction of consciousness? Who benefits from perpetuating this fiction, 
and how does he benefit? 

We must emphasize that the above critique is not applicable to every 
philosophical outlook. It applies specifically to science-- because the scientist 
wants to have the benefits of two incompatible conceptual frameworks. 
Some of the common sense about other minds is necessary in the operational 
preliminaries to formal science; and the scientist's role as observer is 
indispensable to formal science. Yet the conceptual framework of science is 
essentially physicalistic, and can allow only for external objects. What this 
difficulty reveals is that the cognitive model of science has stabilized and 
prevailed even though it has blatent discrepancies in its foundations. The 
foremost discrepancy, of course, is that the scientist is willing to have his 
enterprise rest on a fiction, that of the common objective world. Thus, the 
example of science suggests an additional way of dealing with the problems 
which arise for our theory: we can allow discrepancies to persist unresolved. 

There is an interesting observation to be made about one's own dreams 
in connection with multiple existences. I have found that the person I am in 
my dreams is significantly different from the waking identity I take for 
granted, as in my dream of 2/1/1974. As for the problem of other people's 
dreams, one way of handling them would be simply to reject the existence of 
other people's dream worlds and of their consciousnesses, and to limit one's 
consideration to one's own dreams. But perhaps the most productive way to 
handle the problem would be to construe it as one involving language in the 
way that the problems concerning quasi-dreams did. Our descriptive language 
is a language of stable object gestalts, of scientific-pragmatic reality. If we 
accept reports of other people's dreams in language which blocks any 
implications concerning objective reality, then our perceptual interpretations 
will be different and the quality of the events will be fundamentally 
different. The experience-world will be flatter. But maybe this is a 
revolutionary advance. Maybe reports of our appearances in other people's 
dreams, in language which blocks any implications about reality, are what we 
should strive for. And if ve cease to be stable object gestalts for others, 
maybe our stable object gestalts will not even appear in their dreams. 


Note on how to remember dreams 

The trick in remembering a dream is to fix in your mind one incident or 
theme in the dream immediately upon awaking from it. You will then be 
able to remember the whole dream well enough to write a description of it 
the next day, and you will probably find that for weeks afterwards you can 


add to the description and correct it. 


174 


SOCIAL PHILOSOPHY 


ey 


16. On Social Recognition 


The most important tasks which the individual can undertake arise not 
from personal considerations but from the general conditions of society. The 
standards of accomplishment for these tasks are implicit in the tasks, and are 
objective in the sense that they can be applied without reference to public 
opinion. For example, given that humans express themselves in statements 
which are supposedly true or false, there arises a fundamental philosophical 
"problem of knowledge." Then, the fact that societies are organized in 
different ways at different times and places poses fundamental problems of 
"political" thought and action. Sometimes the most important task posed by 
the conditions of society is to invent a whole new activity. The origination 
of experimental science in Europe in the seventeenth century is an example. 
For lack of a better term, these tasks will be referred to as 'fundamental! 
tasks." 

The fact that a fundamental task is posed by the general conditions of 
society does not mean that public opinion will be aware of the task, or that 
the ruling class will commission someone to undertake it. It may well be that 
the first person to perceive the problem is the person who solves it; and 
public opinion may not catch up with him for decades or centuries. 

The person who devotes himself to a fundamental task is, more often 
than not, persecuted or ignored by society. Society puts up an immense 
resistance to solutions of fundamenta! problems, even when, as in the cases 
of Galois and Mendel, those solutions are politically innocuous. There is no 
evidence that this state of affairs is limited to some particular organization of 
society. Further, there are cases in which an objectively valid result is 
known, and yet apparently society can never adopt the result institutionally. 
Art is objectively inferior to brend, as I have shown, and yet all indications 
are that art will always be a major institution. The persecution of individuals 
who undertake fundamental tasks is an instance of a general human social 
irrationality which runs throughout history, from human sacrifice in ancient 
times to present-day war between communist countries. The conclusion is 
that for an individual to commit himself to a fundamental task tends to 
preclude social approval for his activities. 

Quite apart from the fundamental tasks which are posed by general 
social conditions, the ruling class needs a continual supply of new talent at 


177 


al! levels of society. At the lower levels, this supply is assured by the 
necessity of selling one's labor power in order to eat. At the higher levels of 
accomplishment, the ruling class assures itself of a continual supply of new 
talent by offering publicity or fame--social recognition--as a reward for 
accomplishing the tasks specified by the ruling class. Famous men such as 
Einstein are held up to children as examples of the proper relationship 
between the talented individual and society; and an internationa! institution, 
the Nobel Prize, exists to implement this system of supplying talent. 
According to the doctrine, the individual has a duty to benefit society, to 
choose a task posed by the ruling class as his occupation. (His publicly 
known occupation is supposed to correspond to his real goals.) If he 
performs successfully, he will receive publicity as an indication that he is 
indeed benefiting society. 

Our analysis of fame is the opposite of that of Ben Vautier. Vautier 
asserts that the desire for personal publicity is an instinctive drive of human 
beings, and that the accumulation of publicity is a genuinely selfish act like 
the accumulation of food. In fact, Vautier goes so far as to make no 
distinction between what Gypsy Rose Lee and Lenin, for example, did to 
gain fame; and he assumes that a pacifist, for example, would welcome 
military honors equally as much as he would a peace award. We assert, on 
the contrary, that the desire for publicity is not instinctive; it is inculcated in 
the young so that the ruling class may have a continual supply of new talent 
to serve its purposes. The desire for publicity, far more than the desire for 
money, is establishment-serving more than self-serving. (We suggest that the 
principal reason why Vautier seeks publicity is not instinct, but economics. 
Vautier has no inherited source of income, and has never been trained for a 
profession. For him, the alternative to the art/publicity racket would be 
common labor. !f he had the opportunity for a life of leisure, he might feel 
differently about publicity.) 

The issues which are raised here are extremely important for the person 
who perceives a fundamental task, because his sanity may depend on 
whether he understands the rationality of his motives for undertaking the 
task. He will already have been inculcated with the establishment's concepts 
of service and recognition, concepts which are epitomized in the image of 
Einstein's career. What we suggest is that it is vital to disabuse oneself of 
these concepts. To repeat, fundamental tasks are posed by the general 
conditions of society. Yet the individual who undertakes such a task will 
probably be persecuted or ignored. Given these circumstances, the doctrine 
that the individual has a duty to benefit society is a hypocritical fraud, an 
obscenity. For the individual to commit himself to a fundamental task tends 
to preclude social recognition for his activities; or, to reverse the remark, 


178 


social recognition is not a reward to accomplishment of a fundamental task 
(just as military honors are not a reward to pacifism). Thus, it is not rational 
for the individual to undertake a fundamental task in order to gain fame. 

The motive for undertaking a fundamental task should be genuine 
selfishness. (We will continue our argument that the striving for fame is not 
genuinely selfish below.) The individual who perceives a fundamental task 
should undertake it for his private gratification. The task is of primary 
importance to society. By accomplishing it, the individual gains the privilege 
of knowing something which is socially important, but which society cannot 
deal with honestly. The individual should undertake the task in order to 
utilize his real abilities, to develop his potentiality for its own sake. The 
undertaking of a significant task which utilizes one's real abilities is the true 
source of happiness. To perceive a fundamental task and not to undertake it 
is to be stunted: one loses one's self-respect and becomes progressively 
demoralized. (Another rational motive for undertaking a fundamental task is 
to transform the social environment by methods which do not depend on 
society's approval or comprehension.) 

We do not mean to suggest that the individual who undertakes a 
fundamental task should conceal his results. Even though such tasks may 
seem individualistic, they require cooperative, social activity for their 
accomplishment. A proposed solution to a fundamental problem can hardly 
develop without being scrutinized from a variety of perspectives. It is 
essential to have qualified critics, and it is unfortunate that they are so rare. 
Solutions to fundamental problems are social consumption goods (their 
consumption is not exclusionary), so that critics or collaborators have as 
much opportunity to benefit from them as their originators do. As an 
example, most of my writings are really collaborations with Tony Conrad. I 
often find that I do not understand my own position until I know how it 
appears to him. When communication of results is essentially a form of 
collaboration, it is very different from the attempt to gain publicity or fame. 

It is precisely in the context of the generalized social irrationality which 
runs throughout history that the attempt to gain fame must be seen as 
foolishly un-selfish. What difference can it possibly make whether the masses 
venerate one's name a hundred years after one's death? The adulation of the 
masses after one is dead is of no conceivable value to oneself. It is society 
which indoctrinates one to worry about one's reputation after one is dead, in 
order to condition one to serve the interests of the ruling class. 

Then, what does it mean to the individual who solves a fundamental 
problem to have his name publicized in the mass media, to be a celebrity 
among people who cannot possibly understand what he has done? Even 
more important, we must recognize that publicity carries a definte risk for 


179 


the individual committed to a fundamental task. The solution of such a 
problem must usually be expressed in categories which are incommensurate 
and incompatible with the categories of thought which are common coin at 
the time. In order for the solution of a fundamental! problem to be exposed 
in the mass media, it has to be translated into media categories and this 
usually results in irreparable distortion. In fact, the solution is distorted in 
precisely such a manner that it begins to serve the interests of the ruling 
class. One encounters an immense pressure which tends to harness one to 
goals which have nothing to do with objective value. More precisely, when an 
individual who has solved a fundamental problem is publicized in the mass 
media, a process of mutual subversion takes place as between the 
establishment/media and the individual. In the process, the establishment is 
likely to come out far ahead. 

There are two other reasons why it is actually advantageous to the 
individual who undertakes a fundamental! task to avoid publicity. Since one's 
activity is likely to be treated as a threat by society, one can minimize the 
energy required to defend it, and can carry the activity further, if one 
receives no publicity. Then, there will unavoidably be false starts made in 
developing the solution to a fundamental problem. If one is not operating in 
the glare of publicity, it is far easier to abandon these false starts. 

It used to be that when I saw publicity being given to an inferior way of 
doing a thing, and I knew a better way, then I reacted with a sense of duty. I 
had to appoint myself as a missionary, to enter the public arena and start a 
campaign to replace the inferior approach with the better approach. But this 
sense of duty must now be called into question. Is it really in my interest to. 
thrust myself on the media as a missionary? The truth is that in the context 
of generalized social irrationality, it is un-selfish and self-sacrificing to believe 
that I must either agree with current fads or else contest them publicly. The 
genuinely selfish attitude is *hat it is sufficient for me to know what the 
superior approach is. I can ignore the false issues which fill the mass media; I 
do not have to participate in public opinion at all. The genuinely selfish 
attitude is that "it does not concern me." Genuine selfishness is living one's 
life on a level which does not communicate with the level of the mass media 
and public opinion. 

If we recognize that it is irrational to undertake a fundamental task in 
order to benefit society and gain social approval, then our very choice of 
fundamental tasks shouid be affected. The most visible fundamental tasks 
are those which the establishment is to some extent aware of, and which if 
accomplished would immediately be rewarded with social approval. (In the 
natural sciences, there literally may be a race to solve a well-known problem). 
But if our motives are genuinely self-serving, and have to do with the 


180 


development of our potentiality for its own sake, then there is no reason to 
limit ourselves to widely understood problems. We can undertake to discover 
timeless results--permanent answers to questions which will be important 
indefinitely--without concerning ourselves with whether society can adopt 
the results institutionally. We can pose problems of which neither the 
establishment, the media, nor public opinion are aware. We can undertake 
tasks which draw on our unique abilities, so that our personal contribution is 
indispensable. 

There is a difficulty which we have postponed mentioning. The 
individual is always compelled to engage in some socially approved activity 
in order to obtain the means of subsistence. We cannot assume that the 
individual will have an inherited source of income. In order to pursue a 
fundamental task, he will have to pursue a legitimate occupation at the same 
time. It may be extremely difficult to lead such a double life, because to do 
so requires precisely the self-assurance. that comes from accomplishing the 
fundamenta! task. Leading a double life is not a game for the person who is 
unsure about his real abilities or his vocation. If the individual is capable of 
leading a double life, our suggestion is to obtain the means of subsistence by 
the most efficient swindle available. Do not hesitate to practice outward 
conformity in order to exploit the establishment for your own purposes. 

There remains the case of the individual who, like Galois, is not 
prepared to lead a double life. His problem is one of destitution. However, 
he is different from an ordinary pauper. By assumption, he is more talented 
than the members of the establishment; he does not belong to the 
establishment because he is overqualified for it. Given that he is more 
talented than members of the establishment, and that his survival is 
threatened, a collateral fundamental task emerges, the task of immediately 
transmuting his talent into power to handle the establishment on his own 
terms. To perceive this task is a major resuit of this essay. The task cannot be 
defined accurately without a perfect understanding of the difference 
between fundamental tasks and the serve-society-and-get-famous fraud. We 
contend that Galois should have regarded the task of immediately 
transmuting his talent into power over the establishment as an inseparable 
collateral problem to his mathematical researches. From a common sense 
point of view, this collateral task will seem utterly impossible. However, we 
are talking about individuals whose vocation is to do the seemingly 
impossible. Thus, we conclude by leaving this unsolved fundamental problem 
for the reader to ponder. 


181 


17. Creep 


When Helen Lefkowitz said I was "such a creep" at Interlochen in 
1956, her remark epitomized the feeling that females have always had about 
me. My attempts to understand why females rejected me and to decide what 
to do about it resulted in years of confusion. In 1961-1962, I tried to 
develop a theory of the creep problem. This theory took involuntary 
celibacy as the defining characteristic of the creep. Every society has its 
image of the ideal young adult, even though the symbols of growing up 
change from generation to generation. The creep is an involuntary celibate 
because he fails to develop the surface traits of adulthood--poise and 
sophistication; and because he is shy, unassertive, and lacks self-confidence 
in the presence of others. The creep is awkward and has an unstylish 
appearance. He seems sexless and childish. He is regarded by the ideal adults 
with condescending scorn, amusement, or pity. 

Because he seems weak and inferior in the company of others, and 
cannot maintain his self-respect, the creep is pressed into isolation. There, 
the creep doesn't have the pressure of other people's presence to make him 
feel inferior, to make him feel that he must be like them in order not te be 
inferior. The creep can develop the morale required to differ. The creep also 
tends to expand his fantasy life, so that it takes the place of the 
interpersonal life from which he has been excluded. The important 
consequence is that the creep is led to discover a number of positive 
personality values which cannot be achieved by the mature, married adult. 
During the period when I developed the creep theory, I was spending almost 
all of my time alone in my room, thinking and writing. This fact should 
make the positive creep values more understandable. 

1. Because of his isolation, the creep has a qualitatively higher sense of 
identity. He has a sense of the boundaries of his personality, and a control of 
what goes on within those boundaries. In contrast, the mature adult, who 
spends all his time with his marriage partner or in groups of people, is a mere 
channel into which thoughts flow from outside; he lives in a state of 
conformist anonymity. 

2. The creep is emotionally autonomous, independent, or 
self-contained. He develops an elaborate world of feelings which remain 


182 


within himself, or which are directed toward inanimate objects. The creep 
may cooperate with other people in work situations, but he does not develop 
emotional attachments to other people. 

3. Although the creep's intellectual abilities develop with education, 
the creep lives in a sexually neutral world and a child's world throughout his 
life. He is thus able to play like a child. He retains the child's capacity for 
make-believe. He retains the child's lyrical creativity in regard to 
self-originated, self-justifying activities. 

4. There is enormous room in the creep's life for the development of 
every aspect of the inner world or the inner life. The creep can devote 
himself to thought, fantasy, imagination, imaging, variegated mental states, 
dreams, internal emotions and feelings towards inanimate objects. The creep 
develops his inner world on his own power. His inner life originates with 
himself, and is controlled and intellectually consequential. The creep has no 
use for meditations whose content is supplied by religious traditions. Nor has 
he any use for those drug experiences which adolescents undertake to prove 
how grown-up they are, and whose content is supplied by fashion. The 
creep's development of his inner life is the summation of all the positive 
creep values. 

After describing these values, the creep theory returned to the problem 
of the creep's involuntary celibacy. For physical reasons, the creep remains a 
captive audience for the opposite sex, but his attempts to gain acceptance by 
the opposite sex always end in failure. On the other hand, the creep may 
well find the positive creep values so desirable that he will want to intensify 
them. The solution is for the creep to seek a medical procedure which will 
sexually neutralize him. He can then attain the full creep values, without the 
disability of an unresolved physical desire. 

Actually, the existence of the positive creep values proves that the 
creep is an authentic non-human who happens to be trapped in human social 
biology. The positive creep values imply a specification of a whole 
non-human: social biology which would be appropriate to those values. 
Finally, the creep theory mentioned that creeps often make good grades in 
school, and can thus do clerical work or other work useful to humans. This 
fact would be the basis for human acceptance of the creep. 

In the years after I presented the creep theory, a number of 
inadequacies became apparent in it. The principal one was that I managed to 
cast off the surface traits of the creep, but that when I did my problem 
became even more intractable. An entirely different analysis of the problem 
was required. 

My problem actually has to do with the enormous discrepancy between 
the ways I can relate to males and the ways I can relate to females. The 


183 


essence of the problem has to do with the social values of females, which are 
completely different from my own. The principal occupation of my life has 
been certain self-originated activities which are embodied in "writings." Now 
most males have the same social values that I find in all females. But there 
have always been a few males with exceptional values; and my activities have 
developed through exchanges of ideas with these males. These exchanges 
have come about spontaneously and naturally. In contrast, I have never had 
such an exchange of ideas with females, for the following reasons. Females 
have nothing to say that applies to my activities. They cannot understand 
that such activities are possible. Or they are a part of the "masses" who 
oppose and have tried to discourage my activities. 

The great divergence between myself and females comes in the area 
where each individual is responsible for what he or she is; the area in which 
one must choose oneself and the principles with which one will be identified. 
This area is certainly not a matter of intelligence or academic degrees. 
Further, the fact that society has denied many opportunities to females at 
one time or another is not involved here. (My occupation has no formal 
prerequisites, no institutional barriers to entry. One enters it by defining 
oneself as being in it. Yet no female has chosen to enter it. Or consider such 
figures as Galileo and Galois. By the standards of their contemporaries, these 
individuals were engaged in utterly ridiculous, antisocial pursuits. Society 
does not give anybody the "opportunity" to engage in such pursuits. Society 
tries to prevent everybody from being a Galileo or Galois. To be a Galileo is 
really a matter of choosing sides, of choosing to take a certain stand.) 

Let me be specific about my own experiences. When I distributed the 
prospectus for The Journal of Indeterminate Mathematical Investigations to 
graduate students at the Courant Institute in the fall of 1967, the most 
negative reactions came from the females. The mere fact that I wanted to 
invent a mathematics outside of academic mathematics was in and of itself 
offensive and revolting to them. Since the academic status of these females 
was considerably higher than my own, the disagreement could only be 
considered one of values. 

The field of art provides an even better example, because there are 
many females in this field. In the summer of 1969 I attended a meeting of 
the women's group of the Art Workers Coalition in New York. Many of the 
women there had seen my Down With Art pamphlet. Ail the females who 
have seen this pamphlet have reacted negatively, and it is quite clear what 
their attitude is. They believe that they are courageously defending modern 
art against a philistine. They consider me to be a crank who needs a "modern 
museum art appreciation course." The more they are pressed, the more 
proudiy do they defend "Great Art." Now the objective validity of my 


184 


opposition to art is absolutely beyond question. To defend modern art is 
precisely what a hopeless mediocrity would consider courageous. Again, it is 
clear that the opposition between myself and females is in the area where 
one must choose one's values. 

I have found that what I really have to do to make a favorable 
impression on females is to conceal or suspend my activities--the most 
important part of my life; and to adopt a facade of conformity. Thus, I 
perceive females as persons who cannot function in my occupation. I 
perceive them as being like an employment agency, like an institution to 
which you have to present a conformist facade. Females can he counted on to 
represent the most "social, human" point of view, a point of view which, as I 
have explained, is distant from my own. {In March 1970, at the Institute for 
Advanced Study, the mathematician Dennis Johnson said to me that he 
would murder his own mother, and murder ail his friends, if by doing so he 
could get the aliens to take him to another star and show him a higher 
civilization. My own position is the same as Johnson's.) 

It follows that my perception of sex is totally different from that of 
others. The depictions of sex in the mass media are completely at variance 
with my own experience. I object to pornography in particular because it is 
like deceptive advertising for sex; it creates the impression that the physical 
aspect of sex can be separated from human personalities and social 
interaction. Actually, if most people can separate sex from personality, it is 
because they are so average that their values are the same as everybody else's. 
In my case, although I am a captive audience for females for physical 
reasons, the disparity between my values and theirs overrides the physical 
attraction I feel for them. It is hard enough to present a facade of 
conformity in order to deal with an employment agency, but the thought of 
having to maintain such a facade in a more intimate relationship is 
completely demoralizing. 

What conclusions can be drawn by comparing the creep theory with my 
later experience? First, some individuals who are unquestionably creeps as 
far as the surface traits are concerned simply may not be led to the deeper 
values I described. They may not have the talent to get anything positive out 
of their involuntary situation; or their aspirations may be so conformist that 
they do not see their involuntary situation as a positive opportunity. Many 
creeps are female, but all the evidence indicates that they have the same 
values I have attributed to other females--values which are hard to reconcile 
with the deeper creep values. 

As for the positive creep values, I may have had them even before I 
began to care about whether females accepted me. For me, these values may 
have been the cause, not the effect, of surface creepiness. They are closely 


185 


related to the values that underlie my activities. It is not necessary to appear 
strangely dressed, childish, unassertive, awkward, and lacking in confidence 
in order to achieve the positive creep values. (1 probably emphasized surface 
creep traits during my youth in order to dissociate myself from conformist 
opinion at a time when I hadn't yet had the chance to make a full 
substantive critique of it.) Even sex, in and of itself, might not be 
incompatible with the creep inner life; what makes it incompatible is the 
female personality and female social values, which in real life cannot be 
separated from sex and are the predominant aspect of it. 


Having cast off the surface traits of the creep, I can now see that 
whether I make a favorable impression on females really depends on whether 
I conceal my occupation. Celibacy is an effect of my occupation; it does not 
have the role of a primary cause that the creep theory attributed to it. 
However, it does have consequences of its own. In the context of the entire 
situation I have described, it constitutes an absolute dividing line between 
myself and humanity. It does seem to be closely related to the deeper creep 
values, especially the one of living in a child's world. 

As for the sexual neutralization advocated in the creep theory, to find a 
procedure which actually achieves the stated objective without having all 
sorts of unacceptable side effects would be an enormous undertaking. It is 
not feasible as a minor operation developed for a single person. Further, as 
the human species comes to have vast technological capabilities, many 
special interest groups will want to tinker with human social biology, each in 
a different way, for political reasons. I am no longer interested in petty 
tinkering with human biology. As I make it clear in other writings, I am in 
favor of building entities which are actially superior to humans, and which 
avoid the whole fabric of human biosocial defects, not just one or two of 
them. 


186 


2/22/1963 
Henry Flynt and Jack Smith demonstrate against Lincoln Center, February 22, 1963 
(photo by Tony Conrad) 


18. The Three Levels of Politics 


Political activity and its results can occur on three levels. The first level 
is the personal one. An individual may vote to re-elect a local politician 
because of patronage he has received, for example. On this level the 
individual's motivation is narrow, immediate self-interest. Often the action 
has a defensive character; the individual is trying to hold on to something he 
already possesses. 

The second level may be called the historical level. It is exemplified by 
the Civil War in the United States. Certain political movements result in 
largescale, irreversible social change. The Civil War set in motion the 
industrialization of the United States, as well as abolishing slavery. In 1860, 
slavery was viewed by large numbers of Americans as a legitimate institution. 
One hundred years later, even American conservatives did not often defend 
it. To re-establish a plantation economy in the South today would be out of 
the question. These observations prove that on the second level, society 
really does change. On this level, political action does make a difference. 

However, there is a further aspect to the Civil War which indicates that 
politics does not make the difference people think it makes. According to 
the ideology of the abolitionists, the accomplishment of the Civil War would 
be to raise the slaves to a position of equality with whites. In fact, nothing of 
the sort happened. The real accomplishment of the Civil War was to 
transform the United States into an industria! capitalist society (and to 
abolish an institution which was incompatible with the capitalists' need for a 
free labor market). By the time the Northern businessmen brought 
Reconstruction to an end, it was clear that the position of blacks in 
American society was where it had always been: at the bottom. The Civil 
War changed American society, but is did not make the society any more 
utopian. On the contrary, it brought into prominence still another violent 
social conflict--the conflict between labor and capital. 

The third level of politics has to do with the utopian aspect of modern 
political ideologies, the aspect which calls not only for society to change, but 
to change for the better. Typical third-level political goals are the abolition 
of war, the abolition of the oligarchic structure of society, and the abolition 
of economic institutions which value human lives in terms of money. in all 
of human history, society has never changed on this third level. 


188 


The successful Communist revolutionists of the twentieth century (in 
the underdeveloped countries) have repeatedly claimed to have accomplished 
third-level change in their societies. However, these claims of third-level 
change have always turned out to be illusions which cover a recapitulation of 
capitalist development. Communist revolutions are typical examples of real 
second-level change which is accomplished under the cover of claims of 
third-level change, claims which are pure and simple frauds. 

By -introducing the concept of levels of politics, we can resolve the 
apparent paradox that society certainly changes, but that it really does not 
change. It is important to understand that empirical evidence on the 
question of the levels of politics can only be drawn from the past, the 
present, and the immediate future (five to ten years). Recent technological 
developments have brought into question the very existence of the human 
species. In addition, technology is developing much faster than society is. It 
is meaningless to discuss the issue of second versus third-level social change 
with reference to the more distant future, because there may not be any 
human society in the more distant future. 

This essay is concerned with the politics of the third level. The first and 
second levels are certainly rea! enough, but we are not the least interested in 
them. As we have just said, we make the restriction that any empirical 
analysis of the third level must refer to the past, the present, or the 
immediate future. Our purpose is to present a substitute for the politics of 
the third level. 

There are a number of present-day political tendencies which hold out 
the promise of third-level social change. These tendencies are all descended 
from the leftist working-class movements of nineteenth century Europe, 
most of them by way of the early Soviet regime. The promises of third-level 
change held out by these tendencies are nothing but cheap illusions. What is 
more, a careful examination of leftist ideologies in relation to the historical 
record will show that the promises of third-level change are extremely vague 
and without substance. Beneath the surface of vague promises, leftist 
ideologies do not even favor third-level change; they are opposed to it. 

One example will serve to demonstrate this contention. In my capacity 
as a professional economist, I have become familiar with the official 
economic policies--the doctrines of the professional economists--of the 
various socialist governments and leftist movements throughout the world. It 
should be mentioned that most of the followers of leftism are not familiar 
with these technical economic policies; they are aware only of vague, 
meaningless promises of future bliss coming from leftist political 
speechmakers. When we turn to technical economic realities, we find that 
virtually every leftist tendency in the world today accepts economic 


189 


principles which in the parlance of the layman are referred to as 
"capitalism." The most important principle is stated by Ernest Mandel: "the 
economy continues to be fundamentally a money economy, with the 
satisfaction of the bulk of people's needs depending on the number of 
currency tokens a person possesses." When it comes to the realities of 
technical economics, virtually every leftist in the world accepts this 
principle. So far as the third level is concerned, there is no such thing as a 
non-capitalist polical tendency, and there is no point in hoping for one. A 
similar conclusion holds for virtually every aspect of third-level politics. 
Leftists claim that Communism eliminates the causes of war; while at the 
same time war breaks out beween China and the Soviet Union. 

We propose to draw a far-reaching conclusion from these 
considerations. Returning to the example of first-level politics, it is rational 
for the patronage-seeker to be in favor of the election of one focal politican 
and against the election of his opponent. This is a matter which is within the 
scope of human responsibility, and with respect to which individual action 
can make a difference. But it is not rationa! to be either for against 
"capitalism," to be either for or against war. As we have seen, "capitalism" 
and war are permanent aspects of human society, and no political tendency 
genuinely opposes them. {t is meaningless to treat them as if they were 
within the scope of human responsibility in the sense that the election of a 
local politician is. in other words, the third-level aspects of society are not 
partial, limited aspects which can be eliminated by conscious human action 
while the bulk of human life is retained. The only way you can meaningfully 
be against the third-level aspects of human society is by adopting a different 
attitude to the human species as such. 

This attitude is the one you would adopt if you were suddenly thrown 
into a society of apes-apes which perpetually preyed within their own 
ecological niche. It is clear that if you proposed to be "against" such a 
situation, and to do something about it, then politics as it is normally 
conceived would be out of the question. To anticipate our later discussion, 
the first thing you must do is to protect yourself against society. The way to 
do this is to create an invisible enclave for yourself within the Establishment. 
Having such an enclave certainly does not imply loyalty to the 
Establishment. On the contrary, there is no reason why you should be toyal 
to any faction among the apes. You only pretend to be loyal to one faction 
or another when it is necessary for self-defense. If there is a change of regime 
in the country where you are living, you either leave or join the winning side. 
Transfer your invisible enclave to whatever Establishment is available. But all 
this is an external, defensive tactic which has nothing to do with the primary 
goals of our strategy. 


190 


We will finish our critique of third-level politics, and then continue the 
description of the substitute which we propose. In addition to making vague 
promises of third-level change, leftism encourages indignation at social 
conditions which are beyond anyone's power to affect. Leftism attributes 
great ethical merit to such indignation and morally condemns anyone who 


does not share it. But this attitude is totally irrational and dishonest. In 
philosophy and mathematics, it is possible for a proposition to be valid even 
though it has no chance of institutional acceptance. But in social, economic, 
and political matters, attitudes which have policy implications are nonsense 
unless the policies are actually implemented. Institutional acceptance is the 
only arena of validation of a social doctrine. It is absurd to attribute ethical 
merit to a longing for the impossible. Indignation at a social condition which 
is beyond anyone's power to affect is meaningless. (Indeed, to the extent 
that such indignation diverts social energy into a dead end, it is 
"counter-revolutionary.") To be more radical in social matters than society 
can possibly be is not virtuous; it is idiotic. 

Although third-level politics is a fraud, it is the contention of this essay 
that there exists a rational substitute for it. Once you perceive that you exist 
in a society of apes who attack their own ecological niche, there are rational 
goals which you can adopt for your life that correspond to third-level change 
even though they have nothing to do with leftism. The preliminary step, as 
we have said, is to create an invisible enclave for yourself within. the 
Establishment. The remainder of the strategy is in two parts which are in 
fact closely related. 

The first part is based on a consideration of the effects which such 
figures as Galileo, Galois, Abel, Lobachevski, and Mendel have had on 
society. These men devoted themselves to researches which seemed to be 
purely abstract, without any relevance to the practical world. Yet, through 
long, tortuous chains of events, their researches have had disruptive effects 
on society which go far beyond the effects of most political movements. The 
reason has to do with the peculiar role which technology has in human 
society. Society's attitude in relation to technology is like that of a child 
who cannot refrain from playing with matches. We find that 
the abstract researches of the men being considered accomplished a dual 
result. On the one hand, they represented inner escape, the achievement of a 
private utopia now. Of course, the general public will not understand this; 
only the few who are capable of participating in such activities will 
appreciate the extent to which they can constitute inner escape. On the 
other hand, they have had profoundly disruptive effects on society, effects 
which still have not run their course. 

Thus, the first part of our strategy is to follow the example of these 


191 


individuals. Of course, we do not stay within the bounds of present-day 
academic research, any more than Galileo or Mendel did in their time. What 
we have in mind is activities in the intellectual modality represented by the 
rest of this book. 

It should be clear that such activities do represent a private utopia, and are at 
the same time the seeds of disruptive future technologies which lead directly 
to the second part of our strategy. 


It is important to realize that by speaking of inner escape we do not 
mean fashionable drug use, or Eastern religions, or occultism. These 
threadbare superstitions are embraced by the cosmopolitan middle 
classes--intellectually spineless fools who are always grasping for spiritual 
comfort. Superstitious fads are escapism in the worst sense, as they only 
serve to further muddle the heads of the fools who embrace them. In 
contrast, the inner escape which we propose is origina! and consequential, 
leading to an increase in man's manipulative power over the world. It has 
nothing to do with irrationality or superstition. 

The second part of our strategy is predicated on the following states of 
affairs. First, it is the human species as such which is the obstacle to 
third-level political change. Secondly, technology is developing far more 
rapidly than society is, and no feature of the natural world need any longer 
be taken for granted. Society cannot help but foster technology in the 
pursuit of military and economic supremacy, and this includes technology 
which can contribute to the making of artificial superhuman beings. Every 
fundamental advance in logic, physics, neurophysiology, and 
neurocybernetics obviously leads in this direction. Thus, the second part of 
the strategy is to participate in the making of artificial superhumans, 
possibly by infiltrating the military-scientific establishment and diverting 
research in the appropriate direction. 


Note: This essay provides a specific, practical strategy for the present 
environment. It also shows that certain types of opposition to the status quo 
are meaningless. Subversion Theory, on the other hand, was a general theory 
which was not limited to any one environment, but also which failed to 
provide a specific strategy for the present environment. 


192 


SCIENCE (LOGIC) 


19. The Logic of Admissible Contradictions--work in progress 
Chapter [1!. A Provisional Axiomatic Treatment 


In the first and second chapters, we developed our intuitions 
concerning perceptions of the logically impossible in as much detail as we 
could. We decided, on intuitive grounds, which contradictions were 
admissible and which were not. As we proceeded, it began to appear that the 
results suggested by intuition were cases of a few general principles. In this 
chapter, we will adopt these principles as postulates. The restatement of our 
theory does not render the preceding chapters unnecessary. Only by 
beginning with an exhaustive, intuitive discussion of perceptual illusions 
could we convey the substance underlying the notations which we call 
admissble contradictions, and motivate the unusual collection of postulates 
which we will adopt. 

All properties will be thought of as 'parameters,' such as time, 
location, color, density, acidity, etc. Different parameters will be represented 
by the letters x, y, z, .... Different values of one parameter, say x, will be 
represented by x1, X9, .... Each parameter has a domain, the set of all values 
it can assume. An ensembie (Xo, Yo: Zo, ...) will stand for the single possible 
phenomenon which has x-value xg, y-value yo, etc. Several remarks are in 
order. My ensembles are a highly refined version of Rudolph Carnap's 
intensions or intension sets (sets of all possible entities having a given 
property). The number of parameters, or properties, must be supposed to be 
indefinitely large. By giving a possible phenomenon fixed values for every 
parameter, I assure that there will be only one such possible phenomenon. In 
other words, my intension sets are all singletons. Another point is that if we 
specify some of the parameters and specify their ranges, we limit the 
phenomena which can be represented by our "ensembles." If our first 
parameter is time and its range ts R, and our second parameter is spatial 


location and its range is R , then we are limited to phenomena which are 
point phenomena in space and time. !f we have a parameter for speed of 
motion, the motion will have to be infinitesimal. We cannot have a 
parameter for weight at all; we can only have one for density. The physicist 
encounters similar conceptual problems, and does noi find them 
insurmountable. 

Let (x4, y, Z, ...), (x9, y, Z, -..), etc. stand for possible phenomena 


195 


which all differ from each other in respect to parameter x but are identical in 
respect to every other parameter y, z, ... . {If the ensembles were intension 
sets, they would be disjoint precisely because x takes a different value in 
each.) A "simple contradiction family" of ensembles is the family [(x4,y, 2, 
aay (x9, y, Z, ...), «J. The family may have any number of ensembles. It 
actually represents many families, because y, z, ... are allowed to vary; but 
each of these parameters must assume the same value in all ensembles in any 
one family. x, on the other hand, takes different values in each ensemble in 
any one family, values which may be fixed. A parameter which has the same 
value throughout any one family will be referred to as a consistency 
parameter. A parameter which has a different value in each ensemble in a 
given family will be referred to as a contradiction parameter. 
"Contradiction" will be shortened to "con." A simple con family is then a 
family with one con parameter. The consistency parameters may be dropped 
from the notation, but the reader must remember that they are implicitly 
present, and must remember how they function. 

A con parameter, instead of being fixed in every ensemble, may be 
restricted to a different subset of its domain in every ensemble. The subsets 
must be mutually disjoint for the con family to be well-defined. The con 
family then represents many families in another dimension, because it 
represents every family which can be formed by choosing a con parameter 
value from the first subset, one from the second subset, etc. 

Con families can be defined which have more than one con parameter, 
i.e. more than one parameter satisfying all the conditions we put on x. Such 
con families are not "simple." Let the cardinality of a con family be 
indicated by a number prefixed to "family," and let the number of con 
parameters be indicated by a number prefixed to "con." Remembering that 
consistency parameters are understood, a 2-con °-family would appear as 
(x4, Yq). (x9, y), sei. 

A "contradiction" or "y - object" is not explicitly defined, but it is 
notated by putting "y" in front of a con family. The characteristics of y 
-objects, or cons, are established by introducing additional postulates in the 
theory. 

In this theory, every con is either "admissible" or "not admissible." 
"Admissible" will be shortened to "am." The initial amcons of the theory 
are introduced by postulate. Essentially, what is postulated is that cons with 
a certain con parameter are am. (The cons directly postulated to be am are 
on 1-con families.) However, the postulate will specify other requirements for 
admissibility besides having the given con parameter. The requisite 
cardinality of the con family will be specified. Also, the subsets will be 
specified to which the con parameter must be restricted in each ensemble in 


196 


the con. A con must satisfy all postulated requirements before it is admitted 
by the postulate. 

The task of the theory is to determine whether the admissibility of the 
cons postulated to be am implies the admissibility of any other cons. The 
method we have developed for solving such problems will be expressed as a 
collection of posiulates for our theory. 

Postulate 1. Given y[(x € A), (x € B}, ...] am, where x ¢ A, xe B, ... are the 
restrictions on the con parameter, and given A1CA, By CB, ..., where Ay, By, 
.. & @, then gl(x € Ay), (x € By),...] is am. This postulate is obviously 
equivalent to the postulate that y[{x € ANC), (xe BNC),...] is am, where C is 
a subset of x's domain end the intersections are non-empty. (Proof: Choose 
C= A, UB... .) 

Postulate 2. If x and y are simple amcon parameters, then a con with con 
parameters x and y is am if it satisfies the postulated requirements 
concerning amcons on x and the postulated requirements concerning amcons 
on y. 

The effect of all! our assumptions up to now is to make parameters 
totally independent. They do not interact with each other at all. 

We will now introduce some specific amcons by postulate. If s is speed, 
consideration of the waterfall illusion suggests that we postulate y[(s>O), 
{s=O)] to be am. (But with this postulate, we have come a long way from 
the literary description of the waterfall illusion! } Note the implicit 
requirements that the con family must be a 2-family, and that s must be 
selected from [O] in one ensemble and from [s: s>O] in the other ensemble. 

If tis time, t € R, consideration of the phrase "b years ago," which is an 
amcon in the natural language, suggests that we postulate y[(t): a-b<t<v-b & 
av] to be am, where a is a fixed time expressed in years A.D., bisa fixed 
number of years, and v is a variable--the time of the present instant in years 
A.D. The implicit requirements are that the con family must have the 
cardinality of the continuum, and that every value of t from a-b to v-b must 
appear in an ensemble, where v is a variable. Ensembles are thus continually 
added to the con family. Note that there is the non-trivial possibility of using 
this postulate more than once. We could admit a con for a = 1964, b=, 
then admit another for a=1963, b=2, and admit stifl another for a=1963, 
b=1; etc. 

Let p be spatial location, p é R2. Let P; be a non-empty, bounded, 
connected subset of R2. Restriction subsets will be selected from the P;. 
Specifically, let Py APs = ¢. Consideration of a certain dreamed illusion 


suggests that we admit y[(p € P;), (p € Py)]. The implicit requirements are 
obvious. But in this case, there are more requirements in the postulate of 


197 


admissibility. Vay we apply the postulate twice? May we admit first y[(pe 
P4), (pe P5)} and then y[(peP3), (pePg)], where P2 and Py are arbitrary 
P;'s different from P; and Po? The answer is no. We may admit y [(p € P4), 
(p € Po)] for arbitrary Py and Po, Py OP = «3, but having made this "initial 
choice," the postulate cannot be reused for arbitrary P3 and Pg. A second 
con y[(p € Pa), (p € P4)], PgNP4 = 6, may be postulated to be am only if 
P4UP3, PoUP3, PUP, and PoUP4 are not connected. In other words, you 
may postulate many cons of the form y[(p é Pi), (p € Pi)] to be am, but 
your first choice strongly circumscribes your second choice, etc. 

We will now consider certain results in the logic of amcons which were 
established by extensive elucidation of our intuitions. The issue is whether 
our present axiomization produces the same results. We will express the 
results in our latest notation as far as possible. Two more definitions are 
necessary. The parameter @ is the angle of motion of an infinitesimally 
moving phenomenon, measured in degrees with respect to some chosen axis. 
Then, recalling the set Py, choose Ps and Pa so that Py = P5UPs and 
PEOPe=¢. 

The results by which we will judge our axiomization are as follows. 

1: glS, C,UCs] can be inferred to be am. 

Our present notation cannot express this result, because it does not 
distinguish between different types of uniform motion throughout a finite 
region, i.e. the types M, Cy, Co, Dy, and Do. Instead, we have infinitesimal 
motion, which is involved in all the latter types of motion. Questions such as 
"whether the admissibility of » [M, S] implies the admissibility of y[C,, S}" 
drop out. The reason for the omission in the present theory is our choice of 
parameters and domains, which we discussed earlier. Our present version is 
thus not exhaustive. However, the deficiency is not intrinsic to our method; 
and it does not represent any outright falsification of our intuitions. Thus, 
we pass over the deficiency. 

2: [(pe Py, SQ), (pe Po, SqQ)] and other such cons can be inferred to be am. 
With our new, powerful approach, this result is trivial. It is guaranteed by 
what we said about consistency parameters. 

3: There is no way to infer that y[C1, Cg] is am; and no way to infer that 
y[ (45°, SQ>O), (60°,s=s¢)] is am. 

The first part of the result drops out. The second part is trivial with our new 
method as long as we do not postulate that cons on @ are am. 

4: p [(pe Po), (p € P5)] can be inferred to be am. 

Yes, by Postulate 1. 

5: v [(s>O, p € Py), (s=O, pe Po)] and y [(s>O, pe Po), (s=O, p € P4)] can 


198 


be inferred to be am. 

Yes, by Postulate 2. These two amcons are distinct. The question of whether 
they should be considered equivalent is closely related to the degree to 
which con parameters are independent of each other. 

6: There is no way to infer that y [(p € Ps), (pe Pg)] or p[(pe Py), (p € P3) 
] is am. Our special requirement in the postulate of admissibility for y [(p € 
P+), (p € Po)] guarantees this result. 

The reason for desiring this last result requires some discussion. [In 
heuristic terms, we wish to avoid admitting both location in New York in 
Greensboro and location in Manhattan and Brooklyn. We also wish to avoid 
admitting location in New York in Greensboro and location in New York in 
Boston. If we admitted either of these combinations, then the intuitive 
rationale of the notions would indicate that we had admitted triple location. 
While we have a dreamed illusion which justifies the concept of double 
location, we have no intuitive justification whatever for the concept of triple 
location. It must be clear that admission of either of the combinations 
mentioned would not imply the admissibility of a con on a 3-family with 
con parameter p by the postulates of our theory. Our theory is formally safe 
from this implication. However, the intuitive meaning of either combination 
would make them proxies for the con on the 3-family. 

A closely related consideration is that in the preceding chapter, it 
appeared that the admission of y[(p € P;), (pe Po)] and y[(p € Ps), (pe Pe)] 
would tend to require the admission of the object y[(p € Po), e [(p € Ps), (p 
€ Pg) ]] {a Type 1 chain). Further, it this implication held, then by the same 
rationale the admission of y[(p € P4}, (p € Pa)] and y[(s> O, Pg € Py), (s=O, 
P=Ppo)1], both of which are am, would require the admission of the object 
vl{p € Pa), yl(s> O, pg € Py), (s=O, P=PqQ)]]. We may now say, however, 
that the postulates of our theory emphatically do not require us to accept 
these implications. If there is an intuitively valid notion underlying the chain 
on s and p, it reduces to the amcons introduced in result 5. As for the chain 
on p alone, we repeat that simultaneous admission of the two cons 
mentioned would tend to justify some triple location concept. However, we 
do not have to recognize that concept as being the chain. It seems that our 
present approach allows us to forget about chains for now. 

Our conclusion is that the formal approach of this chapter is in good 
agreement with our intuitively established results. 


199 


Note on the overall significance of the logic of amcons: 

When traditional logicians said that something was logically impossible, 
they meant to imply that it was impossible to imagine or visualize. But this 
implication was empirically false. The realm of the logically possible is not 
the entire realm of connotative thought; it is just the realm of normal 
perceptual routines. When the mind is temporarily freed from normal 
perceptual routines--especially in perceptual illusions, but also in dreams and 
even in the use of certain "illogical" natural language phrases--it can imagine 
and visualize the "logically impossible." Every text on perceptual 
psychology mentions this fact, but logicians have never noticed its immense 
significance. The logically impossible is not a blank; it is a whole layer of 
meaning and concepts which can be superimposed on conventional logic, but 
not reduced or assimilated to it. The logician of the future may use a drug or 
some other method to free himself from normal perceptual routines for a 
sustained period of time, so he can freely think the logically impossible. He 
will then perform rigorous deductions and computations in the logic of 
amcons. 


200 


20. Subjective Propositional Vibration-work in progress 


Up until the present, the scientific study of language has treated 
language as if it were reducible to the mechanical manipulation of counters 
on a board. Scientists have avoided recognizing that language has a mental 
aspect, especially an aspect such as the 'understood meaning" of a linguistic 
expression. This paper, on the other hand, will present linguistic constructs 
which inescapably involve a mental aspect that is objectifiable and can be 
subjected to precise analysis in terms of perceptual psychology. These 
constructs are not derivable from the models of the existing linguistic 
sciences. In fact, the existing linguistic sciences overlook the possibility of 
such constructs. 

Consider the ambiguous schema 'ADB&C', expressed in words as 'C and 
B if A'. An example is 

Jack will soon leave and Bill will laugh if Don speaks. (1) 

In order to get sense out of this utterance, the reader has to supply it with a 
comma. That is, in the jargon of logic, he has to supply it with grouping. Let 
us make the convention that in order to read the utterance, you must 
mentally supply grouping to it, or 'bracket' it. If you construe the schema 
as 'AD (B &C)', you will be said to bracket the conjunction. If you construe 
the. schema as '(ADB) & C', you will be said to bracket the conditional There 
is an immediate syntactical issue. If you are asked to copy (1), do you write 
"Jack will soon leave and Bill will laugh if Don speaks"; or do you write 
"Jack will soon leave, and Bill will laugh if Don speaks" if that is the way 
you are reading (1) at the moment? A distinction has to be made between 
reading the proposition, which involves bracketing; and viewing the 
proposition, which involves reacting to the ink-marks solely as a pattern. 
Thus, any statement about an ambiguous grouping proposition must specify 
whether the reference is to the proposition as read or as viewed. 

Some additional conventions are necessary. With respect to (1), we 
distinguish two possibilities: you are reading it, or you are not looking at it 
(or are only viewing it). Thus, a "single reading' of (1) refers to an event 
which separates two consecutive periods of not looking at {1) (or only 
viewing it). During a single reading, you may switch between bracketing the 
conjunction and bracketing the conditional. These switches demarcate a 
series of "states" of the reading, which alternately correspond to 'Jack will 


201 


¢ 


soon leave, and Bill will laugh if Don speaks' or 'Jack will soon leave and Bill 
will laugh, if Don speaks'. Note that a state is like a complete proposition. 
We stipulate that inasmuch as (1) is read at all, it is the present meaning or 
state that counts--if you are asked what the proposition says, whether it is 
true, etc. 

Another convention is that the logical status of 
(Jack will soon leave and Bill will laugh if Don speaks) if and only if (Jack 
will soon leave and Bill will laugh if Don speaks) 
is not that of a normal tautology, even though the biconditional when 
viewed has the form 'A=A'. The two ambiguous cemponents wil! not 
necessarily be bracketed the same way in a state. 

We now turn to an example which is more substantial that (1). 
Consider 
Your mother is a whore and you are now bracketing the conditional! in (2) if 
you are now bracketing the conjunction in (2). (2) 
If you read this proposition, then depending on how you bracket it, the 
reading wil! either be internally false or else wil! call your mother a whore. In 
general, ambiguous grouping propositions are constructs in which the mental 
aspect plays a fairly explicit role in the language. We have included (2) to 
show that the contents of these propositions can provide more complications 
than would be suggested by (1). 


There is another way of bringing out the mental! aspect of language, 
however, which is incomparably more powerful than ambiguous grouping. 
We will turn to this approach immediately, and will devote the rest of the 
paper to it. The cubical frame is asimple reversible perspective figure 
which can either be seen oriented upward like Q _ or oriented downward 
like ©, . Both positions are implicit in the same ink-on-paper image; it is 
the subjective psychological response of the perceiver which differentiates 
the positions. The perceiver can deliberately cause the perspective to reverse, 
or he can allow the perspective to reverse without resisting. The perspective 
can also reverse against his will. Thus, there are three possibilities: deliberate, 
indifferent, and involuntary reversal. 

Suppose that each of the positions is assigned a different meaning, and 
the figure is used as a notation. We will adopt the following definitions 
because they are convenient for our purposes at the moment. 

> (for '3' if it appears to be oriented like Q 

for 'O' (zero) if it appears to be oriented like @! 

We may now write 


1 +B = 4 (3) 
We must further agree that (3), or any proposition containing such 


202 


notation, is to be read to mean just what it seems to mean at any given 
instant. [f, at the moment you read the proposition, the cube seems to be 
up, then the proposition means 1+3=4; but if the cube seems to be down, 
the proposition means 1+O=4. The proposition has an unambiguous 
meaning for the reader at any given instant, but the meaning may change in 
the next instant due to a subjective psychological change in the reader. The 
reader is to accept the proposition for what it is at any instant. The result is 
subjectively triggered propositional vibration, or SPV for short. The 
distinction between reading and viewing a proposition, which we already 
made in the case of ambiguous grouping, is even more important in the case 
of SPV. Reading now occurs only when perspective is imputed. In reading 
(3) you don't think about the ink graph any more than you think about the 
type face. 

in a definition such as that of ' 8 '3° and 'OQ' will be called the 
assignments. A single reading is defined as before. During a single reading, (3) 
will vibrate some number of times. The series of states of the reading, which 
alternately correspond to '1 + 3 = 4' or '1+ O = 4', are demarcated by 
these vibrations. The portion of a state which can change when vibration 
occurs will be called a partial. It is the partials in a reading that correspond 
directly to the assignments in the definition. 

Additional conventions are necessary. Most of the cases we are 
concerned with can be covered by two extremely important rules. First, the 
ordinary theory of properties which have to do with the form of expressions 
as viewed is not applicable when SPV notation is present. Not only is a 
biconditional not a tautology just because its components are the same when 
viewed; it cannot be considered an ordinary tautology even if the one 
component's states have the same truth value, as in the case of '1 + & # 
2'. Secondly, and even more important, SPV notation has to be present 
explicitly or it is not present at all. SPV is not the idea of an expression with 
two meanings, which is commonplace in English; SPV is a double meaning 
which comes about by a perceptual experience and thus has very special 
properties. Thus, if a quantifier should be used in a proposition containing 
SPV notation, the "range" of the "variable" will be that of conventional 


ser 


logic. You cannot write ' RS ' for 'x' in the statement matrix 'x 
= we ' 

We must now elucidate at considerable length the uniqué properties of 
SPV. When the reader sees an SPV figure, past perceptual training will cause 
him to impute one or the other orientation to it. This phenomenon is not a 
mere convention in the sense in which new terminology is a convention. 
There are already two clear-cut possibilities. Their reality is entirely mental; 
the external. ink-on-paper aspect does not change in any manner whatever. 


203 


The change that can occur is completely and inherently subjective and 
mental. By mental effort, the reader can consciously control the orientation. 
If he does, involuntary vibrations will occur because of neural noise or 
attention lapses. The reader can also refrain from control and accept 
whatever appears. In this case, when the figure is used as a notation, 
vibrations may occur because of a preference for one meaning over the 
other. Thus, a deliberate vibration, an involuntary vibration, and an 
indifferent vibration are three distinct possibilities. 

What we have done is to give meanings to the two pre-existing 
perceptual possibilities. In order to read a proposition containing an SPV 
notation at all, one has to see the ink-on-paper figure, impute perspective to 
it, and recall the meaning of that perspective; rather than just seeing the 
figure and recalling its meaning. The imputation of perspective, which will 
happen anyway because of pre-existing perceptual training, has a function in 
the language we are developing analogous to the function of a letter of the 
alphabet in ordinary language. The imputation of perspective is an aspect of 
the notation, but it is entirely mental. Our language uses not only 
graphemes, but "psychemes" or "mentemes". One consequence is that the 
time structure of the vibration series has a distinct character; different in 
principle from external, mechanical randomization, or even changes which 
the reader would produce by pressing a button. Another consequence is that 
ambiguous notation in general is not equivalent to SPV. There can be mental 
changes of meaning with respect to any ambiguous notation, but in general 
there is no psycheme, no mental change of notation. It is the clear-cut, 
mental, involuntary change of notation which is the essence of SPV. Without 
psychemes, there can be no truly involuntary mental changes of meaning. 


In order to illustrate the preceding remarks, we will use an SPV 
notation defined as follows. 
« fis an affirmative, read "definitely," if it appears to be oriented 
BH ijlike O 

is a negative, read "not," if it appears to be oriented like fy 
The proposition which follows refers to the immediate past, not to all past 
time; that is, it refers to the preceding vebration. 

You have i deliberately vibrated (4). (4) 


This proposition refers to itself, and its truth depends on an aspect of the 
reader's subjectivity which accompanies the act of reading. However, the 
same can be said for the next proposition. 

The bat is made of wood, and you have just decided that the second 


word in (5) refers to a flying mammal. (5) 


204 


Further, the same can be said for (2). We must compare (5), (2), and (4) in 
order to establish that (4) represents an order of language entirely different 
from that represented by (5) and (2). (5) is a grammatical English sentence 
as it stands, although an abnormal one. The invariable, all-ink notation 'bat' 
has an equivocal referental structure: it may have either of two mutually 
exclusive denotations. In reading, the native speaker of English has to choose 
one denotation or the other; contexts in which the choice is difficult rarely 
occur. (2) is not automatically grammatical, because it lacks a comma. We 
have agreed on a conventional process by which the reader mentally supplies 
the comma. Thus, the proposition lacks an element and the reader must 
supply it by a deliberate act of thought. The comma is not, strictly speaking, 
a notation, because it is entirely voluntary. The reader might as well be 
supplying a denotation io an equivocal expression: (5) and (2) can be 
reduced to the same principle. As for (4), it cannot be mistaken for ordinary 
English. It has an equivocal "proto-notation," ' 74] ". You automatically 
impute perspective to the proto-notation before you react to it as language. 
Thus, a notation with a mental component comes into being involuntarily. 
This notation has an unequivocal denotation. However, deliberate, 
inditferent, and most important of all, involuntary mental changes in 
notation can occur. 

We now suggest that the reader actually read (5), (2), and (4), in that 
order. We expect that (5) can be read without noticeable effort, and that a 
fixed result will be arrived at {unless the reader switches in an attempt to 
find a true state). The reading of (2) involves mentally supplying the comma, 
which is easy, and comprehending the logical compound which . results, 
which is not as easy. Again, we expect that a fixed result will be arrived at 
(unless the reader vacillates between the insult and the internally false state). 
In order to read (4), center your sight on the SPV notation, with your 
peripheral vision taking in the rest of the sentence. A single reading should 
last at least half a minute. If the reader will seriously read (4), we expect that 
he will find the reading to be an experience of a totally different order from 
the reading of (5) and (2). It is like looking at certain confusing visual 
patterns, but with an entire dimension added by the incorporation of the 
pattern into language. The essence of the experience, as we have indicated, is 
that the original imputation of perspective is involuntary, and that the reader 
has to contend with involuntary changes in notation for which his own mind 
is responsible. We are relying on this experience to convince the reader 
empirically that (4) represents a new order of language to an extent to which 
(5) and (2) do not. 

To make our point even clearer, let us introduce an operation, called 
"collapsing," which may be applied to propositions containing SPV 


205 


proto-notation. The operation consists in redefining the SPV figure in a given 
proposition so that its assignments are the states of the original proposition. 
Let us collapse (4). We redefine 

for 'You have deliberately vibrated (4)' if it appears to be oriented 
t_* like @J 

for 'You have not deliberately vibrated (4)' if it appears to be oriented 

like 


(4) now becomes 


# (4) 


We emphasize that the reader must actually read (4), for the effect is 
indescribable. The reader should learn the assignments with flash cards if 
necessary. 

The claim we want to make for (4) is probably that it is the most 
clear-cut case yet constructed in which thought becomes an object for itself. 
Just looking at a reversible perspective figure which is not a linguistic 
utterance--an approach which perceptual psychologists have already 
tried--does not yield results which are significant with respect to "thought." 
In order to obtain a significant case, the apparent orientation or imputed 
perspective must be a proposition; it must be true or false. Then, (5) and (2) 
are not highly significant, because the mental act of supplying the missing 
element of the proposition is all a matter of your volition; and because the 
element supplied is essentially an "understood meaning." We already have an 
abundance of understood meanings, but scientists have been able to ignore 
them because they are not "objectifiable." In short, reversible perspective by 
itself is not "thought"; equivocation by itself has no mental aspect which is 
objectifiable. Only in reading (4) do we experience an "objectifiable aspect 
of thought." We have invented an instance of thought (as opposed to 
perception) which can be accomodated in the ontology of the perceptual 
psychologist. 


206 


¥ 


Henry Flynt, Blueprint for a Higher Civilization 
(Milano, Multhipla Edizioni, 1975) 
ERRATA 


p. 4 delete 5/15/1962 
Adams House 
p.- 24 delete 5/15/1962 
audience, 
ppe 26-32 middle of p. 26 to top of p. 32 
should come after p. 60 
pe 27 line 5 fact it 
line 7 of them, which 
pe 42 line 4 bodies 
"statements", it 
pe 53 delete 2/22/1963 
February 27, 1963 
pe 55 line 7 mind', 
pe 72 delete third line from bottom 
pe 74 delete 2/22/1963 
February 27, 1963 
p. 84 delete 2/22/1963 
February 27, 1963 


(photo 
pe 86 line 26 transformation 
p. 94 line 2 from bottom is true, 
p. 96 lines 12-14 all S to have superscript D 
line 13 250 
under the figure: given 25 S X5yy 
pe 97 line 14 D-Memory 
p. 99 lines 13, 14, 15 right-hand 
p. 100 line 3 from bottom 1962 
p. 101 line 19 Chicago." 
line 25 sun," 
p. 102 line 4 from bottom assertion." 
pe 104 line 8 switch 
line 26 A, ar 
i 
line 28 A." 
; as 


pe 105 between lines 25, 26 


Conclusion 3.1. Conscious remembering occurs in 
some mental state. 


I j 7 *j-a 
p. 108 line 20. x.,--x. 


j-1 j 
lines 4, 5 from bottom j+4 


p. 109 line 2 2.4 %-Memories 
pe 114 line 5 from bottom "A single 
pe 120 line 5 26 


pe 106 line 7 x 


Page 1 


Henry Flynt, Blueprint for a Higher Civilization Page 2 
(Milano, Multhipla Edizioni, 1975) 


ERRATA 


pe. 125 bottom line table. See Carnap, Meaning and Necessity. 


p. 129 line 1 —s. 7 
line 12 from bottom 
fotally determinate innperseq' iff an innpersea 
line 10 from bottom 
Tantecedentally indeterminate innperseq! iff an innperseq 
line 8 from bottom 
*halpointally indeterminate innperseq' iff an innperseq 


pp. 134-151 These pages should have tab pagination identifying 
them as pp. 1-18 of the "Guidebook." 


Also, the Guidebook must start on a right-hand 
page. 


p. 139 line 13 a_lb 
p. 141 line 15 NOW--CLOSE 
pe 145 in Instr. 1-3. (t SS ) 
line 6 from bottom 9. 
p. 147 line 3 'a 
p. 152 delete 2/22/1963 
Februery 27, 1963 
(photo 
p. 158 line 23 most fears 
line 24 imposed 
p.- 179 bottom line definite 
p. 180 line 5 categories, 
p. 187 delete 2/22/1963 
February 27, 1963 
p. 195 line 12 admissible 
p. 201 line 19 'AD (BEC)', 
. line 20 conditional. 
p. 202 line 12 than (1). 
p. 204 line 7 from bottom vibration 
p. 206 lines 4-7 definitions in braces { } 


