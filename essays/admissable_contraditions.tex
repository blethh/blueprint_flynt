\chapter{The Logic of Admissible Contradictions (Work in Progress)}

\section{Chapter III. A Provisional Axiomatic Treatment}


In the first and second chapters, we developed our intuitions 
concerning perceptions of the logically impossible in as much detail as we 
could. We decided, on intuitive grounds, which contradictions were 
admissible and which were not. As we proceeded, it began to appear that the 
results suggested by intuition were cases of a few general principles. In this 
chapter, we will adopt these principles as postulates. The restatement of our 
theory does not render the preceding chapters unnecessary. Only by 
beginning with an exhaustive, intuitive discussion of perceptual illusions 
could we convey the substance underlying the notations which we call 
admissble contradictions, and motivate the unusual collection of postulates 
which we will adopt. 

All properties will be thought of as "parameters," such as time, 
location, color, density, acidity, etc. Different parameters will be represented 
by the letters x, y, z, .... Different values of one parameter, say x, will be 
represented by $x_1$, $x_2$, .... Each parameter has a domain, the set of all values 
it can assume. An ensembie ($x_0$, $y_0$, $z_0$, ...) will stand for the single possible 
phenomenon which has x-value $x_0$, y-value $y_0$, etc. Several remarks are in 
order. My ensembles are a highly refined version of Rudolph Carnap's 
intensions or intension sets (sets of all possible entities having a given 
property). The number of parameters, or properties, must be supposed to be 
indefinitely large. By giving a possible phenomenon fixed values for every 
parameter, I assure that there will be only one such possible phenomenon. In 
other words, my intension sets are all singletons. Another point is that if we 
specify some of the parameters and specify their ranges, we limit the 
phenomena which can be represented by our "ensembles." If our first 
parameter is time and its range is $R$, and our second parameter is spatial 
location and its range is $R^2$, then we are limited to phenomena which are 
point phenomena in space and time. If we have a parameter for speed of 
motion, the motion will have to be infinitesimal. We cannot have a 
parameter for weight at all; we can only have one for density. The physicist 
encounters similar conceptual problems, and does noi find them 
insurmountable. 

Let ($x_1$, $y$, $z$, ...), ($x_2$, $y$, $z$, ...), etc. stand for possible phenomena 
which all differ from each other in respect to parameter x but are identical in 
respect to every other parameter $y$, $z$, ... . (If the ensembles were intension 
sets, they would be disjoint precisely because $x$ takes a different value in 
each.) A "simple contradiction family" of ensembles is the family [($x_1$,$y$,$z$, 
...), ($x_2$, $y$, $z$, ...), ...]. The family may have any number of ensembles. It 
actually represents many families, because $y$, $z$, ... are allowed to vary; but 
each of these parameters must assume the same value in all ensembles in any 
one family. $x$, on the other hand, takes different values in each ensemble in 
any one family, values which may be fixed. A parameter which has the same 
value throughout any one family will be referred to as a consistency 
parameter. A parameter which has a different value in each ensemble in a 
given family will be referred to as a contradiction parameter. 
"Contradiction" will be shortened to "con." A simple con family is then a 
family with one con parameter. The consistency parameters may be dropped 
from the notation, but the reader must remember that they are implicitly 
present, and must remember how they function. 

A con parameter, instead of being fixed in every ensemble, may be 
restricted to a different subset of its domain in every ensemble. The subsets 
must be mutually disjoint for the con family to be well-defined. The con 
family then represents many families in another dimension, because it 
represents every family which can be formed by choosing a con parameter 
value from the first subset, one from the second subset, etc. 

Con families can be defined which have more than one con parameter, 
i.e. more than one parameter satisfying all the conditions we put on x. Such 
con families are not "simple." Let the cardinality of a con family be 
indicated by a number prefixed to "family," and let the number of con 
parameters be indicated by a number prefixed to "con." Remembering that 
consistency parameters are understood, a 2-con $\infty$-family would appear as 
[($x_1$, $y_1$). ($x_2$, $y_2$), ...].

A "contradiction" or "$\varphi$-object" is not explicitly defined, but it is 
notated by putting "$\varphi$" in front of a con family. The characteristics of $\varphi$-objects, 
or cons, are established by introducing additional postulates in the 
theory. 

In this theory, every con is either "admissible" or "not admissible." 
"Admissible" will be shortened to "am." The initial amcons of the theory 
are introduced by postulate. Essentially, what is postulated is that cons with 
a certain con parameter are am. (The cons directly postulated to be am are 
on 1-con families.) However, the postulate will specify other requirements for 
admissibility besides having the given con parameter. The requisite 
cardinality of the con family will be specified. Also, the subsets will be 
specified to which the con parameter must be restricted in each ensemble in 
the con. A con must satisfy all postulated requirements before it is admitted 
by the postulate. 

The task of the theory is to determine whether the admissibility of the 
cons postulated to be am implies the admissibility of any other cons. The 
method we have developed for solving such problems will be expressed as a 
collection of posiulates for our theory. 

\postulate{1} Given $\varphi[(x\in A),(x\in B),\ldots]$ am, where $x\in A$, $x\in B$, ... are the 
restrictions on the con parameter, and given $A_1\subset A$, $B_1\subset B$, ..., where $A_1,B_1,...\neq\emptyset$, then 
$\varphi[(x\in A_1),(x\in B_1),...]$ is am. This postulate is obviously 
equivalent to the postulate that $\varphi[(x\in A\cap C),(x\in B\cap C),...]$ is am, where $C$ is 
a subset of $x$'s domain end the intersections are non-empty. (Proof: Choose 
$C=A_1\cup B_1\cup\ldots$ .) 

\postulate{2} If $x$ and $y$ are simple amcon parameters, then a con with con 
parameters $x$ and $y$ is am if it satisfies the postulated requirements 
concerning amcons on $x$ and the postulated requirements concerning amcons 
on $y$. 

The effect of all our assumptions up to now is to make parameters 
totally independent. They do not interact with each other at all. 

We will now introduce some specific amcons by postulate. If $s$ is speed, 
consideration of the waterfall illusion suggests that we postulate 
$\varphi[(s>O),(s=O)]$ to be am. (But with this postulate, we have come a long way from 
the literary description of the waterfall illusion!) Note the implicit 
requirements that the con family must be a 2-family, and that $s$ must be 
selected from $[O]$ in one ensemble and from ${s:s>O}$ in the other ensemble. 

If $t$ is time, $t\in R$, consideration of the phrase "b years ago," which is an 
amcon in the natural language, suggests that we postulate $\varphi[(t):a-b\leq t\leq v-b \&a\leq v]$ to be am,
where $a$ is a fixed time expressed in years A.D., $b$ is a fixed 
number of years, and $v$ is a variable---the time of the present instant in years 
A.D. The implicit requirements are that the con family must have the 
cardinality of the continuum, and that every value of $t$ from $a-b$ to $v-b$ must 
appear in an ensemble, where $v$ is a variable. Ensembles are thus continually 
added to the con family. Note that there is the non-trivial possibility of using 
this postulate more than once. We could admit a con for $a=1964$, $b=\sfrac{1}{2}$
then admit another for $a=1963$, $b=2$, and admit still another for $a=1963$,
$b=1$; etc. 

Let $p$ be spatial location, $p\in R^2$. Let $P_i$ be a non-empty, bounded, 
connected subset of $R^2$. Restriction subsets will be selected from the $P_i$.
Specifically, let $P_1\cap P_2=\emptyset$. Consideration of a certain dreamed illusion 
suggests that we admit $\varphi[(p\in P_1),(p\in P_2)]$. The implicit requirements are 
obvious. But in this case, there are more requirements in the postulate of 
admissibility. May we apply the postulate twice? May we admit first 
$\varphi[(p\in P_1),(p\in P_2)]$ and then $\varphi[(p\in P_3),(p\in P_4)]$, where $P_3$ and $P_4$ are arbitrary 
$P_i$'s different from $P_1$ and $P_2$? The answer is no. We may admit 
$\varphi[(p\in P_1),(p\in P_2)]$ for arbitrary $P_1$ and $P_2$, $P_1\cap P_2=\emptyset$, but having made this "initial 
choice," the postulate cannot be reused for arbitrary $P_3$ and $P_4$. A second 
con $\varphi[(p\in P_3),(p\in P_4)]$, $P_3\cap P_4=\emptyset$, may be postulated to be am only if 
$P_1\cup P_3$,$P_2\cup P_3$,$P_1\cup P_4$, and $P_2\cup P_4$ are not connected. In other words, you 
may postulate many cons of the form $\varphi[(p\in P_i),(p\in P_j)]$ to be am, but 
your first choice strongly circumscribes your second choice, etc. 

We will now consider certain results in the logic of amcons which were 
established by extensive elucidation of our intuitions. The issue is whether 
our present axiomization produces the same results. We will express the 
results in our latest notation as far as possible. Two more definitions are 
necessary. The parameter $\theta$ is the angle of motion of an infinitesimally 
moving phenomenon, measured in degrees with respect to some chosen axis. 
Then, recalling the set $P_1$, choose $P_5$ and $P_6$ so that $P_1=P_5\cup P_6$ and 
$P_5\cap P_6=\emptyset$. 

The results by which we will judge our axiomization are as follows. 

\begin{enumerate} % TODO with colons?

	\item $\varphi[S, C_1\cup C_2]$ can be inferred to be am. 

Our present notation cannot express this result, because it does not 
distinguish between different types of uniform motion throughout a finite 
region, \ie the types $M$, $C_1$, $C_2$, $D_1$, and $D_2$. Instead, we have infinitesimal 
motion, which is involved in all the latter types of motion. Questions such as 
"whether the admissibility of $\varphi[M,S]$ implies the admissibility of $\varphi[C_1,S]$" 
drop out. The reason for the omission in the present theory is our choice of 
parameters and domains, which we discussed earlier. Our present version is 
thus not exhaustive. However, the deficiency is not intrinsic to our method; 
and it does not represent any outright falsification of our intuitions. Thus, 
we pass over the deficiency. 

\item $\varphi[(p\in P_1,s_0),(p\in P_2,S_0)]$ and other such cons can be inferred to be am. 
With our new, powerful approach, this result is trivial. It is guaranteed by 
what we said about consistency parameters. 

\item There is no way to infer that $\varphi[C_1,C_2]$ is am; and no way to infer that 
$\varphi[(45^\circ,s_0\greater O),(60^\circ,s=s_0)]$ is am. 

The first part of the result drops out. The second part is trivial with our new 
method as long as we do not postulate that cons on $\theta$ are am. 

\item $\varphi[(p\in P_2),(p\in P_5)]$ can be inferred to be am. 

Yes, by Postulate 1. 

\item $\varphi[(s>O, p\in P_1),(s=O, p\in P_2)]$ and $\varphi[(s>O, p\in P_2),(s=O, p\in P_1)]$ can 
be inferred to be am. 

Yes, by Postulate 2. These two amcons are distinct. The question of whether 
they should be considered equivalent is closely related to the degree to 
which con parameters are independent of each other. 

\item There is no way to infer that $\varphi[(p\in P_5),(p\in P_6)]$ or $\varphi[(p\in P_1),(p\in P_3)]$
is am. Our special requirement in the postulate of admissibility for 
$\varphi[(p\in P_1),(p\in P_2)]$ guarantees this result. 
\end{enumerate}

The reason for desiring this last result requires some discussion. In 
heuristic terms, we wish to avoid admitting both location in New York in 
Greensboro and location in Manhattan and Brooklyn. We also wish to avoid 
admitting location in New York in Greensboro and location in New York in 
Boston. If we admitted either of these combinations, then the intuitive 
rationale of the notions would indicate that we had admitted triple location. 
While we have a dreamed illusion which justifies the concept of double 
location, we have no intuitive justification whatever for the concept of triple 
location. It must be clear that admission of either of the combinations 
mentioned would not imply the admissibility of a con on a 3-family with 
con parameter p by the postulates of our theory. Our theory is formally safe 
from this implication. However, the intuitive meaning of either combination 
would make them proxies for the con on the 3-family. 

A closely related consideration is that in the preceding chapter, it 
appeared that the admission of $\varphi[(p\in P_1),(p\in P_2)]$ and $\varphi[(p\in P_5),(p\in P_6)]$
would tend to require the admission of the object $\varphi[(p\in P_2),\varphi[(p\in P_5),(p\in P_6)]]$
(a Type 1 chain). Further, it this implication held, then by the same 
rationale the admission of $\varphi[(p\in P_1),(p\in P_2)]$ and $\varphi[(s>O,p_0\in P_1),(s=O,p=p_0)]$,
		both of which are am, would require the admission of the object 
$\varphi[(p\in P_2), \varphi[(s>O,p_0\in P_1),(s=O, p=p_0)]]$. 
We may now say, however, 
that the postulates of our theory emphatically do not require us to accept 
these implications. If there is an intuitively valid notion underlying the chain 
on s and p, it reduces to the amcons introduced in result 5. As for the chain 
on p alone, we repeat that simultaneous admission of the two cons 
mentioned would tend to justify some triple location concept. However, we 
do not have to recognize that concept as being the chain. It seems that our 
present approach allows us to forget about chains for now. 

Our conclusion is that the formal approach of this chapter is in good 
agreement with our intuitively established results. 

\section*{Note on the overall significance of the logic of amcons:}

When traditional logicians said that something was logically impossible, 
they meant to imply that it was impossible to imagine or visualize. But this 
implication was empirically false. The realm of the logically possible is not 
the entire realm of connotative thought; it is just the realm of normal 
perceptual routines. When the mind is temporarily freed from normal 
perceptual routines---especially in perceptual illusions, but also in dreams and 
even in the use of certain "illogical" natural language phrases---it can imagine 
and visualize the "logically impossible." Every text on perceptual 
psychology mentions this fact, but logicians have never noticed its immense 
significance. The logically impossible is not a blank; it is a whole layer of 
meaning and concepts which can be superimposed on conventional logic, but 
not reduced or assimilated to it. The logician of the future may use a drug or 
some other method to free himself from normal perceptual routines for a 
sustained period of time, so he can freely think the logically impossible. He 
will then perform rigorous deductions and computations in the logic of 
amcons. 

