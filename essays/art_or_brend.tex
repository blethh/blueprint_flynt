\chapter{\textsc{Art} or \textsc{Brend}?}

\begin{enumerate}[label=\textbf{\arabic*.}, wide, itemsep=1em]
\item Perhaps the most diseased justification the artist can give of his profession 
is to say that it is somehow scientific. LaMonte Young, Milton Babbitt, and 
Stockhausen are exponents of this sort of justification. 

The law which relates the mass of a body to its velocity has predictive value 
and is an outstanding scientific law. Is the work of art such a law? The 
experiment which shows that the speed of light is independent of the motion 
of its source is a measurement of a phenomenon crucial to the confirmation of 
a scientific hypothesis; it is an outstanding scientific experiment. Is the work 
of art such a measurement? The invention of the vacuum tube was an 
outstanding technological advance. Is the work of art such a technological 
advance? Differential geometry is a deductive analysis of abstract relations 
and an outstanding mathematical theory. ts the work of art such an 
analysis? 

The motives behind the \enquote{scientific} justification of art are utterly sinister. 
Perhaps LaMonte Young is merely rationalizing because he wants an 
academic job. But Babbitt is out to reduce music to a pedantic 
pseudo-science. And Stockhausen, with his \enquote{scientific music}, intends 
nothing less than the suppression of the culture of \enquote{lower classes} and 
\enquote{ower races.} 

It is the creative personality himself who has the most reason to object to 
the \enquote{scientific} justification of art. Again and again, the decisive step in 
artistic development has come when an artist produces a work that shatters 
all existing 'scientific' laws of art, and yet is more important to the 
audience than all the works that \enquote{obey} the laws. 

\item The artist or entertainer cannot exist without urging his product on other 
people. In fact, after developing his product, the artist goes out and tries to 
win public acceptance for it, to advertise and promote it, to sell it, to force it 
on people. If the public doesn't accept it at first, he is disappointed. He 
doesn't drop it, but repeatedly urges the product on them. 

People have every reason, then, to ask the artist: Is your product good for 
me even if I don't like or enjoy it? This question really lays art open. One of 
the distinguishing features of art has always been that it is very difficult to 
defend art without referring to people's liking or enjoying it. (Functions of 
art such as making money or glorifying the social order are real enough, but 
they are rarely cited in defense of art. Let us put them aside.) When one 
artist shows his latest production to another, all he can usually ask is \enquote{Do 
you like it?} Once the \enquote{scientific} justification of art is discredited, the 
artist usually has to admit: If you don't like or enjoy my product, there's no 
reason why you should \enquote{consume} it. 

There are exceptions. Art sometimes becomes the sole channel for political 
dissent, the sole arena in which oppressive social relations can be 
transcended. Even so, subjectivity of value remains a feature which 
distinguishes art and entertainment from other activities. Thus art is 
historically a leisure activity. 

\item But there is a fundamental contradiction here. Consider the object which 
one person produces for the liking, the enjoyment of another. The value of 
the object is supposed to be that you just like it. It supposedly has a value 
which is entirely subjective and entirely within you, is a part of you. Yet---the 
object can exist without you, is completely outside you, is not you or your 
valuing, and has no inherent connection with you or your valuing. The 
product is not personal to you. 

Such is the contradiction in much art and entertainment. it is unfortunate 
that it has to be stated so abstractly, but the discussion is about something 
so personal that there can be no interpersonal examples of it. Perhaps it will 
help to say that in appreciating or consuming art, you are always aware that 
it is not you, your valuing---yet your liking it, your valuing it is usually the 
only thing that can justify it. 

In art and entertainment, objects are produced having no inherent 
connection with people's liking, yet the artist expects the objects to find 
their value in people's liking them. To be totally successful, the object would 
have to give you an experience in which the object is as personal to you as 
your valuing of it. Yet you remain aware that the object is another's 
product, separable from your liking of it. The artist tries to \enquote{be oneself} for 
other people, to \enquote{express oneself} for them. 

\item There are experiences for each person which accomplish what art and 
entertainment fail to. The purpose of this essay is to make you aware of 
these experiences, by comparing and contrasting them with art. I have 
coined the term \term{brend} for these experiences. 

Consider all of your doings, what you already do. Exclude the gratifying of 
physiological needs, physically harmful activities, and competitive activites. 
Concentrate on spontaneous self-amusement or play. That is, concentrate on 
everything you do just because you like it, because you just like it as you do 
it. 

Actually, these doings should be referred to as your just-likings. In saying 
that somebody likes an art exhibit, it is appropriate to distinguish the art 
exhibit from his liking of it. But in the case of your just-likings, it is not 
appropriate to distinguish the objects valued from your valuings, and the 
single term that covers both should be used. When you write with a pencil, 
you are rarely attentive to the fact that the pencil was produced by 
somebody other than yourself. You can use something produced by 
somebody else without thinking about it. In your just-likings, you never 
notice that things are not produced by you. The essence of a just-liking is 
that in it, you are not aware that the object you value is less personal to you 
than your very valuing. 

These just-likings are your \term{brend.} Some of your dreams are brend; and 
some children's play is brend (but formal children's games aren't). In a sense, 
though, the attempt to give interpersonal examples of brend is futile, 
because the end result is neutral things or actions, cut off from the valuing 
which gives them their only significance; and because the end result suggests 
that brend is a deliberate activity like carrying out orders. The only examples 
for you are your just-likings, and you have to guess them by directly 
applying the abstract definition. 

Even though brend is defined exclusively in terms of what you like, it is not 
necessarily solitary. The definition simply recognizes that valuing is an act of 
individuals; that to counterpose the likes of the community to the likes of 
the individuals who make it up is an ideological deception. 

\item It is now possible to say that much art and entertainment are 
pseudo-brend; that your brend is the total originality beyond art; that your 
brend is the absolute self-expression and the absolute enjoyment beyond art. 
Can brend, then, replace art, can it expand to fill the space now occupied by 
art and entertainment? To ask this question is to ask when utopia will 
arrive, when the barrier between work and leisure will be broken down, 
when work will be abolished. Rather than holding out utopian promises, it is 
better to give whoever can grasp it the realization that the experience 
beyond art already occurs in his life---but is totally suppressed by the general 
repressiveness of society. 
\end{enumerate}

\vfill

\textsc{Note:} The avant-garde artist may raise a final question. Can't art or 
entertainment compensate for its impersonality by having sheer newness as a 
value? Can't the very foreignness of the impersonal object be entertaining? 
Doesn't this happen with \essaytitle{Mock Risk Games}, for example? The answer is 
that entertainmental newness is also subjective. What is entertainingly 
strange to one person is incomprehensible, annoying, or irrelevant to 
another. The only difference between foreignness and other entertainment 
values is that brend does not have more foreignness than conventional 
entertainment does. 

As for objective newness, or the objective value of \essaytitle{Mock Risk Games}, these 
issues are so difficult that I have been unable to reach final conclusions 
about them. 

