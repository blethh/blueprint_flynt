\chapter{Creep}

When Helen Lefkowitz said I was "such a creep" at Interlochen in 
1956, her remark epitomized the feeling that females have always had about 
me. My attempts to understand why females rejected me and to decide what 
to do about it resulted in years of confusion. In 1961-1962, I tried to 
develop a theory of the creep problem. This theory took involuntary 
celibacy as the defining characteristic of the creep. Every society has its 
image of the ideal young adult, even though the symbols of growing up 
change from generation to generation. The creep is an involuntary celibate 
because he fails to develop the surface traits of adulthood--poise and 
sophistication; and because he is shy, unassertive, and lacks self-confidence 
in the presence of others. The creep is awkward and has an unstylish 
appearance. He seems sexless and childish. He is regarded by the ideal adults 
with condescending scorn, amusement, or pity. 

Because he seems weak and inferior in the company of others, and 
cannot maintain his self-respect, the creep is pressed into isolation. There, 
the creep doesn't have the pressure of other people's presence to make him 
feel inferior, to make him feel that he must be like them in order not te be 
inferior. The creep can develop the morale required to differ. The creep also 
tends to expand his fantasy life, so that it takes the place of the 
interpersonal life from which he has been excluded. The important 
consequence is that the creep is led to discover a number of positive 
personality values which cannot be achieved by the mature, married adult. 
During the period when I developed the creep theory, I was spending almost 
all of my time alone in my room, thinking and writing. This fact should 
make the positive creep values more understandable. 

\begin{enumerate}
\item Because of his isolation, the creep has a qualitatively higher sense of 
identity. He has a sense of the boundaries of his personality, and a control of 
what goes on within those boundaries. In contrast, the mature adult, who 
spends all his time with his marriage partner or in groups of people, is a mere 
channel into which thoughts flow from outside; he lives in a state of 
conformist anonymity. 

\item The creep is emotionally autonomous, independent, or 
self-contained. He develops an elaborate world of feelings which remain 
within himself, or which are directed toward inanimate objects. The creep 
may cooperate with other people in work situations, but he does not develop 
emotional attachments to other people. 

\item Although the creep's intellectual abilities develop with education, 
the creep lives in a sexually neutral world and a child's world throughout his 
life. He is thus able to play like a child. He retains the child's capacity for 
make-believe. He retains the child's lyrical creativity in regard to 
self-originated, self-justifying activities. 

\item There is enormous room in the creep's life for the development of 
every aspect of the inner world or the inner life. The creep can devote 
himself to thought, fantasy, imagination, imaging, variegated mental states, 
dreams, internal emotions and feelings towards inanimate objects. The creep 
develops his inner world on his own power. His inner life originates with 
himself, and is controlled and intellectually consequential. The creep has no 
use for meditations whose content is supplied by religious traditions. Nor has 
he any use for those drug experiences which adolescents undertake to prove 
how grown-up they are, and whose content is supplied by fashion. The 
creep's development of his inner life is the summation of all the positive 
creep values. 
\end{enumerate}

After describing these values, the creep theory returned to the problem 
of the creep's involuntary celibacy. For physical reasons, the creep remains a 
captive audience for the opposite sex, but his attempts to gain acceptance by 
the opposite sex always end in failure. On the other hand, the creep may 
well find the positive creep values so desirable that he will want to intensify 
them. The solution is for the creep to seek a medical procedure which will 
sexually neutralize him. He can then attain the full creep values, without the 
disability of an unresolved physical desire. 

Actually, the existence of the positive creep values proves that the 
creep is an authentic non-human who happens to be trapped in human social 
biology. The positive creep values imply a specification of a whole 
non-human: social biology which would be appropriate to those values. 
Finally, the creep theory mentioned that creeps often make good grades in 
school, and can thus do clerical work or other work useful to humans. This 
fact would be the basis for human acceptance of the creep. 

In the years after I presented the creep theory, a number of 
inadequacies became apparent in it. The principal one was that I managed to 
cast off the surface traits of the creep, but that when I did my problem 
became even more intractable. An entirely different analysis of the problem 
was required. 

My problem actually has to do with the enormous discrepancy between 
the ways I can relate to males and the ways I can relate to females. The 
essence of the problem has to do with the social values of females, which are 
completely different from my own. The principal occupation of my life has 
been certain self-originated activities which are embodied in "writings." Now 
most males have the same social values that I find in all females. But there 
have always been a few males with exceptional values; and my activities have 
developed through exchanges of ideas with these males. These exchanges 
have come about spontaneously and naturally. In contrast, I have never had 
such an exchange of ideas with females, for the following reasons. Females 
have nothing to say that applies to my activities. They cannot understand 
that such activities are possible. Or they are a part of the "masses" who 
oppose and have tried to discourage my activities. 

The great divergence between myself and females comes in the area 
where each individual is responsible for what he or she is; the area in which 
one must choose oneself and the principles with which one will be identified. 
This area is certainly not a matter of intelligence or academic degrees. 
Further, the fact that society has denied many opportunities to females at 
one time or another is not involved here. (My occupation has no formal 
prerequisites, no institutional barriers to entry. One enters it by defining 
oneself as being in it. Yet no female has chosen to enter it. Or consider such 
figures as Galileo and Galois. By the standards of their contemporaries, these 
individuals were engaged in utterly ridiculous, antisocial pursuits. Society 
does not give anybody the "opportunity" to engage in such pursuits. Society 
tries to prevent everybody from being a Galileo or Galois. To be a Galileo is 
really a matter of choosing sides, of choosing to take a certain stand.) 

Let me be specific about my own experiences. When I distributed the 
prospectus for \journaltitle{The Journal of Indeterminate Mathematical Investigations} to 
graduate students at the Courant Institute in the fall of 1967, the most 
negative reactions came from the females. The mere fact that I wanted to 
invent a mathematics outside of academic mathematics was in and of itself 
offensive and revolting to them. Since the academic status of these females 
was considerably higher than my own, the disagreement could only be 
considered one of values. 

The field of art provides an even better example, because there are 
many females in this field. In the summer of 1969 I attended a meeting of 
the women's group of the Art Workers Coalition in New York. Many of the 
women there had seen my Down With Art pamphlet. Ail the females who 
have seen this pamphlet have reacted negatively, and it is quite clear what 
their attitude is. They believe that they are courageously defending modern 
art against a philistine. They consider me to be a crank who needs a "modern 
museum art appreciation course." The more they are pressed, the more 
proudiy do they defend "Great Art." Now the objective validity of my 
opposition to art is absolutely beyond question. To defend modern art is 
precisely what a hopeless mediocrity would consider courageous. Again, it is 
clear that the opposition between myself and females is in the area where 
one must choose one's values. 

I have found that what I really have to do to make a favorable 
impression on females is to conceal or suspend my activities----the most 
important part of my life; and to adopt a facade of conformity. Thus, I 
perceive females as persons who cannot function in my occupation. I 
perceive them as being like an employment agency, like an institution to 
which you have to present a conformist facade. Females can he counted on to 
represent the most "social, human" point of view, a point of view which, as I 
have explained, is distant from my own. (In March 1970, at the Institute for 
Advanced Study, the mathematician Dennis Johnson said to me that he 
would murder his own mother, and murder all his friends, if by doing so he 
could get the aliens to take him to another star and show him a higher 
civilization. My own position is the same as Johnson's.) 

It follows that my perception of sex is totally different from that of 
others. The depictions of sex in the mass media are completely at variance 
with my own experience. I object to pornography in particular because it is 
like deceptive advertising for sex; it creates the impression that the physical 
aspect of sex can be separated from human personalities and social 
interaction. Actually, if most people can separate sex from personality, it is 
because they are so average that their values are the same as everybody else's. 
In my case, although I am a captive audience for females for physical 
reasons, the disparity between my values and theirs overrides the physical 
attraction I feel for them. It is hard enough to present a facade of 
conformity in order to deal with an employment agency, but the thought of 
having to maintain such a facade in a more intimate relationship is 
completely demoralizing. 

What conclusions can be drawn by comparing the creep theory with my 
later experience? First, some individuals who are unquestionably creeps as 
far as the surface traits are concerned simply may not be led to the deeper 
values I described. They may not have the talent to get anything positive out 
of their involuntary situation; or their aspirations may be so conformist that 
they do not see their involuntary situation as a positive opportunity. Many 
creeps are female, but all the evidence indicates that they have the same 
values I have attributed to other females---values which are hard to reconcile 
with the deeper creep values. 

As for the positive creep values, I may have had them even before I 
began to care about whether females accepted me. For me, these values may 
have been the cause, not the effect, of surface creepiness. They are closely 
related to the values that underlie my activities. It is not necessary to appear 
strangely dressed, childish, unassertive, awkward, and lacking in confidence 
in order to achieve the positive creep values. (I probably emphasized surface 
creep traits during my youth in order to dissociate myself from conformist 
opinion at a time when I hadn't yet had the chance to make a full 
substantive critique of it.) Even sex, in and of itself, might not be 
incompatible with the creep inner life; what makes it incompatible is the 
female personality and female social values, which in real life cannot be 
separated from sex and are the predominant aspect of it. 

Having cast off the surface traits of the creep, I can now see that 
whether I make a favorable impression on females really depends on whether 
I conceal my occupation. Celibacy is an effect of my occupation; it does not 
have the role of a primary cause that the creep theory attributed to it. 
However, it does have consequences of its own. In the context of the entire 
situation I have described, it constitutes an absolute dividing line between 
myself and humanity. It does seem to be closely related to the deeper creep 
values, especially the one of living in a child's world. 

As for the sexual neutralization advocated in the creep theory, to find a 
procedure which actually achieves the stated objective without having all 
sorts of unacceptable side effects would be an enormous undertaking. It is 
not feasible as a minor operation developed for a single person. Further, as 
the human species comes to have vast technological capabilities, many 
special interest groups will want to tinker with human social biology, each in 
a different way, for political reasons. I am no longer interested in petty 
tinkering with human biology. As I make it clear in other writings, I am in 
favor of building entities which are actially superior to humans, and which 
avoid the whole fabric of human biosocial defects, not just one or two of 
them. 

\clearpage
{
	

2/22/1963 
Henry Flynt and Jack Smith demonstrate against Lincoln Center, February 22, 1963 
(photo by Tony Conrad) 
}
\clearpage


