\chapter{The Perception-Dissociation of Physics}







From the physicist's point of view, the human dichotomy of sight and 
touch is a coincidence. It does not correspond to any dichotomy in the 
objective physical world. Light exerts pressure, and substances hot to the 
touch emit infrared light. It is just that the range of human receptors is too 
limited for them to register the tactile effect of light or the visual effect of 
moderate temperatures. 

Our problem is to determine what observations or experiences would 
cause the physicist to say that the objective physical world had split along 
the humen sight-touch boundary, to say that the human sight-touch 
dichotomy was an unavoidable model of objective physical reality. Our 
discussion is not about perfectly transparent matter, or light retlection and 
emission in the absence of matter, or the dissociation of electromagnetic and 
inertial phenomena, or the fact that human sight registers light, while touch 
registers inertia, bulk modulus, thermal conduction, friction, adhesion, and 
so on. (However, these concepts may have to be introduced to complete our 
discussion.) Our discussion is about a change in the physicist's observations 
or experiences, such that the anomalous state of affairs would be an 
experimental analogue to the sight-touch dichotomy of philosophical 
subjectivism. Of course, philosophical subjectivism itself will not enter the 
discussion. 

Because of the topic, our discussion will often seem psychological and 
even philosophical. However, the psychology involved always has to do with 
experimentally demonstrable aspects of perception. The philosophy involved 
is always scientific concept formation, the relating of concepts to 
experiments. Sooner or later it will be clear that our only concern is with 
experiences that would cause a physicist to modify physics. 

Throughout much of the discussion, we have to assume that the human 
physicist exists before the sight-touch split occurs, that he continues to exist 
after it occurs, and that he functions as a physicist after it occurs. Therefore, 
we begin as follows. A healthy human has a realm of sights, and a realm of 
touches: and there is a correlation between the two which receives its highest 
expression in the concept of the object. (In psychological jargon, intermodal 
organization contributes to the object Gestalt. Incidentally, for us \enquote{touch} 
includes just about every sense except sight, hearing, smell.) Suppose there is 
a change in which the tactile realm remains coherent, if not exactly the same 
as before, and the visual realm also remains coherent; but the correlation 
between the two becomes completely chaotic. A totally blind person does 
not directly experience any incomprehensible dislocation, nor does a person 
with psychogenic tactile anesthesia (actually observed in hysteria patients). 
Let us define such a change. Consider the sight-touch correlation identified 
with closing one's eyes. The point is that there is a whole realm of sights 
which do not occur when one can feel that one's eyes are closed. 

Let $T$ indicate tactile and $V$ indicate visual. Let the tactile sensation of 
open eyes be $T_1$, and of closed eyes be $T_2$. Now anything that can be seen 
with closed eyes---from total blackness, to the multicolored patterns produced 
by waving the spread fingers of both hands between closed eyes and direct 
sunlight---can no doubt be duplicated for open eyes. Closed-eye sights are a 
subset of open-eye sights. Thus, let sights seen only with open eyes be $V_1$, 
and sights seen with either open or closed eyes be $V_2$: If there are sights seen 
only with closed eyes, they will be $V_3$; we want disjoint classes. We are 
interested in the temporal concurrence of sensations. Combining our 
definitions with information about our present world, we find there are no 
intrasensory concurrences (eyes open and closed at the same time). Further, 
our change will not produce intrasensory concurrences, because each realm 
will remain coherent. Thus, we will drop them from our discussion. There 
remain the intersensory concurrences, and four can be imagined; let us 
denote them by the ordered pairs $(T_1, V_1)$, $(T_1, V_2)$, $(T_2, V_1)$, $(T_2, V_2)$. In 
reality, some concurrences are permitted and others are forbidden, Let us 
designate each ordered pair as permitted or forbidden, using the following 
notation. Consider a rectangular array of \enquote{places} such that the place in the 
$i$th row and $j$th column corresponds to $(T_i, V_j)$, and assign a $p$ or $f$ (as 
appropriate) to each place. Then the following state array is a description of 
regularities in our present world. 

\begin{equation}
\begin{pmatrix}
	p & p\\
	f & p 
\end{pmatrix}
\end{equation}


So far as temporal successions of concurrences (within the present 
world) are concerned, any permitted concurrence may succeed any other 
permitted concurrence. The succession of a concurrence by itself is 
excluded, meaning that at the moment, a $V_1$, is defined as lasting from the 
time the eyes open until the time they next close. 

We have said that our topic is a certain change; we can now indicate 
more precisely what this change is. As long as we have a $2\times2$ array, there are 
16 ways it can be filled with $p$'s and $f$'s. That is, there are 16 imaginable 
states. The changes we are interested in, then, are specific changes from the 
present state (\ref{physpresent}) to another state such as \ref{physafter}.

\vskip 1em
{\centering
\parbox{0.9\textwidth}{\centering
	\parbox{1.5in}{
		\begin{equation}\label{physpresent}
\begin{pmatrix}
	p & p \\
	f & p
\end{pmatrix}
		\end{equation}}\parbox{1.5in}{\begin{equation}\label{physafter}
\begin{pmatrix}
	p & f \\
	p & p
\end{pmatrix}\end{equation}}\par}\par}
\vskip 1em

However, 
we want to exclude some changes. The change that changes nothing is 
excluded. We aren't interested in changing to a state having only $f$'s, which 
amounts to blindness. A change to a state with a row or column of $f$'s leaves 
one sight or touch completely forbidden (a person becomes blind to 
open-eye sights); such an \enquote{impairment} is of little interest. Of the remaining 
changes, one merely leaves a formerly permitted concurrence forbidden: 
closed-eye sights can no longer be seen with open eyes. The rest of the 
changes are the ones most relevant to perception-dissociation. They are 
changes in the place of the one $f$; the change to the state having only $p$'s; 
and finally 

\vskip 1em
{\centering
\parbox{0.9\textwidth}{\centering
	\parbox{0.75in}{\raggedleft $\begin{pmatrix} p & p \\ f & p \end{pmatrix}$}
		\parbox{0.5in}{\centering \huge $\rightarrow$ }
		\parbox{0.75in}{$\begin{pmatrix} f & p \\ p & f \end{pmatrix}$}}}
\vskip 1em

In general, we speak of a partition of a sensory realm into disjoint 
classes of perceptions, so that the two partitions are $[T_j]$ and $[V_j]$. The 
number of classes in a partition, m for touch and n for sight, is its 
detailedness. The detailedness of the product partition $[T_j]\times [V_j]$ is written 
$m\times n$. This detailedness virtually determines the $(mn)^2$ imaginable states, 
although it doesn't determine their qualitative content. Now suppose one 
change is followed by another, so that we can speak of a change series. It is 
important to realize that by our definitions so far, a change series is not a 
conposition of functions; it is a temporal phenomenon in which each state 
lasts for a finite time. (A function would be a general rule for rewriting 
states. A $2\times2$ rule might say, rotate the state clockwise one place, from \ref{physegcwa} to \ref{physegcwb}.

\vskip 1em
			{\centering\parbox{0.9\textwidth}{\centering
\parbox{1.25in}{\raggedleft\begin{equation}\label{physegcwa}\begin{pmatrix}a & b \\ c & d\end{pmatrix}\end{equation}}
\parbox{1.25in}{\begin{equation}\label{physegcwb}\begin{pmatrix}c & a \\ d & b\end{pmatrix}\end{equation}}}}
	\vskip 1em

But a composition of rules would not be a temporal series; it would be a new 
rule.) Returning to the sorting of changes, we always exclude the no-change 
changes, and states having only $f$'s. We are unenthusiastic about \enquote{impairing}
changes, changes to states with rows or columns of $f$'s. Of the remaining 
changes, some merely forbid, replacing $p$'s with $f$'s. The rest of the changes 
are the most perception-dissociating ones. 

As for changes in the succession state in the eye case, either they leave 
the forbidden concurrence permitted; or else they merely leave permitted 
successions forbidden---for example, in order to open your eyes in the dark 
you might have to open them in the light and then turn the light off. These 
secondary changes are of secondary interest. 

If we simply continue with the material we already have, two lines of 
investigation are possible. The first investigation is mathematical, and 
apparently amounts to combinatorial algebra. The second investigation 
concerns the relation between concurrences and commands of the will 
(observable as electrochemical impulses along efferent neurons). If a change 
occurs, and the perceptual feedback from a willed command consists of a 
formerly forbidden concurrence, is it $T$ or $V$ that conflicts with the 
command? Is it that you tried to close your eyes but couldn't get the sight 
to go away, or that you were trying to look at something but felt your eyes 
close anyway? 

Before we carry out these investigations, however, we must return to 
our qualitative theory. If one of our eye changes happens to a physicist, he 
may immediately conclude that the cause of the anomaly is in himself, that 
the anomaly is psychological. But suppose that starting with a state for an 
extremely detailed product partition describing the present world, a whole 
change series occurs. Let $p$'s be black dots and $f$'s be white dots, and imagine 
a continuously shaded gray rectangle whose shading suddenly changes from 
time to time. We evoke this image to impress on the reader the 
extraordinary qualities of our concept, which can't be conveyed in ordinary 
English. Suppose also that to the extent that communication between 
scientists is still possible, perhaps in Braille, everybody is subjected to the 
same changes. If the physicist turns to his instruments, he finds that the 
anomalies have spread to his attempts to use them. The changes affect 
everything---everything, that is, except the intrasensory coherence of each 
sensory realm. Intrasensory coherence becomes the only stable reference 
point in the \enquote{world.} The question of \enquote{whether the anomalies are really 
outside or only in the mind} comes to have less and less scientific meaning. 
If physics survived, it would have to recognize the touch-sight dichotomy as 
a physical one! This scenario helps answer a question the reader may have 
had: what is the methodological status of our states? They don't seem to be 
either physics or psychology, yet it is quite clear how we would know if the
asserted regularities had changed; in fact, that is the whole point of the 
states. The answer is that the states are perfectly good assertions (of 
observed regularities) which would acquire primary importance if the 
changes actually occurred. In fact, the changes would among other things 
shift the boundaries of physics and psychology; but we insist that our 
interest is in the physicist's side of the boundary. To complete the 
investigation we have outlined, the relation between what the states say and 
what existing physics says should be established, so that we will know what 
has to be done to the photons and electrons to produce the changes. It is the 
same as with time travel: the hard part is deciding what it is and the even 
harder part is making it happen. 

\visbreak

However, the foundations of our qualitative theory are not yet 
satisfactory, We have assumed that the physicist will be able to identify the 
subjective concurrences of perceptions, and will be able to identify his 
perceptions themselves, even if sense correlation becomes completely 
chaotic. We have assumed that the physicist will be able to say \enquote{I see a book 
in my hand but I concurrently feel a pencil.} These assumptions may not be 
justified at all. It is quite likely that the physicist will say, \enquote{I don't even 
know whether the sight and the touch seem concurrent; I don't even know 
whether I think I see a book; I don't even know whether this sensation is 
visual.} In fact, the anomalies may cause the physicist to decide that books 
never looked like books in the first place. In this case, the occurrence of the 
changes would render meaningless the terms in which the changes are 
defined. Alternately, if the changes produce a localized chaos, so that 
everything fits together except the book seen in the hand, the physicist may 
literally force himself to re-see that-book as a pencil, and in time this 
compensation may become habitual and \enquote{pre-conscious.} In this case, if the 
physicist remembers the changes, he will be convinced that they were a 
temporary psychological malfunction. 

These criticisms are based on the fact that our simple perceptions are 
actually learned, \enquote{unconscious} interpretations of raw data which by 
themselves don't look like anything. This fact is demonstrated by a vast 
number of standard experiments in which the raw data are distorted, the 
subject perceptually adapts to the distorted data, and then the subject is 
confronted with normal sensations again. The subject finds that the old 
familiar sensation of a table looks quite wrong, and that he has to make an 
effort to see the table which he knows is there. 

Consider a modification of the clock-bell simultaneity experiment. The 
subject sits facing a large clock with a second-hand. His hearing is blocked in 
some way. Behind him, completely unseen, is a device which can give hima 
quick tap, a tactile sensation. There is also an unseen movie camera which 
photographs both the tactile contact and the clock face. The subject is 
tapped, and must call out the second-hand reading at the time of the tap. We 
expect a discrepancy between what the subject says and what the film says; 
but even if there is none, the experiment can proceed. Tell the subject that 
he always placed the tap earlier than it actually occurred, and that he will be 
given a reward if he learns to perceive more accurately. The purpose of the 
experiment is to demonstrate to the subject that even his perception of 
subjective simultaneity can be consciously modified. In the course of 
modification, he may not even know whether two perceptions seem 
simultaneous. 

This criticism of the changes defined earlier is important, but it may 
not be insurmountable. Although Stratton became used to his trick 
eyeglasses, the image continued to seem distorted. There is some stability to 
our identification of our perceptions. Also, the physicist in our earlier 
scenario might ultimately adapt to the changes. He might realize that it is 
possible separately to identify sights and touches. Only the sight-touch 
correlation is unidentifiable; and the concept of such a correlation might 
become an abstract concept of physics just as the concept of particle 
resonance is today. 

Time is inescapably involved in our discussion; so we must decide what 
happens to time as a distinct physical category, and as a sense, in 
perception-dissociation. Here, we will simply distinguish three sorts of time. 
First, there is subjective concurrence, which we have already begun to 
discuss. Secondly, there is the physicist's operational definition of time. 
There must be two repeating processes, which to the best of our knowledge 
are causally independent, so that irregularities in one process aren't 
automatically introduced in the other. If the ratio of the repetitions of the 
two processes is constant, we assume that the repetitions divide time into 
equal intervals. Eventually the physicist arrives at a concept of time as a real 
line along which movement can be both forward and backward (Feynman). 
One effect of perception-dissociation relating to this sort of time would be 
to disrupt the ratios of visual clocks (such as electric wall clocks) to tactile 
clocks (such as the pulse). The third idea of time comes from an unpublished 
manuscript by John Alten, a Harvard classmate of mine. According to Alten, 
our most intimate sensation of futurity is associated with our acts of will. 
\enquote{The future} is simply the time of willing. In comparison with volitional 
futurity, the physicist's linear, reversible time is a mere spatial concept. The 
empirical importance of Alten's idea is that it raises the question of what the 
perceptual frustration of the will (as we defined it) would do to the sense of 
futurity. 

\visbreak

We now come to some considerations which will help us develop the 
state descriptions, and which also show that from one point of view, the 
states are actually necessary for the operational definition of physical 
language. Let parallel but separated sheets of clear plastic and colored plastic 
be mounted in lighting conditions so that the subject can't see the clear 
plastic. He touches the clear plastic, but from what he sees, he believes he is 
touching the colored plastic. The lighting is then changed and his error is 
exposed. In some sense, the sight-touch concurrence identifying an object 
was a mere coincidence. Next, we produce another colored sheet for the 
subject to touch, and we are able to convince him that this time the 
object-identifying concurrence is more than a coincidence. 

The physicist interprets this latter case by saying that the matter which 
resists the pressure of the subject's finger also reflects the light into his eyes. 
To the extent that the physicist's interpretation is causal, it employs the 
concept of \enquote{matter,} a concept which is not really either visual or tactile. 
The physicist explains a sight and a touch with a reference beyond both sight 
and touch. It is important, then, to know the operational definition of the 
physicist's statement, the testing procedures which give the statement its 
immediate meaning. What is significant is that the testing procedures cannot 
be reduced to purely visual procedures or purely tactile procedures. 
Affecting the world requires tactile operations; and the visual \enquote{reading} of 
the world is so woven into physics that it can't be given up. Yet our 
experiment showed that the subject can be fooled by object-identifying 
concurrences, and the physicist is supposed to tell us how to avoid being 
fooled. 

We find, then, that there is nothing the physicist can appeal to, in 
testing object-identifying concurrences, that doesn't immediately rely on 
other object-identifying concurrences, the very concurrences which are 
suspect. It is as if the physicist proposed to prove that clicks come from a 
certain metronome by manipulating a detecting device that outputs its data 
as sounds. But suppose the physicist proves that the clicks come from the 
metronome by showing (1) that the metronome has to be stopped or 
removed to stop the clicks, and (2) that the clicks stop if the metronome is 
stopped or removed. The physicist proves that the object-identifying 
concurrence is not a coincidence by demonstrating that certain related 
concurrences are forbidden. We suggest that the physicist ultimately handles 
touch-sight concurrences in just this way. The operational basis of the 
physicist's activity comes down to our states. (But note that the physicist 
has tests, which do not rely directly on his hearing, to determine whether the 
clicks come from the metronome!) One way to develop our states, then, 
may be to develop substates which express the differences between those 
object-identifying concurrences that are coincidental and those that 
aren't---the differences illustrated by the plastic sheet experiment. 

