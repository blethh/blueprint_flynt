\chapter{The Dream Reality}

\section{Memo on the Dream Project}

Original aim: To recreate the effect of e.g. Pran Nath's singing---transcendent 
inner escape---in direct life rather than art. I needed material which could 
function as an alien civilization (since the source of Pran Nath's expression is 
an alien civilization relative to me); yet which was encultured in me and not 
an affectation or pretense. I decided to use dreams as the material, assuming 
that my dreams would take me to alien worlds. But mostly they did not. 
Mostly my dreams consist of long periods of tawdry, familiar life interrupted 
occasionally by senseless, unmotivated anomalies. In contrast, my original 
aim required alluring, psychically gratifying material. 

The emphasis shifted to redefining reality so that dreams were on the same 
level as waking life; so that they were apprehended as what they seem to be: 
literal reality (and not memory, precognition, or symbolism). The project 
was still arcane, but in a drastically different way. I was getting into an 
alternate reality which was extremely bizarre but not psychically gratifying. 
It was boringly frightful and sometimes obscene. I became concerned with 
analytical study of the natural order of the dream world, a para-scientific 
investigation. As I grappled with the rational arguments against treating 
dreams as literal reality, the project became a difficult analytical exercise in 
the philosophy of science. The original sensuous-esthetic purpose was lost. 

Now I would like to return to the original aim, but how to do it? Obtain 
other people's dreams---see if they are more suitable? Work only with my 
very rare dreams which do take me to alien worlds? Try to alter the content 
of my raw dreams? Attempt to affect content of dreams by experiment in 
which many people sleep in same room and try to communicate in their 
sleep? The most uncertain approach to a solution: set up a transformation 
on my banal dreams, so that to the first-order activity of raw dreaming is 
added a second-order activity. The transformation procedure to somehow 
combine conscious ideational direction---coding of the banal dreams---with 
alteration of my experience, my esthesia, my lived experience. 


\section{Dreams and Reality---An Experimental Essay}

Excerpts from my dream diary which are referred-to in the essay that 
follows. 

\dreamdate{12/11/1973}

I notice a state between waking and dreaming: a waking dream. I have 
been asleep; I wake up; I close my eyes to sleep again. While not yet asleep, I 
experience isolated objects before me as in a dream, but with no 
background, only a dark void. In this case, there are two pocket combs, both 
with teeth broken. In the waking world, I threw away one of my two pocket 
combs because I broke it; the other comb is still in good condition. 

\dreamdate{12/30/1973}

I am chased by the police for one block west on West Market Street in 
Greensboro. I reach the intersection with Eugene Street, and in the north 
direction there is a steep hill rather than the street. The surface of the hill is 
bare ground and grass. I run up the hill, sensing that if I can get over the hill 
I will find Friendly Road and the general neighborhood of my mother's 
houses on the other side. The police start shooting. If I can get a few yards 
farther on the top of the hill I will be past the line of fire. I take a headlong 
dive and awaken in the middle of the dive to find myself diving forward on 
my mattress in the front room of my apartment. The action is carried on 
continuously through waking up and through the associated change of 
setting. 


\dreamdate{1/12/1974}

Just before I go to sleep for the night, I am lying in bed drowsy. I think 
of being, and suddenly am, at the south edge of the Courant Institute plaza, 
which is several feet above the sidewalk. The edge of the plaza and the drop 
are all I see. It is night; and there is only a void where the peripheral 
environment should be. (Comment: It is of great theoretical importance that 
while most of the internal reality cues were present in this experience, some, 
like the peripheral environment, were not. In my dream experiences, all 
reality cues are present.) The drop expands to twenty or thirty feet, and I 
start to fall off. Fright jolts me completely awake. I have had something like 
a waking nightmare and have awakened from being awake. I thought of the 
scene, was suddenly in it (except for peripheral reality cues), lost control and 
became endangered by it, and then snapped back to my bedroom. 

\dreamdate{1/1-/1974}

One or two nights after 1/12/74 I was lying in bed just before going to 
sleep. I could see women standing on a sidewalk. The scene was real, but I 
was not in it; I was a disembodied spectator. Also, the peripheral 
environment was absent. The reality was between that of a waking 
visualization and that of the Courant Institute incident of 1/12/74. 
Comment: The differences between this experience and a waking 
visualization are that the latter is less vivid than seeing and is accompanied 
by waking reality cues such as cues of bodily location. 


\dreamdate{1/16/1974}

\begin{enumerate}
\item I am in an apartment vaguely like the first place in which I lived, at 
1025 Madison Avenue in Greensboro. I am a spy. I am teen-aged and short; 
and I am in the apartment with several enemy men, who are middle-aged and 
adult-sized. My code sheets look like the sheets of Yiddish I have been 
copying out in waking life. Eventually the men discover me in the front 
room with the code sheets on a fold-up desk. They chase me out the front 
door and onto the west side of the lawn, and shoot me with a needle gun. At 
that moment my consciousness jumps from my body and becomes that of a 
disembodied spectator watching from an eastward location, as if I were 
watching a film. 

\item I am living in a dormitory in a rural setting with other males. At one 
point I walking barefoot in weeds outside the dormitory, and Supt. Toro 
tells me I am walking in poison ivy. My feet begin to show the rash, but I 
recognize that I am in a dream and think that the rash will not carry over to 
the waking state. I then begin to will away the rash in the dream, and I 
succeed, 
\end{enumerate}


\dreamdate{1/20/1974}

For some reason the dream associates Simone Forti with flute-like 
music. It is shortly before midnight. In the dream I believe that Simone lives 
in a loft on the east side of Wooster Street. The blocks in SOHO are very 
small. If I walk through the streets and whistle, she will hear me. I start to 
whistle but can only whistle a single high note. I half awaken but continue 
whistling, or trying to; the dream action continues into waking. But I cannot 
change pitch or whistle clearly because my mouth is taped. As I realize this, I 
awaken fully. 

Comments: I tape my mouth at night so I will sleep with my mouth closed. I 
experimented at trying to whistle with the tape on while fully awake. The 
breath just hisses against the tape. The pitch of the hiss can be varied. 


\dreamdate{2/1/1974}

1. I try to assist a man in counterfeiting ten dollar bills by taking half 
of a ten, scotch taping it to half of a one, and then coloring over the one 
until it looks like the other half of the ten. The method fails because I bring 
old crumpled tens rather than new tens, and the one doilar bills are new. 


Comments: There are no natural anomalies in this dream at all. What is 
anomalous is that this counterfeiting method seems perfectly sensible, and I 
only begin to question it when we try to fit the crumpled half-bill to the 
crisp half-bill. Why am I so foolish in this dream? I retain my identity as 
Henry Flynt, and yet my outlook, my sense of what is rational, is so 
different that it is that of a different person. More generally, the person I am 
in my dreams is much more limited in certain ways that I am in waking life. 
My waking preoccupations are totally absent from my dreams. Instead there 
is bland material about my early life which could apply to any child or 
teen-ager. Thus, I must warn readers who know me only from this diary not 
to try to make the image of me here fit my waking life. 


\dreamdate{2/3/1974}

3. I have had several dreams that I am taking the last courses of my 
student career. (In waking life I have completed all course work.) I am 
usually failing them. Tonight I dream that I have gone all semester without 
studying (in a course in English?). Now I am in the final exam and sinking. I 
will have to repeat these courses. Subsequently, I am sitting in a school 
office (of a professor or psychologist?), giving him a long list (of words, a 
foreign vocabulary?). (I mention this episode because I remember that while 
I retained my nominal identity as Henry Flynt, I had the mind of a different 
person. I experienced another person's existence instead of mine. Professor 
Nell also appeared somewhere in this dream; as he has in several school 
dreams I have had recently. 


\dreamdatecomment{2/3/1974}{This is the date I recorded, but it seems that it would have to be later.}

I get up in the morning and decide to have a self-indulgent breakfast 
because of the unpleasantness of working on my income tax the day before. 
So I put two slices of pizza in the oven, and also eat two bakery sweets, 
possibly \'{e}clairs. Then I think that a Mexican TV dinner would have been 
better all around, but it is too late; I have to eat what I am already preparing. 
Subsequently, I go with John Alten to a Shoreham Cafeteria at Houston and 
Mercer Streets. The cafeteria chain is a good one, but this cafeteria is dark 
and extremely dingy upstairs where the serving line is. John complains that 
there is no ventilation and that he is suffocating, and he stalks out. 

Comment: When I awoke, my first thought was that the pizza in the oven 
would be burning. (I assumed that I had arisen, put the pizza in the oven, 
and gone back to sleep.) But then I realized that the breakfast was a dream. I 
got up and prepared the Mexican dinner which I had decided was best in the 
dream, but I also ate one \'{e}clair. 

\dreamdate{7/8/1974}

I am caught out in a theft of money, and I feel that the rest of my life 
will be ruined. 

Comment: The quality of the episode depended on my 
strong belief in the reality of the social future and in my ability to form 
accurate expectations about it. When I awakened, the whole misadventure 
vanished. 


End of excerpts from my dream diary.

\begin{quotation}
"... It is correct to say that the objective world is a synthesis of private views 
or perceptions... But ... inasmuch as it is the common objective world that 
renders ... general knowledge possible, it will be this world that the scientist 
will identify with the world of reality. Henceforth the private views, though 
just as real, will be treated as its perspectives. ... the common objective 
world, whether such a thing exists or is a mere convenient fiction, is 
indispensable to science ... 
."\footnote{A. d'Abro, The Evolution of Scientific Thought (New York, Dover, 1950), pp. 176--7}
\end{quotation}


\textbf{A.} We wish to postulate that dreams are exactly what they seem to be 
while we are dreaming, namely, literal reality. Naively, we want to get closer 
to literal empiricism than natural science is. But science has worked out a 
very comfortable world-view on the assumption that both dreams and 
semi-conscious quasi-dreams are mere subjective phenomena of individual 
consciousness. If we wish to carry through the postulate that dreams are 
literal reality, then we will have to adopt a cognitive model quite different 
from that of natural science. It is of crucial importance that we are not 
interested in superstition. We do not wish to adopt a cognitive model which 
would simply be defeated in competition with science. We wish to be at least 
as rational, as empirical, and as cognitively parsimonious as science is. We 
want our cognitive model to be compelling, and not to be a plaything which 
is easily taken up and easily discarded. 

The question is whether there can be a rational empiricism which 
differs from science in placing dreamed episodes on the same level as waking 
episodes, but which stops short of the "nihilistic empiricism" of my 
philosophical essay entitled \essaytitle{The Flaws Underlying Beliefs}. (In effect, the 
latter essay rejects other minds, causality, persistent objective entities, past 
time, the possibility of objective categories and significant language, and so 
forth, ending up with ungraded immediate experience.) 

As an example of our problem, the waking scientific outlook assumes 
that a typewriter continues to exist even when we turn our backs on it 
(persistence of objective entities). In many of our dreams we make the same 
sort of assumption. In other words, in some of our dreams the natural order 
is not noticeably different from that of the waking world; and in many 
dreams our conscious world-view has much in common with waking 
common sense or scientific pragmatism. On 2/3/1974 I had a dream in which 
a typewriter was featured. I certainly assumed that the typewriter continued 
to exist when my back was turned to it. On 7/8/1974 I dreamed that I was 
caught out in a theft of money, and I felt my life would be ruined because of 
it. I certainly assumed the reality of the social future, and my ability to form 
accurate expectations about it. These examples illustrate that we are not 
nihilistic empiricists in our dreams. The question is whether acceptance of 
the pragmatic outlook which we have in dreams is consistent with not 
regarding the dream-world as a subjective phenomenon of individual 
consciousness. Can we accept dreams as "literal reality"; or must we reject 
the very concept of "reality" on order to defend the placing of the dream 
world on the same level as the waking world? 

In summary, the question is whether we can place dreams on the same 
fevel as the waking world while stopping short of nihilistic empiricism. A 
further difficulty in accomplishing this aim is that neurological science might 
succeed in gaining complete experimental control of dreams. Scientists might 
become able to produce dreams at will and to monitor them. The whole 
phenomenon of dreaming would then tend to be totally assimilated to the 
outlook of scientists. Their decision to treat dreams as subjective phenomena 
of individual consciousness would be greatly supported by these 
developments. Would we have to go all the way to nihilistic empiricism in 
order to have a basis for rejecting the neurologists' accomplishments? 

Still another difficulty is presented for us by semi-conscious 
quasi-dreams such as the ones described in my diary. Semi-conscious 
quasi-dreams exhibit some reality cues, but lack other important internal 
reality cues. Science handles these experiences easily, by dismissing them 
along with dreams as subjective phenomena of individual consciousness. 
Suppose we accept that the semi-conscious quasi-dreams are illusory reality. 
But if they can be illusory reality, how can we exclude the possibility that 
dreams might be also? If, on the other hand, we accept the quasi-dreams as 
literal reality, what about the missing reality cues? Can we justify different 
treatment for dreams and quasi-dreams by saying that all reality cues have to 
be present before an experience is accepted as non-illusory? If we propose 
to do so, the question then becomes whether we should accept the weight 
which common sense places on reality cues. 

Why do we wish to stop short of nihilistic empiricism? Because we do 
wish to assert that dreams can be remembered; that they can be described in 
permanent records; that they can be compared and studied rationally. We do 
want to cite the past as evidence; we do want to distinguish between actual 
dream experience and waking fabrications, waking lies about what we have 
dreamed; and we do want to describe what we experience in intersubjective 
language.

As easy way out which would offend nobody would be to treat dreams 
as simulations of alternate universes. But this approach is a cowardly evasion 
for several reasons. It excludes the phenomenon of the semi-conscious 
quasi-dream, which poses the problem of internal reality cues in the sharpest 
way. Further, we cannot give up the notion that our project is nearer to 
literal empiricism than natural science is. We cannot accept the notion that 
we must dismiss some of our experiences as mere illusions, but not all of 
them. We do not see dreams as simulations of anything. Some of the most 
interesting observations I have made about connections between adjacent 
dreamed and waking episodes in my own experience are noticeable only 
because I take both dreamed and waking experience literally. 

\gap


\textbf{B.} Before we continue our attempt to resolve our methodological 
problem, we will provide more detail on topics which we have mentioned in 
passing. We begin with the purported empiricism of natural science. The 
philosopher Hume postulated that experience was the only raw material of 
reality or cognition. However, he did not content himself with ungraded 
experience. He insisted on draping the experiential raw material on an 
intellectual framework in such a way that experience was used to simulate 
the inherited conception of. reality, a conception which we will call 
Aristotelian realism. Similarly for the purported empiricism of natural 
science. In fact, the working scientist learns to think of the framework or 
model as primary, and of experiences and verification procedures as ancillary 
to it. The quotation by d'Abro which heads this essay concedes as much. 

What we are investigating is whether experiences can be draped on a 
different intellectual framework in which dreamed and waking life come out 
as equally real. Some examples of alternate verification conventions follow. 

\begin{enumerate}
\item Accept intersubjective confirmation of my experience of the dream world 
which occurs within the dream as confirmation of the reality of the dream 
world. 

\item Accept intersubjective confirmation of the past of the dream world which 
occurs in the dream itself as confirmation of the reality of the dreamed past. 

\item Recognize that there is no infallible way to tell whether other people are 
lying about their dreamed experience or their waking experience. 

\item Develop sophisticated interrogation techniques as a limited test of 
whether people are telling the truth about their dreams. 

\item Accept that a certain category of anomalies occurs in dreams only when 
several people have reported experiences in that category. 
\end{enumerate}

The principal characteristic of the approach which these conventions 
represent is that each dream is treated as a separate world. There is no 
attempt to arrive at an account, for a given "objective" time period, which is 
consistent with more than one dream or with both dreamed and waking 
periods. Thus, many parallel worlds could be confirmed as real. As our 
discussion proceeds, we will move away from this approach, probably out of 
a sense that it is pointless to maintain a strong notion of reality and yet to 
forego the notion of the consistency of all portions of reality. 

\textbf{C.} Something that I have learned from a study of my dream records is 
that while dreams are not chaotic, while they can be compared and 
classified, it is not possibie to apply the method of natural science to them in 
the sense of discerning a consistent, impersonal natural order in the dream 
world. It is not that the natural order is different in dreams from what it is in 
the waking world; it is that the dream worlds are incommensurate with the 
discernment of a natural order in the scientific sense. Here are some specific 
observations which relate to this whole question. 

\begin{enumerate}
	\item Some dreams are not noticeably anomalous. The laws of science are not 
violated in them. This observation is important in giving us a normal base for 
our investigation. Dreams are not all crazy and chaotic. 

\item In some dreams, it is impossible to abstract an impersonal natural order 
from personal experiences and anecdotes. There are no impersonal events. 
There is no nature whose order can be defined impersonally. The dreams are 
full of personal magic which cannot be generalized to a characteristic of an 
impersonal natural order. 

\item As a special case of (2), in some dreams, we jump back in time and move 
discontinuously in time and space. Chronological personal magic. 

\item In dreams, the distinction between myself and other people is blurred in 
many different ways. Also, I sometimes become a disembodied 
consciousness. 

\item As a generalization of (4), sometimes it becomes impossible to distinguish 
objects from our sensing and perceiving function. The mediating sensory 
function becomes obtrusively anomalous. Stable object gestalts cannot be 
identified. 

\item Sometimes we experience the logically impossible in dreams. My father 
was both dead and buried, and alive and walking around, in one dream. 

\item The possibility of identifying causal relationships is sometimes lacking in 
dreams. It is not just that actions have unexpected effects. It is that events 
are strung together like beads on a string. There is no sense of willful acting 
on the world or manipulation of the world which can be objectified as a 
causal relation between impersonal events. 
\end{enumerate}

The possibility arises of using dreams as philosophical experiments in 
worlds in which one or more of the preconditions for application of the 
scientific method is absent. (But in the one case in which Alten and I tried 
this, we reached opposite conclusions. Alten said that dreams in which one 
can jump around in time proved that the irreversibility of time is the basis 
for distinguishing between time and space; I said that the dreams proved that 
time and space can be distinguished even when the irreversibility of time is 
lacking.) 

Observation (2) above can lead us to an insight about the waking world. 
Perhaps science insists on the elimination of personal anecdotes from the 
natural order which it recognizes because the scientist wants results which 
can be transferred from one life to another and which will give one person 
power over another. At any rate, science excludes anecdotal anomalies which 
cannot be made somehow into "objective" events. As an example, I may be 
walking down the street and suddenly find myself on the other side of the 
street with no awareness of any act of crossing the street. 

What dreams provide us with is worlds in which anecdotal anomalies 
cannot be relegated to limbo as they are in waking science. They are so 
prominent in dreams that we can become accustomed to identifying them 
there. We may then learn to recognize analogous anomalies in the waking 
world, where we had overlooked them before because of our scientific 
indoctrination. 

Of course, we run the risk that superstitious people will misuse our 
theory to justify their folly. But the difference between our theory and 
superstition is clear. When the superstitious person says that he 
communicates with spirits, he either lies outright; or alse he misinterprets his 
experiences---embedding them in an extraneous pre-scientific belief system, 
or treating them as controversions of scientific propositions. We, on the 
other hand, maintain more literally than science does that the only raw 
material of cognition is experience. We differ from science in draping 
experiences on a different organizational framework. The "reality" we arrive 
at is incommensurate with science; it does not falsify any scientific 
proposition. As for science and superstition, we headed this essay with the 
quotation by d'Abro to emphasize that the scientist himself is superstitious: 
he is determined to believe in the common objective world, even though it is 
a fiction, because it is necessary to science. The superstitious person wants 
you to believe that his communication with spirits is intersubjectively 
consequential. Thus our theory, which tends toward the attitude that 
nothing is intersubjectively consequential, offers him even less comfort than 
science does. 

\textbf{D.} We next turn to semi-conscious quasi-dreams. Referring to my 
experience on the morning of 1/12/1974, I describe the experience by saying 
that I was on the Courant Institute plaza. But I cannot conclude that I was 
on the Courant Institute plaza. The reason is that important internal reality 
cues are missing in the experience. For one thing, the peripheral environment 
is missing; in its place is a void. Referring to my experience on 1/1-/1974, 
still other cues are missing. I am awake, and the scene is unstable and 
momentary. The slightest attention shift will cause the scene to vanish. 

When we recognize that we have disallowed falling asleep, awaking, and 
anomalous phenomena in dreams as evidence of unreality, a careful analysis 
yields only two types of reality cues. 

\begin{enumerate}
\item Presence of the peripheral environment. 

\item "Single consciousness." This cue is missing when we see a 
three-dimensional scene and move about in it, and yet have a background 
awareness that we are awake in bed; and lose the scene through a mere shift 
of attention. Its absence is even more marked if the scene is a momentary 
one between two waking periods. 
\end{enumerate}

Let us recall our earlier discussion of the empiricism of science. Science 
does not content itself with ungraded experience. it drapes experience on an 
intellectual framework in such a way as to simulate Aristotelian realism. It 
feeds experience into a maze of verification procedures in order to confirm a 
model which is not explicit in ungraded experience. It short, science grades 
experience as to its reality on the basis of standards which are 
"intellectually" supplied. Internal reality cues are thus characteristics of 
experience which are given special weight by the grading procedure. The 
immediate problem for us is that ordinary descriptive language implicitly 
recognizes these reality cues; one would never say without qualification that 
one was on the Courant Institute plaza if the peripheral environment was 
missing and if one was also aware of being awake in bed at the time. (In 
contrast, it is fair to use ordinary descriptive language with respect to 
dreamed episodes when our consciousness is singulary, that is, when 
everything seems real and unqualified.)

For purposes of further comparison I may mention an experience I 
have had on rare occasions while lying on my back in bed fully awake. It is 
as if colored spheres whose centers are located a few feet or yards in front of 
my chest expand until they press against me, one after the other. I use the 
phrase "as if" because reality cues are missing in this experience, and thus I 
cannot use the language of stable object gestalts without qualification in 
describing it. The colors are not vivid as real colors are. They are like 
visualized colors. The spheres pass through each other, and through me---with 
only a moderate sensation of pressure. I can turn the experience off by 
getting out of bed. The point, again, is that it is inherent in ordinary 
language not to use unqualified object descriptions in these circumstances. 
Yet the only language I have for such sensory configurations is the language 
of stable object gestalts-this is particularly obvious in the example of the 
Courant Institute plaza. (Is "ringing in the ears' in the same class of 
phenomena?)

An insight that is crucial in elucidating this problem is that when I 
describe episodes, the descriptions implicitly convey not only sensations but 
beliefs, as when I speak of a typewriter in a dream on the assumption that it 
persisted while I was not looking at it. The peculiar quality of a quasi-dream 
comes about not only because it is an anomaly in my sensations but because 
it is an anomaly in the scientific-pragmatic cognitive model which underlies 
ordinary language. If I discard this cognitive model and then report the 
event, it will not be the same event: the beliefs implicit in ordinary language 
helped give the event its quality. As a further example, now that I have 
recognized experiences such as that of 1/12/1974, I am willing to entertain 
the possibility that they are the basis for claims by superstitious persons to 
have projected astrally. But to use the phrase "astral projection" is to embed 
the experiences in a pre-scientific belief system extraneous to the 
experiences themselves. If we learn to report such experiences by using 
idioms like "ringing in the ears" and blocking any comparison with notions 
of objective reality or intersubjective import, we will have flattened out 
experience and will have moved in the direction of ungraded experience and 
nihilistic empiricism. 

\textbf{E.} We next take up connections between adjacent dreamed and waking 
periods. As a preliminary, we reject conventional notions that dreams are 
fabricated from memories of waking reality; or that dreams are precognitions 
of waking reality; or that dreams are mental phenomena which symbolize 
waking reality. We reject these notions because they conflict with the placing 
of the dream world on the same level as the waking world. 

Connections between dream and waking periods are important in this 
study because we may wish to create such connections deliberately, and even 
to attribute causal significance to them. Initially, we define the concept of 
dream control: it is to conduct one's waking life so that it is supportive of 
one's dreamed life in some sense. We also define controlled dreaming: it is to 
manipulate a person "from outside" before sleep (or during sleep) so as to 
influence the content of that person's dreams. (An example would be to give 
somebody a psychoactive sleeping pill.) 

A careful analysis of connections between dream and waking periods 
yields the following classification of such connections. 

\begin{enumerate}
	\item I walk around the kitchen in a dream, then awaken and walk around the 
kitchen. Voluntary continued action. 

\item Given a project with causally separate components, voluntarily 
assembled, I can carry out the project entirely while awake, entirely in 
dreams, or partly while awake and partly in dreams. 

\item I walk around the kitchen while awake, then sleep. I may then walk 
around the kitchen in a dream. Also, I draw a glass of water while awake. I 
may have the glass of water to use in the dream. We could postulate that 
such connections are not mere coincidences, if they occur. However, we 
certainly cannot produce such connections at will. We call these connections 
echoes of waking actions in dreams. Note the case in which I taped my 
mouth shut before sleeping, and could not whistle in the subsequent dream. 

\item We next have connections from dreamed to waking periods which can be 
postulated to have causal significance. First, misfortune or danger in dreams 
is regularly followed by immediate awaking. Secondly, I have had 
experiences in which a headlong dive or an attempt to whistle continued 
from dream to waking, right through waking up. These experiences are 
causally continuous actions. However, I cannot bring them about at will. 

\item We can manipulate a person "from outside" before sleep (or during sleep) 
so as to influence the content of that person's dreams. The dream is not an 
echo of the waking action; the causal relationship is manipulative. Examples 
are to give someone a psychoactive sleeping drug or to create a special 
environment for sleep. The case in which I taped my mouth shut before 
sleeping was a remarkable borderline case between an echo and a 
manipulation. 
\end{enumerate}

in conclusion, dream control is any of the connections described in 
(1)--(4). Controlled dreaming is (5). We have analyzed these concepts 
meticulously because we want to exclude all attempts at magic, all 
superstition from the project of placing dreamed and waking life on the same 
level. There must be no rain dancing, no false causality, in this project. 

\textbf{F.} Until now, we have analyzed our experience episode by episode. We 
could make this approach into a principle by assuming that each episode is a 
separate and complete world, which has its reality confirmed internally. In 
particular, the notion of objective location in space and time would be 
maintained if it appeared in a dream and was intersubjectively confirmed in 
the dream, but the notion would be purely internal to each episode. The 
objection to these assumptions, as we mentioned at the end of (B), is that 
they propose to maintain the notion of objective location, and yet they 
forego the notion of the consistency of all portions of reality. if we adopt 
these assumptions and then compare all the reports of our dreamed and 
waking periods, we may find that we have experienced different events 
attributed to the same location---and indeed, that is exactly what we do 
experience. 

One of the main discoveries of this essay has been that dreamed and 
waking periods are more symmetrical than our scientific-pragmatic 
indoctrination would have us suppose. The reality of the dream world is 
intersubjectively confirmed---within the dream. Anecdotal anomalies can be 
found in waking periods as well as in dreams. Entities which resemble 
common object gestalts but which lack some of the reality cues of object 
gestalts can be encountered whicle we are fully awake. Now we can 
recognize a further symmetry between dreamed and waking life. A dreamed 
misfortune is usually "lost" when we awaken, and its disappearance is taken 
as evidence of the unreality of the dream (the nightmare). But we can also 
"lose" a waking misfortune by going to sleep and dreaming. Further, just as 
a waking misfortune can persist from one waking period to another, a 
dreamed misfortune can persist from one dream to another (recurrent 
nightmares). Thus, we conclude that in regard to the consistency of episodes 
with each other, there is no basis for preferring any one episode, dreamed or 
waking, as the standard by which the reality of other episodes will be judged. 
Of course, rather than maintaining the reality of each episode as a separate 
world, we can block all attributions of events to objective locations. This 
approach would alter the quality of the events and bring us closer to 
nihilistic empiricism. 

A further problem arises if we take the dream reports of other people as 
reports of reality. Suppose I am awake in my apartment at 3 AM on 
2/6/1974, but that someone dreams at that time that I am out of my 
apartment. Multiple existences which I do not even experience are now being 
attributed to me. (My own episodes also pose a problem of whether 
"multiple existences" are being attributed to me, but that problem concerns 
events I experience myself.) What we should recognize is that the problem of 
"multiple existences" is not as unique to our investigation as may at first 
appear. Natural science has an analogous problem in disposing of the notion 
of other minds. The notion of the existence of many minds, none of which 
can experience any other, is difficult to assimilate to the cognitive model of 
science. On the other hand, to deny the existence of any mind, as 
behaviorists do, is to repudiate the scientist's observations of his own mental 
life. And if the scientist's observations of his own mental life are repudiated, 
then there is no good reason not to repudiate the scientist's observations of 
his budily sensations and of external phenomena also; that is, to repudiate 
the very possibility of scientific observation. Further, when behaviorists try 
to convince people that they have no awareness, whom (or what) are they 
trying to convince? And what is the behaviorist explanation of the origin of 
the fiction of consciousness? Who benefits from perpetuating this fiction, 
and how does he benefit? 

We must emphasize that the above critique is not applicable to every 
philosophical outlook. It applies specifically to science---because the scientist 
wants to have the benefits of two incompatible conceptual frameworks. 
Some of the common sense about other minds is necessary in the operational 
preliminaries to formal science; and the scientist's role as observer is 
indispensable to formal science. Yet the conceptual framework of science is 
essentially physicalistic, and can allow only for external objects. What this 
difficulty reveals is that the cognitive model of science has stabilized and 
prevailed even though it has blatent discrepancies in its foundations. The 
foremost discrepancy, of course, is that the scientist is willing to have his 
enterprise rest on a fiction, that of the common objective world. Thus, the 
example of science suggests an additional way of dealing with the problems 
which arise for our theory: we can allow discrepancies to persist unresolved. 

There is an interesting observation to be made about one's own dreams 
in connection with multiple existences. I have found that the person I am in 
my dreams is significantly different from the waking identity I take for 
granted, as in my dream of 2/1/1974. As for the problem of other people's 
dreams, one way of handling them would be simply to reject the existence of 
other people's dream worlds and of their consciousnesses, and to limit one's 
consideration to one's own dreams. But perhaps the most productive way to 
handle the problem would be to construe it as one involving language in the 
way that the problems concerning quasi-dreams did. Our descriptive language 
is a language of stable object gestalts, of scientific-pragmatic reality. If we 
accept reports of other people's dreams in language which blocks any 
implications concerning objective reality, then our perceptual interpretations 
will be different and the quality of the events will be fundamentally 
different. The experience-world will be flatter. But maybe this is a 
revolutionary advance. Maybe reports of our appearances in other people's 
dreams, in language which blocks any implications about reality, are what we 
should strive for. And if ve cease to be stable object gestalts for others, 
maybe our stable object gestalts will not even appear in their dreams. 


\section*{Note on how to remember dreams}

The trick in remembering a dream is to fix in your mind one incident or 
theme in the dream immediately upon awaking from it. You will then be 
able to remember the whole dream well enough to write a description of it 
the next day, and you will probably find that for weeks afterwards you can 
add to the description and correct it. 


