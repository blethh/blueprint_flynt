\chapter{Representation of the Memory of an Energy Cube Organism (Original 1961 Version)}

\section*{Foreward}

I have refrained from editing the Original Version except where 
absolutely necessary. It is full of inconsistencies and inadequate 
explanations, but I have flagged only two major ones, by placing them 
between the signs $\ltimes$ and $\rtimes$. Part of the fourth paragraph is flagged because a 
sequence of units is not analogous to a sequence of inflected words; it is 
rather more like permutations of letters which form words ('rat', 'tar', 'art'). 
Most of the seventh paragraph is flagged because I promise to define intervals 
by their lengths and ends, but instead give their beginnings and ends. 

In the fourth paragraph, there are two different versions of the 
correspondence between possible methods and sequences of units, and of 
why any sequence is acceptable. Passages belonging exclusively to the 
"multiplex" version are set off by the sign \#. Passages which belong 
exclusively to the "style" version and which should be deleted if the 
"multiplex" version is used are placed between slashes (\slash). The "style" version is 
the main version. In the fifth paragraph, a notion appears which is 
interesting, but unconvincingly explained. It is not clear whether this notion 
relates only to the "multiplex" version, or whether it would relate to the 
"style" version if the word 'multiplex' were omitted. The passages suggesting 
this notion are placed in brackets. 

\begin{enumerate}
\item Energy cube organisms are conscious organisms which are cubical 
spaces containing only energy. The particular energy cube organism of 
concern here has, for an indefinitely long time, been in a body of liquid, 
"resting on' a rectangular energy slab also in the body of liquid; the 
organism's "bottom" face is separated from the slab by only a very thin film 
of the liquid. The "universe" the organism and slab are in is made up of four 
infinite triangular right prisms, prismatic spaces, as defined geometrically by 
two intersecting planes almost perpendicular to each other. The prismatic 
spaces defined by the vertical obtuse dihedral angles are empty. The other 
spaces, defined by the vertical acute dihedral angles, are infinite bodies of a 
stationary, colorless liquid--the "upper" body of liquid being what the 
organism and slab are in. The two opposite shorter edges of the slab are at 
the faces of the body of liquid, the planes, near their intersection; the slab is 
"slanted," so that the edges are at slightly different distances from the line 
of intersection. The organism and slab are the only "objects" in the bodies 
of liquid. (See the illustration.) The organism can move (the energy cube can 
continuously change position) without creating currents in the liquid. For 
almost as long as it has been in the liquid, the organism has devoted all its 
"intelligence," all its "energies," to moving across the slab, from one of the 
shorter edges to (any point on) the other. 

\item The organism's conscious, distinct memory is entirely concerned 
with, is entirely of, its efforts to cross the slab. (I am using 'memory' 
narrowly to refer to an organism's memory of its past. I am counting its 
"general information," for example knowing a language, not as part of its 
memory but as imagings not memories. Thinking the sequence 1, 2, 1, 2 is 
not in itself remembering.) The total memory consists of a large number of 
units (tens of thousands), of which the organism can be attentive to precisely 
one at a time. "Total recall," the total memory, involves considering, having, 
all units in any succession, which the organism can do very rapidly. Now 
from one point of view, the memory consists of its content; from another, it 
consists of symbols, just as human memories often consist of language. In 
describing the memory, I will go from considering primarily the content, 
what the memory is of; to considering the specific character of the units, 
specific symbolism used in the memory, and specific content. Each unit is 
first a memory of the amount of progress made toward the destination edge 
in a particular interval of time. The amount of progress is the difference 
between the minimum distance of the organism from the destination edge at 
the beginning of the interval, and the minimum distance at the end of the 
interval. The total of intervals, in the total of units, cover the "absolute" 
interval of time from the earliest to the most recent remembered event; as 
time passes, more units are added to the memory. 

\item Now the memory is temporally dual: the interval of time for each 
unit is first, an interval of 'absolute' time; defined by its duration, and the 
"absolute" time of its end (stated with respect to an "absolute event" such 
as the appearance of the organism on the slab); and secondly, an interval 
defined by its duration, and how far from the present instant its end is. It is 
like remembering that so much progress was made during one year which 
ended at January 1, 1000 A.D.; as well as remembering that it was made 
during one year which ended 1,000 years ago. In the second temporal 
memory, the absolute time of the end of the interval to which the progress is 
assigned changes according as the absolute time of the present instant 
changes. For example, it is like remembering \said{that so much progress was 
made during one year ending 1,000 years ago,} and, 100 years later, 
remembering---\said{that so much progress was made during one year ending 
1,000 years ago}; and in general, always remembering \said{that so much 
progress was made during one year ending 1,000 years ago.} Both temporal 
memories are in their own ways "natural," the first being anchored at an 
"absolute beginning," the second at the present instant. When a unit is added 
to the memory, the interval of time of the first temporal memory is added at 
the end, exactly covers the time not already covered, up to the absolute time 
when the unit is added; so that the total of intervals of the first temporal 
memory exactly cover, without overlap, the absolute total time. In contrast, 
although the intervals of the second temporal memory do not overlap at any 
time, there can be gaps between them; so that when a unit is added to the 
memory, the interval for the second temporal memory may be placed 
between existing intervals and not have to cover an absolute time which they 
have left behind, that is, not have to be placed farther back than all of them. 
Intervals of both temporal memories are of different sizes, a "natural 
complexity." (See the graph.) Incidentally, the condition for coincidence of 
the two temporal intervals of a unit is: if the two intervals are of the same 
duration, they will coincide at the absolute time which is the sum of the 
absolute time of the end of the first interval, and the distance from the 
present instant of the end of the second interval. The two temporal 
memories complement each other; aside from this comment I will not be 
concerned to "explain" the duality with respect to when the amounts of 
progress were made, whether when they were "really" made stayed the same 
and changed, or whether the memory is inconsistent about it, or what. 

\item I will now turn to the aspect of the memory concerned with the 
method the organism has used to move itself. \# Methodologically, the 
memory is a multiplex symbol. \# A "single method" is everything to be done 
by the organism, to move itself, throughout the total time it takes to reach 
the destination edge; so that the organism could not use two different 
"single methods," must, after it chooses its method, continue with it alone 
throughout. The organism has available different (single) methods, has 
different methods it could try. The different sequences, of all units, are 
assigned to the different (single) methods available to the organism to signify 
them; are symbols for them. (Thus, the number of available methods 
increases as units are added to the memory.) \slash Now all this only approximates 
what is the case, because contrary to what I may have implied, which 
method is used is not a matter of "fact" as are the temporal intervals and 
amounts of progress. As I have said, having all units in any succession 
constitutes the total memory, total recall ("factually")--different sequences 
of all units are each the total memory, total recall, $\ltimes$ but, as language, the 
total memory in different styles (like words in different orders in a highly 
inflected language); and the matter of method (which might better be said to 
be "manner") corresponds to the matter of style, rather than factual 
content, of language. Different styles exclude each other, but not what is 
said in each other's being true.$\rtimes$ Thus it is that the number of available 
methods can increase; and that any sequence of all units can constitute the 
total memory, total recall ("factually"), although different sequences signify 
different methods used. \slash \# As an indicator of the method used, the whole 
memory is a multiplex symbol. Names for each of the methods are combined 
in a single symbol, the totality of units. In remembering, the organism 
separates any single name by going through all the units in succession, and 
that name is the complete reading of the multiplex symbol, the complete 
information about the method used. I will not be concerned to "explain" 
the matter of the increasing number of available methods; or the matter of 
any sequence of all units' constituting the complete reading, the total 
memory, total recall, but different sequences' signifying different methods 
used. \#

\item I will give just an indication of what the available methods [and 
their relations through the multiplex memory] are like. Throughout this 
description, there has been the difficulty that English lacks a vocabulary 
appropriate for describing the "universe" I am concerned with, but the 
difficulty is particularly great here, in the case of the methods [and their 
relations through the multiplex memory]; so that I will just have to 
approximate a vocabulary with present English as best as I can. The 
methods, instruments of autokinesis, are all mental, teleportation, result in 
teleportation. The "consciousnesses" available to the organism to be 
combined into methods are infinitely many. It has available many states of 
mind (as humans have non-consciousness, autohypnotic trance, dizziness, 
dreaming, clear-headed calculation, and so forth), corresponding to different 
forms its energy can assume. To give this description more content I will 
differentiate its states of mind by referring to them with the names of the 
human states of mind (rather than just with letters). It has available an 
indefinite variety of contents, as humans have particular imagings, in its 
conscious states of mind. I will outline the principal contents. There are 
"visualized" fluid regions of color (like colored liquids), first-order contents. 
There are 'visualized' radient surfaces, and non-radient surfaces or regions 
("holes"), the intermediate contents. The second-order contents are 
"projective" constructs of imaged geometric surfaces, "covers," "lattices," 
and "shells." Fluid colors can be stationary or flowing. They can occur in 
certain series, "channels"; and in certain arrays, "reservoirs." A channel can 
be "closed" or "open"; two channels can be "crossed," or 
"screw-connected" (earlier members of each channel flowing into later 
members of the other). First-order contents (fluid colors) often occur on or 
within second-order ones (projective surfaces). Second-order contents can be 
"held" or "growing." States of mind have depth, 'deeper' being 'farther from 
the forefront of attention'; and contents can be at different depths. A state 
of mind as a unity can be "frozen," which is more than just unchanging (in 
particular having its contents stationary or held). It can be projected into 
"superstate," remaining a state of mind but being superenergized. [Most 
interesting, states of mind, in different methods signified by different 
symbols combined in the multiplex methodological memory, can have 
contact with each other, for example be "interfrozen."] A partial description 
of a method will give an idea of the complexity of the methods. Channels are 
generated by a frozen non-conscious state, and become fixed in the surface 
layer of an [inter] melted trance. The screw-crossed channels erode crevices 
in a held shell, which breaks into growing sheets (certain covers). The sheets 
are stacked, and held in a frozen dream thawed at intervals for reshuffling. 
The dream becomes melted, and proceeds in a trajectory which shears, and 
closes, open channels. If no violation of the channels cross-mars the melt, the 
stack meshes with the sharp-open channels. The dream becomes [inter] 
frozen, and mixed calculation states compress the closed channels which 
were not surface-fixed in it. A fused exterior double-flash (a certain 
maximally radient surface) is expand-enveloped by a trance, which becomes 
dizziness; and oblique lattices are projected from the paralinear deviation of 
guided open channels in it. Growing shells are dreamed into violet 
sound-slices (certain fluid colors) by the needed jumped drag (a certain 
consciousness), a [cross] frozen dream. Channels in a growing trance enspiral 
concentric shells having intermixed reservoirs between them, during cyclic 
intersection of the trance in superstate. I will not say more about the 
available methods, because in a sense the memory does not: a sequence of 
units is a marker arbitrarily assigned to a method to signify it, like an 
arbitrary letter, say 'q', assigned to a certain table to signify it; it no more 
gives characteristics of the method than 'q' does of the table. In fact, the 
available methods and sequences do not have any particular order; one 
cannot speak of the "first" method, the "second," or the like. 

\item I will now concentrate on the character of the memory as a mental 
entity, and the rest of the symbolism used in it and specific content. A unit 
is a rectangular plane ("visualized") radient surface (! ---the terminology is 
that introduced in the last paragraph), which has two stationary plane 
reservoirs (!) on it, and has a triangular hole (!) in it. The triangular hole is 
a simple symboi not yet explained: its perimeter equals the amount of the 
organism's progress, the difference in its minimum distances from the 
destination edge, in the interval the unit is concerned with. Absence of the 
hole indicates zero perimeter and no progress. 

\item As for the symbols for the temporal interval. The colors in each of 
the two reservoirs on each unit are primary, and are mixed together. 
Speaking as accurately as possible in English, in each reservoir there is 
precisely one point of "maximum mixture' of the primary colors. (The rest 
of the reservoirs are not significant: the primary colors are mentally mixed in 
any way to get the right amount of mixture, as pigments are mixed on a 
palette.) $\ltimes$ For the first temporal memory, these points are two points on a 
scale of amounts of color mixture. For the second memory, the points are 
two points on a scale of vertical distances from the imaginary horizontal line 
which bisects the rectangular surface, divides it into lower and upper halves. 
The units are marked in their lower halves only; because for the second 
memory the imaginary dividing line represents the present instant, distances 
below it represent distances into the past, and distances above it distances 
into the future (lower and upper edges representing equal distances from the 
present). Now a scale is required so that it can be told what temporal 
intervals the interval on the amount of mixture scale and the interval on the 
distance scale represent. The parts of the scale which may vary from unit to 
unit and have to be specified in each unit are the "absolute" time 
corresponding to the maximum possible color mixture, the number of units 
of absolute duration per unit difference in amounts of mixture, and the 
number of units of absolute duration per unit difference in distances from 
the imaginary dividing line. The markers arbitrarily assigned to the triples of 
information giving these parts of the scale are average radiences per unit 
areas of the units (excepting the holes). $\rtimes$

\item A final aspect of interest. Not too surprisingly, the transformation 
which is inverting all units gives, if one considers not the first temporal 
memory but its reflection in the present instant, the organism's precognized 
course of action in the future, specifically, what progress will be made when. 
\end{enumerate}


\section*{The Representation}

With this background, it is not surprising that the method of 
representation I have chosen is visual representation of the units, the 
"visualizations." Units are represented by rectangular sheets of paper of 
different translucencies with mixtures of inks of primary colors on them and 
holes cut in them, together in an envelope. Only one sheet should be out of 
the envelope at a time. A sheet should be viewed while placed before a white 
light in front of a black background, so that the light illuminates the whole 
sheet as evenly as possible without being seen through the hole, only the 
black being seen at the hole. The ultimate in fidelity would be to learn to 
visualize these sheets as they look when viewed properly; then one could 
have the memory as nearly as possible as the organism does. I have 
represented eleven of the tens of thousands of units in the total memory. 

