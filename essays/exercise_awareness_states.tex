\chapter{Exercise Awareness-States (July 1961)}

{
\itshape
The July 1967 issue of IKON contained Henry Flynt's \essaytitle{Mock Risk Games}.
This work was a reconstruction, from memory, of Flynt's 1961 work, 
\essaytitle{Exercise Awareness-States}, which Flynt had disavowed and discarded in 1962. 
In 1981 Flynt obtained a copy of the original 1961 piece in the possession 
of Tom Constanten. In this issue IKON publishes, for the first time, the 
original \essaytitle{Exercise Awareness-States.} Preceding the text, we reprint the 
introduction from \essaytitle{Mock Risk Games} because of its clarity in explaining the 
work. Flynt read \essaytitle{Exercise Awareness-States} during his July 15, 1961 
appearance in the legendary series at George Maciunas' A/G Gallery, NYC. 
This was the only documented public presentation of the work in that period. 
The reconstruction \essaytitle{Mock Risk Games} has been printed a number of 
times --- it was included in Flynt's book, \booktitle{Blueprint For a Higher Civilization}\footnote{That's this!}
(Milan, 1975). 
}


\section*{INTRODUCTION (from "Mock Risk Games"---1967 Version)}


Suppose you stand in front of a swinging door with a nail sticking out of it 
pointing at your face; and suppose you are prepared to jump back if the door 
suddenly opens in your face. You are deliberately taking a risk on the 
assumption that you can protect yourself. Let us call such a situation a "risk 
game." Then a "mock risk game" is a risk game such that the misfortune 
which you risk is contrary to the course of nature, a freak misfortune; and 
thus your preparation to evade it is correspondingly superficial. 

lf the direction of gravity reverses and you fall on the ceiling, that is a freak 
misfortune. If you don't want to risk this misfortune, then you will anchor 
yourself to the floor in some way. But if you stand free so that you can fall, 
and yet try to prepare so that if you do fall, you will fall in such a way that 
you won't be hurt, then that is a mock risk game. If technicians could actually 
effect or simulate gravity reversal in the room, then the risk game would be 
a real one. But I am not concerned with real risk games. I am interested in 
dealing with gravity reversal in an everyday environment, where everything 
tells you it can't possibly happen. Your "preparation" for the fall is thus 
superficial, because you still have the involuntary conviction that it can't 
possibly happen. 

Mock risk games constitute a new area of human behavior, because they 
aren't something people have done before you don't know what they will be 
like until you try them, and it took a very special effort to devise them. They 
have a tremendous advantage over other activities of comparable 
significance, because they can be produced in the privacy of your own room 
without special equipment. Let us explore this new psychological effect; and 
let us not ask what use it has until we are more familiar with it. 

Instructions for a variety of mock risk games follow. (I have played each 
game many times in developing it, to ensure that the experience of playing 
it will be compelling.) For each game, there is a physical action to be 
performed in a physical setting. Then there is a list of freak misfortunes 
which you risk by performing the action, and which you must be prepared 
to evade. The point is not to hallucinate the misfortunes, or even to fear 
them, but rather to be prepared to evade them. First you work with each 
misfortune separately. For example, you walk across a room, prepared to 
react self-protectingly if you are suddenly upside down, resting on the top 
of your head on the floor. In preparing for this risk, you should clear the path 
of objects that might hurt you if you fell on them, you should wear clothes 
suitable for falling, and you should try standing on your head, taking your 
hands off the floor and falling, to get a feeling for how to fall without getting 
hurt. After you have mastered the preparation for each misfortune 
separately, you perform the action prepared to evade the first misfortune and the 
second (but not both at once). You must prepare to determine instantly 
which of the two misfortunes befalls you, and to react appropriately. After 
you have mastered pairs of misfortunes, you go on to triples of misfortunes, 
and so forth. 

\section*{Exercise Awareness-States (July, 1961)}

I am concerned here to introduce an activity which I will call, for want of 
a better term, "exercise," and the states of awareness one has in exercise, 
"exercise awareness-states." Incidentally, this activity is based on wrong, 
although common, philosophical assumptions, but I hope the reader will play 
along with them for the sake of the activity; philosophical rightness is not 
the main concern here. Exercise should be thought of first as training to help 
prepare one for dangerous situations of a very special kind (which the 
reader is admittedly not likely to encounter). (Incidentally, 'danger' here 
should not be an emotive word; my concern is with the theory of defense, 
not with giving the reader vicarious experience.) Suppose that the adults in 
a society occasionally have to be in situations, such as walking across a 
bare metal floor in a certain "building," during which \emph{dangers}, very unusual 
and \emph{unpredictable}, may arise. Suppose that they know nothing of the 
provenance of the dangers, just that they may be there, so that they can't 
prevent them (or predict what they will be); the persons are somewhat like 
animals trying to defend themselves against a variety of modern (human) 
weapons. They cannot adequately prepare for the dangers by practicing 
responses to specific dangers so that they become habitual, because of the 
extreme unpredictability of the actual dangers. However, the dangers are 
such that when one arises a person \emph{can} figure out what he needs to do to 
defend himself \emph{fast enough} and carry it out. 

Finally, suppose that although it is desired to train persons [to be prepared] 
to defend themselves in the situations, there isn't the technology to simulate 
dangers, so that they can't be given a chance to actually figure out and carry 
out defenses against simulated dangers. Then it would seem that the best 
preparation in the situations (until a danger appeared) would be the \emph{state 
of mind}---"unpredictably-dangerous-situation awareness state"---of lack of 
preconceptions as to what one might encounter, emotionlessness (except 
for the small amount of fear and confidence needed to make one maximally 
alert), very very heightened awareness of all sensory data, and readiness 
to figure out (quickly) whether they indicated a danger and [to figure out] a 
defense against it. After all, it might be best to stay away, or at least get 
away, from the preparation resulting from practice with simulated dangers, 
just because the actual ones are so unpredictable. Training for the situations 
would then be to help persons achieve this best dangerous-situation 
awareness-state when in the situations. Then (one should first think) the purpose 
of "exercise," or the "exercises," is to help persons to achieve the best 
dangerous-situation awareness-state in the situations by teaching them to 
achieve "\emph{ultimate} exercise awareness-states," which are as similar as 
possible to the best dangerous-situation states within the limitations I have 
given. 

Exercise may secondly be thought of as something to be done for its own 
sake, so that ultimate exercise awareness-states are achieved for their own 
sake, in particular, as an unusual way of "appreciating" the sensory date 
while in them. This is the way I suppose the reader will regard exercise. Thus 
exercise, rather than unpredictably dangerous situations, is the principal 
subject of this paper. However, it should not be lost sight of that exercise 
could be useful in the first way; and the development of exercises should be 
controlled by concern with whether they are useful in the first way. 

I will now give some explanations and general instructions for exercise. 
An "exercise" is what the general instructions, and a specification of a(n 
exercise) "\emph{situation}" one is to place oneself in and of several 
"\emph{given dangers}" to anticipate in the situation, refers to;
an "\emph{exercise awareness-state}" 
is any state of mind throughout an exercise. In first doing an exercise, one 
anticipates given dangers; the point of having specific dangers to anticipate 
at first is to keep one from anticipating nothing, being indifferent in the 
situation and thus not achieving an interesting awareness-state. In a good 
exercise, the dangers should be interesting to anticipate, one should find it 
easy to anticipate them strongly, and it should be clear what is dangerous 
in them and how they can defended against. It is only when one can 
anticipate the given dangers strongly that one does the exercise, places 
oneself in the situation, without thinking of specific dangers, trying to
strongly anticipate unpredictable danger; when one can do this one will be 
achieving "ultimate exercise awareness-states." 

The general instructions for the exercises follow. First place oneself in the 
situation, anticipate one of the given dangers as strongly as possible (short 
of getting oneself in a state of fright), be very very aware of all sensory data, 
and be ready to figure out (quickly) whether they indicate the danger and to 
start defending against it. Try to achieve the greatest anticipation of and 
readiness for the one danger. The result is an "initial exercise 
awareness-state." Finally one can do the exercise forgetting the given dangers; place 
oneself in the situation, try to anticipate [unpredictable] danger strongly 
(short of getting oneself frightened), without preconceptions as to what form 
it will take, be very very aware of all sense data, and be ready to figure out 
(quickly) whether they indicate a danger, and a defense against it. This is 
an "\emph{ultimate} exercise awareness-state." A final point. So that one will not be 
distracted from the exercise, there must be a minimum of familiar events 
extraneous to it during it, such as the sight of a door opening, talking, 
cooking smells. For this reason, unless otherwise stated exercise should be 
taken in environments as inanimate, quiet, odorless, etc. as possible. One 
will fail to achieve interesting exercise awareness-states if one cannot play 
along and (for the sake of the exercise) strongly anticipate danger; [because 
one doesn't expect it,] but rather remains relaxed, indifferent, or worse is 
sleepy, physiologically depressed (indifferent, depressed exercise states). 
It should be clear that one has to really try the exercises, not just read about 
them, in order to appreciate them. 

\section*{Exercise 1}

The situation: You walk across the floor of a medium-sized brightly lighted 
square room, from the middle of one side to the middle of the other, in a 
straight line. There should be no other animal [fauna, animate creature] in 
the room and the path of walking should be clear [of obstructions]; ideally 
the room should be bare. 

The given dangers to anticipate: 
\begin{enumerate}
\item Heavy invisible objects falling around you, making a whirring noise as 
they do. 
\item Immovability of whichever foot presses most strongly on the floor, and 
a steel cylinder two feet in diameter with sharp edges' falling down, around 
you (hopefully). 
\item Instantaneous inversion of yourself so that you rest on whatever part of 
your surface was uppermost in walking, and doubling of the gravitational 
force on you. 
\item Sudden dizziness, change of equilibrium to that of one who has been 
turning around for a long time, and the floor's vanishing except for a narrow 
strip, where you have walked, shortening from the front. 
\item Change of field of vision to behind your head, instead of in front, 
something's coming to hit you from the side in an erratic path, and loud 
noises on the side of you opposite it. 
\item What you see's suddenly becoming two-dimensional instead of three so 
that you bump into it, while the room fills from behind with a mildly toxic gas; 
and going forward's requiring that you \emph{guess} the unpredictable action, 
symbolic of getting past the barrier, which will enable you to get forward. 
\end{enumerate}

\section*{Exercise 2}

You stand, in a dark room, facing a wall and pulling medium hard with both 
hands on a horizontal bar running along the wall and attached to it, for five 
minutes. Have an alarm clock to let you know when the time is up. There 
should be no other animal in the room; ideally the room should be bare. You 
must not let up on the pulling; the assumption is that if you do your eyes 
and ears will be assaulted with a blinding light and a deafening sound, 
except in the case of certain dangers. 

The dangers: 
\begin{enumerate}
\item Loss of your kinesthetic sense. (body-movement or muscle sense) 
\item Suspension of the "normal" "cause and effect" relationship between 
pulling and keeping the light and sound from appearing, so that you just 
have to guess what to do to keep them from appearing and it changes with 
time, with the restriction that it will be closely related to pulling on the bar, 
\eg\ letting go of the bar. 
\item Suspension of the "normal" "cause and effect" relationship between what 
you will and what your body does, so that you just have to guess what to 
will to keep your arms (and hands) pulling on the bar and it changes with 
time, with the restriction that it will be closely related to willing to pull on 
the bar, \eg willing to let go of the bar. 
\item Having the tactile, cutaneous sensation of being under water, so that 
you will "drown"---"cutaneously"---unless you cutaneously swim to the top; 
your sight and hearing being lost except for sensitivity to the light and sound 
if you stop pulling. 
\end{enumerate}

\section*{Exercise 3}

The situation: You lie on your back, barefoot, on a bunk, your arms more 
or less at your sides, with a pillow on your face so that you can breathe but 
not easily, for five minutes. Do not change your position; the assumption is 
that you can't except in the case of certain dangers. Have an alarm clock 
to let you know when the time is up. The room should be dark and there 
should be no other animal in it. Ideally you should be lying, in the middle 
and along the longitudinal axis of a not uncomfortably hard rectangular 
surface a yard above the floor and having an area almost that of the room, 
in a very long room, which should otherwise be bare. 

The dangers: 
\begin{enumerate}
\item The gravitational force's becoming zero and the room's getting 
unbearably hot towards the ceiling. 
\item Having to press the pillow against your face with your arms and hands, 
except for one angle of your face wherein you can roll your face from under 
the pillow, your head and neck becoming movable. 
\item The surface you are lying on's and the pillow's turning into a two-part 
living organism, of which the lower part is so delicate that unless you 
distribute your bodily pressure on it as evenly as possible, it will be injured 
and the upper part will pull you off of it by the skin of your face in 
self-protection, the organism being sufficiently telepathic that you can 
sense when it is hurting. 
\item Division of your body (and clothing) just below the ribs. The two halves 
separate by 114 feet and a metal wall one inch thick appears between them. 
Matter and so forth are transmitted between halves and they remain in the 
usual position relative to each other so that it is rather as if you simply grew 
in the middle by 112 feet. Your consciousness suddenly seems to be located 
in the pillow, where the pillow is, rather than in your head; nothing that 
happens to the pillow materially affects your consciousness. Two kinds of 
metal blocks come crashing against the wall from far in front of and behind 
it, starting slowly and speeding up as they get near the wall, and then draw 
back to where they came from. Blocks of the first kind come from the front 
(the side the upper part of your body and the pillow are on) only; they are 
"vertical," tall and narrow so that they can be avoided by moving from side 
to side. Blocks of the second kind come in pairs, one in front, one behind. 
They are "horizontal," two feet high (thick), and very wide (long). The ones 
in front hit low and the ones in back high, so they can be avoided by standing 
up (necessarily in a stooped position). Each time the pair hits higher and 
higher. There are long indentations in the back side of the wall in which one 
can get footholds to climb the wall. If one gets to the top of the wall, gets 
both halves of one's body above the wall, they will rejoin. 
\end{enumerate}
