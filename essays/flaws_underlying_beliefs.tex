\chapter{The Flaws Underlying Beliefs}

We begin with the question of whether there is a realm beyond my 
\enquote{immediate experience.} Does the \textsc{Empire State Building} continue to exist 
even when I am not looking at it? If either of these questions can be asked, 
then there must indeed be a realm beyond my experience. If I can ask 
whether there is a realm beyond my experience, then the answer must be 
yes. The reason is that there has to be a realm beyond my experience in 
order for the phrase \enquote{a realm beyond my experience} to have any meaning. 
Russell's theory of descriptions will not work here; it cannot jump the gap 
between my experience and the realm beyond my experience. The assertion 
\speech{There is a realm beyond my experience} is true if it is meaningful, and that 
is precisely what is wrong with it. There are rules implicit in the natural 
language as to what is semantically legitimate. Without a rule that a 
statement and its negation cannot simultaneously be true, for example, the 
natural language would be in such chaos that nothing could be done with it. 
Aristotle's \booktitle{Organon} was the first attempt to explicate this structure formally, 
and Supplement D of Carnap's \booktitle{Meaning and Necessity} shows that hypotheses 
about the implicit rules of a natural language are well-defined and testable. 
An example of implicit semantics is the aphorism that \enquote{saying a thing is so 
doesn't make it so.} This aphorism has been carried over into the semantics 
of the physical sciences: its import is that there is no such thing as a 
substantive assertion which is true merely because it is meaningful. If a 
statement is true merely because it is meaningful, then it is too true. It must 
be some kind of definitional trick which doesn't say anything. And this is 
our conclusion about the assertion that there is a realm beyond my 
experience. Since it would be true if it were meaningful, it cannot be a 
substantive assertion. 

The methodology of this paper requires special comment. Because we 
are considering ultimate questions, it is pointless to try to support our 
argument on some more basic, generally accepted account of logic, language, 
and cognition. After all, such accounts are being called into question here. 
The only possible approach for this paper is an internal critique of common 
sense and the natural language, one which judges them by reference to 
aspects of themselves. 

As an example of the application of our initial result to specific 
questions of belief, consider the question of whether the \textsc{Empire State 
Building} continues to exist when I am not looking at it. If this question is 
even meaningful, then there has to be a realm in which the nonexperienced 
\textsc{Empire State Building} does or does not exist. This realm is precisely the 
realm beyond my experience. The question of whether the \textsc{Empire State 
Building} continues to exist when I am not looking at it depends on the very 
assertion, about the existence of a realm beyond my experience, which we 
found to be nonsubstantive. Thus, the assertion that the \textsc{Empire State 
Building} continues to exist when I am not looking at it must also be 
considered as nonsubstantive or meaningless, as a special case of a 
definitional trick. 

We start by taking questions of belief seriously as substantive questions, 
which is the way they should be taken according to the semantics implicit in 
the natural language. The assertion that God exists, for example, has 
traditionally been taken as substantive; when American theists and Russian 
atheists disagree about its truth, they are not supposed to be disagreeing 
about nothing. We find, however, that by using the rules implicit in the 
natural language to criticize the natural language itself, we can show that 
belief-assertions are not substantive. 

Parallel to our analysis of belief-assertions or the realm beyond my 
experience, we can make an analysis of beliefs as mental acts.\footnote{We 
understand a belief to be an assertion referring to the realm beyond my 
experience, or to be the mental act of which the assertion is the verbal 
formulation.} Introspectively, what do I do when I believe that the \textsc{Empire 
State Building} exists even though I am not looking at it? I imagine the 
\textsc{Empire State Building}, and I have the attitude toward this mental picture 
that it is a perception rather than a mental picture. Let us bring out a 
distinction we are making here. Suppose I see a table. I have a so-called 
perception of a table, a visual table-experience. On the other hand, I may 
close my eyes and imagine a table. Independently of any consideration of 
\enquote{reality,} two different types of experiences can be distinguished, 
non-mental experiences and mental experiences. A belief as a mental act 
consists of having the attitude toward a mental experience that it is a 
non-mental experience. The \enquote{attitude} which is involved is not a 
proposition. There are no words to describe it in greater detail; only 
introspection can provide examples of it. The attitude is a self-deceiving 
psychological trick which corresponds to the definitional trick in the 
belief-assertion. 

The entire analysis up until now can be carried a step farther. So far as 
the formal characteristics of the problem are concerned, we find that 
although the problem originally seems to center on \enquote{nonexperience,} it 
turns out to center on \enquote{language.} Philosophical problems exist only if there 
is language in which to formulate them. The flaw which we have found in 
belief-assertions has the following structure. A statement asserts the 
existence of something of a trans-experiential nature, and it turns out that 
the statement must be true if it is merely meaningful. The language which 
refers to nonexperience can be meaningful only if there is a realm beyond 
experience. The entire area of beliefs reduces to one question: are linguistic 
expressions which refer to nonexperience meaningful? We remark 
parenthetically that practically all language is supposed to refer to 
nonexperiences. Even the prosaic word \enquote{table} is supposed to denote an 
object, a stable entity which continues to exist when I am not looking at it. 
Taking this into account, we can reformulate our fundamental question as 
follows. Is language meaningful? Is there a structure in which symbols that 
we experience (sounds or marks) are systematically connected to objects, to 
entities which extend beyond our experience, to nonexperiences? In other 
words, is there language? (To say that there is language is to say that half of 
all belief-assertions are true. That is, given any belief-assertion, either it is 
true or its negation is true.) Thus, the only question we need to consider is 
whether language itself exists. But we see immediately, much more 
immediately than in the case of \enquote{nonexperience,} that this question is 
caught in a trap of its own making. The question ought to be substantive. (Is 
there a systematic relation between marks and objects, between marks and 
nonexperiences? Is there an expression, \enquote{\textsc{Empire State Building,}} which is 
related to an object outside one's experience, the \textsc{Empire State Building}, and 
which therefore has the same meaning whether one is looking at the \textsc{Empire 
State Building} or not?) However, it is quite obvious that if one can even ask 
whether there is language, then the answer must be affirmative. Further, the 
distinction of language levels which is made in formal languages will not help 
here. Before you can construct formal languages, you have to know the 
natural language. The natural language is the infinite level, the container of 
the formal languages. If the container goes, everything goes. And this 
container, this infinite level language, must include its own semantics. There 
is no way to \enquote{go back before the natural language.} As we mentioned 
before, the aphorism that \enquote{saying a thing is so doesn't make it so} is an 
example of the natural language's semantics in the natural language. 

In summary, the crucial assertion is the assertion that there is language, 
made in the natural language. This assertion is true if it is meaningful. It is 
too true; it must be a definitional trick. Beliefs stand or fall on the question 
of whether there is language. There is no way to get outside the definitional 
trick and ask this question in a way that would be substantive. The question 
simply collapses. 

