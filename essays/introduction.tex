\chapter{Introduction}

This essay is the third in a series on the rationale of my career. It 
summarizes the results of my activities, the consistent outlook on a whole 
range of questions which I have developed. The first essay, 
\essaytitle{On Social Recognition}, noted that the official social philosophy of practically every 
regime in the world says that the individual has a duty to serve society to the 
best of his abilities. Social recognition is supposed to be the reward which 
indicates that the individual is indeed serving society. Now it happens that 
the most important tasks the individual can undertake are tasks (intellectual, 
political, and otherwise) posed by society. However, when the individual 
undertakes such tasks, society's actual response is almost always persecution 
(Galileo) or indifference (Mendel). Thus, the doctrine that the'individual has 
a duty to serve society is a hypocritical fraud. I reject every social 
philosophy which contains this doctrine. The rational individual will obtain 
the means of subsistence by the most efficient swindle he can find. Beyond 
this, he will undertake the most important tasks posed by society for his 
own private gratification. He will not attempt to benefit society, or to gain 
the recognition which would necessarily result if society were to utilize his 
achievements. 

The second essay, \essaytitle{Creep}, discussed the practices of isolating oneself; 
carefully controlling one's intake of ideas and influences from outside; and 
playing as a child does. I originally saw these practices as the effects of 
certain personality problems. However, it now seems that they are actually 
needed for the intellectual approach which I have developed. They may be 
desirable in themselves, rather than being mere effects of personality 
problems. 

I chose fundamental philosophy as my primary subject of investigation. 
Society presses me to accept all sorts of beliefs. At one time it would have 
pressed me to believe that the earth was flat; then it reversed itself and 
demanded that I believe the earth is round. The majority of Americans still 
consider it \enquote{necessary} to believe in God; but the Soviet government has 
managed to function for decades with an atheistic philosophy. Thus, which 
beliefs should I accept? My analysis is presented in writings entitled 
\essaytitle{Philosophy Proper}, \essaytitle{The Flaws Underlying Beliefs}, and 
\essaytitle{Philosophical Aspects of Walking Through Walls}. 
The question of whe\-ther a given belief is valid 
depends on the issue of whether there is a realm beyond my \enquote{immediate 
experience.} Does the Empire State Building continue to exist even when I 
am not looking at it? If such a question can be asked, there must indeed be 
a realm beyond my experience, because otherwise the phrase 'a realm 
beyond my experience' could not have any meaning. (Russell's theory of 
descriptions does not apply in this case.) But if the assertion that there is a 
realm beyond my experience is true merely because it is meaningful, it 
cannot be substantive; it must be a definitional trick. In general, beliefs 
depend on the assertion of the existence of a realm beyond my experience, 
an assertion which is nonsubstantive. Thus, beliefs are nonsubstantive or 
meaningless; they are definitional tricks. Psychologically, when I believe that 
the Empire State Building exists even though I am not looking at it, I 
imagine the Empire State Building, and I have the attitude toward this 
mental picture that it is a perception rather than a mental picture. The 
attitude involved is a self-deceiving psychological trick which corresponds to 
the definitional trick in the belief assertion. The conclusion is that all beliefs 
are inconsistent or self-deceiving. It would be beside the point to doubt 
beliefs, because whatever their connotations may be, logically beliefs are 
nonsense, and their negations are nonsense also. 

The important consequence of my philosophy is the rejection of truth 
as an intellectual modality. I conclude that an intellectual activity's claim to 
have objective value should not depend on whether it is true; and also that 
an activity may perfectly well employ false statements and still have 
objective value. I have developed activities which use mental capabilities that 
are excluded by a truth-oriented approach: descriptions of imaginary 
phenomena, the deliberate adoption of false expectations, the thinking of 
contradictions, and meanings which are reversed by the reader's mental 
reactions; as well as illusions, the deliberate suspension of normal beliefs, and 
phrases whose meaning is stipulated to be the associations they evoke. It 
must be clear that these activities are not in any way whatever a return to 
pre-scientific irrationalism. My philosophy demolishes astrology even more 
than it does astronomy. The irrationalist is out to deceive you; he wants you 
to believe that his superstitions are truths. My activities, on the other hand, 
explicitly state that they are using non-true material. My intent is not to get 
you to believe that superstitions are truths, but to exploit non-true material 
for rational purposes. 

The other initial subject of investigation I chose was art. The art which 
claims to have cognitive value is already demolished by my philosophical 
results. However, art at its most distinctive does not need to claim cognitive 
value; its value is claimed to be entertainmental or amusemental. What about 
art whose justification is simply that people like it? Consider things which 
are just liked, or whose value is purely subjective. I point out that each 
individual already has experiences, prior to art, whose value is purely 
subjective. (Call these experiences \term{brend.}) The difference between brend 
and art is that in art, the thing valued is separated from the valuing of it and 
turned into an object which is urged on other people. Individuals tend to 
overlook their brend, and they do so because of the same factors which 
perpetuate art. These factors include the relation between the socialization 
of the individual and the need for an escape from work. The conditioning 
which causes one to venerate \enquote{great art} is also a conditioning to dismiss 
one's own brend. If one can become aware of one's brend without the 
distortion produced by this conditioning, one finds that one's brend is 
superior to any art, because it has a level of personalization and originality 
which completely transcends art. 

Thus, I reject art as an intellectual or cultural modality. In rejecting 
truth, I advocated in its place intellectual activities which have an objective 
value independent of truth. In rejecting art, I do not propose that it be 
replaced with any objective activity at all. Rather, I advocate that the 
individual become aware of his just-likings for what they are, and allow them 
to come out. If I succeed in getting the individual to recognize his own 
just-likings, then I will have given him infinitely more than any artist ever 
can. 

We are not finished with art, however. Ever since art began to 
disintegrate as an institution, modern art has become more and more of a 
repository for activities which represent pure waste, but which counterfeit 
innovation and objective value. A two-way process is involved here. On the 
one hand, the modern artist, faced with the increasing gratuitousness of his 
profession, desperately incorporates superficial references to science in his 
products in the hope of intimidating his audience. On the other hand, art 
itself has become an institution which invests waste with legitimacy and even 
prestige; and it offers instant rewards to people who wish to play the game. 
What is innovation in modern art? You take a poem by Shelly, cut it up into 
little pieces, shake the pieces up in a box, then draw them out and write 
down whatever is on them in the order in which they are drawn. If you call 
the result a \enquote{modern poem,} people will suddenly be awed by it, whereas 
they would not have been awed otherwise. This sort of innovation is utterly 
mechanical and superficial. When artists incorporate scientific references in 
their products, the process is similarly a mechanical, superficial 
amalgamation of routine artistic material with current gadgets. 

Now there may be some confusion as to what the difference is between 
the products which result from this attempt to \enquote{save} art, and activities in 
the intellectual modality which I favor. There may be a tendency to confuse 
activities which are neither science nor art, but have objective value, with art 
products which are claimed to be \enquote{scientific} and therefore objectively 
valuable. To dispel this confusion, the following questions may be asked 
about art products. 
\begin{enumerate}[itemsep=3pt, parsep=0pt, topsep=3pt, leftmargin=1cm]
\item If the product were not called art, would it immediately be seen to be 
worthless? Does the product rely on artistic institutions to \enquote{carry} it? 

\item Suppose that the artist claims that his product embodies major scientific 
discoveries, as in the case of a ballet dancer who claims to be working in the 
field of antigravity ballet. If the dancer really has an antigravity device, 
why can it only work in a ballet theater? Why can it 
only be used to make dancers jump higher? Why do you have to be able to 
perform \enquote{Swan Lake} in order to do antigravity experiments? 
\end{enumerate}
To use a phrase from medical research, I contend that a real scientist would seek to 
isolate the active principle---not to obscure it with non-functional mumbo-jumbo. 

Both of these sets of questions make the same point, from somewhat 
different perspectives. Given an individual with a product to offer, does he 
actively seek out the lady art reporters, the public relations contracts, the 
museum officials, or does he actively dissociate himself from them? Does he 
seek artistic legitimation of his product, or does he reject it? The objective 
activities which I have developed stand on their own feet. They are not art, 
and to construe them as art would make it impossible to comprehend them. 

A definition of the intellectual modality which I favor is now in order. 
Until now, this modality has involved the construction of ideas such that the 
very possibility of thinking these ideas is a significant phenomenon. In other 
words, the modality has consisted of the invention of mental abilities. The 
ideas involve physical language, that is, language which occurs in beliefs 
about the physical world. Such language is philosophically meaningless, but 
it has connotations provided by the psychological trick involved in believing. 
The connotations are what are utilized; factual truth is irrelevant. Then, the 
ideas cannot be reduced to the mechanical manipulation of marks or 
counters---unlike ordinary mathematics. Also, logical truth, which happens to 
be discredited by my philosophical results, is irrelevant to the ideas. 

But the defining requirement of the modality is that each activity in it 
must have objective value. The activity must provide one with something 
which is useful irrespective of whether one likes it; that is, which is useful 
independently of whether it produces emotional gratification. 

We can now consider the following principle. \enquote{spontaneously and 
without any prompting to sweep human culture aside and to carry out 
elaborate, completely self-justifying activities.} Relative to the social context 
of the individual's activities, this principle is absurd. We have no reason to 
respect the eccentric hobbyist, or the person who engages in arbitrary 
antisocial acts. If an action is to have more than merely personal significance, 
it must have a social justification, as is explained in On Social Recognition. 
In the light of The Flaws Underlying Beliefs and the brend theory, however, 
the principle mentioned above does become valid when it is interpreted 
correctly, because it becomes necessary to invent ends as well as means. The 
activity must provide an objective value, but this value will no longer be 
standardized. 

The modality I favor is best exemplified by \essaytitle{Energy Cube Org\-an\-ism},
\essaytitle{Concept Art}, and the \essaytitle{Perception-Dissociator Model}. 
\essaytitle{Energy Cube Org\-an\-ism} is a perfect example of ideas such that the very 
possibility of thinking them is a significant phenomenon. It is also a perfect example of an 
activity which is useful irrespective of whether it provides emotional 
gratification. It combines the description of imaginary physical phenomena 
with the thinking of contradictions. It led to \essaytitle{Studies in Constructed 
Memories}, which in turn led to \essaytitle{The Logic of Admissible Contradictions}.
With this last writing, it becomes obvious that the activity has applications 
outside itself. 

\essaytitle{Concept Art}\footnote{published in An Anthology ed. LaMonte Young, 1963}
uses linguistic expressions which are changed by the reader's mental 
reactions. It led to \essaytitle{Post-Formalism in Constructed Mem\-or\-ies}, and this led 
in turn to \essaytitle{Subjective Propositional Vibration}.

The \essaytitle{Perception-Dissociator Model}\footnote{published in I-KON, Vol. 1, No. 5} 
was intended to exploit the realization that humans are the most 
advanced machines (or technology) that we have. I wanted to build a model 
of a machine out of humans, using a minimum of non-human props. Further, 
the machine modelled was to have capabilities which are physically 
impossible according to present-day science. I still think that the task as I 
have defined it is an excellent one; but the model does not yet completely 
accomplish the objective. The present model uses the deliberate suspension 
of normal beliefs to produce its effects. 

\essaytitle{Post-Formalism in Constructed Memories} and \essaytitle{Studies in 
Constructed Memories} together make up \booktitle{Mathematical Studies} (1966). In 
this monograph, the emphasis was on extending the idea of mathematics as 
formalistic games to games involving subjectivity and contradiction. In two 
subsequent monographs, the material was developed so as to bring out its 
potential applications in conjunction with science. 
\essaytitle{Subjective Propositional Vibration} investigates the logical 
possibilities of expressions which are changed by the reader's mental responses.
\essaytitle{The Logic of Admissible Contradictions} starts with the experiences 
of the logically impossible which 
we have when we suffer certain perceptual illusions. These illusions enable us 
to imagine certain logical impossibilities just as clearly as we imagine the 
logically possible. The monograph models the content of these illusions to 
obtain a system of logic in which some (but not all) contradictions are 
\enquote{admissible.} The theory investigates the implications of admitting some 
contradictions for the admissibility of other contradictions. A theory of 
many-valued numbers is also presented. 

The \essaytitle{Perception-Dissociator Model} led to 
\essaytitle{The Perception-Dissociation of Physics.} Again, here is an essay whose 
significance lies in the very possibility of thinking the ideas at all. The essay 
defines a change in the pattern of experience which would make it 
impossible for physicists to \enquote{construct the object from experience.} Finally, 
\essaytitle{Mock Risk Games} is the activity which involves the deliberate adoption of 
false expectations. It is on the borderline of the intellectual modality which I 
favor, because it seems to me to have objective value, and yet has not 
generated a series of applications as the other activities have. 

To summarize my general outlook, truth and art are discredited. They 
are replaced by an intellectual modality consisting of non-true activities 
having objective value, together with each individual's brend. Consider the 
individual who wishes to go into my intellectual modality. What is the 
significance to him of the academic world, professional occupations, and the 
business of scholarships, fellowships, and grants? From the perspective of 
the most socially important tasks, these institutions have always rewarded 
the wrong things, as I argued in \essaytitle{On Social Recognition}. But in addition, the 
institutions as now organized are obstacles specifically to my intellectual 
modality. In fact, society in general has the effect of a vast conspiracy to 
prevent one from achieving the kind of consequential intellectual play which 
I advocate. The categories of thought which are obligatory in the official
intellectual world and the media are categories in which my outlook cannot 
be conceived. And here is where the creep practices mentioned at the 
beginning of this essay become important. Isolation from society is 
presumably not inherent in my intellectual modality; but under present 
social conditions isolation is a prerequisite for its existence. 

