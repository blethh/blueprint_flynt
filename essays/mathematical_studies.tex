\chapter{1966 Mathematical Studies: Introduction}

Pure mathematics is the one activity which is intrinsically formalistic. It 
is the one activity which brings out the practical value of formal 
manipulations. Abstract games fit in perfectly with the tradition and 
rationale of pure mathematics; whereas they would not be appropriate in 
any other discipline. Pure mathematics is the one activity which can 
appropriately develop through innovations of a formalistic character. 

Precisely because pure mathematics does not have to be immediately 
practical, there is no intrinsic reason why it should adhere to the normal 
concept of logical truth. No harm is done if the mathematician chooses to 
play a game which is indeterminate by normal logical standards. All that 
matters is that the mathematician clearly specify the rules of his game, and 
that he not make claims for his results which are inconsistent with his rules. 

Actually, my pure philosophical writings discredit the concept of 
logical truth by showing that there are flaws inherent in all non-trivial 
language. Thus, no mathematics has the logical validity which was once 
claimed for mathematics. From the ultimate philosophical standpoint, all 
mathematics is as \enquote{indeterminate} as the mathematics in this monograph. 
All the more reason, then, not to limit mathematics to the normal concept 
of logical truth. 

Once it is realized that mathematics is intrinsically formalistic, and need 
not adhere to the normal concept of logical truth, why hold back from 
exploring the possibilities which are available? There is every reason to 
search out the possibilities and present them. Such is the purpose of this 
monograph. 

The ultimate test of the non-triviality of pure mathematics is whether it 
has practical applications. I believe that the approaches presented on a very 
abstract level in this monograph will turn out to have such applications. In 
order to be applied, the principles which are presented here have to be 
developed intensively on a level which is compatible with applications. The 
results will be found in my two subsequent essays, \essaytitle{Subjective Propositional 
Vibration} and \essaytitle{The Logic of Admissible Contradictions}.

