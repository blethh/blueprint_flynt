\newcommand\aside[1]{
	{\raggedleft \vskip 1em \parbox{3.5in}{\itshape [#1]} \par\vskip 1em}}
\newcommand\lilskip{\vskip 0.3em}

\chapter{Mock Risk Games}


Suppose you stand in front of a swinging door with a nail sticking out of it 
pointing at your face; and suppose you are prepared to jump back if the 
door suddenly opens in your face. You are deliberately taking a risk on the 
assumption that you can protect yourself. Let us call such a situation a "risk 
game." Then a mock risk game is a risk game such that the misfortune which 
you risk is contrary to the course of nature, a freak misfortune; and thus 
your preparation to evade it is correspondingly superficial. 

If the direction of gravity reverses and you fall on the ceiling, that is a 
freak misfortune. If you don't want to risk this misfortune, then you will 
anchor yourself to the floor in some way. But if you stand free so that you 
can fall, and yet try to prepare so that if you do fall, you will fall in such a 
way that you won't be hurt, then that is a mock risk game. if technicians 
could actually effect or simulate gravity reversal in the room, then the risk 
game would be a real one. But I am not concerned with real risk games. I am 
interested in dealing with gravity reversal in an everyday environment, where 
everything tells you it can't possibly happen. Your 'preparation' for the fall 
is thus superficial, because you still have the involuntary conviction that it 
can't possibly happen. 

Mock risk games constitute a new area of human behavior, because they 
aren't something people have done before, you don't know what they will be 
like until you try them, and it took a very special effort to devise them. 
They have a tremendous advantage over other activities of comparable 
significance, because they can be produced in the privacy of your own room 
without special equipment. Let us explore this new psychological effect; and 
let us not ask what use it has until we are more familiar with it. 

Instructions for a variety of mock risk games follow. (I have played 
each game many times in developing it, to ensure that the experience of 
playing it will be compelling.) For each game, there is a physical action to be 
performed in a physical setting. Then there is a list of freak misfortunes 
which you risk by performing the action, and which you must be prepared 
to evade. The point is not to hallucinate the misfortunes, or even to fear 
them, but rather to be prepared to evade them. First you work with each 
misfortune separately. For example, you walk across a room, prepared to 
react self-protectingly if you are suddenly upside down, resting on the top of 
your head on the floor. In preparing for this risk, you should clear the path 
of objects that might hurt you if you fell on them; you should wear clothes 
suitable for falling; and you should try standing on your head, taking your 
hands off the floor and falling, to get a feeling for how to fail without 
getting hurt. After you have mastered the preparation for each misfortune 
separately, you perform the action prepared to evade the first misfortune 
and the second (but not both at once). You must prepare to determine 
instantly which of the two misfortunes befalls you, and to react 
appropriately. After you have mastered pairs of misfortunes, you go on to 
triples of misfortunes, and so forth. 

The principal games are for a large room with no animals or distracting 
sounds present. 

\begin{itemize}[label={}. wide, nosep, itemsep=0.5em]
	\item[\textbf{A.}] Walk across the lighted room from one corner to the diagonally 
opposite one, breathing normally, with your eyes open. 
\lilskip
\begin{enumerate}[label=\arabic*., nosep, itemsep=0.5em]
\item You are suddenly upside down, resting on the top of your head on the 
floor. You must get down without breaking your neck. 

\item Although the floor looks unbroken and solid, beyond a certain point 
nothing is there. If you step onto that area, you will take a fatal fall. Thus, as 
you walk, you must not shift your weight to your forward foot until you are 
sure it will hold. Put the ball of the forward foot down before the heel. 

\item Something happens to the cohesive forces in your neck so that if your 
head tips in any direction, it will come right off your body, killing you 
immediately. Otherwise everything remains normal. Thus, as you walk, you 
must \enquote{balance} your head on your neck. When you reach the other side of 
the room, your neck will be restored to normal. (Prepare beforehand by 
walking with a book balanced on your head.) 

\item Invisible conical weights fall around you with their points down, each 
whistling as it falls. You must evade them by ear in order not to be stabbed. 
Walk softly and fast. 

\item The room is suddenly filled with water. You have to control your lungs 
and swim to the top. Wear clothes suitable for swimming. 
\end{enumerate}

\item[\textbf{A'.}] Play game \textbf{A} while on a long walk on an uncrowded street. The floor 
is replaced by the sidewalk. The fifth misfortune becomes for space suddenly 
to be filled with water to a height of fifteen feet above the street. 

\item[\textbf{B.}] Lie on your back on a pallet in the dimly lit room, hands at your 
sides, with a pillow on your face so that it is slightly difficult to breathe, for 
thirty seconds at a time. 

\lilskip
\begin{enumerate}
\item The pillow suddenly hardens and becomes hundreds of pounds heavier. It 
remains suspended on your face for a split second and then \enquote{falls,} bears 
down with full weight. You must jerk your head out from under it in that 
split second. 

\item The pillow adheres to your skin with a force greater than your skin's 
cohesion, and begins to rise. You must rise with it in such a way that your 
skin is not torn. 
\end{enumerate}

\item[\textbf{C.}] Lie on your back on the pallet in the dimly lit room. 

\lilskip

\begin{enumerate}
\item Gravity suddenly disappears completely, so that nothing is held down by 
it; and the ceiling becomes red-hot. You must avoid drifting up against the 
ceiling. 

\item The surface you are lying on becomes a vast lighted open plane. From the 
distance, giant steel spheres come rolling in your direction. You must evade 
them. 

\item Your body is split in half just above the waist by an indefinitely long, 
rather high, foot-thick wall. Your legs and lower torso are on one side, and 
your upper torso, arms, and head are on the other side. Matter normally 
exchanged between the two halves of your body continues to be exchanged 
through the. wall by telekinesis. It is as if you are a foot longer above the 
waist. In order to reunite your body, you must first roll over and get up, 
bent way forward. There are depressions in the wall on the same side as your 
feet. You have to climb the wall, putting your feet in the depressions and 
balancing yourself. You will be reunited when you reach the top and your 
waist passes above the wall. 
\end{enumerate}

\item[\textbf{D.}] Sit in a plain, small, straight chair, on the edge of the seat, hands 
hanging at the sides of the seat, feet together in front of the chair, in the 
lighted room, for about thirty seconds at a time. 

\lilskip

\begin{enumerate}
\item The chair is suddenly out from under you and sitting on you with Its legs 
straddling your lap and legs. You have to get your weight over your feet so 
you won't take a hard fall. 

\item The direction of gravity reverses and the chair remains anchored to the 
floor. You have to grab the seat and hold on in order not to fall on the 
ceiling. 

\item You are suddenly in a contra-terrene universe, in which the atmosphere is 
unbreathable and prolonged contact with either the atmosphere or the 
ground will disintegrate you. The seat and back of the chair become a 
penetrable hyperspatial sheet between the alien universe and your own. As 
soon as you feel the alien atmosphere, you must jerk your feet off the 
ground and deliberately sink or plunge through the seat and back of the chair 
in the best way that you can. You will end up on the floor under the chair in 
your universe. 

\item You are suddenly in dark empty space in a three-dimensional lattice of 
gleaming wires. Segments of the lattice alternately burst into flame and cool 
off. You adhere to the chair as if it were part of you. With your hands 
holding onto the seat, you can move yourself and the chair forward by 
pushing the seat forward with your hands; you can move backward by
pulling backward; you can move up by pulling up on the seat; and so on.
The lattice is formed in such a way that in order to move from one cell to
the next, you always haave to turn to some extent. Flames immediately
spring up next to you, and you have to maneuver yourself through the 
lattice to escape them.
\end{enumerate}

\item[\textbf{D`.}] Play Game \textbf{D} in situations where you have to sit and wait.

\end{itemize}

\aside{Note: The original version of \essaytitle{Mock Risk Games} was entitled \essaytitle{Exercise Awareness-States}. It was written during April--July, 1961; and read at the AG Gallery in New York on July 15, 1961. I subsequently turned against amusemental compositions, and around June 25, 1962 I sent the only copy of \essaytitle{Exercise Awareness-States} to the young musician Tom Constanten, at 1650 Michael Way in Las Vegas. I later wrote Constanten asking him to return the MS, but I never heard from him. The present revival analyses the activity better than the original version did. I am unable, though, to remember some of the most elegant misfortunes for the original games (\textbf{A}, \textbf{B}, \textbf{C}); and it seems that they are permanently lost.

In developing the original games---and the present games---I had two objectives in mind. First, the experience of playing the games (as opposed to reading or analyzing them) must involve or compel you, must be vivid and immediate. Secondly, the misfortunes must be elegant, undreamed-of \enquote{explosions} of the natural order. These objectives, though, do not constitute a use for the games. The games can have many uses, beginning with amusement; and it remains to be seen what the most significant use will be.}

\section{Intrusions}

A noise in an adjacent room may intrude on a person playing a mock
risk game, and affect his experience or state of being in a variety of ways.
Let us consider the effects of such \enquote{intrusions} on the player's state. There
are several kinds of intructions. \enquote{Distractions} are perceived by the player to
be unrelated to the game, and tend simply to take his mind off it. \enquote{Bogies}
are surprises which so fit in with the game that the player momentarily
thinks a freak misfortune has really begun; they tend to frighten the player 
and halt the game. \enquote{Modulations} are changes in the player's state or mood
which may enhance the game; they are typically induced with drugs.

The player himself can turn the radio on, bring in a cat, or otherwise
create distractions for himself. Here the object of study is how compelling
the game is. Through how much distraction can the game hold the player's 
attention? Turning to modulations, the player can also produce them for
himself.

More elaborate investigations require an experimenter outside a room
where the subject is playing mock risk games. The experimenter needs a
one-way window and an intercom to observe and talk with the subject. Here
the effects of bogies can be studied. (The experimenter has a problem 
though, in that after he frightens the subject, the subject will forget about
the game and just watch out for the bogies.) Here are some sample bogies,
for game \textbf{A}:
\lilskip
\begin{enumerate}
	\item Trip the subject with an invisible thread.
	\item Cause the floor to shift.
	\item Throw a pingpong ball at the subject from the side.
	\item Squirt water on him from behind.
\end{enumerate}
\lilskip
The mechanics of the experiments can readily be 
worked out by anyone interested in them. After an intrusion, the 
experimenter should question the subject about his reaction if it is
appropriate.

\section{Mock Risk Games for Couples (Duo Games)}

In order for these games to 
be successful,  each of you has to have confidence that the other is actually
playing. If you lack this confidence, you forget the game and just watch out
for intrusions created by the other.

\vskip 0.5em
\begin{itemize}[label={}. wide, nosep, itemsep=0.5em]
	\item[\textbf{AA.}] Face each other at a distance and walk toward each other.
		\lilskip
		\begin{enumerate}[nosep,itemsep=0.5em]
	\item The other's head flies off and hurtles at you like a cannonball. It can
		swerve up or down, so that you will be hit unless you jump aside. The time
		you have to jump is about the same no matter what your distance from the
		other is, because the head accelerates rapidly.
	\item Just as the other is putting his foot down to make a step, he suddenly
		becomes so large that his foot is descending right over your head. At the 
		same time, the mental commands of each of you to your muscles begin to be
		transmitted to the other's muscles rather than your own, and to be executed
		by his muscles rather than by yours. Thus, you must jerk \enquote{your} / \enquote{his} foot
		back, rather than complete the step, in order not to "step on your own
		head." The two of you should walk in step, right foot with right and left 
		with left. Watch the other's feet and also watch above yourself---using your
		vertical peripheral vision to do so. In short, if you suddenly see a giant foot
		coming down on you, jerk \enquote{your} forward foot back.
	\item (This misfortune is exceptionally complex, but there are good reasons for
		the complexity, and it will repay study.) The consciousness of each of you
		suddenly becomes located in the other's body and becomes hooked into the 
		other's receptors and muscles. At the same time, your body, which is now
		\enquote{outside you} and which is under the other's control, becomes surrounded
		by slowly moving beams of tissue-destroying radiation coming from the sides
		of the room. The radiation is invisible, but the eyes you are seeing through
		become sensitive to it. At the same time, the other mind loses its knowledge
		of language. In order to save your body, under the other's blind control,
		from blundering into a radiation beam, you have to communicate 
		pre-verbally to the other mind by every means from vocal cries to 
		pantomine, and get your-body/his-mind out of range of the radiation. When 
		the body is out, you will both be restored to normal. (The first thing to 
		anticipate is the basic shift in viewpoint by which you will be looking at 
		your own body from the other's position. There is no point in tensing your 
		muscles in preparatiton for the misfortune, because if it occurs, you will be 
		working with a strange set of muscles anyway. The next thing to prepare to 
		do is to spot the radiation beams; and then to yell, gesture, or 
		whatever--anything to get the \enquote{other} to avoid the radiation. Note finally 
		that neither player prepares for the possibility that he will be surrounded by 
		radiation. Each player prepares for the same role in an asymmetrical pas de 
		deux.) 
\end{enumerate}

\vskip 0.5em

\emph{Asymmetry:} The two of you play a given duo game, but each prepares 
to evade a different misfortune. 

\vskip 0.5em

\item[\textbf{AB.}] Stay awake with eyes closed for an agreed upon time between one 
and fifteen minutes. Use a timer with an alarm. 
\vskip 0.5em
\begin{enumerate}[nosep,itemsep=0.5em]
\item Each suddenly has the other's entire present consciousness in addition to 
his own, from perceptions to memories, ideologies, ambitions, and 
everything else---threatening both with psychological shock. 
\lilskip
The couple must take up positions such that their sensory perceptions 
are as nearly identical as possible. Beforehand, each must discuss with the 
other the aspects of the other's attitude to the world which each must fears 
having impused on his consciousness. During the game, each must think 
about these aspects and try to prepare for them. 

\item Each suddenly relives the other's most intense past feelings of depression 
and suicidal impulses. In other words, if five years ago the other attempted 
suicide because he failed out of college, you suddenly have the consciousness 
that \enquote{you} have just failed out of college, are totally worthless, and should 
destroy yourself. Presumably the other has since learned to live with his past 
disasters, but you do not have the defenses he has built up. You are 
overwhelmed with a despair which the other felt in the past, and which is 
incongruous with the rest of your consciousness. In summary, both of you 
risk shock and suicidal impulses. Beforehand, of course, each must tell the 
other of his worst past suicidal or depressed episode; and discuss anything 
else that may minimize the risk of shock. 
\end{enumerate}
\end{itemize}

\section{Intrusions in Duo Games}

As before, distractions and modulations can be openly studied by 
consent of the players. As for bogies, it is possible in duo games for one 
player to create a bogy without warning, in effect acting as a saboteur. As 
soon as a game is sabotaged, though, confidence is lost, and each player just 
watches out for the other's bogies. Here are some sample intrusions. 

\newcommand\rP[2]{\parbox{#1}{\raggedleft #2}}
\newcommand\cP[2]{\parbox{#1}{\centering #2}}

% \begin{tabular}{4in}{|r|p{1.25in}|p{1.25in}|}
	\begin{tabular}{|r|p{1.25in}|p{1.25in}|l|}\cline{1-4}
		\textsc{Game} & \textsc{Distraction} & \textsc{Bogy} & \textsc{Modulation}  \\\cline{1-4}
		\textbf{AA.} 1. & cough & shout in other's face& \\\cline{1-3}
		2. & talk and laugh \vskip 0.5em get out of step & stamp hard& \\\cline{1-3}
		3. & spin around & & \rotatebox{70}{each take a different drug} \\\cline{1-3}
	\textbf{AB.} 1. &  cough \vskip 0.5em talk and laugh &
		gasp \vskip 0.5em silently pass palm back \& forth in front of others face& \\\cline{1-3}
		2. & && \\\cline{1-4}
\end{tabular}
