\chapter{Philosophical Reflections I}

\begin{enumerate}[label=\textbf{\Alph*.}, wide]
\item If language is nonsense, why do we seem to have it? How do these 
intricate pseudo-significant structures arise? If beliefs are self-deceiving, why 
are they there? Why are we so skilled in the self-deceptive reflex that I find 
in language and belief? Why are we so fluent in thinking in self-vitiating 
concepts? Granting that language and belief are mistakes, are mistakes of 
this degree of complexity made for nothing? Is not the very ability to 
concoct an apparently significant, self-vitiating and self-deceiving structure a 
transcendent ability, one that points to something non-immediate? Do not 
these conceptual gymnastics, even if self-vitiating, make us superior to the 
mindless animals? 

Such questions tempt one to engage in a sort of philosophical 
anthropology, using in part the method of introspection. Beliefs could be 
explained as arising in an attempt to deal with experienced frustrations by 
denying them in thought. The origin of Christian Science and magic would 
thereby be explained. Further, we could postulate a primal anxiety-reaction 
to raw experience. This anxiety would be lessened by mythologies and 
explanatory beliefs. The frustration and the anxiety-reaction would be 
primal non-cognitive needs for beliefs. 

Going even farther, we could suppose that a being which could 
apprehend the whole universe through direct experience would have no need 
of beliefs. Beliefs would be a rickety method of coping with the limited 
range of our perception, a method by which our imperfect brains cope with 
the world. There would be an analogy with the physicist's use of phantom 
models to make experimental observations easier to comprehend. 

However, there are two overwhelming objections to this philosophical 
anthropology. First, it purports to study the human mind as a derivative 
phenomenon, to study it from a God-like perspective. The philosophical 
anthropology thus consists of beliefs which are subject to the same 
objections as any other beliefs. It is on a par with any other beliefs; it has no 
privileged position. Specifically, it is in competition not only with my 
philosophy but with other accounts of the mind-reality relation, such as 
behaviorism, Platonism, and Thomism. And my philosophy provides me with 
no basis to defend my philosophical anthropology against their philosophical 
anthropologies. My philosophy doesn't even provide me with a basis to 
defend my philosophical anthropology against its own negation. 

In short, the paradoxes which my philosophy uncovers must remain 
unexplained and unresolved. 

The other objection to my philosophical anthropology is that its 
implications are unnecessarily conservative. An explanation of why people 
do something wrong can become an assertion that it is necessary to do wrong 
and finally a justification for doing wrong. But just because I tend, for 
example, to construe my perceptions as confirmations of propositions about 
phenomena beyond my experience does not mean that I must think in this 
way. To explain the modern cognitive orientation by philosophical 
anthropology tends to absolutize it and to conceal its dispensability. 

\item There are more legitimate tasks for the introspective "anthropology" 
of beliefs than trying to find primal non-cognitive needs for beliefs. 
Presupposing the analysis of beliefs as mental acts and self-deception which I 
have made elsewhere, we need to examine closely the boundary line between 
beliefs and non-credulous mental activity. 

Is my fear of jumping out of the window a belief? Strictly speaking, 
no. In psychological terms, a conditioned reflex does not require 
propositional thought. 

Is my identification of an object in different spatial orientations 
(relative to my field of vision) as "the same object" a belief? Apparently, 
but this is very ambiguous. 

Is my identification of tactile and visual "pencil-perceptions" as aspects 
of a single object (identity of the object as it is experienced through 
different senses) a belief? Yes. 

It is possible to subjectively classify bodily movements according to 
whether they are intentional, because drunken awkwardness, adolescent 
awkwardness, and movements under ESB are clearly unintentional. Then 
does intentional movement of my hand require a belief that I can move my 
hand? Definitely not, although in rare cases some belief will accompany or 
precede the movement of my hand. But believing itself will not get the hand 
moved! 

Is there any belief involved in identifying my leg, but not the leg of the 
table at which I am sitting, as part of my body? Maybe---another ambiguous 
case.

Are my emotions of longing and dread beliefs in future time? Is my 
emotion of regret belief in past time? Philosophical anthropology: these 
temporal feelings precede and give rise to temporal beliefs. (?) 

How can I introspectively analyze my dread as dread of future injury if 
my belief in the existence of the future is invalid to begin with? Easily---the 
object of the fear is a belief or has a belief associated with it. 

\gap

\item At one point Alten claimed that his dialectical approach does not 
take any evidence as being more immediate, more primary, than any other 
evidence. Our "immediate experience" is mediated; it is a derived 
phenomenon which only subsists in an objective reality that is outside our 
subjective standpoint. 

\begin{enumerate}[label=\textbf{\arabic*.}, leftmargin=2em]

\item But Alten does not seriously defend the claim that he does not 
distinguish between immediate and non-immediate. The claim that there is 
no distinction would be regarded as demented in every human culture. Every 
culture supposes that I may be tricked or cheated: there is a realm, the 
non-immediate or non-experienced, which provides an arena for surreptitious 
hostility to me. Every culture supposes that it is easier for me to tell what I 
am thinking than what you are thinking. Every culture supposes that I will 
hear things which I should not accept before I go and see for myself. Alten is 
simply not iconoclastic enough to reject these commonplaces. What he 
apparently does is, like the perceptual psychologist, to accept the distinction 
between immediate and non-immediate, and to accept the former as the only 
way of confirming a model, but to construct a model of the relation between 
the two in which the former is analyzed as a derivative phenomenon. 

\item Alten proposes to analyze his own awareness as a derivative 
phenomenon, to take a stance outside all human awareness. But this is the 
pretense of the God-like perspective. He postulates both his own limitedness 
and his ability to step outside it! This is an overt contradiction. Indeed, it is
the archetype of the overt self-deception in beliefs which my philosophy 
exposes. "I can tell the Empire State Building exists now even though I 
cannot now perceive it." 
\end{enumerate}

\item In my technical philosophical writings, I call attention to certain 
self-vitiating "nodes" in the logic of common sense. These nodes include the 
concept of non-experience and the assertion that there is language. I often 
find that others dismiss these examples as jokes that can be isolated from 
cognition or the logic of common sense, rather than acknowledging that they 
are self-vitiating nodes in the logic of common sense. As a result, I have 
concluded that it is probably futile to debate the abstract validity of my 
analysis of these nodes. It does indeed appear as if I am debating over an 
abstract joke, and it is not apparent why I would attribute such great 
importance to a joke. 

\essaytitle{Philosophical Aspects of Walking Through Walls} represents my 
present approach. The advantage of this approach is that it makes 
unmistakable the reason why I attribute so much importance to these 
philosophical studies. I am not merely debating the abstract validity of a few 
isolated linguistic jokes; I seek to overthrow the life-world. The only 
significance of my technical philosophical writings is to offer an explanation 
of why the life---world is subject to being undermined. 

When I speak of walking through walls, the mistake is often made of 
trying to understand this reference within the framework of present-day 
scientific common sense. Walking through walls is understood as it would be 
pictured in a comic-book episode. But such an understanding is quite beside 
the point. What I am advocating---to skip over the intermediate details and go 
directly to the end result---is a restructuring of the whole modern cognitive 
orientation such that one doesn't even engage in scientific hypothesizing or 
have "object perceptions," and thus wouldn't know whether one was 
walking through a wail or not. 

At first this suggestion may seem like another joke, a triviality. But my 
genius consists in recognizing that it is not, that there is a residue of 
non-vacuity and non-triviality in this proposal. There may be only a 
hair's-breadth of difference between the state I propose and mental 
incompetance or death---but still, there is all of a hair's-breadth. I magnify 
this hair's-breadth many times, and use it as a lever to overturn civilization. 

\item I am often asked in philosophical discussion how it is that we are 
now talking if language is vitiated. Let me comment that merely pointing 
over and over to one of the two circumstances which create a paradox does 
not resolve the paradox. Indeed, a paradox arises when there are two 
circumstances in conflict. The "fact" that we are talking is one of the two 
circumstances which conjoin in the paradox of language; the other 
circumstance being the self-vitiating "nodes" I have mentioned. To repeat 
over and over that we are now talking does not resolve any paradoxes. 

Contrary to what the question of how it is that we are now talking 
suggests, we do not "see" language. (That is, we do not experience an 
objective relation between words and things.) The language we "see" is a 
shell whose "transcendental reference" is provided by self-deception. 

\item Does the theory of amcons show that the contradiction exposed in 
\essaytitle{The Flaws Underlying Beliefs} is admissible and thus loses its philosophical 
force? No. An amcon is between two things that you see, e.g. stationary 
motion. It is between two sensed qualities, the simultaneous experiencing of 
contradictory qualities. (But "He left an hour ago" begins to be a borderline 
case. Here the point is the ease with which we swallow an expression which 
violates logical rules. Also expansion of an arc: a case even more difficult to 
classify.) The contradiction in \essaytitle{The Flaws Underlying Beliefs} has to do first 
with the logic of common sense, with the logical rules of language. It has to 
do, secondly, with the circumstance that you don't see something, yet act as 
if you do. Amcons should not be used to justify self-deception in the latter 
sense, to rescue every cheap superstition. 
\end{enumerate}
