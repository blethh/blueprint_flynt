\newcommand{\stress}[1]{\textbf{#1}}

\chapter[Philosophy Proper (\enquote{Version 3,} 1961)][Philosophy Proper]{Philosophy Proper (\enquote{Version 3,} 1961)}
\renewcommand*{\thesection}{\Alph{section}}
\subsection[Chapter 1: Introduction (Revised, 1973)][Introduction]{Chapter 1: Introduction (Revised, 1973)}

This monograph defines philosophy as such---philosophy proper---to be 
an inquiry as to which beliefs are \enquote{true,} or right. The right beliefs are 
tentatively defined to be \emph{the beliefs one does not deceive oneself by holding.}
Although beliefs will be regarded as mental acts, they will be identified by 
their propositional formulations. Provisionally, beliefs may be taken as 
corresponding to \emph{non-tautologous propositions.}

Philosophy proper is an ultimate activity in the sense that no belief or 
supposed knowledge is conceded to be above philosophical examination. It is 
also an unavoidable activity in the sense that the notion of a belief, and the 
notion of judging the truth of a belief, are intrinsic to common sense and the 
natural language. Philosophers may not have achieved convincing results in 
philosophy proper; but the question of which beliefs are right is 
continuously posed for us even if we do not respect the way in which 
philosophers have dealt with it. 

All of the obstacles to philosophy proper arise because beliefs are 
normally held in order to satisfy non-cognitive needs. It will be helpful to 
examine this situation at some length. However, nothing can be done here 
beyond examining the situation. It is already clear that the interest of this 
monograph in beliefs is cognitive. It would be inappropriate to try to gain 
approval for philosophy proper by appealing to the values of those who hold 
beliefs in order to satisfy non-cognitive needs. 

It is implicit in beliefs that they correspond to cognitive claims, that 
they are subject to being judged true or false, and that their value rests on 
their truth. Nevertheless, beliefs can and do satisfy non-cognitive needs, 
quite apart from whether they are true. In order for a belief to satisfy some 
non-cognitive need, it is not necessary for the belief to be true; it merely has 
to be held. Concern with the ultimate philosophical validity of beliefs is rare. 
Concern with beliefs is normally concern with their ability to satisfy 
non-cognitive needs. 

To be specific, the literature of credulity contains remarks such as \enquote{\emph{I
could not stand to live if I did not believe so-and-so,}} or \enquote{\emph{Even if so-and-so is 
true I don't want to know it.}} These remarks manifest the needs with which 
we are concerned. To take note of these remarks is already to uncover a level 
of self-deception. It is important to realize that this self-deception is explicit 
and self-admitted. To recognize it has nothing to do with imputing 
subconscious motives to behavior, as is done in psychoanalysis. Further, to 
recognize it is by no means to advance a theory of the ultimate origin of 
beliefs, a theory which would presuppose a judgment as to the philosophical 
validity of the beliefs. To theorize that the ultimate origin of beliefs lies in 
the denial of frustrating experiences, or in primal anxieties which are 
alleviated by mythological inventions, would be inappropriate when we have 
not even begun our properly philosophical inquiry. The only self-deceptions 
being considered here are admitted self-deceptions. 

A partial classification of the circumstances in which beliefs are held for 
non-cognitive reasons follows. 
\vskip 0.5em
\begin{enumerate}[nosep, itemsep=0.5em]
\item Beliefs may be directly tied to one's morale. \enquote{\emph{I couldn't stand to live if I didn't believe in God.}} \enquote{\emph{If President Nixon is guilty I don't want to know it.}}

\item One may believe for reasons of conformity. The conversion of Jews to Catholicism in late medieval Spain was an extreme example. 

\item The American philosopher Santayana said that he believed in Ca\-tho\-li\-cism for esthetic reasons. 

\item Moral doctrines are sometimes justified on the grounds of their efficacy in maintaining public order, rather than their philosophical validity. 

\item A more complicated and more interesting situation arises when one 
who claims to be engaged in a cognitive inquiry somehow circumscribes the 
inquiry so as to ensure in advance that it will yield certain preferred results. 
Such a circumscribed inquiry will be called \term{theologizing,} in recognition of 
the archetypal activity in this category. 
\end{enumerate}

When we raise the question of whether the natural sciences are 
instances of theologizing, it becomes apparent that the issue of non-cognitive 
motives for beliefs is no light matter. According to writers on the scientific 
method such as A. d'Abro, the scientist is compelled to operate as if he 
believed in the \enquote{\emph{real existence of a real absolute objective universe---a 
common objective world, one existing independently of the observer who 
discovers it bit by bit.}} The scientist holds this belief, even though it is a 
commonplace of college philosophy courses that it is unprovable, because he 
must do so in order to get on to the sort of results he considers desirable. 
The scientist claims to be engaged in a cognitive inquiry; yet the inquiry 
begins with an act of faith which it is impermissible to scrutinize. It follows 
that science is an instance of \term{theologizing.} If scientists cannot welcome a 
demonstration that their \enquote{metaphysical} presuppositions are invalid, then 
their interest in science cannot be cognitive. 

The scientist's non-cognitive motive for believing differs from the 
non-cognitive motives described earlier in one notable respect. Each of the 
non-cognitive needs described earlier required a given belief, and could not 
be satisfied by that belief's negation. But inside a science's circumscribed 
area of inquiry, the scientist can welcome the establishment of either of two 
contradictory propositions; in other words, his non-cognitive need can be 
satisfied by either proposition. It is in this sense that he can impartially test 
or decide between two propositions, or make new discoveries. On the other 
hand, with regard to the metaphysical presuppositions of science, only a 
single alternative is welcome. 

\vskip 0.5em
\begin{enumerate}[resume, nosep, itemsep=0.5em]
\item Academicians will readily acknowledge that they are not interested 
in scholarly work by unknown persons with no academic credentials. To 
academic mathematicians and biologists, whether Galois and Mendel had 
made valid discoveries was irrelevant. Thus, academicians as academicians 
circumscribe their purported interest in the cognitive in two ways---once as 
scientists; and once for reasons of personal gain and prestige. 

\item The strangest instance of a non-cognitive need for a belief is 
provided by the person who holds a fearful belief which is widely considered 
to be superstitious, such as belief in Hell. As always, the test of whether the 
motive for the belief is cognitive is the question of whether the person would 
welcome a demonstration that the belief is invalid. There is reason to suspect 
that persons who cling to fearful beliefs would not welcome such a 
demonstration, perverse as their attitude may seem. After all, they take no 
comfort in the widespread rejection of the belief as superstitious. Thus, it 
seems that a masochistic need for fearful beliefs must be recognized. 
\end{enumerate}

\vskip 0.5em
This examination of non-cognitive motives for beliefs is, to repeat, 
limited to circumstances in which there is explicit self-deception, or 
self-deception that can be demonstrated directly from internal evidence. The 
examination cannot be carried further unless we become able to judge 
whether the beliefs referred to are, after all, valid. Thus, we will now turn to 
our properly philosophical inquiry, which will occupy the remainder of this 
monograph. 

\plainbreak{2}

\signoffnote{(Note: Chapters 2--7 were written in 1961, at a time when I used 
unconventional syntax and punctuation. They are printed here without 
change.)}

\section{The Linguistic Solution of Properly Philosophical Problems}
\subsection[Chapter 2: Preliminary Concepts][Preliminary Concepts]{Chapter 2: Preliminary Concepts}

In this part of the book I will be concerned to solve the problem of 
philosophy proper, the problem of which beliefs are right, by discussing 
language, certain linguistic expressions. To motivate what follows I might 
tentatively say that I will consider beliefs as represented by statements, 
formulations of them (for example, \enquote{Other persons have minds} as 
representing the belief that other persons have minds), so that the problem 
will be which statements are true. Actually, to solve this problem we will be 
driven far beyond answers to the effect that given statements are true (or 
false). 

To make this book as engaging as possible, I would like to start right 
into the solution of the problem, to begin with the material in the next 
chapter. However, it effects, I think, a considerable clarification and 
simplification of the presentation of the solution if I first introduce certain 
concepts in an extended discussion. Then, when they enter into the solution 
they won't have to be just suggested in a condensed explanation which has 
to be repeated over and over. Thus, this chapter will be a properly 
philosophically neutral introduction of the concepts, an introduction which 
doesn't in itself say anything about the rightness of given beliefs (or the 
truth of given statements). The chapter is as a result not so interesting as the 
others, but I hope the reader will bear with me through it. 

The first concept is a new one, that of \emph{explication}. Explication of a 
familiar linguistic expression is what might traditionally be said to be finding 
a definition of the expression; it amounts partly to determining what it is 
wanted that the expression \enquote{mean}. To explain: I will be discussing 
philosophically important expressions, familiar to the reader, such that their 
\enquote{meaning} needs clarifying, such that it is not clear to him how he wants to 
use them. I will be concerned with the suggestion of expressions, of which 
the \enquote{meanings}, uses, are clear, which will be acceptable to the reader as 
replacements for the expressions of which the uses are obscure; that is, 
which have the uses that, it will turn out, the expressions of which the uses 
are obscure are supposed to have. Since the expressions which are to be 
replacements can be equivalent as expressions (sounds, bodies of marks) to 
the expressions they are to replace, it can also be said that I will be 
concerned with the suggestion of clear \emph{uses}, of the expressions of which the 
uses are obscure, which are, it will turn out, the uses the reader wants the 
expressions to have. To be more specific about the conditions of 
acceptability of such replacements, if the familiar expressions (expressions of 
which the uses were obscure) were supposed to be names, have referents 
(and non-referents), then the new expressions must clearly have referents. 
Further, the new expressions must deserve (by having appropriate referents 
in the case of names) the principal connotations of the familiar expressions, 
especially the distinctive, honorific connotations of the familiar expressions. 
(I will not say here just how I use \enquote{connotation}. What the connotations of 
an expression are will be suggested by giving sentences about, in the case of a 
supposed name for example, what the referents of the expression are 
supposed to be like.) \enquote{Finding}, or constructing, an expression (with its use) 
supposed to be acceptable to oneself as a replacement, of the kind described, 
for an expression familiar to oneself, will be said to be \term{explicating} the 
expression familiar to oneself. The expression to be replaced will be said to 
be the \term{explicandum}, and the suggested replacement, the \term{explication}. 
Incidentally, if clarification shows that the desired use of the explicandum is 
inconsistent, then it can't have an explication at all acceptable, or what is the 
same thing, any explication will be as good as any other. 

I should mention that my use of \term{explication} is different from that of 
Rudolph Carnap, from whom I have taken the word rather than use the very 
problematic \term{definition}. For him, explication is a scientist's, or philosopher 
of science's, devising a new precise concept, useful in natural science, 
suggested by a vague, unclear common concept (for example, that of 
\enquote{work}); whereas for me it is in effect constructing (if possible) that precise, 
clear concept which is the nearest equivalent to an unclear common concept. 

Here is an example in the acceptability of explications. Suppose that an 
expression is suggested, as an explication for \enquote{thing having a mind} (if 
supposed to be a name, have referents), which has as referents precisely the 
things which have certain facial expressions, or talk, or have certain other 
\enquote{overt} behavior, or even certain brain electricity. Then I expect that this 
expression will not be acceptable to the reader as an explication for \enquote{thing 
having a mind}, since \enquote{thing having a mind} presumably has the connotations 
for the reader \enquote{\emph{that having a mind is not the same as, is very different from, 
higher than, having certain facial expressions, talking, certain other overt 
behaving, or having certain brain electricity---the mind is observable only by 
the thing having it}}, and the explication doesn't deserve these connotations: 
the connotations of the explicandum are exclusive of the referents of the 
proposed explication. It doesn't make any difference if there's a causal 
connection between having a mind and the other things, because the 
expression \enquote{thing having a mind} itself, and not the supposed effects of 
having a mind, is what is under discussion. 

As the reader can tell from the example, I will, in evaluating 
expressions, have to speak of what I assume the connotations of words are 
for the reader. If any of my assumptions are incorrect, the book will be 
slightly less relevant to the reader's philosophical problems than it would be 
otherwise. Even so, the reader should get from this part the method of 
finding good explications, and its use in solving properly philosophical 
problems. 

Especially important in deciding whether an explication for a supposed 
name is good is the check of the referents of the explication against the 
connotations of the explicandum. Traditional philosophers, in the rare cases 
when they have suggested explications for expressions in dealing with 
philosophical problems, have suggested absurdly bad ones, which can quickly 
be shown up by such a check. Examples which are typically horrible are the 
explications for \enquote{thing having a mind} mentioned above. 

The second concept I will discuss is that of true statement. As I will be 
discussing the \enquote{truth} of formulations of beliefs, statements, in the next two 
chapters, and as the concept of true statement is quite obscure (making it a 
good example of one needing explication), it will be helpful for me to clarify 
the concept beforehand, to give a partial explication for \enquote{true statement}. 
(Partial because the explication, although much clearer than the 
explicandum, will itself have an unclear word in it.) 

Well, what is a \term{statement}? How do what are usually said to be 
\term{statements} state? Take a book and look through it, a book in a language 
you don't read, so you won't assume that it's obvious what it means. What 
does the book, the object, do? How does it work? Note that talking just 
about the marks in the book, or what seem (!) to be the rules of their 
arrangement, or the like, won't answer these questions. In fact, I expect that 
when the reader really thinks about them, the questions won't seem easy 
ones to answer. Now to begin answering them, one of the most important 
connotations of \term{true statement}, and, more generally, of \emph{statement}, as 
traditionally and commonly used, is that a \term{statement} is an \enquote{assertion 
which has truth value} (is true or false) (or \enquote{has content}, as it is sometimes 
said, rather misleadingly). That is, the \enquote{verbal} part of a statement is 
supposed to be related in a certain way to something \enquote{non-verbal}, or at 
least not in the language the verbal part of the statement is in. Further, a 
statement is supposed to be \enquote{true} or not because of something having to do 
with the non-verbal thing to which the verbal part of the statement is 
related. (The exceptions are the \enquote{statements} of formalist logic and 
mathematics, which are not supposed to be assertions; they are thus 
irrelevant to statements of the kind ordinary persons and philosophers are 
interested in.) Thus, if \enquote{\term{true statement}} is to be explicated, \enquote{assertion having 
truth value} and \enquote{is true} (and \enquote{has content} in a misleading use) have to be 
explicated, as they are obscure, and as it must be clear that the explication 
for \enquote{\term{true statement}} deserves the connotations which were suggested with 
\enquote{assertion having truth value} and \enquote{is true}. One important conclusion from 
these observations is that although \enquote{sentences} (the bodies of sound or 
bodes of marks such as \enquote{The man talks}) are often said to be \enquote{statements}, 
would not be sufficient (to say the least) to explicate \enquote{\term{statement}} by simply 
identifying it with \enquote{sentence} (in my sense); something must be said about 
such matters as that of being an assertion having truth value. For the same 
reason, it is not sufficient (to say the least) to simply identify \enquote{\term{statement}}
with \enquote{sentence}, the latter being explicated in terms of the (\enquote{formal}) rules 
for the formation of (grammatical) sentences, as these rules have no 
reference to such matters as that of being an assertion having truth value. 

In explicating \enquote{\term{true statement}} I will use the most elegant approach, one 
relevant to the interest in such matters as that of being an assertion having 
truth value. This is to begin by describing a simple, if not the simplest, way 
to make an assertion. As an example, I will describe the simplest way to 
make the assertion that a thing is a table. The way is to \enquote{apply} \uline{table} to 
the thing. It is supposed that \uline{table} has been \enquote{interpreted}, that is, that it is 
\enquote{\emph{determinate}} to which, of all things, applications of \uline{table} are (to be said 
to be) \enquote{true}. (It is good to realize that it is also supposed that it is 
\enquote{determinate} which, of all things (events), are \enquote{occurrences of the word 
\enquote{table}}, are expressions \enquote{equivalent to} \uline{table}.) 
The word \enquote{\emph{determinate}} is 
the intentionally ambiguous one in this explication; I don't want to commit 
myself yet on how an expression becomes interpreted. As for \enquote{apply}, one 
can \enquote{apply} the word to the thing by pointing out \enquote{first} the word and 
\enquote{then} the thing. \enquote{point out} is restricted to refer to \term{ostension}, pointing 
out things in one's presence, things one is perceiving, and not to \enquote{directing 
attention to things not in one's presence} as well. The assertion is \enquote{true}, of 
course, if and only if the thing to which \uline{table} is applied is one of the things 
to which it is determinate that the application of \uline{table} is (to be said to be) 
\enquote{true}, otherwise \enquote{false}. It should be clear that such a pointing out of a 
\enquote{first} thing and a \enquote{second}, the first being an interpreted expression, is an 
assertion of a simple kind, does have truth value and so forth. Let me further 
suggest \enquote{\term{interpreted expression}} as an explication for \enquote{name}; with respect to 
this explication, the things to which equivalent names (\enquote{occurrences of a 
name}) may be truthfully applied are the referents of the equivalent names, 
other things being non-referents. (Incidentally, I could have started with the 
concept of a name and its referents, and then said how to make a simple 
assertion using a name.) Then what I have intentionally left ambiguous is 
\emph{how a name has referents}; I have not said, for example, whether the relation 
between name and referents is an \enquote{objective, metaphysical entity}, which 
would be getting into philosophy proper. 

The point of describing this simple way of making an assertion is that 
what one wants to say are \term{statements}, namely sentences used in the 
context of certain conventions, can be regarded as assertions of the \enquote{simple} 
kind; thus an explication for \enquote{\term{true statement}} can be found. To do so, first 
let us say that the \term{complex name} gotten by replacing a sentence's \enquote{main 
verb} with the corresponding participle is the \term{associated name} of the 
sentence. For example, the associated name of \enquote{Boston is in Massachusetts} is 
\enquote{Boston being in Massachusetts}. In the case of a sentence with coordinate 
clauses there may be a choice with respect to what is to be taken as the main 
verb, but this presents no significant difficulty. 

\vskip 0.5em

Example:  \\
\textbf{sentence:} \enquote{The 
table in the room will have been black only if it had been pushed by one 
man while the other man talked}; \\
\textbf{main verb:} \enquote{will have been} or \enquote{had been 
pushed}. 

\vskip 0.5em

Also, English may not have a participle to correspond to every verb, 
but this is in theory no difficulty; the lacking participle could obviously be 
invented. Now what we would like to say one does, in using a sentence to 
make a statement, is to so to speak \enquote{assert} its associated name; this 
\enquote{asserted name} being \enquote{true} if and only if it has a referent. However, one 
doesn't \emph{assert} names; names just have referents---it is statements that one 
makes, \enquote{asserts}, and that are \enquote{true} or \enquote{false}. How, then, do we explicate 
this \enquote{\term{asserting}} of a name? By construing it as that assertion, of the simple 
kind, which is the application of \uline{having a referent} to the name. In other 
words, from our theoretical point of view, to use a sentence to make a 
statement, one begins with a name (the sentence's associated name), and 
puts it into the sentence form, an act equivalent by convention to applying 
\uline{having a referent} to it. For example, the sentence \enquote{Boston is in 
Massachusetts} should be regarded as the simple assertion which is the 
application of \uline{having a referent} to \enquote{Boston being in Massachusetts}. 

Now this approach may seem \enquote{unnatural} or incomplete to the reader 
for several reasons. First there is the syntactical oddity: the sentence is 
replaced by a statement \enquote{about} it (or to be precise its associated name). 
Well, all I can say is that this oddity is the inevitable result of trying to 
describe explicitly all that happens when one uses a sentence to make a 
statement; I can assure the reader that the alternate approaches are even 
more unnatural. Secondly, it may seem natural enough to speak of 
interpreting \enquote{simple names} (Fries' \term{Class 1 words}), but not so natural to 
speak of interpreting complex names (what could their referents be?). Of 
course, this is because complex names are to be regarded as formed from 
simpler names by specified methods; that is, their interpretations (and thus 
referents) are in specified relations to those of the simple names from which 
they are formed. The relations are indicated by the words, in the complex 
names, which are not names, and by the order of the words in the complex 
names. An example worth a comment is associated names containing such 
words as \enquote{the}; in making statements, these names have to be in the context 
of additional conventions, understandings, to have significance. It will be 
clear that what these relations (and referents) are, the explication of these 
relations, is not important for my purposes. Thirdly, I have not said anything 
about what the \enquote{meaning} (intension), as opposed to the referents (and 
non-referents), of a name is. (I might say that a thing can't have an intension 
unless it has referents or non-referents.) This matter is also not important for 
my purposes (and gets into philosophy proper). Finally, my approach tells 
the reader no more than he already knew about whether a given statement is 
true. Quite so, and I said that the discussion would be properly 
philosophically neutral. In fact, it is so precisely because of the ambiguous 
word \enquote{determinate}, because I haven't said anything about how names get 
referents. Even so, we have come a long way from blank wonder about how 
one (sounds, marks) could ever state anything, a long way towards 
explicating how asserting works. (And to the philosopher of language with 
formalist prejudices, the discussion has been a needed reminder that if 
language is to be assertional, say something, then names and referring in 
some form must have the central role in it.) 

\term{Statements}, then, can be regarded as assertions of the \enquote{simple} kind 
which are made in the special, conventional way, involving sentences, I have 
described. I could thus explicate \enquote{\emph{true statement}} as referring to those true 
\enquote{simple} assertions made in the special way, and it should be clear that this 
would be a good explication. However, as the connotations of \enquote{true 
statement} having to do with the method of applying the first member to the 
second are, I expect, of secondary importance compared to those having to 
do with such matters as being an assertion having truth value, it ts more 
elegant to explicate \enquote{true statement} as referring to all true assertions of the 
\enquote{simple} kind. For the purposes of this book it is not important which of 
the two explications the reader prefers. 

So much for the preliminaries. 

\subsection[Chapter 3: \enquote{Experience}][\enquote{Experience}]{Chapter 3: \enquote{Experience}}

I will introduce in this chapter some basic terminology, as the main step 
in taking the reader from ordinary English and traditional philosophical 
language to a language with which my philosophy can be exposited. This 
terminology is important because one of the main difficulties in expositing 
my philosophy (or any new philosophy) is that current language is based on 
precisely some of the assumptions, beliefs, I intend to question. It will, I 
think, be immediately clear to the reader at all familiar with modern 
philosophy that the problems of terminology I am going to discuss are 
relevant to the problem of which beliefs are right. 

First, consider the term \enquote{\term{non-experience}}. Although the concept of a 
non-experience is intrinsically far more \enquote{difficult} than the concept of 
\enquote{\term{experience}} which I will be discussing presently, it is, I suppose, 
presupposed in all \enquote{natural languages} and throughout philosophy, is so 
taken for granted that it is rarely discussed in itself. Thus, the reader should 
have no difficulty understanding it. Examples of \term{non-experiences} are 
perceivable objects---for example, a table (as opposed to one's perceptions of 
it), existing external to oneself, persisting when one is not perceiving it; the 
future (future events); the past; space (or better, the distantness of objects 
from oneself); minds other than one's own; causal relationships as ordinarily 
understood; referential relationships (the relationships between names and 
their referents as ordinarily understood; what I avoided discussing in the 
second chapter); unperceivable \enquote{things} (microscopic objects (of course, 
viewing them through microscopes does not count as perceiving them), 
essences, Being); in short, most of the things one is normally concerned with, 
normally thinks about, as well as the objects of uncommon knowledge.\footnote{To 
simplify the explanation of the concept, make it easier on the reader, I am 
speaking as if I believed that there are non-experiences, that is, introducing 
the concept in the context of the beliefs usually associated with it.}
Non-experiences are precisely what one has beliefs about. One believes that 
there are microscopic living organisms, or that there are none (or that one 
can not know whether there are any---this is \stress{not} a \term{non-belief} but a complex 
belief about the relation of the realm where non-experiences could be to the 
mind). Incidentally, that other minds, for example, are non-experiences is 
presumably a connotation of \enquote{other minds} for the reader, as explained in the 
second chapter. 

In the history of philosophy, the concept of \term{non-experience} comes first. 
Then philosophers begin to develop theories of how one knows about 
non-experiences (epistemological theories). The concept of a \term{perception}, or 
\term{experience} of something, is introduced into philosophy. The theory is that 
one knows about \term{non-experiences} by perceiving, having experiences of, some 
of them. For example, one knows that there is a table before one's eyes 
(assuming that there is) by having a visual perception or experience of it, by 
having a \enquote{visual-table-experience}. The theory goes on to say that these 
perceptions are in the mind. Then, if one has a visual-table-experience in 
one's mind when there is no table, one is hallucinated. And so forth. Now 
there are two sources of confusion in all this for the naive reader. First, 
saying that perceptions of objects are in one's mind is not saying that they 
are, for example, visualizations, imaginings, such as one's visualization of a 
table with one's eyes closed. Perceptions of objects do not seem \enquote{mental}. 
The theory that they are in the mind is a \textbf{belief}. This point leads directly to 
the second source of confusion. Does the English word \enquote{table}, as ordinarily 
used to refer to a table when one is looking at it, refer to the table, an entity 
external to one's perceptions which persists when not perceived, or to one's 
perception of it, to the visual-table-experience? If distinguishing between 
the two, and the notion that the table-experience is in his mind, seem silly to 
the reader, then he probably uses \enquote{table}, \enquote{perceived table}, and 
\enquote{table-experience} as equivalent some of the time. The distinction, however, 
is not just silly; anyone who believes that there are tables when he is not 
perceiving them must accept it to be consistent. At any rate there is this 
confusion, that it is not always clear whether English object-names are being 
used to refer to perceived non-experiences or to experiences, the 
perceptions. 

Now let us ignore for a moment the connotations that experiences are 
experiences, perceptions, of non-experiences, and are in the mind. The term 
\enquote{\term{experience}} is important here because with it philosophers finally made a 
start at inventing a term for the things one knows directly, unquestionably 
knows, or, better, which one just has, or are just there (whether they are 
experiences, perceptions, of non-experiences or not). A traditional 
philosopher would say that if one is having a table-experience, one may not 
know whether it's a true perception of a table, whether there's an objective 
table there; or whether it's an hallucination; but one unquestionably knows, 
has, the table-experience. And of course, with respect to one's experiences 
not supposed to be perceptions of anything, such as visualizations, one 
unquestionably knows, has them too. A better way of putting it is that \stress{there 
is no question as to whether one has one's experiences or what they are like.}
One doesn't believe (that one has) one's experiences; to try to do so would 
be rather like trying to polish air. In fact, \enquote{thinking} that one doesn't have 
one's experiences, if this is possible, is a belief, a wrong one (as will be 
shown, although it should already be obvious if the reader has the slightest 
idea of what I am talking about), and in fact a perfectly insane one. Now the 
reader must not think that because I say experiences are unquestionably 
known I am talking about tautologies, or about beliefs which some 
philosophers say can be known by intuition even though unprovable, or say 
cannot really be doubted without losing one's sanity.\footnote{For example, some 
philosophers say this about the belief that other persons have minds.} In 
speaking of experiences I am not trying to trick the reader into accepting a 
lot of beliefs I am not prepared to justify, as many philosophers do by 
appealing to intuition or sanity or what not, a reprehensible hypocrisy 
which shows that they are not the least interested in philosophy proper. One 
does not have other-persons'-having-minds-experiences (nor are the objective 
tables one supposedly perceives table-experiences); one believes that other 
persons have minds (or that there is an objective table corresponding to one's 
table-experience), and this belief could very well be wrong (in fact, it is, as 
will be shown). 

I have explained the current use of the term \enquote{\term{experience}}. Now I want 
to propose a new use for the term, which, except where otherwise noted, 
will be that of the rest of this book. (Thus whereas in discussing 
\enquote{\term{non-experience}} I was merely explaining and accepting the current use of 
the term, in the case of \enquote{\term{experience}} I am going to suggest a new use for the 
term.) As I explained, the concept of \term{non-experience} preceded that of 
\term{experience}, and the latter was developed to explain how one knows the 
former. What I am interested in, however, is not \enquote{experience} as it implies. 
\enquote{perceptions, of non-experiences, and in the mind}, but as it refers to \stress{that 
which one unquestionably knows, is immediate, is just there, is not 
something one believes exists}. I am going to use \enquote{\term{experience}} to refer, as it 
already does, to that immediate \enquote{world}, but \stress{without the implication that 
experience is perception of non-experience, and in the mind: the same 
referents but without the old connotations}. In other words, in my use 
\enquote{\term{experience}} is completely neutral with respect to relationships to 
non-experiences, is not an antonym for \enquote{\term{non-experience}} as conventionally 
used, does not presuppose a metaphysic. The reader is being asked to take a 
leap of understanding here, because there is all the difference in philosophy 
between \enquote{experience} as implying, connoting, relatedness to non-experiences 
or in particular the realm where they could be, and \enquote{\term{experience}} without 
these connotations. 

Viewing this discussion of terminology in retrospect, it should be 
obvious that although my term \enquote{\term{experience}} was introduced last, it is 
intrinsically, logically, the simplest, most immediate, most inevitable of the 
terms, and should be the easiest to understand. In contrast, the notions I 
discussed in reaching it may seem a little arbitrary. As a matter of fact, I 
have used the perspective of the Western philosophical tradition to explain my 
term, but this doesn't mean that it is relevant only to that tradition or, 
especially, the theory of knowing about \term{non-experiences}. Even if the reader's 
conceptual background does not involve the concept of \term{non-experience,} and 
especially the modern Western theory of knowing about \term{non-experiences,} he 
ought to be able to understand, and realize the \enquote{primacy} of, my term 
\enquote{\term{experience}}. The term should be supra-cultural. 

I have gone to some length to explain my use of the term \enquote{experience}. 
As I have said, it is \enquote{intrinsically} the simplest term, but I can not define it 
by just equating it to some English expression because all English, including 
the traditional term \enquote{experience}, the antonym of \enquote{\term{non-experience}}, is based 
on metaphysical assumptions, does have implications about non-experience, 
in short, is formulations of beliefs. These implications are different for 
different philosophers according as their metaphysics\footnote{Or, as is sometimes 
(incorrectly) said, \enquote{ontologies}.} differ. Even such a sentence as \enquote{The table is 
black} implies the formulation \enquote{\uline{Material objects are real}} (to the materialist), 
or \enquote{\uline{So-called objects are ideas in the mind}} (to the idealist), or 
\enquote{\uline{Substances 
and attributes are real}}, and so forth, traditionally. As a result, in order to 
explain the new term I have had to use English in a very special way, 
ultimately turning it against itself, so as to enable the reader to guess how I 
use the term. That is, although there is nothing problematic about my use of 
\term{experience}, about its referents, there is about my English, for example 
when I say that the connotation of relatedness to \term{non-experience} is to be 
dropped from \term{experience}. There can be this new term, the philosopher is 
not irrevocably tied to English or other natural language and its implied 
philosophy, as some philosophers claim; because a term is able to be a name, 
to be used to make assertions, not by being a part of conventional English or 
other natural language, but by having referents. 

As I suggested at the beginning of this chapter, I need to introduce my 
\term{experience} because without it I cannot question all beliefs, everything 
about \term{non-experiences}, since in English there is always the implication that 
there could be \term{non-experiences}. The term is a radical innovation; one of the 
most important in this book. The fact that although it is the \enquote{simplest} and 
least questionable term, it is a radical innovation and is difficult to explain 
using English, shows how philosophically inadequate English and the 
philosophies it implies are. Now if the reader has not understood my 
\term{experience} he is likely to precisely mis-understand the rest of the book as 
an attempt to show that there are no non-experiences.\footnote{It's good that this 
isn't what I'm trying to show, because it is self-contradictory: for there to be 
no non-experiences there would have to be a realm empty of them, and this 
realm would have to be a non-experience.} If he is lucky he will just find the 
book incomprehensible, or possibly even come to understand the term from 
the rest of what I say, using it. But if he does understand the term, then he is 
past the greatest difficulty in understanding the book; in fact, he may 
already realize what I'm going to say. 


\subsection[Chapter 4: The Linguistic Solution][The Linguistic Solution]{Chapter 4: The Linguistic Solution}

Now that I have explained the key terminology for this part of the 
book, I can give the solution to properly philosophical problems, the 
problems of which beliefs are right, in the form of conclusions about the 
language in which the beliefs are formulated. My concern here is to present 
the solution as soon as possible, so as to make it clear to the reader that my 
work contains important results, is an important contribution to philosophy, 
and not just admirable sentiments or the formulation of an attitude or a 
philosophically neutral analysis of concepts or the like. For this reason I will 
not be too concerned to make the solution seem natural, or intuitive, or to 
explore all its implications; that will come later. 

However, in the hope that it will make the main \enquote{argument} of this 
chapter easier to understand, I will precede it with a short, non-rigorous 
version of it, which should give the \enquote{intuitive insight} behind the main 
argument. Consider the question of whether one can know if a given belief is 
true. Now a given belief is cognitively arbitrary in that it cannot be justified 
from the standpoint of having no beliefs, cannot be justified without 
appealing to other beliefs. Thus the answer must be skepticism: one cannot 
know if a given belief is true. However, this skepticism is a belief---a 
contradiction. The ultimate conclusion is that to escape inconsistency, to be 
right, one must, at the linguistic level, reject all talk of beliefs, of knowing if 
they are true, reject all formulations of beliefs. The \enquote{necessity}, but 
inconsistency, of skepticism \enquote{shows} my conclusion in an intuitively 
understandable way. 

To get on to the definitive version of my \enquote{argument}. I will say that 
one name \enquote{\term{depends}} on another if and only if it has the logical relation to 
that other that \uline{black table} has to \uline{table}: a referent of the former is 
necessarily a referent of the latter (one of the relations between names 
mentioned in the second chapter). Now the associated name of any 
statement, or formulation, of a belief of necessity \term{depends} on 
\term{non-experience}, since non-experiences are what beliefs are about. For 
example, \enquote{Other persons having minds}, the associated name of the 
formulation \enquote{Other persons have minds}, certainly \term{depends} on 
\term{non-experience}. Thus, anything true of \term{non-ex\-per\-ience} will be true of the 
associated name of any formulation of a belief. 

In the last chapter I introduced, explained the concepts of 
non-ex\-per\-ience and experience (in the traditional sense, as the antonym of 
\term{non-experience}), showed the connotations of the expressions 
\term{non-experience} and experience (traditional). What I did not go on to 
show, left for this chapter, is that if one continues to analyze these concepts, 
one comes on crucial implications which result in contradictions. What 
follows is perhaps the most concentrated passage in this book, so that the 
reader must be willing to read it slowly and thoughtfully. Consider one's 
\term{experience} (used in my, \enquote{neutral}, sense unless I say otherwise). Could there 
be something in one's experience, a part of one's experience, which was 
awareness of whether it's experience (traditional), of whether it's related to 
non-experience, of whether there is non-experience, awareness of 
non-experience? No, as should be obvious from the connotations shown in 
the last chapter.\footnote{Compare this with the point that one cannot (cognitively) 
justify a belief from the standpoint of having no beliefs, cannot justify it 
without appealing to other beliefs.} If there could be, if such awareness were 
just an experience, the distinctness of \term{experience} from experience 
(traditional) and so forth would disappear. The concepts of experience 
(traditional) and so forth would be superfluous, in fact, one couldn't have 
them: experience (traditional) and so forth would just be absorbed into 
\term{experience}. One concludes that there cannot be anything in one's experience 
which is awareness of whether it's experience (traditional), of whether there 
is non-experience. But then this awareness, which is in part about experience 
(traditional) and non-experience and thus involves awareness of them, is in 
one's \term{experience}---a contradiction. In fact, the same holds for the awareness 
which is \enquote{understanding the concepts} of non-experience and the rest as 
they are supposed to be understood. And for \enquote{understanding}
\term{non-experience} (and the rest) as it is supposed to be, being aware of its 
referents (and non-referents); since to name non-experience, it must be an 
experience (traditional). And even for being aware of the referents (and 
non-referents) of \enquote{\term{non-experience}}, which to name an experience 
(traditional) must be one. One mustn't assume that one understands 
\term{non-experience}---and \enquote{\term{non-experience}}---and \enquote{\enquote{\term{non-experience}}}; but here 
one is, using \enquote{\term{non-experience}} and \enquote{\enquote{\term{non-experience}}} to say so (which 
certainly implies that one assumes one understands them). It is impossible 
for there to be non-experiences. When one begins to examine closely the 
concept of \term{non-experience}, it collapses. 

(A final point for the expert. This 
tangle of contradictions is intrinsic in the concept of non-experience; it does 
not result because I have introduced a violation of the law that names cannot 
name themselves. This should be absolutely clear from the two sentences 
about names, which show contradictions---that one must not assume that 
one understands certain expressions, but that one uses the expressions to say 
so (does assume it)---with explicit stratification.) 

My exposition has broken down in a tangle of contradictions. Now 
what is important is that it has done so precisely because I have talked about 
experience (traditional), non-experience, and the rest, because I have spoken 
as if there could be non-experiences, because I have used \enquote{experience} 
(traditional), \enquote{non-experience}, and the rest. Thus, even though what I have 
said is a tangle of contradictions, it is not by any means valueless. Since it is 
a tangle of contradictions precisely because it involves \enquote{experience} 
(traditional), \enquote{non-experience}, and the rest, it shows that one who \enquote{accepts} 
the expressions, supposes that they are valid language, has inconsistent 
desires with respect to how they are to be used. The expressions can have no 
explications at all acceptable to him. He cannot consistently use the 
expressions (the way they're supposed to be). The expressions, and, 
remembering the paragraph before last, any formulation of a belief, are 
completely discredited. (What is not discredited is language referring to 
experiences (my use). If it happens that an expression I have said is a 
formulation of a belief does have a good explication for the reader, then it is 
not a formulation of a belief for him but refers to experiences.) Now there is 
an important point about method which should be brought out. If all 
\enquote{non-experiential language}, \enquote{belief language}, is inconsistent, how can I 
show this and yet avoid falling into contradiction when I say it? The answer 
is that I don't have to avoid falling into contradiction; that I fall into 
contradiction precisely because I use formulations of beliefs shows what I 
want to show. This, then, is the linguistic solution; as I said we would, we 
have been driven far beyond any such conclusion as \enquote{all formulations of 
beliefs are false}. 

Now what do these conclusions about formulations of beliefs, about 
belief language, say about beliefs themselves, about whether a given belief is 
right? Well, to the extent that a belief is tied up with its formulation, since 
the formulation is discredited, the belief is, must be wrong. After all, if a 
belief were right, its formulation would necessarily have an acceptable 
explication which was true; in short, the belief would have a true 
formulation (to see this, note that the contrary assertion is itself a 
formulation of a belief---leading to a contradiction). Incidentally, this point 
answers those who would say, that the inconsistency of their statements of 
belief taken literally does not discredit their beliefs, as the statements are not 
to be taken literally, are metaphorical or symbolic truths. To continue, one 
who because of having a belief took its formulation seriously, expected that 
it could have an acceptable explication for him, could not turn out to be an 
expression he could not properly use, must be deceiving himself in some 
way. Now there is another important point about \enquote{method} to be made. 
The question will probably continually recur to the critical reader how one 
can \enquote{know}, be aware that any given belief is wrong, without having beliefs. 
The answer is that one way one can be aware of it is simply to be aware of 
the inconsistency of belief language, which awareness is not a belief. 
(Whether belief language is inconsistent is not a matter of belief but of the 
way one wants expressions used; being aware of the inconsistency is like 
being aware with respect to a table, \enquote{that in my language, this is to be said to 
be a \enquote{table}}.) Incidentally, to wrap things up, the common belief as to how 
a name has referents is that there is a relation between the name and its 
referents which is an objective, metaphysical entity, a non-experience; this 
belief is wrong. How, in what sense a name can have referents will not be 
discussed here. 

The unsophisticated reader may react to all of this with a lot of \enquote{Yes, 
but...} thoughts. If he doesn't more or less identify beliefs with their 
formulations, and doesn't have an intuitive appreciation of the force of 
linguistic arguments, he my tend to regard my result as a mere (if 
embarrassing) curiosity. (Of course, it isn't, but I am concerned with how 
well the reader understands that.) And there does remain a lot to be said 
about beliefs themselves (as mental acts), and where the self-deception is in 
them; it is not even clear yet just what the relation of a belief to its 
formulation is. Then the reader might ask whether there aren't beliefs whose 
rejection as wrong would conflict with experience, or which it would be 
impossible or dangerous not to have. I now turn to the discussion of these 
matters. 

\subsection*{Chapter 5 : Beliefs as Mental Acts}

In this chapter I will solve the problems of philosophy proper by 
discussing believing itself, as a (\enquote{conscious}) mental act. Although I will be 
talking about mental acts and experience, it must be clear that this part of 
the book, like the fast part, is not epistemology or phenomenology. I will 
not try to talk about \enquote{perception} or the like, in a mere attempt to justify 
\enquote{common-sense} beliefs or what not. Of course, both parts are incidentally 
relevant to epistemology and phenomenology, since in discussing beliefs I 
discuss the beliefs which constitute those subjects. 

I should say immediately that \enquote{belief}, in its traditional use as supposed 
to refer to \enquote{mental acts, often unconscious, connected with the realm of 
non-experience}, has no explication at all satisfactory, has been discredited. 
This point is important, as it means that one does not want to say that one 
does or does not \enquote{have beliefs}, in the sense important to those having 
beliefs, that beliefs (in my sense) will not do as referents for \enquote{belief} in the 
use important to those having beliefs; helping to fill out the conclusion of 
the last part. Now when I speak of a \enquote{belief} I will be speaking of an 
experience, what might be said to be \enquote{an act of consciously believing, of 
consciously having a belief}, of what is \enquote{in one's head} when one says that 
one \enquote{believes a certain thing}. Further, I will, for convenience in 
distinguishing beliefs, speak of belief \enquote{that others have minds}, for example, 
or in general of belief \enquote{that there are non-experiences} (with quotation 
marks), but I must not be taken as implying that beliefs manage to be 
\enquote{about non-experiences}. (Thus, what I say about beliefs will be entirely 
about experiences; I will not be trying to talk \enquote{about the realm of 
non-experience, or the relation of beliefs to it}.) I expect that it is already 
fairly clear to the reader what his acts of consciously believing are (if he has 
any); I will be more concerned with pointing out to him some features of his 
\enquote{beliefs} (believing) than with the explication of \enquote{act of consciously 
believing}, although I will need to make a few comments about that too. 
What I am trying to do is to get the reader to accept a useful, possibly new, 
use of a word (\enquote{belief}) salvaged from the unexplicatible use of the word, 
rather than rejecting the word altogether. 

There is a further point about terminology. The reader should 
remember from the third chapter that quite apart from the theory \enquote{that 
perceptions are in the mind}, one can make a distinction between mental 
and non-mental experiences, between, for example, visualizing a table with 
one's eyes closed, and a \enquote{seen} table, a visual-table-experience. Now I am 
going to say that visualizations and the like are \enquote{imagined-experiences}. For 
example, a visualization of a table will be said to be an 
\enquote{imagined-visual-table-experience}. The reader should not suppose that by 
\enquote{imagined} I mean that the experiences are 
\enquote{hallucinations}, are \enquote{unreal}. I 
use \enquote{imagined} because saying \enquote{mental-table-experience} is too much like 
saying \enquote{table in the mind} and because just using \enquote{visualization} leaves no way 
of speaking of mental experiences which are not visualizations. Speaking of 
an \enquote{imagined-table-experience} seems to be the best way of saying that it is 
a mental experience, and then distinguishing it from other mental 
experiences by the conventional method of saying that it is an imagining \enquote{of 
a (non-mental) table-experience} (better thought of as meaning an imagining 
like a  (non-mental)  table-experience). In other words, an 
imagined-x-experience (to generalize) is a \enquote{valid} experience, all right, but it 
is not a non-mental x-experience; it is a mental experience which is like a 
(non-mental) x-experience in a certain way. Incidentally, an \enquote{imagined-imagined-experience} is impossible by definition; or is no different from an 
imagined-experience, whichever way you want to look at it. If this 
terminology is a little confusing, it is not my fault but that of the 
conventional method of distinguishing different mental experiences by 
saying that they are imaginings \enquote{of one or another non-mental experiences}. 

I can at last ask what one does when one believes \enquote{that there is a table, 
not perceived by oneself, behind one now}, or anything else. Well, in the 
first place, one takes note of, gives one's attention to, an 
imagined-experience, such as an imagined-table-experience or a visualization 
of oneself with one's back to a table; or to a linguistic expression, a supposed 
statement, such as \lexpression{There is a table behind me}. This is not all one does, 
however; if it were, what one does would not in the least deserve to be said 
to be a \enquote{belief} (a point about the explication of my \enquote{belief}). The 
additional, \enquote{essential} component of a belief is a self-deceiving \enquote{attitude} 
toward the experience. What this attitude is will be described below. Observe 
that one does not want to say that the additional component is a belief 
about the experience because of the logical absurdity of doing so, or, in 
other words, because it suggests that there is an infinite regress of mental 
action. Now the claim that the attitude is \enquote{self-deceiving} is not, could not 
be, at all like the claim \enquote{that a belief as a whole, or its formulation, fails to 
correspond in a certain way to non-experience, to reality, or is false}. The 
question of \enquote{what is going on in the realm of non-experience} does not arise 
here. Rather, my claim is entirely about an experience; it is that the attitude, 
the experience not itself a belief but part of the experience of believing, is 
\enquote{consciously, deliberately} self-deceiving, is a \enquote{self-deception experience}. I 
don't have to \enquote{prove that the attitude is self-deceiving by reference to what 
is going on in the realm of non-experience}; when I have described the 
attitude and the reader is aware of it, he will presumably find it a good 
explication, unhesitatingly want, to say that it is \enquote{self-deceiving}. 

I will now say, as well as can be, what the attitude is. In believing, one 
is attentive primarily to the imagined-experience or linguistic expression as 
mentioned above. The attitude is \enquote{peripheral}, is a matter of the way one is 
attentive. Saying that the attitude is \enquote{conscious, deliberate}, is a little 
strong if it seems to imply that it is cynical self-brainwashing; what I am 
trying to say is that it is not an \enquote{objective} or \enquote{subconscious} self-deception 
such as traditional philosophers speak of, one impossible to be aware of. This 
is about as much as I can say about the attitude directly, because of the 
inadequacy of the English descriptive vocabulary for mental experiences; 
with respect to English the attitude is a \enquote{vague, elusive} thing, very difficult 
to describe. I will be able to say more about what it is only by suggestion, by 
saying that it is the attitude \enquote{that such and such} (the reader must not think 
I mean the belief \enquote{that such and such}). If the experience to which the 
attention is primarily given in believing is an imagined-x-experience, then the 
self-deceiving attitude is the attitude \enquote{that the imagined-x-experience is a 
(non-mental) x-experience}. As an example, consider the belief \enquote{that there is 
a table behind one}. If one's attention in believing is not on a linguistic 
expression, it will be on an imagined-experience such as an 
imagined-table-experience or a visualization of a person representing oneself 
(to be accurate) with his back to a table, and one will have the self-deceiving 
attitude \enquote{that the imagined-experience is a table or oneself with one's back 
to a table}. Of course, if one is asked whether one's imagined-x-experience is 
a (non-mental) x-experience, one will say that it is not, that it is admittedly 
an imagined-experience but \enquote{corresponds to a non-experience}. This is not 
inconsistent with what I have said: first, I don't say that one believes \enquote{that 
one's imagined-x-experience is an x-experience}; secondly, when one is asked 
the question, one stops believing \enquote{that there is a table behind one} and starts 
believing \enquote{that one's imagined-experience corresponds in a certain way to a 
non-experience}, a different matter (different belief). 

lf one's attention in believing is primarily on a linguistic expression 
(which if a sentence, will be pretty much regarded as its associated name), 
the self-deceiving attitude is the attitude \enquote{that the expression has a 
referent}. With respect to the belief \enquote{that there is a table behind one}, one's 
attention in believing would be primarily on the expression \expression{There is a table 
behind me}, pretty much regarded as 'There being a table behind me', and 
one would have the self-deceiving attitude \enquote{that this name has a referent}. 
Unexplicatible expressions, then, function as principal components of 
beliefs. 

\inlineaside{This paragraph is complicated and inessential; if it begins to confuse 
the reader it can be skipped.} I will now describe the relation between the 
version, of a belief, involving language and the version not involving 
language. In the version not involving language, the attention is on an 
imagined-x-experience which is \enquote{regarded} as an x-experience, whereas in 
the version involving language, the attention is on something which is 
\enquote{regarded} as having as referent \enquote{something} (the attitude is vague here). 
For the latter version, the idea is \enquote{that the reality is at one remove}, and 
correspondingly, one whose \enquote{language} consists of formulations of beliefs 
doesn't desire to have as experiences, or perceive, or even be able to imagine, 
referents of expressions---which, for the more critical person, may make 
believing easier. Thus, just as one takes note of the imagined-x-experience in 
the version of the belief not involving language, has something which 
functions as the thing the belief is about, so in the version involving language 
one has the attitude that the expression has a referent. Further, just as one 
has the attitude that the imagined-x-experience is an x-experience in the 
version not involving language, does not recognize that what functions as the 
thing believed in is a mere imagined-experience, so in the version involving 
\enquote{language} one takes note of an \enquote{expression} not having a referent, since a 
referent could only be a (mere) experience. One who expects an expression, 
which is the principal component of a belief, to have a good explication does 
so on the basis of the self-deceiving attitude one has towards it in having the 
belief. In trying to explicate the expression, one finds inconsistent desires 
with respect to what its referents must be. These desires correspond to the 
way the expression functions in the belief: the desire that it be possible for 
awareness of the referent to be part of one's experience corresponds to the 
attitude, in believing, that the expression has a referent; and the desire that it 
not be possible for awareness of the referent to be (merely) part of one's 
experience corresponds to the expression's not having a referent in believing. 
Pointing out that the expression is unexplicable discredits the belief of which 
it is the principal component, just as pointing out that a belief not involving 
language consists of being attentive to an imagined-experience and having the 
attitude that it is not an imagined-experience, discredits that belief. 

Such, then, is what one does when one believes. If the reader is rather 
unconvinced by my description, especially because of my speaking of 
\enquote{attitudes}, then let him consider the following summary: there must be 
something more to a mental act than just taking note of an experience for it 
to be a \enquote{belief}; this something is \enquote{peripheral and elusive}, so that I am 
calling the something an \enquote{attitude}, the most appropriate way in English to 
speak of it; the attitude, an experience not itself a belief but part of the 
experience which is the belief, is thus isolated; the attitude is 
\enquote{self-deceiving}, is a \enquote{(conscious) self-deception experience}, because when 
aware of it the reader will presumably want to say that it is. The attitude just 
about has to be a (\enquote{conscious}) self-deception experience to transform mere 
taking note of an experience into something remotely deserving to be said to 
be a \enquote{belief}. The decision as to whether the attitude is to be said to be 
\enquote{self-deceiving} is to be made without trying to think \enquote{about the relation of 
the belief as a whole to the realm of non-experience}, to do which would be 
to slip into having beliefs, other than the one under consideration, which 
would be irrelevant to our concern here. Ultimately, the important thing is 
to observe what one does in believing, and particularly the attitude, more 
than to say that the attitude is \enquote{self-deceiving}. 

In order for my description of believing to be complete, I must mention 
some things often associated with believing but not \enquote{essential} to it. First, 
one may take note of non-mental and imagined-experiences other than the 
one to which attention is primarily given. If one has a table-experience and 
believes \enquote{that it is a table-perception corresponding to an objectively existing 
table', one may give much of his attention to the table-experience in so 
believing, associate the table-experience strongly with the belief. One may in 
believing give attention to non-mental experiences supposed to be "evidence 
for, confirmation of, one's belief} (more will be said about confirmation 
shortly). If one's attention in believing is primarily on the linguistic 
expression "x", one may give attention to a referent of 
"imagined-x(-experience)", an \enquote{imagined-referent} of "x"; or to 
imagined-y-experiences such that y-experiences are supposed, said, to be 
\enquote{analogous to the referent of "x"}. In the latter case the y-experiences will be 
mutually exclusive, and less importance will be given to them than would be 
to imagined-referents. An example of imagined-referents in believing is 
visualizing oneself with one's back to a table, as the imagined-referent of 
"There being a table behind one". An example of imagined-y-experiences 
(such that y-experiences are mutually exclusive) which are said to be 
\enquote{analogous to referents}, in believing, is the visualizations associated with 
beliefs \enquote{about entities wholly other than, transcending, experience, such as 
Being}. 

Secondly, there are associated with beliefs logical \enquote{justifications}, 
\enquote{arguments}, for them, \enquote{defenses} of them. I will not bother to explicate 
the different kinds of justifications because it is so easy to say what is wrong 
with all of them. There are two points to be made. First, explication would 
show that the matter of justifications for beliefs is just a matter of language 
and beliefs of the kind already discussed. Secondly, as I have suggested 
before, whether a statement or belief is right is not dependent on what the 
justifications, arguments for it are. (If this seems to fail for inductive 
justification, the kind involving the citing of experience supposed to be 
evidence for, confirmation of, the belief, it is because the metaphysical 
assumptions on which induction is based are rarely stated. Without them 
inductive justifications are just non sequiturs. An example: this table has 
four legs; therefore (\enquote{it is more probable that}) any other table has four 
legs.) Justification of a statement or belief does nothing but conjoin to it 
superfluous statements or beliefs, if anything. The claim that a justification, 
argument can show that a belief is not arbitrary, gratuitous, in that it can 
show that to be consistent, one must have the belief if one has a Sesser, 
weaker belief, is simply self-contradictory. If a justification induces one to 
believe what one apparently did not believe before hearing the justification, 
then one already had the belief \enquote{implicitly} (it was a conjunct of a belief 
one already had), or one has accepted superfluous beliefs conjoined with it. 

I will conclude this chapter first with a list of philosophical positions 
my position is not. Although I have already suggested some of this material, 
I repeat it because it is so important that the reader not misconstrue my 
position as some position which is no more like mine than its negation is, 
and which I show to be wrong. My position is not disbelief. (Incidentally, it 
is ironic that "disbeliever", without qualification, has been used by believers 
as a term of abuse, since, as disbelief is belief which is the negation of some 
belief, any belief is disbelief.) In particular, I am not concerned to deny \enquote{the 
existence of non-experience}, to \enquote{cause non-experiences to vanish}, so to 
speak, to change or cause to vanish some of the reader's non-mental 
experiences, \enquote{perceived objects}. My position is not skepticism of any kind, 
is not, for example, the belief \enquote{that there is a realm where there could either 
be or not be certain entities not experiences, but our means of knowing are 
inadequate for finding which is the case.} My position is not a mere 
\enquote{decision to ignore non-experiences, or beliefs}. The philosopher who denies 
\enquote{the existence of non-experiences}, or denies any belief, or who is skeptical 
of any belief, or who merely \enquote{decides to ignore non-experiences, or beliefs}, 
has some of the very beliefs I am concerned to discredit. 

What I have been concerned to do is to discredit formulations of 
beliefs, and beliefs as mental acts, by pointing out some features of them. In 
the first part of the book I showed the inconsistency of linguistic expressions 
dependent on "non-experience", and pointed out that those who expect them 
to have explications at all acceptable are deceiving themselves; discrediting 
the beliefs of which the expressions are formulations. In this chapter, I have 
described the mental act of believing, calling the reader's attention to the 
self-deception experience involved in it, and thus showing that it is wrong. 
To conclude, in discrediting beliefs I have shown what the right 
philosophical position is: it is not having beliefs (and realizing, for any belief 
one happens to think of, that it is wrong (which doesn't involve having beliefs)). 

\subsection*{Chapter 6 --- Discussion of Some Basic Beliefs}

In the preceding chapters I have been concerned, in discrediting any 
given belief, to show what the right philosophical position is. In this chapter 
I will turn to particular beliefs, supposed knowledge, to make it clear just 
what, specifically, have been discredited. Now if the reader will consider the 
entire \enquote{history of world thought}, the fantastic proliferation of activities at 
least partly \enquote{systems of knowledge} which constitute it, Platonism, 
psychoanalysis, Tibetian mysticism, physics, Bantu witchcraft, 
phenomenology, mathematical logic, Konko Kyo, Marxism, alchemy, 
comparative linguistics, Orgonomy, Thomism, and so on indefinitely, each 
with its own kind of conclusions, method of justifying them, applications, 
associated valuations, and the like, he will quickly realize that I could not 
hope to analyze even a fraction of them to show just how \enquote{non-experiential 
language}, and beliefs, are involved in them. And I should say that it is not 
always obvious whether the concepts of non-experiential language, and 
belief, are relevant to them. Zen is an obvious example (although as a matter 
of fact is unquestionably does involve beliefs, is not for example an 
anticipation of my position). Further, many quasi-systems-of-knowledge are 
difficult to discuss because the expositions of them which are what one has 
to work with, are badly written, in particular, fail to state the insights behind 
what is presented, the real reasons why it can be taken seriously, and are 
incomplete and confused. 

What I will do, then, to specifically illustrate my results, is to discuss a 
few particular beliefs which are found in almost all systems of \enquote{knowledge}; 
have been given especial attention in modern Western philosophy and are 
thus especially relevant to the immediate audience for this book; and are so 
\enquote{basic} (accounting for their ubiquity) that they are either just assumed, as 
too trivially factual to be worthy the attention of a profound thinker, or if 
they are explicit are said to be so basic that persons cannot do without them. 
The discussion will make it specifically clear that it is not necessary to have 
these beliefs, that not having them is not \enquote{inconsistent} with one's 
experience; and is thus important for the reader who is astonished at the idea 
of rejecting any given belief, the idea of any given belief's being wrong and 
of not having it. 

Consider beliefs to the effect \enquote{that the world is ordered}, beliefs 
formulated in \enquote{natural laws}, beliefs \enquote{about substance}, and the like. 
Rejection of them may seem to lead to a problem. After all, one's \enquote{perceived 
world} is not \enquote{chaotic}, is it? The reader should observe that in rejecting 
beliefs \enquote{that the world is ordered} I do not say that his \enquote{perceived world} is 
(\enquote{subjectively}) chaotic (that is, extremely unfamiliar, strange). The 
non-strange character of one's \enquote{perceived world} is associated with beliefs 
\enquote{about substance} and beliefs formulated in natural laws, but it is not \enquote{the 
world being ordered}; and taking note of the non-strange character of one's 
\enquote{perceived world} is not part of what is \enquote{essential} in these beliefs. 

Rejection of \enquote{spatio-temporal} beliefs may seem to lead to a problem. 
After all, cannot one watch oneself wave one's hand towards and away from 
oneself? Of course one can \enquote{watch oneself wave one's hand} (in a non-strict 
sense---and if the reader uses the expression in this sense it will not be a 
formulation of a belief for him). However, that one can \enquote{watch oneself wave 
one's hand} (in the non-strict sense) does not imply \enquote{that there are spatially 
distant, and past and future events}; and although experiences such as a 
visual---\enquote{moving}---hand experience are associated with spatio-temporal 
beliefs, taking note of them is not part of what is essential in those beliefs. 

Rejection of beliefs \enquote{about the objectivity of linguistic referring} may 
seem to lead to a problem. After all, when one says that a table is a \enquote{table}, 
doesn't one do so unhesitatingly, with a feeling of satisfaction, a feeling that 
things are less mysterious, strange, when one has done so, and without the 
slightest intention of saying that it is a \enquote{non-table}? The reader should 
observe that I do not deny this. These experiences are associated with beliefs 
\enquote{about the objectivity of referring}, but they are not \enquote{objective referring}; 
and taking note of them is not part of what is essential in those beliefs. 

Rejection of the belief \enquote{that other humans (better, things) than oneself 
have minds} my seem to lead to a problem. After all, \enquote{perceived other 
humans} talk and so forth, do they not? The reader should observe that in 
rejecting the belief \enquote{that others have minds} I do not deny that \enquote{perceived 
other humans} talk and so forth. Other humans' talking and so forth is 
associated with the belief \enquote{that others have minds}, but it is not \enquote{other 
humans having minds}; and taking note of others talking and so forth is not 
part of what is essential in believing \enquote{that others have minds}, points I 
anticipated in the second chapter. 

Finally, many philosophers will violently object to rejection of 
temporal beliefs of a certain kind, namely beliefs of the form \enquote{If \x, then \y\ 
will follow in the future}, especially if \y\ is something one wants, and \x\ is 
something one can do. (After all, doesn't it happen that one throws the 
switch, and the light goes on?) They object so strongly because they fear 
\enquote{that one cannot live unless one has and uses such knowledge}. They say, 
for example, \enquote{that one had better know that one must drink water to live, 
and drink water, or one won't live}. Now \enquote{one's throwing the switch and the 
light's coming on} (in a non-strict sense) is like the experiences associated 
with other temporal beliefs; that one can do it (in the non-strict sense) does 
not imply \enquote{that there are past or future events}, and taking note of it is not 
part of what is essential in the belief \enquote{that if one throws the switch, then the 
light will come on}. As for what the philosophers say, fear, believe \enquote{about 
the necessity of such knowledge for survival}, it is just more beliefs of the 
same kind, so that rejection of it is similarly unproblematic. If this abrupt 
dismissal of the fears as wrong is terrifying to the reader, then it just shows 
how badly he is in need of being straightened out philosophically. 
Incidentally, all this should make it clear that it is futile to try to \enquote{save} 
beliefs (render them justifiable) by construing them as predictions. 

By now the reader has probably observed that the beliefs, and their 
formulations, which I have been discussing, the ones he is presumably most 
suspicious of rejecting, are all strongly (but not essentially) associated with 
non-mental experiences of his. The reader may no longer seriously have the 
beliefs, but have problems in connection with them, get involved in 
defending them, and be suspicious of rejecting them, merely because he 
continues to use the formulations of the beliefs, but to refer to the 
experiences associated with them (as there's no other way in English to do 
so), and confusedly supposes that to reject the beliefs and formulations is to 
deny that he has the experiences. Now I am not denying that he has the 
experiences. As I said in the last chapter, I am not trying to convince the 
reader that he doesn't have experiences he has, but to point out to him the 
self-deception experiences involved in his beliefs. The reader should be wary 
of thinking, however, on reading this, that maybe he doesn't have any beliefs 
after all, just uses the belief language he does to refer to experiences. It 
sometimes happens that people who have beliefs and as a result use belief 
language excuse themselves on the basis that they are just using the language 
to refer to experiences, an hypocrisy. If one uses belief formulations, it's 
usually because one has beliefs. 

The point that the language which one may use to describe experiences 
is formulations of beliefs, is true generally. As I said in the third chapter, all 
English sentences are, traditionally anyway, formulations of beliefs. As a 
result, those who want to talk about experiences (my use) and still use 
English are forced to use formulations of beliefs to refer to strongly 
associated experiences, and this seems to be happening more and more; often 
among quasi-empiricists who naively suppose that the formulations have 
always been used that way, except by a few \enquote{metaphysicians}. I have had to 
so use belief language throughout this book, the most notable example being 
the introduction of my use of \enquote{experience} in the third chapter. Thus, some 
of what I say may imply belief formulations for the reader when it doesn't 
for me, and be philosophically problematic for him; he must understand the 
book to some extent in spite of the language, as I suggested in the third 
chapter. I have tried to make this relatively easy by choosing, to refer to 
experiences, language with which they are very strongly associated and 
which is only weakly associated with beliefs, and, the important thing, by 
announcing when the language is used for that purpose. 

It is time, though, that I admit, so as not to be guilty of the hypocrisy I 
was exposing earlier, that most of the sentences in this book will be 
understood as formulations of beliefs, that, in other words, I have presented 
my philosophy to the reader by getting him to have a series of beliefs. This 
does not invalidate my position, because the beliefs are not part of it. They 
are for the heuristic purpose of getting the reader to appreciate my position, 
which is not having beliefs (and realizing, for any belief one happens to think 
of, that it is wrong (which doesn't involve believing)); and they may well not 
be held when they have accomplished that purpose. I hope I will eventually 
get around to writing a version of this book which presents my position by 
suggesting to the reader a series of imaginings (and no more), rather than 
beliefs; developing a new language to do so. The reason I stick with English 
in this book is of course (!) that readers are too \enquote{unmotivated} (lazy!) to 
learn a language of an entirely new kind to read a book, having 
unconventional conclusions, in philosophy proper. 

\subsection*{Chapter 7 --- Summary}

The most important step in understanding my work is to realize that I 
am trying neither to get one to adopt a system of beliefs, nor to just ignore 
beliefs or the matter of whether they are right. Once the reader does so, he 
will find that my position is quite simple. The reader has probably tended to 
construe the body of the book, the second through the sixth chapters, as a 
formulation of a system of beliefs; or as a proposal that he ignore beliefs or 
the matter of whether they are right. Even if he has, a careful reading of 
them will, I hope, have prepared him for a statement of my position which is 
supposed to make it clear that the position is simple and right. This 
statement is a summary, and thus cannot be understood except in 
connection with the second through the sixth chapters. First, I reiterate that 
my position is not a system of beliefs, supported by a long, plausible 
argument. This means, incidentally, that it is absurd to \enquote{remain 
unconvinced} of the rightness of my position, or to \enquote{doubt, question} it, or 
to take a long time to decide whether it is right: one can \enquote{question} (not 
believe) disbelief, but not unbelief. (Not to mention that it is a wrong belief 
to be \enquote{skeptical} of my position in the sense of believing \enquote{that although the 
position may subjectively seem right, there is always the possibility that it is 
objectively wrong}.) I am trying, not to get one to adopt new beliefs but to 
reject those one already has, not to make one more credulous but less 
credulous. If one \enquote{questions my position} then one is misconstruing it as a 
belief for which I try to give a long, plausible argument, and is trying to 
decide which is more plausible, my argument that all beliefs are false, say, or 
the arguments that beliefs are true. It may well take one a long time to 
understand my position, but if one is taking a long time to decide whether it 
is right then one is wasting one's time thinking about a position I show to be 
wrong. Secondly, my position is not a proposal that one ignore beliefs or the 
matter of whether they are right. Thus, it is absurd to conclude that my 
position is irrefutable but trivial, that one who has beliefs can also be right. 

Now for the statement of the position. Imagine yourself without 
beliefs. One certainly is without beliefs when one is not thinking, for 
example (although not only then). This being without beliefs is my position. 
Now this position can't be wrong inasmuch as you aren't doing anything to 
be \enquote{true or false}, to be self-deceiving. Now imagine that someone asks you 
to believe something, for example, to believe \enquote{that there is a table behind 
you}. Then if you are going to do what he asks, and believe (as opposed to 
continuing not to think; or only imagining---for example, \enquote{visualizing 
yourself with your back to a table}), you are going to have to have the 
attitude that you are in effect perceiving what you don't perceive, that is, 
deceive yourself. (What else could he be asking you to do?) You are going 
to have to be wrong. That's all there is to it. 

As for my language here, it is primarily intended to be suggestive, 
intended, at best, to suggest imaginings to you which will enable you to 
realize what the right philosophical position is (as in the last paragraph). The 
important thing is not whether the sentences in this book correspond to true 
statements in your language (although I expect the key ones will, the 
expressions in them being construed as referring to the experiences 
associated with them); it is for you to realize, observe what you do when 
you don't have beliefs and when you do. You are not so much to study my 
language as to begin to ask what one who asks you to believe wants you to 
do, anyway. The language isn't sufficiently flawless to absolutely force the 
complete realization of what the right position is on you (it doesn't have to 
be flawless to unquestionably discredit \enquote{non-experiential language}); if you 
don't want to realize where the self-deception is in believing you can just 
ignore the book, and \enquote{justify} your doing so on the basis of what I have said 
about language such as I have used. The point is that the book is not 
therefore valueless. 

So much for what the right philosophical position is. From having 
beliefs to not having them is not a trivial step; it is a complete 
transformation of one's cognitive orientation. Yet astonishing as the latter 
position is when first encountered, does it not become, in retrospect, 
\enquote{obvious}? What other position could be the resolution of the fantastic 
proliferation of conflicting beliefs, and of the \enquote{profound} philosophical 
problems (for example, \enquote{Could an omnipotent god do the literally 
impossible?}, \enquote{Are statements about what I did in the past while alone 
capable of intersubjective verification?}) arising from them? And again, one 
begins to ask, when one is asked to believe something, what it is that one is 
wanted to do, anyway; and one's reaction to the request comes to be \enquote{Why 
bother? Cognitively, what is the value of doing so? I'd just be deceiving 
myself}. Also, how much simpler my position is than that of the believer. 
And although in a way the believer's position is the more natural, since one 
\enquote{naturally} tends to deceive oneself if there's any advantage in doing so 
(that is, being right tends not to be valued), in another way my position is, 
since it is simple, and since the non-believer isn't worried by the doubts 
which arise for one who tries to keep himself deceived. 

