\chapter{Subjective Propositional Vibration (Work in Progress)}

Up until the present, the scientific study of language has treated 
language as if it were reducible to the mechanical manipulation of counters 
on a board. Scientists have avoided recognizing that language has a mental 
aspect, especially an aspect such as the 'understood meaning" of a linguistic 
expression. This paper, on the other hand, will present linguistic constructs 
which inescapably involve a mental aspect that is objectifiable and can be 
subjected to precise analysis in terms of perceptual psychology. These 
constructs are not derivable from the models of the existing linguistic 
sciences. In fact, the existing linguistic sciences overlook the possibility of 
such constructs. 

Consider the ambiguous schema '$A\supset B\&C$', expressed in words as '$C$ and 
$B$ if $A$'. An example is 

\begin{equation}
	\label{firstvib}
	\parbox{4in}{Jack will soon leave and Bill will laugh if Don speaks.}
\end{equation}

In order to get sense out of this utterance, the reader has to supply it with a 
comma. That is, in the jargon of logic, he has to supply it with grouping. Let 
us make the convention that in order to read the utterance, you must 
mentally supply grouping to it, or "bracket" it. If you construe the schema 
as '$A\supset (B\&C)$', you will be said to bracket the conjunction. If you construe 
the schema as '$(A\supset B)\&C$', you will be said to bracket the conditional. There 
is an immediate syntactical issue. If you are asked to copy \ref{firstvib}, do you write 
"Jack will soon leave and Bill will laugh if Don speaks"; or do you write 
"Jack will soon leave, and Bill will laugh if Don speaks" if that is the way 
you are reading \ref{firstvib} at the moment? A distinction has to be made between 
reading the proposition, which involves bracketing; and viewing the 
proposition, which involves reacting to the ink-marks solely as a pattern. 
Thus, any statement about an ambiguous grouping proposition must specify 
whether the reference is to the proposition as read or as viewed. 

Some additional conventions are necessary. With respect to \ref{firstvib}, we 
distinguish two possibilities: you are reading it, or you are not looking at it 
(or are only viewing it). Thus, a "single reading" of \ref{firstvib} refers to an event 
which separates two consecutive periods of not looking at \ref{firstvib} (or only 
viewing it). During a single reading, you may switch between bracketing the 
conjunction and bracketing the conditional. These switches demarcate a 
series of "states" of the reading, which alternately correspond to "Jack will 
soon leave, and Bill will laugh if Don speaks" or "Jack will soon leave and Bill 
will laugh, if Don speaks". Note that a state is like a complete proposition. 
We stipulate that inasmuch as \ref{firstvib} is read at all, it is the present meaning or 
state that counts---if you are asked what the proposition says, whether it is 
true, \etc

Another convention is that the logical status of 
\begin{quotation}
(Jack will soon leave and Bill will laugh if Don speaks) if and only if (Jack 
will soon leave and Bill will laugh if Don speaks) 
\end{quotation}
is not that of a normal tautology, even though the biconditional when 
viewed has the form '$A\equiv A$'. The two ambiguous components will not 
necessarily be bracketed the same way in a state. 

We now turn to an example which is more substantial than \ref{firstvib}.

Consider 

\begin{quotation}
Your mother is a whore and you are now bracketing the conditional in (2) if 
you are now bracketing the conjunction in (2). (2) 
\end{quotation}

If you read this proposition, then depending on how you bracket it, the 
reading will either be internally false or else will call your mother a whore. In 
general, ambiguous grouping propositions are constructs in which the mental 
aspect plays a fairly explicit role in the language. We have included (2) to 
show that the contents of these propositions can provide more complications 
than would be suggested by \ref{firstvib}.

There is another way of bringing out the mental aspect of language, 
however, which is incomparably more powerful than ambiguous grouping. 
We will turn to this approach immediately, and will devote the rest of the 
paper to it. The cubical frame \cubeframe\ is a simple reversible perspective figure 
which can either be seen oriented upward like \cubeup\ or oriented downward 
like \cubedown. Both positions are implicit in the same ink-on-paper image; it is 
the subjective psychological response of the perceiver which differentiates 
the positions. The perceiver can deliberately cause the perspective to reverse, 
or he can allow the perspective to reverse without resisting. The perspective 
can also reverse against his will. Thus, there are three possibilities: deliberate, 
indifferent, and involuntary reversal. 

Suppose that each of the positions is assigned a different meaning, and 
the figure is used as a notation. We will adopt the following definitions 
because they are convenient for our purposes at the moment. 

$$ \cubeframe \left\{\parbox{4in}{for '3' if it appears to be oriented like \cubeup \linebreak
for '0' if it appears to be oriented like \cubedown}\right\} $$

We may now write 

\begin{equation}
	\label{cubefour}
1+\cubeframe = 4 
\end{equation}

We must further agree that \ref{cubefour}, or any proposition containing such 
notation, is to be read to mean just what it seems to mean at any given 
instant. If, at the moment you read the proposition, the cube seems to be 
up, then the proposition means $1+3=4$; but if the cube seems to be down, 
the proposition means $1+O=4$. The proposition has an unambiguous 
meaning for the reader at any given instant, but the meaning may change in 
the next instant due to a subjective psychological change in the reader. The 
reader is to accept the proposition for what it is at any instant. The result is 
subjectively triggered propositional vibration, or SPV for short. The 
distinction between reading and viewing a proposition, which we already 
made in the case of ambiguous grouping, is even more important in the case 
of SPV. Reading now occurs only when perspective is imputed. In reading 
\ref{cubefour} you don't think about the ink graph any more than you think about the 
type face. 

in a definition such as that of '\cubeframe', '3' and 'O' will be called the 
assignments. A single reading is defined as before. During a single reading, \ref{cubefour}
will vibrate some number of times. The series of states of the reading, which 
alternately correspond to '$1+3=4$' or '$1+O=4$', are demarcated by 
these vibrations. The portion of a state which can change when vibration 
occurs will be called a partial. It is the partials in a reading that correspond 
directly to the assignments in the definition. 

Additional conventions are necessary. Most of the cases we are 
concerned with can be covered by two extremely important rules. First, the 
ordinary theory of properties which have to do with the form of expressions 
as viewed is not applicable when SPV notation is present. Not only is a 
biconditional not a tautology just because its components are the same when 
viewed; it cannot be considered an ordinary tautology even if the one 
component's states have the same truth value, as in the case of '$1+\cubeframe\neq2$'. 
Secondly, and even more important, SPV notation has to be present 
explicitly or it is not present at all. SPV is not the idea of an expression with 
two meanings, which is commonplace in English; SPV is a double meaning 
which comes about by a perceptual experience and thus has very special 
properties. Thus, if a quantifier should be used in a proposition containing 
SPV notation, the "range" of the "variable" will be that of conventional 
logic. You cannot write '\cubeframe' for '$x$' in the statement matrix 
'$x=\cubeframe$'.

We must now elucidate at considerable length the uniqué properties of 
SPV. When the reader sees an SPV figure, past perceptual training will cause 
him to impute one or the other orientation to it. This phenomenon is not a 
mere convention in the sense in which new terminology is a convention. 
There are already two clear-cut possibilities. Their reality is entirely mental; 
the external, ink-on-paper aspect does not change in any manner whatever. 
The change that can occur is completely and inherently subjective and 
mental. By mental effort, the reader can consciously control the orientation. 
If he does, involuntary vibrations will occur because of neural noise or 
attention lapses. The reader can also refrain from control and accept 
whatever appears. In this case, when the figure is used as a notation, 
vibrations may occur because of a preference for one meaning over the 
other. Thus, a deliberate vibration, an involuntary vibration, and an 
indifferent vibration are three distinct possibilities. 

What we have done is to give meanings to the two pre-existing 
perceptual possibilities. In order to read a proposition containing an SPV 
notation at all, one has to see the ink-on-paper figure, impute perspective to 
it, and recall the meaning of that perspective; rather than just seeing the 
figure and recalling its meaning. The imputation of perspective, which will 
happen anyway because of pre-existing perceptual training, has a function in 
the language we are developing analogous to the function of a letter of the 
alphabet in ordinary language. The imputation of perspective is an aspect of 
the notation, but it is entirely mental. Our language uses not only 
graphemes, but "psychemes" or "mentemes". One consequence is that the 
time structure of the vibration series has a distinct character; different in 
principle from external, mechanical randomization, or even changes which 
the reader would produce by pressing a button. Another consequence is that 
ambiguous notation in general is not equivalent to SPV. There can be mental 
changes of meaning with respect to any ambiguous notation, but in general 
there is no psycheme, no mental change of notation. It is the clear-cut, 
mental, involuntary change of notation which is the essence of SPV. Without 
psychemes, there can be no truly involuntary mental changes of meaning. 

In order to illustrate the preceding remarks, we will use an SPV 
notation defined as follows. 

\begin{equation*}
	\cubeframe \left\{\parbox{4in}{is an affirmative, read "definitely," if it appears to be oriented 
	like \cubeup\linebreak
	is a negative, read "not," if it appears to be oriented like \cubedown}\right\}
\end{equation*}

The proposition which follows refers to the immediate past, not to all past 
time; that is, it refers to the preceding vibration. 

\begin{quotation}
You have \cubeframe deliberately vibrated (4). (4) 
\end{quotation}


This proposition refers to itself, and its truth depends on an aspect of the 
reader's subjectivity which accompanies the act of reading. However, the 
same can be said for the next proposition. 

\begin{quotation}
The bat is made of wood, and you have just decided that the second 
word in (5) refers to a flying mammal. (5) 
\end{quotation}


Further, the same can be said for (2). We must compare (5), (2), and (4) in 
order to establish that (4) represents an order of language entirely different 
from that represented by (5) and (2). (5) is a grammatical English sentence 
as it stands, although an abnormal one. The invariable, all-ink notation 'bat' 
has an equivocal referental structure: it may have either of two mutually 
exclusive denotations. In reading, the native speaker of English has to choose 
one denotation or the other; contexts in which the choice is difficult rarely 
occur. (2) is not automatically grammatical, because it lacks a comma. We 
have agreed on a conventional process by which the reader mentally supplies 
the comma. Thus, the proposition lacks an element and the reader must 
supply it by a deliberate act of thought. The comma is not, strictly speaking, 
a notation, because it is entirely voluntary. The reader might as well be 
supplying a denotation io an equivocal expression: (5) and (2) can be 
reduced to the same principle. As for (4), it cannot be mistaken for ordinary 
English. It has an equivocal "proto-notation," '\cubeframe'. You automatically 
impute perspective to the proto-notation before you react to it as language. 
Thus, a notation with a mental component comes into being involuntarily. 
This notation has an unequivocal denotation. However, deliberate, 
inditferent, and most important of all, involuntary mental changes in 
notation can occur. 

We now suggest that the reader actually read (5), (2), and (4), in that 
order. We expect that (5) can be read without noticeable effort, and that a 
fixed result will be arrived at (unless the reader switches in an attempt to 
find a true state). The reading of (2) involves mentally supplying the comma, 
which is easy, and comprehending the logical compound which . results, 
which is not as easy. Again, we expect that a fixed result will be arrived at 
(unless the reader vacillates between the insult and the internally false state). 
In order to read (4), center your sight on the SPV notation, with your 
peripheral vision taking in the rest of the sentence. A single reading should 
last at least half a minute. If the reader will seriously read (4), we expect that 
he will find the reading to be an experience of a totally different order from 
the reading of (5) and (2). It is like looking at certain confusing visual 
patterns, but with an entire dimension added by the incorporation of the 
pattern into language. The essence of the experience, as we have indicated, is 
that the original imputation of perspective is involuntary, and that the reader 
has to contend with involuntary changes in notation for which his own mind 
is responsible. We are relying on this experience to convince the reader 
empirically that (4) represents a new order of language to an extent to which 
(5) and (2) do not. 

To make our point even clearer, let us introduce an operation, called 
"collapsing," which may be applied to propositions containing SPV 
proto-notation. The operation consists in redefining the SPV figure in a given 
proposition so that its assignments are the states of the original proposition. 
Let us collapse (4). We redefine 

\begin{equation*}
	\cubeframe \left\{\parbox{4in}{for 'You have deliberately vibrated (4)' if it appears to be oriented 
	like \cubeup\linebreak
	for 'You have not deliberately vibrated (4)' if it appears to be oriented 
	like \cubedown}\right\}
\end{equation*}

(4) now becomes 

\begin{quotation}
\cubeframe (4) 
\end{quotation}


We emphasize that the reader must actually read (4), for the effect is 
indescribable. The reader should learn the assignments with flash cards if 
necessary. 

The claim we want to make for (4) is probably that it is the most 
clear-cut case yet constructed in which thought becomes an object for itself. 
Just looking at a reversible perspective figure which is not a linguistic 
utterance---an approach which perceptual psychologists have already 
tried---does not yield results which are significant with respect to "thought." 
In order to obtain a significant case, the apparent orientation or imputed 
perspective must be a proposition; it must be true or false. Then, (5) and (2) 
are not highly significant, because the mental act of supplying the missing 
element of the proposition is all a matter of your volition; and because the 
element supplied is essentially an "understood meaning." We already have an 
abundance of understood meanings, but scientists have been able to ignore 
them because they are not "objectifiable." In short, reversible perspective by 
itself is not "thought"; equivocation by itself has no mental aspect which is 
objectifiable. Only in reading (4) do we experience an "objectifiable aspect 
of thought." We have invented an instance of thought (as opposed to 
perception) which can be accomodated in the ontology of the perceptual 
psychologist. 

