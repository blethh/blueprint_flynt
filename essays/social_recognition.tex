\chapter{On Social Recognition}

The most important tasks which the individual can undertake arise not 
from personal considerations but from the general conditions of society. The 
standards of accomplishment for these tasks are implicit in the tasks, and are 
objective in the sense that they can be applied without reference to public 
opinion. For example, given that humans express themselves in statements 
which are supposedly true or false, there arises a fundamental philosophical 
"problem of knowledge." Then, the fact that societies are organized in 
different ways at different times and places poses fundamental problems of 
"political" thought and action. Sometimes the most important task posed by 
the conditions of society is to invent a whole new activity. The origination 
of experimental science in Europe in the seventeenth century is an example. 
For lack of a better term, these tasks will be referred to as "fundamental 
tasks." 

The fact that a fundamental task is posed by the general conditions of 
society does not mean that public opinion will be aware of the task, or that 
the ruling class will commission someone to undertake it. It may well be that 
the first person to perceive the problem is the person who solves it; and 
public opinion may not catch up with him for decades or centuries. 

The person who devotes himself to a fundamental task is, more often 
than not, persecuted or ignored by society. Society puts up an immense 
resistance to solutions of fundamental problems, even when, as in the cases 
of Galois and Mendel, those solutions are politically innocuous. There is no 
evidence that this state of affairs is limited to some particular organization of 
society. Further, there are cases in which an objectively valid result is 
known, and yet apparently society can never adopt the result institutionally. 
Art is objectively inferior to brend, as I have shown, and yet all indications 
are that art will always be a major institution. The persecution of individuals 
who undertake fundamental tasks is an instance of a general human social 
irrationality which runs throughout history, from human sacrifice in ancient 
times to present-day war between communist countries. The conclusion is 
that for an individual to commit himself to a fundamental task tends to 
preclude social approval for his activities. 

Quite apart from the fundamental tasks which are posed by general 
social conditions, the ruling class needs a continual supply of new talent at 
all levels of society. At the lower levels, this supply is assured by the 
necessity of selling one's labor power in order to eat. At the higher levels of 
accomplishment, the ruling class assures itself of a continual supply of new 
talent by offering publicity or fame---social recognition---as a reward for 
accomplishing the tasks specified by the ruling class. Famous men such as 
Einstein are held up to children as examples of the proper relationship 
between the talented individual and society; and an international institution, 
the Nobel Prize, exists to implement this system of supplying talent. 
According to the doctrine, the individual has a duty to benefit society, to 
choose a task posed by the ruling class as his occupation. (His publicly 
known occupation is supposed to correspond to his real goals.) If he 
performs successfully, he will receive publicity as an indication that he is 
indeed benefiting society. 

Our analysis of fame is the opposite of that of Ben Vautier. Vautier 
asserts that the desire for personal publicity is an instinctive drive of human 
beings, and that the accumulation of publicity is a genuinely selfish act like 
the accumulation of food. In fact, Vautier goes so far as to make no 
distinction between what Gypsy Rose Lee and Lenin, for example, did to 
gain fame; and he assumes that a pacifist, for example, would welcome 
military honors equally as much as he would a peace award. We assert, on 
the contrary, that the desire for publicity is not instinctive; it is inculcated in 
the young so that the ruling class may have a continual supply of new talent 
to serve its purposes. The desire for publicity, far more than the desire for 
money, is establishment-serving more than self-serving. (We suggest that the 
principal reason why Vautier seeks publicity is not instinct, but economics. 
Vautier has no inherited source of income, and has never been trained for a 
profession. For him, the alternative to the art\slash publicity racket would be 
common labor. If he had the opportunity for a life of leisure, he might feel 
differently about publicity.) 

The issues which are raised here are extremely important for the person 
who perceives a fundamental task, because his sanity may depend on 
whether he understands the rationality of his motives for undertaking the 
task. He will already have been inculcated with the establishment's concepts 
of service and recognition, concepts which are epitomized in the image of 
Einstein's career. What we suggest is that it is vital to disabuse oneself of 
these concepts. To repeat, fundamental tasks are posed by the general 
conditions of society. Yet the individual who undertakes such a task will 
probably be persecuted or ignored. Given these circumstances, the doctrine 
that the individual has a duty to benefit society is a hypocritical fraud, an 
obscenity. For the individual to commit himself to a fundamental task tends 
to preclude social recognition for his activities; or, to reverse the remark, 
social recognition is not a reward to accomplishment of a fundamental task 
(just as military honors are not a reward to pacifism). Thus, it is not rational 
for the individual to undertake a fundamental task in order to gain fame. 

The motive for undertaking a fundamental task should be genuine 
selfishness. (We will continue our argument that the striving for fame is not 
genuinely selfish below.) The individual who perceives a fundamental task 
should undertake it for his private gratification. The task is of primary 
importance to society. By accomplishing it, the individual gains the privilege 
of knowing something which is socially important, but which society cannot 
deal with honestly. The individual should undertake the task in order to 
utilize his real abilities, to develop his potentiality for its own sake. The 
undertaking of a significant task which utilizes one's real abilities is the true 
source of happiness. To perceive a fundamental task and not to undertake it 
is to be stunted: one loses one's self-respect and becomes progressively 
demoralized. (Another rational motive for undertaking a fundamental task is 
to transform the social environment by methods which do not depend on 
society's approval or comprehension.) 

We do not mean to suggest that the individual who undertakes a 
fundamental task should conceal his results. Even though such tasks may 
seem individualistic, they require cooperative, social activity for their 
accomplishment. A proposed solution to a fundamental problem can hardly 
develop without being scrutinized from a variety of perspectives. It is 
essential to have qualified critics, and it is unfortunate that they are so rare. 
Solutions to fundamental problems are social consumption goods (their 
consumption is not exclusionary), so that critics or collaborators have as 
much opportunity to benefit from them as their originators do. As an 
example, most of my writings are really collaborations with Tony Conrad. I 
often find that I do not understand my own position until I know how it 
appears to him. When communication of results is essentially a form of 
collaboration, it is very different from the attempt to gain publicity or fame. 

It is precisely in the context of the generalized social irrationality which 
runs throughout history that the attempt to gain fame must be seen as 
foolishly un-selfish. What difference can it possibly make whether the masses 
venerate one's name a hundred years after one's death? The adulation of the 
masses after one is dead is of no conceivable value to oneself. It is society 
which indoctrinates one to worry about one's reputation after one is dead, in 
order to condition one to serve the interests of the ruling class. 

Then, what does it mean to the individual who solves a fundamental 
problem to have his name publicized in the mass media, to be a celebrity 
among people who cannot possibly understand what he has done? Even 
more important, we must recognize that publicity carries a definte risk for 
the individual committed to a fundamental task. The solution of such a 
problem must usually be expressed in categories which are incommensurate 
and incompatible with the categories of thought which are common coin at 
the time. In order for the solution of a fundamental problem to be exposed 
in the mass media, it has to be translated into media categories and this 
usually results in irreparable distortion. In fact, the solution is distorted in 
precisely such a manner that it begins to serve the interests of the ruling 
class. One encounters an immense pressure which tends to harness one to 
goals which have nothing to do with objective value. More precisely, when an 
individual who has solved a fundamental problem is publicized in the mass 
media, a process of mutual subversion takes place as between the 
establishment\slash media and the individual. In the process, the establishment is 
likely to come out far ahead. 

There are two other reasons why it is actually advantageous to the 
individual who undertakes a fundamental task to avoid publicity. Since one's 
activity is likely to be treated as a threat by society, one can minimize the 
energy required to defend it, and can carry the activity further, if one 
receives no publicity. Then, there will unavoidably be false starts made in 
developing the solution to a fundamental problem. If one is not operating in 
the glare of publicity, it is far easier to abandon these false starts. 

It used to be that when I saw publicity being given to an inferior way of 
doing a thing, and I knew a better way, then I reacted with a sense of duty. I 
had to appoint myself as a missionary, to enter the public arena and start a 
campaign to replace the inferior approach with the better approach. But this 
sense of duty must now be called into question. Is it really in my interest to. 
thrust myself on the media as a missionary? The truth is that in the context 
of generalized social irrationality, it is un-selfish and self-sacrificing to believe 
that I must either agree with current fads or else contest them publicly. The 
genuinely selfish attitude is *hat it is sufficient for me to know what the 
superior approach is. I can ignore the false issues which fill the mass media; I 
do not have to participate in public opinion at all. The genuinely selfish 
attitude is that "it does not concern me." Genuine selfishness is living one's 
life on a level which does not communicate with the level of the mass media 
and public opinion. 

If we recognize that it is irrational to undertake a fundamental task in 
order to benefit society and gain social approval, then our very choice of 
fundamental tasks shouid be affected. The most visible fundamental tasks 
are those which the establishment is to some extent aware of, and which if 
accomplished would immediately be rewarded with social approval. (In the 
natural sciences, there literally may be a race to solve a well-known problem). 
But if our motives are genuinely self-serving, and have to do with the 
development of our potentiality for its own sake, then there is no reason to 
limit ourselves to widely understood problems. We can undertake to discover 
timeless results---permanent answers to questions which will be important 
indefinitely---without concerning ourselves with whether society can adopt 
the results institutionally. We can pose problems of which neither the 
establishment, the media, nor public opinion are aware. We can undertake 
tasks which draw on our unique abilities, so that our personal contribution is 
indispensable. 

There is a difficulty which we have postponed mentioning. The 
individual is always compelled to engage in some socially approved activity 
in order to obtain the means of subsistence. We cannot assume that the 
individual will have an inherited source of income. In order to pursue a 
fundamental task, he will have to pursue a legitimate occupation at the same 
time. It may be extremely difficult to lead such a double life, because to do 
so requires precisely the self-assurance. that comes from accomplishing the 
fundamental task. Leading a double life is not a game for the person who is 
unsure about his real abilities or his vocation. If the individual is capable of 
leading a double life, our suggestion is to obtain the means of subsistence by 
the most efficient swindle available. Do not hesitate to practice outward 
conformity in order to exploit the establishment for your own purposes. 

There remains the case of the individual who, like Galois, is not 
prepared to lead a double life. His problem is one of destitution. However, 
he is different from an ordinary pauper. By assumption, he is more talented 
than the members of the establishment; he does not belong to the 
establishment because he is overqualified for it. Given that he is more 
talented than members of the establishment, and that his survival is 
threatened, a collateral fundamental task emerges, the task of immediately 
transmuting his talent into power to handle the establishment on his own 
terms. To perceive this task is a major resuit of this essay. The task cannot be 
defined accurately without a perfect understanding of the difference 
between fundamental tasks and the serve-society-and-get-famous fraud. We 
contend that Galois should have regarded the task of immediately 
transmuting his talent into power over the establishment as an inseparable 
collateral problem to his mathematical researches. From a common sense 
point of view, this collateral task will seem utterly impossible. However, we 
are talking about individuals whose vocation is to do the seemingly 
impossible. Thus, we conclude by leaving this unsolved fundamental problem 
for the reader to ponder. 

