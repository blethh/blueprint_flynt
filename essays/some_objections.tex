\chapter{Some Objections to My Philosophy}


\begin{enumerate}[label=\textbf{\Alph*.}, wide, nosep, itemsep=1em]
\item The predominant attitude toward philosophical questions in 
educated circles today derives from the later Wittgenstein. Consider the 
philosopher's question of whether other people have minds. The 
Wittgensteinian attitude is that in ordinary usage, statements which imply 
that other people have minds are not problematic. Everybody knows that 
other people have minds. To doubt that other people have minds, as a 
philosopher might do, is simply to misuse ordinary language.\footnote{See 
\booktitle{Philosophical Investigations}, \S 420.} Statements which imply that other 
people have minds works perfectly well in the context for which they were 
intended. When philosophers find these statements problematic, it is because 
they subject the statements to criticism by logical standards which are 
irrelevant and extraneous to ordinary usage.\footnote{\S \S 402, 412, 119, 116.}

For Wittgenstein, the existence of God, immortal souls, other minds, 
and the \textsc{Empire State Building} (when I am not looking at it) are all things 
which everybody knows; things which it is impossible to doubt \enquote{in a real 
case.}\footnote{\S 303, Iliv. For Wittgenstein's theism, see Norman Malcolm's 
memoir.} The proper use of language admits of no alternative to belief in 
God; atheism is just a mistake in the use of language. 

In arguing against Wittgenstein, I will concentrate on the real reason 
why I oppose him, rather than on less fundamental technical issues. We read 
that in the Middle Ages, people found it impossible not to believe that they 
would be struck by lightning if they uttered a blasphemy; just as 
Wittgenstein finds the existence of God impossible to doubt \enquote{in a real case.} 
Yet even Wittgenstein does not defend the former belief; while the Soviet 
Union has shown that a government can function which has repudiated the 
latter belief. There is a tremendous discovery here: that beliefs which were as 
inescapable---as impossible to doubt in a real case---as any belief we may have 
today, were subsequently discarded. How was this possible? My essay \essaytitle{The 
Flaws Underlying Beliefs} shows how. Further, it shows that the belief that 
the \textsc{Empire State Building} exists when I am not looking at it, or the belief 
that I would be killed if I jumped out of a tenth story window, are no 
different in principle from beliefs which we have already discarded. It is 
perfectly possible to project a metaphysical outlook on experience which is 
totally different from the beliefs Wittgenstein inherited, and it is also 
possible not to project a metaphysical outlook on experience at all. Let us be 
absolutely clear: the point is not that we do not know with one hundred per 
cent certainty that the \textsc{Empire State Building} exists; the point is that we 
need not believe in the \textsc{Empire State Building} at all. \essaytitle{The Flaws Underlying 
Beliefs} shows that factual propositions, and the propositions of the natural 
sciences, involve outright self-deception. 

These discoveries have consequences far more important than the 
technical issues involved. It is by no means trivial that I do not have to pray, 
or to fast, or to accept the moral dictates of the clergy, or to give money to 
the Church. Because the Church prohibited the dissection of human 
cadavers, it took an atheist to originate the modern subject of anatomy. In 
analogy with this example, the rest of my writings are devoted to exploring 
the consequences of rejecting beliefs that Wittgenstein says are impossible to 
doubt in a real case, as in my essay \essaytitle{Philosophical Aspects of Walking 
Through Walls.} I oppose Wittgenstein because he descended to extremes of 
intellectual dishonesty in order to prevent us from discovering these 
consequences. 

A reply to the Wittgensteinian attitude which is technically adequate 
can be provided in short order, for when Wittgenstein's central philosophical 
maneuver is identified, its dishonesty becomes transparent. It is not 
necessary to enumerate the fallacies in the Wittgensteinian claim that logical 
connections and logical standards are extrinsic to the natural language, or in 
the aphorism that \enquote{the meaning is the use} (as an explication of the natural 
language). In other words, there is no reason why I should bandy descriptive 
linguistics with Wittgenstein. Wittgenstein was wrong at a level more basic 
than the level on which his philosophical discussions were conducted. 

Wittgenstein held that philosophical or metaphysical controversies 
literally would not arise if it were not for bad philosophers. They would not 
arise because there is nothing problematic about sentences, expressing 
Wittgenstein's inherited beliefs, in ordinary usage. This rhetorical maneuver 
is the inverse of what it seems to be. Wittgenstein doesn't prove that the 
paradoxes uncovered by \enquote{bad} philosophers result from a misuse of ordinary 
language; he defines the philosophers' discussions as a misuse of ordinary 
language because they uncover paradoxes in ordinary language propositions. 
Wittgenstein waits to see whether a philosopher uncovers problems in 
ordinary language propositions; and if the philosopher does so, then 
Wittgenstein defines his discussion as improper usage. Wittgenstein waits to 
see whether evidence is against his side, and if it is, he defines it as 
inadmissible. 

Consider the philosopher's question of how I know whether the \textsc{Empire 
State Building} continues to exist when I am not looking at it. The 
Wittgensteinian position on this question would be that it is problematic 
because it is a misuse of ordinary language; and because there is no 
behavioral context which constitutes a use for the question. According to 
this position, we would not encounter such problems if we would use 
ordinary language properly. But what does this position amount to? The 
philosopher's question has not been proved improper; it has been defined as 
improper because it leads to problems. The reason why \enquote{the proper use of 
ordinary language never leads to paradoxes} is that Wittgenstein has defined 
proper use as use in which no paradoxes are visible. Wittgenstein has not 
resolved or eliminated any problems; he has just refused to notice them. 
Wittgenstein attempts to pass off, as a discovery about philosophy and 
language, a gratuitous definition to the effect that certain portions of the 
natural language which embarrass him are inadmissible, a gratuitous ban on 
certain portions of the natural language which embarrass him. His purpose is 
to make criticism of his inherited beliefs impossible, to give them a spurious 
inescapability. Wittgenstein's maneuver is the last word in modish 
intellectual dishonesty. 

\item In philosophy, arguments which start from an immediate which 
cannot be doubted and attempt to prove the existence of an objective reality 
are called transcendental arguments. Typically, such an argument says that if 
there is experience, there must be subject and object in experience; if there 
are subject and object, subject and object must be objectively real; and thus 
there must be objectively real mind and matter. Clearly, the belief which 
leaps the gap from the immediate to the objectively real is smuggled into the 
middle of the argument by a play on the words \enquote{subject} and \enquote{object.} 

When the sophistry is cleared away, it becomes apparent that the 
attempt to attain the trans-experiential or extra-experiential within 
experience faces a dilemma of overkill. If the attempt could succeed, it 
would have only collapsed objective reality to my subjectivity. If it could be 
\enquote{proved} that I know the distant past, other minds, God, angels, archangels, 
etc. from immediate experience, then all these phenomena would be 
trivialized. If other minds were given in my experience, they would only be 
my mind. The interest of the notion of objective reality is precisely its 
otherness and unreachability. If it could be reached from the immediate, it 
would be trivial. We ask how I know that the \textsc{Empire State Building} exists 
when I am not looking at it. If the answer is that I know through immediate 
experience, then objective reality has been collapsed to my subjectivity. The 
dilemma for transcendental arguments is that they propose to overcome the 
gap between the appearance of a thing and the thing itself, yet they do not 
want to conclude that appearances exhaust reality. 

There are two special assumptions which are smuggled into supposedly 
assumptionless transcendental arguments. First, there is the belief that there 
is an objective relationship between descriptive words and the things they 
describe, an objective criterion of the use of descriptive words. Secondly, 
there is the belief that correlations between the senses have an objective 
basis. (It is claimed that this belief cannot be doubted, but the claim is 
controverted by intersensory illusions such as the touching of a pencil with 
crossed fingers.) 

Transcendental arguments are secular theology, because they are 
addressed to a reader who wants only philosophical analyses that have 
conventional conclusions. A transcendental argument will contain a step 
such as the following, for example. We can have \enquote{real knowledge} of 
particular things only if there is an objective relationship between descriptive 
words and the things they describe; thus there must be such a relationship. 
This argument is plausible only if the reader can be trusted to overlook the 
alternative that we don't have this \enquote{real knowledge.} 

In the way of supplementary remarks, we may mention that 
transcendental arguments typically commit the ontological fallacy: inferring 
the existence of a thing from the idea or name of the thing. Finally, 
transcendental arguments share a confusion which originates in the 
empiricism they are directed against: the confusion between doing 
fundamental philosophy and doing the psychology of perception. Many 
transcendental arguments are similar to current doctrines in scientific 
psychology. But they fail as philosophy, because scientific psychology takes 
as presuppositions, and cannot prove, the very beliefs which transcendental 
arguments are supposed to prove. 

\end{enumerate}