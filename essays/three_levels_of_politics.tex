\chapter{The Three Levels of Politics}


Political activity and its results can occur on three levels. The first level 
is the personal one. An individual may vote to re-elect a local politician 
because of patronage he has received, for example. On this level the 
individual's motivation is narrow, immediate self-interest. Often the action 
has a defensive character; the individual is trying to hold on to something he 
already possesses. 

The second level may be called the historical level. It is exemplified by 
the Civil War in the United States. Certain political movements result in 
largescale, irreversible social change. The Civil War set in motion the 
industrialization of the United States, as well as abolishing slavery. In 1860, 
slavery was viewed by large numbers of Americans as a legitimate institution. 
One hundred years later, even American conservatives did not often defend 
it. To re-establish a plantation economy in the South today would be out of 
the question. These observations prove that on the second level, society 
really does change. On this level, political action does make a difference. 

However, there is a further aspect to the Civil War which indicates that 
politics does not make the difference people think it makes. According to 
the ideology of the abolitionists, the accomplishment of the Civil War would 
be to raise the slaves to a position of equality with whites. In fact, nothing of 
the sort happened. The real accomplishment of the Civil War was to 
transform the United States into an industrial capitalist society (and to 
abolish an institution which was incompatible with the capitalists' need for a 
free labor market). By the time the Northern businessmen brought 
Reconstruction to an end, it was clear that the position of blacks in 
American society was where it had always been: at the bottom. The Civil 
War changed American society, but is did not make the society any more 
utopian. On the contrary, it brought into prominence still another violent 
social conflict---the conflict between labor and capital. 

The third level of politics has to do with the utopian aspect of modern 
political ideologies, the aspect which calls not only for society to change, but 
to change for the better. Typical third-level political goals are the abolition 
of war, the abolition of the oligarchic structure of society, and the abolition 
of economic institutions which value human lives in terms of money. in all 
of human history, society has never changed on this third level. 

The successful Communist revolutionists of the twentieth century (in 
the underdeveloped countries) have repeatedly claimed to have accomplished 
third-level change in their societies. However, these claims of third-level 
change have always turned out to be illusions which cover a recapitulation of 
capitalist development. Communist revolutions are typical examples of real 
second-level change which is accomplished under the cover of claims of 
third-level change, claims which are pure and simple frauds. 

By introducing the concept of levels of politics, we can resolve the 
apparent paradox that society certainly changes, but that it really does not 
change. It is important to understand that empirical evidence on the 
question of the levels of politics can only be drawn from the past, the 
present, and the immediate future (five to ten years). Recent technological 
developments have brought into question the very existence of the human 
species. In addition, technology is developing much faster than society is. It 
is meaningless to discuss the issue of second versus third-level social change 
with reference to the more distant future, because there may not be any 
human society in the more distant future. 

This essay is concerned with the politics of the third level. The first and 
second levels are certainly real enough, but we are not the least interested in 
them. As we have just said, we make the restriction that any empirical 
analysis of the third level must refer to the past, the present, or the 
immediate future. Our purpose is to present a substitute for the politics of 
the third level. 

There are a number of present-day political tendencies which hold out 
the promise of third-level social change. These tendencies are all descended 
from the leftist working-class movements of nineteenth century Europe, 
most of them by way of the early Soviet regime. The promises of third-level 
change held out by these tendencies are nothing but cheap illusions. What is 
more, a careful examination of leftist ideologies in relation to the historical 
record will show that the promises of third-level change are extremely vague 
and without substance. Beneath the surface of vague promises, leftist 
ideologies do not even favor third-level change; they are opposed to it. 

One example will serve to demonstrate this contention. In my capacity 
as a professional economist, I have become familiar with the official 
economic policies---the doctrines of the professional economists---of the 
various socialist governments and leftist movements throughout the world. It 
should be mentioned that most of the followers of leftism are not familiar 
with these technical economic policies; they are aware only of vague, 
meaningless promises of future bliss coming from leftist political 
speechmakers. When we turn to technical economic realities, we find that 
virtually every leftist tendency in the world today accepts economic 
principles which in the parlance of the layman are referred to as 
"capitalism." The most important principle is stated by Ernest Mandel: "the 
economy continues to be fundamentally a money economy, with the 
satisfaction of the bulk of people's needs depending on the number of 
currency tokens a person possesses." When it comes to the realities of 
technical economics, virtually every leftist in the world accepts this 
principle. So far as the third level is concerned, there is no such thing as a 
non-capitalist polical tendency, and there is no point in hoping for one. A 
similar conclusion holds for virtually every aspect of third-level politics. 
Leftists claim that Communism eliminates the causes of war; while at the 
same time war breaks out beween China and the Soviet Union. 

We propose to draw a far-reaching conclusion from these 
considerations. Returning to the example of first-level politics, it is rational 
for the patronage-seeker to be in favor of the election of one focal politican 
and against the election of his opponent. This is a matter which is within the 
scope of human responsibility, and with respect to which individual action 
can make a difference. But it is not rational to be either for against 
"capitalism," to be either for or against war. As we have seen, "capitalism" 
and war are permanent aspects of human society, and no political tendency 
genuinely opposes them. It is meaningless to treat them as if they were 
within the scope of human responsibility in the sense that the election of a 
local politician is. in other words, the third-level aspects of society are not 
partial, limited aspects which can be eliminated by conscious human action 
while the bulk of human life is retained. The only way you can meaningfully 
be against the third-level aspects of human society is by adopting a different 
attitude to the human species as such. 

This attitude is the one you would adopt if you were suddenly thrown 
into a society of apes---apes which perpetually preyed within their own 
ecological niche. It is clear that if you proposed to be "against" such a 
situation, and to do something about it, then politics as it is normally 
conceived would be out of the question. To anticipate our later discussion, 
the first thing you must do is to protect yourself against society. The way to 
do this is to create an invisible enclave for yourself within the Establishment. 
Having such an enclave certainly does not imply loyalty to the 
Establishment. On the contrary, there is no reason why you should be toyal 
to any faction among the apes. You only pretend to be loyal to one faction 
or another when it is necessary for self-defense. If there is a change of regime 
in the country where you are living, you either leave or join the winning side. 
Transfer your invisible enclave to whatever Establishment is available. But all 
this is an external, defensive tactic which has nothing to do with the primary 
goals of our strategy. 

We will finish our critique of third-level politics, and then continue the 
description of the substitute which we propose. In addition to making vague 
promises of third-level change, leftism encourages indignation at social 
conditions which are beyond anyone's power to affect. Leftism attributes 
great ethical merit to such indignation and morally condemns anyone who 
does not share it. But this attitude is totally irrational and dishonest. In 
philosophy and mathematics, it is possible for a proposition to be valid even 
though it has no chance of institutional acceptance. But in social, economic, 
and political matters, attitudes which have policy implications are nonsense 
unless the policies are actually implemented. Institutional acceptance is the 
only arena of validation of a social doctrine. It is absurd to attribute ethical 
merit to a longing for the impossible. Indignation at a social condition which 
is beyond anyone's power to affect is meaningless. (Indeed, to the extent 
that such indignation diverts social energy into a dead end, it is 
"counter-revolutionary.") To be more radical in social matters than society 
can possibly be is not virtuous; it is idiotic. 

Although third-level politics is a fraud, it is the contention of this essay 
that there exists a rational substitute for it. Once you perceive that you exist 
in a society of apes who attack their own ecological niche, there are rational 
goals which you can adopt for your life that correspond to third-level change 
even though they have nothing to do with leftism. The preliminary step, as 
we have said, is to create an invisible enclave for yourself within. the 
Establishment. The remainder of the strategy is in two parts which are in 
fact closely related. 

The first part is based on a consideration of the effects which such 
figures as Galileo, Galois, Abel, Lobachevski, and Mendel have had on 
society. These men devoted themselves to researches which seemed to be 
purely abstract, without any relevance to the practical world. Yet, through 
long, tortuous chains of events, their researches have had disruptive effects 
on society which go far beyond the effects of most political movements. The 
reason has to do with the peculiar role which technology has in human 
society. Society's attitude in relation to technology is like that of a child 
who cannot refrain from playing with matches. We find that 
the abstract researches of the men being considered accomplished a dual 
result. On the one hand, they represented inner escape, the achievement of a 
private utopia now. Of course, the general public will not understand this; 
only the few who are capable of participating in such activities will 
appreciate the extent to which they can constitute inner escape. On the 
other hand, they have had profoundly disruptive effects on society, effects 
which still have not run their course. 

Thus, the first part of our strategy is to follow the example of these 
individuals. Of course, we do not stay within the bounds of present-day 
academic research, any more than Galileo or Mendel did in their time. What 
we have in mind is activities in the intellectual modality represented by the 
rest of this book. 

It should be clear that such activities do represent a private utopia, and are at 
the same time the seeds of disruptive future technologies which lead directly 
to the second part of our strategy. 

It is important to realize that by speaking of inner escape we do not 
mean fashionable drug use, or Eastern religions, or occultism. These 
threadbare superstitions are embraced by the cosmopolitan middle 
classes---intellectually spineless fools who are always grasping for spiritual 
comfort. Superstitious fads are escapism in the worst sense, as they only 
serve to further muddle the heads of the fools who embrace them. In 
contrast, the inner escape which we propose is original and consequential, 
leading to an increase in man's manipulative power over the world. It has 
nothing to do with irrationality or superstition. 

The second part of our strategy is predicated on the following states of 
affairs. First, it is the human species as such which is the obstacle to 
third-level political change. Secondly, technology is developing far more 
rapidly than society is, and no feature of the natural world need any longer 
be taken for granted. Society cannot help but foster technology in the 
pursuit of military and economic supremacy, and this includes technology 
which can contribute to the making of artificial superhuman beings. Every 
fundamental advance in logic, physics, neurophysiology, and 
neurocybernetics obviously leads in this direction. Thus, the second part of 
the strategy is to participate in the making of artificial superhumans, 
possibly by infiltrating the military-scientific establishment and diverting 
research in the appropriate direction. 

{ \itshape
Note: This essay provides a specific, practical strategy for the present 
environment. It also shows that certain types of opposition to the status quo 
are meaningless. Subversion Theory, on the other hand, was a general theory 
which was not limited to any one environment, but also which failed to 
provide a specific strategy for the present environment. \par }


