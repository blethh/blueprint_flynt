\chapter{Philosophical Aspects of Walking Through Walls}


We read that in the Middle Ages, people found it impossible not to 
believe that they would be struck by lightning if they uttered a blasphemy. 
Yet I utterly disbelieve that I will be struck by lightning if I utter a 
blasphemy. Beliefs such as the one at issue here will be called fearful beliefs. 
Elsewhere, I have argued that all beliefs are self-deceiving. I have also 
observed that there are often non-cognitive motives for holding beliefs, so 
that a technical, analytical demonstration that a belief is self-deceiving will 
not necessarily provide a sufficient motive for renouncing it. The question 
then arises as to why people would hold fearful beliefs. It would seem that 
people would readily repudiate beliefs such as the one about blasphemy as 
soon as there was any reason to doubt them, even if the reason was abstract 
and technical. Yet fearful beliefs are held more tenaciously than any others. 
Further, when philosophers seek examples of beliefs which one cannot 
afford to give up, beliefs which are not mere social conventions, beliefs 
which are truly objective, they invariably choose fearful beliefs. 

Fearful beliefs raise some subtle questions about the character of beliefs 
as mental acts. If I contemplate blasphemy, experience a strong fear, and 
decide not to blaspheme, do I stand convicted of believing that I will be 
punished if I blaspheme, or may I claim that I was following an emotional 
preference which did not involve any belief? Is there a distinction between 
fearful avoidance and fearful belief? Can the emotion of fear be 
self-deceiving in and of itself? Must a belief have a verbal, propositional 
formulation, or is it possible to have a belief with no linguistic representation 
whatever? 

It is apparent that fearful beliefs suggest many topics for speculation. 
This essay, however, will concentrate exclusively on one topic, which is by 
far the most important. Given that people once held the belief about 
blasphemy, and that I do not, then I have succeeded in dispensing with a 
fearful belief. Two beliefs which are exactly analogous to the one about 
blasphemy are the belief that if I jump out of a tenth story window I will be 
hurt, and the belief that if I attempt to walk through a wall I will bruise 
myself. Given that I am able to dispense with the belief about blasphemy, it 
follows that, in effect, I am able to walk through walls relative to medieval 
people. That is, my ability to blaspheme without being struck by lightning 
would be as unimaginable to them as the ability to walk through walls is 
today. The topic of this essay is whether it is possible to transfer my 
achievement concerning blasphemy to other fearful beliefs. 

\visbreak

I am told that \enquote{if you jump out of a tenth story window you really will 
be hurt.} Yet the analogous exhortation concerning blasphemy is not 
convincing or compelling at all. Why not? I suggest that the nature of the 
"evidence" implied in the exhortation should be examined very closely to 
see if it does not represent an epistemological swindle. In the cases of both 
blasphemy and jumping out of the window, I am told that if I perform the 
action I will suffer injury. But do I concede that I have to blaspheme, in 
order to prove that I can get away with it? Actually, I do not blaspheme; I 
simply do not perform the action at all. Yet I do not have any belief 
whatever that it would be dangerous to do so. Why should anyone suppose 
that because I do not believe something, I have to run out in the street, 
shake my fist at the sky, and curse God in order to validate may disbelief? 
Why should the credulous person be able to put me in in the position of 
having to accept the dare that "you have to do it to prove you don't believe 
it's dangerous"? Could it not be that this dare is some sort of a swindle? 
The structure of the evidence for the supposedly unrelinquishable belief 
should be examined very closely to see if it is not so much legerdemain. 

The exhortation continues to the effect that if I did utter blasphemy I 
really would be struck by lightning. I still do not find this compelling. But 
suppose that I do see someone utter a blasphemy and get struck by lightning. 
Surely this must convert me. But with due apologies to the faithful, I must 
report that it does not. There is no reason why it should make me believe. I 
do not believe that blaspheming will cause me to be struck by lightning, and 
the evocation of frightful images---or for that matter, something that I 
see---would provide no reason whatever for sudden credulity. There is an 
immense difference between seeing a person blaspheme and get struck by 
lightning, and believing that if one blasphemes, one will get struck by 
lightning. This difference should be quite apparent to one who does not hold 
the belief.\footnote{In more conventional terms, the civilization in which I tive is so 
profoundly secular that its secularism cannot be demolished by one 
"sighting."}

In general, the so-called evidence doesn't work. There is a swindle 
somewhere in the evidence that is supposed to make me accept the fearful 
belief. Upon close scrutiny, each bit of evidence misses the target. Yet the 
whole conglomeration of "evidence" somehow overwhelmed medieval 
people. They had to believe something that I do not believe. I can get away 
with something that they could not get away with. 

It is not that I stand up in a society of the faithful and suddenly 
blaspheme. It is rather that the whole medieval cognitive orientation had 
been completely reoriented by the time it was transmitted to me. Or in other 
words, the medieval cognitive orientation was restructured throughout 
during the modern era. In the process, the compelling conglomeration of 
evidence was disintegrated. Isolated from their niches in the old orientation, 
the bits of evidence no longer worked. Each bit missed the target. I do not 
have a head-on confrontation with the medieval impossibility of 
blaspheming. I slip by the impossibility, where they could not, because I 
structure the entire situation, and the evidence, differently. 

The analysis just presented, combined with analyses of beliefs which I 
have made elsewhere, assures me that the belief that "if I try to walk 
through the wall I will fail and will bruise myself" is also discardable. I am 
sure that I can walk through walls just as successfully as I can blaspheme. 
But to do so will not be trivial. As I have shown, escaping the power of a 
fearful belief is not a matter of head-on confrontation, but of restructuring 
the entire situation, of restructuring evidence, so that the conglomeration of 
evidence is disintegrated into isolated bits which are separately powerless. 
Only then can one slip by the impossibility. I cannot exercise my freedom to 
walk through walls until the whole cognitive orientation of the modern era is 
restructured throughout. 

The project of restructuring the modern cognitive orientation is a vast 
one. The natural sciences must certainly be dismantled. In this connection it 
is appropriate to make a criticism about the logic of science as Carnap 
rationalized it. Carnap considered a proposition meaningful if it had any 
empirically verifiable proposition as an implication. But consider an 
appropriate ensemble of scientific propositions in good standing, and 
conceive of it as a conjunction of an infinite number of propositions about 
single events (what Carnap called protocol-sentences). Only a very small 
number of the latter propositions are indeed subject to verification. If we 
sever them from the entire conjunction, what remains is as effectively 
blocked from verification as the propositions which Carnap rejected as 
meaningless. This criticism of science is not a mere technical exercise. A 
scientific proposition is a fabrication which amalgamates a few trivially 
testable meanings with an infinite number of untestable meanings and 
inveigles us to accept the whole conglomeration at once. It is apparent at the 
very beginning of \booktitle{Philosophy and Logical Syntax} that Carnap recognized this 
quite clearly; but it did not occur to him to do anything about it. For us, 
however, it is essential to be assured that science can be dismantled just as 
the proof can be dismantled that I will be struck by lightning if I blaspheme. 

We can suggest some other approaches which may contribute to 
overcoming the modern cognitive orientation. The habitual correlation of 
the realm of sight and the realm of touch which occurs when we perceive 
"objects" is a likely candidate for dismantling.\footnote{The psychological jargon for 
this correlation is "the contribution of intermodal organization to the 
object Gestalt."}

From a different traditon, the critique of scientific fact and of 
measurable time which is suggested in Luk\'{a}cs' \booktitle{Reification and the 
Consciousness of the Proletariat} might be of value if it were developed.\footnote{Lulkacs also implied that scientific truth would disappear in a communist 
society---that is, a society without necessary labor, in which the right to 
subsistence was unconditional. He implied that scientific quantification and 
facticity are closely connected with the work discipline required by the 
capitalist mode of production; and that like the price system, they constitute 
a false objectivity which we accept because the social economic institutions 
deprive us of subsistence if we fail to submit to them. Quite aside from the 
historical unlikelihood of a communist society, this suggestion might be 
pursued as a thought experiment to obtain a more detailed characterization 
of the hypothetical post-scientific outlook.}

Finally, I may mention that most of my own writings are offered as 
fragmentary beginnings in the project of dismantling the modern cognitive 
orientation. 

Someday we will realize that we were always free to walk through 
walls. But we could not exercise this freedom because we structured the 
whole situation, and the evidence, in an enslaving way. 

