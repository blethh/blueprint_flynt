\chapter{Anthology of Non-Philosophical Cultural Works (1961)}

\section{Introduction}

I cannot include here my essays which discuss at length 
the purposes of lingart, audart, strange culture description,
concept art, and so forth, tell what they aro good for. However,
each of the works in this anthology, if considered 
without rigid preconceptions, is pretty obviously good for 
only one thing. (For example, the value of "Innperseqs" could 
hardly be that it has a deep emotional impact, or of "Audart 
Composition" that it gives the experience of awareness of an 
aesthetic complex structure.) Thus, the lack of explanation 
need not prevent the reader from understanding the works.
In fact, the lack of explanation mill be a good thing if it 
leads the reader to decide for himself what the value of the 
works is, rather than depending on their being in a tradition,
or having acadmic certification, or having a standard label
(such as 'music' or 'mathematics') to tell  him. These works
stand by themselves. 

\clearpage 

\section{Lingart: Poem 1 (early 1960 / August 1961)}
{\vskip 2em \centering [Instructions \vskip 2em}

Any lines (a line may be repeated) may be read, in any 
order. A line must be either entirely normally voiced or 
entirely whispered. Two lines may be read "at once", one 
normally voiced, the other whispered. That is, the words of 
one line are inserted between the words of the other; the only 
restriction is that the order of the words from each line must
be preserved. The two lines should be clearly distinguished 
in reading, each being given its proper phonemic intonation 
(light fall at end of sentence, and so forth). Example:
`(w)Mild (n)Oceans (n)prayed (n)sleepily (w)steel (n)in (w)dissolves (n)weary (w)in (w)eats. 
(n)cheeses.', or, in a bettor notatien, 

\setlength\tabcolsep{0.1em}
\begin{tabular}{ r c c c c c c c c c c c }
whispered: & Mild & & & & steel & & dissolves & & in & oats. & \\
normal: & & Oceans & prayed & sleepily & & in & weary & & cheeses. \\
\end{tabular}

Within these limitations, the lines should be read in any may 
which will maximally \uline{bring out their content} (for ex, slowly).] 

\vskip 2em

\noindent
Bitter carbon was moons. \\
Vinegar crushes giant monsters. \\
Moons are viciously decaying cream. \\
Frozen cream erratically crushes eats. \\
Carbon crushes giant moons. \\
Black vinegar sleepily was nights. \\
Black cheese loudly was oceans. \\
Empty sound dissolved cheese. \\
Bitter night is vinegar. \\
Vinegar quiets giant silences. \\
Steel is erratically decaying oceans. \\
Scattered sounds sullenly lay in oceans. \\
Mild steel dissolves in cats. \\
Nights sullenly mocked green monsters. \\
Oceans bled precisely in weary sound. \\
Mild monsters opened in sound. \\
Scattered cheese viciously fell in cats. \\
Silences were erratically decaying carbon. \\
Carbon was sleepily decaying silence. \\
Cats bled in screaming cheeses. \\
Sounds were blue monsters. \\
Sounds clearly grasped green oceans. \\
Monsters bled in weary nights. \\
Cats endlessly burned green steel. \\
Oceans were blue sounds.\\
Empty oceans crush steel.\\
Scattered silences killed analytically in vinegar.\\
Frozen vinegar dulcetly crushed night. \\
Cheeses clearly crush green sounds. \\
Oceans prayed sleepily in weary cheeses.\\
Frozen nights sleepily mocked vinegar.\\
Moons open blindly in weary carbon.\\
Mild cats pray in cream.\\
Empty cream quiets moons.\\
Monsters prayed in screaming vinegar.\\
Bitter moons were carbon,\\

\clearpage
\section{Audart Composition (May 1961)}

To experience this composition, one must be alone in a
quiet, darkened room. Relax, and accustom oneself to breathing
slowly so that one's breathing will be as quiet as possible.
Then put one's fingers in one's ears and close one's eyes.
Listen to the very low sound (subsonic vibration) and the medium
high---high noise (the sound of one's nervous system in operation),
and "look" at the changing pattern of light and dark.
At first the low sound will be the easiest one to hear, then
the high one. Try to have maximum awareness of both sounds
until the low one becomes difficult to hear. Then concentrate
on the high one, being aware that it is the sound of oneself
listening to it (and of oneself being aware that it is the 
sound of oneself listening to it). It is unlikely that the
visual aspect of this composition will be very interesting,
but if it is, as sometimes happens if one has looked at a complex
array of bright lights before starting, the way to appreciate 
it is to seize on images as soon as they appear and concentrate
to bring them cut. If done properly this should be a very strange experience, 

\clearpage
\section{Audart: A way of enjoying a Non-Controlled Acoustical Environment (July 1961)}
Let me distinguish what I will say, for want of better
terms, are "highly select sounds", such as popular music and 
indistinct talking (like a radio in an adjacent room), as 
opposed to "highly unselect sounds", such as the sounds of a
busy street; and "non-controlled sounds", sounds not produced
for one to listen to. Then this activity is a way, of enjoying
an acoustical environment, developed as an alternate to
listening to highly select music when none is available---a
way of enjoying a non-controlled environment. In general what
one does is to listen to the environmental sounds, and by
selective listening and the addition of "sounds heard in one's
head", imagined-sounds, "form them into" highly select sounds;
so that one seems to be hearing, from outside, from the non-mental
environment, highly select sounds and not just the non-controlled
sounds. It would seem that to some extent this 
activity is an exercise in autosuggestion. It can be done more
easily the more unselect the environmental sounds are. One way 
of learning to do it is to have highly select music on; then 
turn it down so that it is almost or just covered by the rest
of the (non-controlled) environmental sounds, but keep on 
"hearing" it, filling in with your imagination when you can't
quite hear it, "hearing" a continuation of what it was when 
clearly audible, modified whenever necessary to fit in with 
the environmental sounds; then cut it off entirely but go on 
"hearing" it in(to) the non-mental environmental sounds. It 
should be clear that this activity is not just listening to the
environmental sounds, and is not just "hearing music in one's 
head". The music is made from the environmental sounds with
imaginings, and seems to be part of the non-mental environmental
sounds, come from the non-mental environment. 

\clearpage
\section{Strange Culture Description: Instructions Accompanying Two Identity Structure Standards (April--May 1961)}

These standards are for determining the type of identity 
(continuity) structure of an exercise awareness-state in a
conscious organism, and thus determining whether the awareness-state
can be annihilated. There are two standards, corresponding
to the two types of identity structure: enclosure\footnote{A rectangular $6\times9$ inch envelope made of tracing paper open at one end. Inside, a rectangular $3x5$ inch sheet of white paper.} and 
organic--rotation\footnote{A blank white $3x5$ inch card.} (the terms are somewhat arbitrary, do not
essentially characterize the structures). If the awareness-state
has enclosure structure, it can be annihilated; if it 
has organic-rotation structure, it cannot be. 

Use of the standards: An entity has an identity structure
of one of the two types if its identity structure is observed 
to be isomorphic to that of the standard for that type for one
hour, and if it is not deflected before or after that hour. 
It must be emphasized that determination of identity structure
type involves use of the actual standard (in non-ruined condition)
accompanying these instructions through the actual hour
in which the determination is made. The specification of the
conditions under which the standard is ruined which will be 
given does not define the standard, for the purpose of determining
identity structure type, as anything which is not ruined. 

Care of the standards: With respect to the enclosure 
structure standard, if the sheet of paper inside the envelope
is turned over in the envelope, or is allowed to project out
of the open and of the envelope, the standard is \uline{ruined}. Thus, 
it is advised that if the shoot is shaken around in the envelope,
a hand or other surface be held against the open end to prevent
the sheet from coming out. With respect to the organic-rotation 
structure standard, from the time the standard comes into one's
possession, it must be turned over once a day or it is \uline{ruined}.
The exception is that if one can make it into a strip by
joining opposite edges of it, without tearing its fibers, that
is, tearing or creasing it, then it does not have to be turned
over, and it is not ruined. If the standard is heated until it
becomes a gas, it is \uline{ruined}. If the standard is transmuted into 
a gold leaf by nuclear bombardment, it is \uline{ruined}. It is advised
that persons who do not know all this or are careless or destructive
not be allowed to handle the standards.

\clearpage
\section{Concept Art: Work Such That No One Knows What's Going On (July 1961)}

[Ono just has to guess whether this work exists and if it does what it is like.]

\clearpage
\section{Concept Art: Innperseqs (May--July 1961)}

A "halpoint" iff whatever is at any point in space, in the fading rainbow halo which appears to surround a small bright light when one looks at it through glasses fogged by having been breathed on, for as long as the point is in the halo. 

An "inittpoint" iff a halpoint in the initial vague outer ring of its halo.

An "inn'perseq" iff a sequence of sequences of halpoints such that all the halpoints are on one (initial) radius of a halo; the members of the first sequence are initpoints; for each of the other sequences, the first member (a "consequent") is got from the non-first members of the preceding sequence (the "anteeedents") by being the inner endpoint of the radial seg-ment in the vague outer ring whom they are on the segment, and the dater members (if any) are initpoints or first mem-bers of preceding sequences; all first members of sequences other than the last appear as non-first members, and halpoints appear only once as non-first members; and the last sequence has one member. 

Indeterminacy A rtotally determinate innperseq' iff an innperseq in which one is aware of (specifies) all halpoints. 

An rantecedentally indeterminate innperseql iff an innperseq in which one is aware of (specifies) only each consequent and the radial segment beyond it. 

A rhalpointally indeterminate innperseq-liff an innporseq in Mich one is aware of (specifies) only the radial segment in the vague outer ring, and its inner endpoint, as it progresses inward. 


