\chapter{Communists Must Give Revolutionary Leadership in Culture (1965)}

\section*{1\hfill}

If the bourgeoisie could keep control indefinitely of the world's most productive 
economies, the U.S. and West Europe, then all cultural leadership, no matter how 
revolutionary, would come to nothing. The culture of the future would be fascist 
barbarism. The discussion of revolutionary cultural leadership is contingent upon 
this state of affairs. 

\section*{2}

\uline{For clarity}, \uline{somewhere}, \uline{sometime}, the \uline{best 
possibility} in culture for the present period has to be defined. As an analogy,
with respect to religion Communists have to state somewhere, sometime that we
are atheists, that the best possibility is the liquidation of religious 
institutions. 

But it is not possible to \uline{implement at once} the best possibility,
before there is a social basis for it. To continue the analogy, a Communist
government cannot have all churches demolished the day it seizes power. Of 
course, when the workers take the initiative, as when the Spanish workers 
burned churches in the Civil War, they must be supported by the leadership
(otherwise it is no leadership). But usually a Communist government must
implement its atheism by guaranteeing \enquote{freedom} of religion; at the same
time closing down the faith healing and religious charity rackets, 
nationalizing church investments, secularizing the schools, teaching atheism,
and in general creating social conditions which tend against escapism among the
masses. Under these conditions atheism should gradually replace religion.

It is the same for culture. A good example is Lenin's directives on the film in
the early years of the Soviet Republic, which were quite advanced. Lenin said to
show the old melodramatic cinemas, yes, to get the people into the movie theatres
and to pay for the his industry; but in the production of Soviet films begin with
socialist documentaries, and require that they appear on every film program along
with the melodramas. 

\section*{3}

The following three conditions for revolutionary cultural leadership are the ones
I will argue from. I will refer to them by letter throughout the text. 
\begin{enumerate}[label=\Alph*., nosep, itemsep=0.5em]
\item To increase the productivity of labor. (For Marxists, a fundamental progressive historical tendency.) 
\item To promote equality and solidarity of the workers; and to tend against the formation of a labor aristocracy, the stratification of the workers by nationality, and the formation of bureaucracies in the proletarian dictatorships. 
\item To satisfy the workers' desire, during a flood tide of class struggle, to come to grips with reality and to be done with escapism in culture. 
\end{enumerate}

\section*{4}

The following is the best possibility in culture for the present 
period. (The many tasks pressing on me allow time only to sketch it.) 
It is important to remember that implementation of the best possibility
by a Communist government can only be in stages. For example, I will say
that the socialist documentary film should replace the filmed melodrama,
\uline{and} that the filmed drama should replace the theatre. In practice this
means to replace the theatre by the filmed drama, while at the same time
to continually increase the proportion of documentaries to theatrical 
films. 


\subsection*{\enquote{\textsc{Applied Arts}}}
These include architecture, automotive products, appliances and
utensils, furniture, graphics, and clothes. Actually, they are nothing
but fields of design engineering which are particularly \uline{retarded} 
by Artistic, national, and stylistic traditions (as well as by the profit
economy). Revolutionary leadership in these fields is to increase the
productivity of labor (Condition A). 

To increase labor productivity in the \enquote{applied arts}, public
ownership is necessary, particularly to escape the forced consumption,
the deliberately wasteful stylisrn required by the capitalist economy;
and a planned economy is necessary, to plan housing and transport as a
whole. But the experience of the Soviet Union shows that these economic
prerequisites do not ensure efficient design. The Soviet Union is 
needlessly backward in the design of automobiles, appliances, furniture,
graphics and clothes. Economic prerequisites and engineering advances
have to be \uline{taken advantage of}: the efficient design has to be chosen 
over the stylized design. 

The oppressed, the poor and illiterate masses as such have nothing to 
contribute to engineering in the \enquote{applied arts}; there can be no 
\enquote{proletarian design}. \enquote{Folk handicrafts}, such as handweaving, pottery,
metalworking must be replaced by mass pro-duction. 

The most efficient architecture, housing today is Soviet prefabricated
concrete architecture. (Appendix 1.) However, this architecture is 
limited by the heaviness of concrete, structural redun-dancy, and 
stylism. Maciunas' prefabricated architecture shows how much farther 
efficiency can be carried. (Appendix 2. A sample of a major material 
in Maciunas' design, expanded polystyrene, is utilized as the back 
cover of this pamphlet.) The Buckminster Fuller geodesic dome is efficient
architecture. (Appendix 3.) 

The Citroen 2CV automobile is efficient, much more so than Soviet 
automobiles. (Appendix 4.) As for clothes, their design is complicated by
their sexual decorative function. However, attractive sports clothes are
often efficient, when they are designed to meet the physical requirements of
the sport. The Soviet decision to copy the freakish Paris fashion industry,
rather than to develop a line of street clothes on the principles of sports
clothes, say exemplifies the reactionary tendency in Soviet \enquote{applied arts}. 

Soviet prefabricated architecture, Maciunas' architecture, the Fuller 
dome, the Citroen 2CV, and this pamphlet show that an efficient artifact
is not a modernistic styled artifact, or even a conventional artifact minus
familiar stylization or decoration. \uline{Efficiency is giving the most performance
for the least cost: to be achieved through complete disregard of 
artistic, national, and stylistic traditions; extensive thought and analysis
in the solution of the design problem; and use of the latest scientific advances.
This principle is completely different, incomparably more advanced and 
intelligent than stylization or decoration; and excludes them.} Efficiency
should replace stylization and decoration. Further, people can be educated to
\uline{like} the achievement of the most performance for the least cost, rather than the more
primitive decoration. 

\subsection*{\textsc{Music (Dancing, \enquote{Poetry})}}

In this area the most oppressed classes and nations have a crucial traditional
culture of their own. For the United States, there are the country blues, 
boogie woogie, R\&B (and Gospel), and the hillbilly music they have influenced.
For Jamaica, there is Blue Beat. For Black Africa, the many tribal musics such
as Baoule and Zulu, either the traditional non-urban varients or the recent 
city-worker variants. For Brazil, macumba music. For India, the analogous music.
For Cuba, bemb\'{e} and as much else as possible from Santer\'{\i}a---if the Government
doesn't wipe it out first. 

In general, for each nation there is a common musical culture which is the
spontaneous creation of those farm-workers and later city-workers who are not
social climbers and can't quit being poor. \emph{Actually, it is usually a fusion of music, dancing, and lyrics---which is done or watched, but not \enquote{performed}. Such a dance of one national grouping comes to another nation not on the Concert Stage, but by people taking it up, doing it.} 
This music-dancing is \uline{already} the indigenous symbols of the oppressed.
In a phrase: \enquote{street-Negro music}. 

To give more examples: For France---during the revolution of 1789, the 
revolutionary masses danced the \uline{carmagnole} in the streets of Paris. Unfortunately,
as a result of the protracted monopoly capitalist stagnation of Europe,
the history of colonialism and the bribing of the workers, there is no native
\enquote{street-Negro music} in France today. The recent ascendancy of an R\&B-inspired
music among British working youth prefigures the path for Europe. 

Street-Negro music cannot be reduced to old or recent European bourgeois Art. 
Nor is it African arts of hundred of years ago which are no longer extant. 
\emph{Further, it must be absolutely clear that street-Negro music is not \enquote{folk art}.}
Street-Negro music is not those \enquote{Negro spirituals} that were \enquote{purified} of 
\enquote{ugly} Africanisms, notated in choral fashion, and performed in the Concert
Hall; or Villa-Lobos'\enquote{Brazilian} Symphonies; or the Kingston Trio; or Odella.
Street-Negro music is not the Russian Moiseyev's Ballets of phony \enquote{folk dances}
performed on the Concert Stage; or the Mexican Ballet Folklorico; or the Cuban
National \enquote{Folk Dance} group; or the Rhumba, \enquote{cleansed} of Africanisrns to become
a ballroom dance. 

Recent trends in street-Negro music prove that it is irreconcilable 
with \enquote{folk art}, that it is hundreds of years \uline{less} antiquated than Bach or 
Mozart, that its modern development is away from European bourgeois Modern 
Art. Such advanced instruments as the many models of electric guitars and 
electric organs (Appendix 5) are standard in R\&B and hillbilly music. Rock'n'roll 
uses such electronic recording techniques as reverberation, dubbing
and over-dubbing. Many numbers are \enquote{studio creations}, exist only in recording.
In addition to the importance of the recording studio in production, there 
is the importance of radio stations in distribution. Street-Negro music rocks;
European bourgeois Modern Art (and \enquote{folk art}) doesn't. 

\enquote{folk art} means something antiquated, humble, and pathetic, and R\&B singer
Buddy Guy's \enquote{First Time I Met the Blues} (Argo LP 4026) comes on too scary for
that put-down. \uline{Bo Diddley} (Chess LP 1431) is good, normal street-Negro music; 
Bo Diddley, use of amplifier reverb and tremolo in \enquote{Pretty Thing} is the 
opposite of \enquote{primitive}. \enquote{Tobacco Road}, by the Nashville musician John D. 
Laudermilk (Columbia Promotion Record 45 RPM 4-41562, JZ SP49232), is the very 
opposite of \enquote{primitive}; and utterly cuts up the \enquote{sophisticated } Nashville Teens'
version. Incidentally, Laudermilk's lyrics are reactionary, but his musical
culture in general is an indigenous symbol of the proletariat. The R\&R which 
jives the snobs and has an electronic sound and is called \enquote{junk music}, such
as \enquote{The Surfin' Bird} by the Trashmen (Garrett Records GA 4002'), is also
by no means \enquote{primitive}. 

For revolutionary leadership, street-Negro music \uline{must be} the 
\uline{point} of \uline{departure. (Condition B.)} Now there is no nonsense about
ringing this music-dancing to the masses, because they created
it. The almost insuperable problem is to bring street-Negro music
to the Communists. Personal aquaintance reveals that \uline{in general}
the instinct of Communists towards street-Negro music is a revulsion
that defies rationality. Communists will say: \enquote{I won't have 
that music in my house.} \enquote{If the Negroes had the opportunity to
hear Fine Music, they would disown that low music.} \enquote{Saxophones
are for playing saxophone \uline{concerti.}} \enquote{Musicians come from 
conservatories.} When people who listen to street-Negro music
become Communists, they sense a hostile and embarrassed reaction 
to the music from their comrades, and they give it up. Communists'
faces are turned \enquote{up} to old European bourgeois Music,
even though they do not know a great deal about it; even though
they cannot analyze the \uline{Grosse Fuge} or a Bruckner Symphony---even
though they have never heard of the insertion syncopations
of \uline{Sumite karissimi}---or of Faugues. Their idea of popular music
is European popular music, Anglo-American music, Mitch Miller 
music. They will fabricate the most tortuous sophistry to prove
that street-Negro music is \enquote{bourgeois corruption}---but that of
course Beethoven's \uline{Solemn Mass in D Major} is not. Communists 
derive self-respect from being too \enquote{high-minded} for those \enquote{vulgar, noisy, trashy} dances. They feel threatened by street-Negro music 
as if it were a bottomless pit of tar. 

But \uline{in general}, the music of the Communists, particularly Anglo-American
popular music, is simply hopeless. It is a cultural expression
of \enquote{white}---chauvinism and racism. It is \enquote{dead} and on
the way out. As I suggested, street-Negro music is already replacing
it among the white masses, wherever they accept rock'n'roll, 
bemb\'{e} or the like. The Communist opposition to this process 
simply will not do. The music of the Communists must be virtually 
replaced by street-Negro music. (Condition B.) 

\framebox[1.1\width]{The amount that U.S. Communists can do now to implement all this
is of course very limited. But they can do some things. They can be
asked to listen to the R\&B and R\&R radio programs rather than the
other music programs. They can be asked to replace their Classical
record collections. They can be asked to play only \enquote{street-Negro music} at parties. 
We must never sponsor Concerts or, worse yet, 
\enquote{folk} Concerts. If anything, we must sponsor dances. Remember 
that street-Negro music is not a narrow category. lt includes every 
authentic popular music in the world today, except the European
or Anglo-American, which is simply all washed up.}

After the Communists are listening to street-Negro music, then it
will be time to consider that the bourgeoisie \uline{does} manipulate it by
controlling it commercially. In addition to other commercial practices,
the bourgeoisie selects to sell, and encourages, lyrics 
which suit itself---especially in rock 'n' roll. (And the religion in
Gospel, Santer\'{\i}a, and the like is reactionary---but the musical
culture is not, and survives through its influence on secular
music.) Somebody will have to encourage an open call to rebellion
in the lyrics (a step which is irreconcilable with composing cop-out
\enquote{Negro freedom} \enquote{folk songs}).Actually, rebellion is what this
music-dancing is all about; it just couldn't come out under the 
overseer, shotgun or the A\&R man's axe. 

As for \enquote{folk music} and \enquote{folk ballet}, they must be replaced by 
street-Negro music. (Condition B.) All ballet should be replaced 
by this music-dancing. (Condition B. Also, the live Concert Stage
performance is not efficiently produced culture --- Condition A.)
Grand Opera should be replaced by this music-dancing. (Conditions
B, A, C.) All other music should be replaced by it. (Condition B.) 

Even more, Poetry should be replaced by the lyrics of street-Negro
music (Conditions B, C.) 

Revolutionary leadership in music, dancing, "poetry" can only be
as I have stated. No alternative can even be considered. Street-Negro
music is already a vital symbol of the most oppressed; all 
that is needed is encouragement so that it can become more so. 

\subsection*{Film}

Leaving aside the \enquote{applied arts}, the three areas of culture which
have the greatest mass significance are music, the film, and the
novel; and the film tends to become more important than the novel. 

Film-and-sound-track as an area of culture is not engineering;
only the design of the equipment is. However, neither is film
something the oppressed originate. They can originate their own
music by singing while they work; but to realize film-and-track
at all, from camera and recorder to sound projection, requires
time and highly complex equipment. Film is also wholly modern.
It simply hasn't been around long enough to have \enquote{Laws of Art}.
The actual making of films is done exclusively by capitalist or
post-capitalist cultural specialists. It is not the oppressed, but 
intellectuals or cultural specialists who originate film-\enquote{ideas}.
All the poor and illiterate masses as such can do is to respond 
to the films that are shown to them. Thus, there can be no \enquote{proletarian}, 
\enquote{folk} cinema. 

Now \enquote{film} or \enquote{The Cinema} commonly means the \enquote{novelistic cinema},
the \enquote{Theatrical or melodramatic cinema}, the \enquote{acted cinema}. 
That is,the cinema is commonly a photograph of an acted theatrical
drama, staged with sets, actors, costumes, and make-up. In
turn, this drama is usually an adaptation of a fictional-plot novel. 
The Hollywood movie industry is a novelistic cinema industry. The
Russian cinema too, before the Revolution and after Lenin's death, 
is novelistic cinema. 

But during the revolutionary flood tide when Lenin was head of
state, a significant change took place in the Russian film. Lenin
issued a number of directives which said in effect that the novelistic
cinema was merely to be tolerated. Soviet film-makers were 
to concentrate on newsreels and documentaries, and these were 
to be included in every film program. Lenin called for a propagandist 
newsreel series \enquote{From the life of peoples in all countries}.
He expected that these scientific films, as he called them, would 
overcome the influence of the novelistic cinema. (Appendix 6. This
was the only time so far that a Communist politician gave revolutionary 
leadership in culture.) 

At the same time, Dziga Vertov (Denis Arkadyevich Kaufman), a
highly sophisticated cultural specialist who at twenty had been 
in charge of all newsreels of the Russian Civil War, called for the 
virtual replacement of the theatrical cinema by documentaries. He 
made a series of documentaries, using footage of actual events,
and developing new organizations of material to replace the fictional plot. (Appendix 7.)
Vertov's films were popular with the 
Soviet massess, because his films were often the only ones that
showed the new age the people lived in. \uline{Pravda} praised Vertov's
films on its front page. Vertov had the support of celebrities such
as Mayakovsky. Vertov's films even influenced the neo-theatrical
cinemas such as \uline{Strike} and \uline{Potemkin}, unfortunately giving them a 
new lease on life. (When Eisenstein started to make his cinema
\uline{October}, or \uline{Ten Days That Shook the World}, in nine months in
1927, the first three months were taken entirely by the writing of 
the scenario, the fiction, the act. Vertov started by filming the
\uline{events} he wanted to portray.) The first screening of Vertov, \uline{A Sixth
of the Earth} was was for the delegates to the Fifteenth Congress of the
CPSU in 1926, showing the extent of his official acceptance.
Another figure during this high tide of the documentary was Esther 
Shub, who in 1926--7 turned from editing melodramas to make two 
\uline{historical} documentaries from old newsreel footage, \uline{Fall of the
Romanov Dynasty} and \uline{The Great Road} (October anniversary). 

But in the ebb of revolution that followed Lenin, death, the novelistic 
cinematographers again came to the fore, and were supported by Stalin.
Although Vertov continued to be a steadfast opponent
of the novelistic cinema, he abandoned documentaries of
Soviet life for films of primarily technical importance, and portrayals
of Lenin as a Russian deity rather than a political figure.
The novelistic cinematographers won out, and Vertov's work was 
suppressed. From 1937 to his death in 1954, Vertov made no 
more feature films, The feature documentary made a limited come-back
in World War II; but has boon in obscurity ever since. 

Further, Vertov had a serious limitation which put him at a slight disadvantage in his struggle with Eisenstein. None of the Soviet artists were able to stand on their own feet politically; but in addition Vertov was hardly class-conscious at all. Eisenstoin made theatrical, acted cinemas which emphasized the class struggle (fat banker---starving workers); Vertov could have made documentaries which did so even more, but he didn't. Vertov's film reflected Soviet life, but not in a highly class-conscious way. Esther Shub was more political. 

The British cinema is mostly novelistic. But in the early 1930's, during the upsurge of the workers, and under the influence of the Soviet filmmakers l have just discussed, the filmmaker John Grierson and others started the British Documentary Film movement, and the Left adopted the documentary as its own. 

When the documentary has been favored, in Russia, Britain, and elsewhere, the cause has been an upsurge of the workers. I conclude that the predominance of the novelistic cinema, which is like a fantasy or dream, is escapism and false consciousness in a period of reaction. In a period of revolution, real life offers more hope to the oppressed than any fantasy or dream. If the documentary is favored, it is because of the oppressed's desire to conic to grips with reality. Vertov's support of the documentary against the acted cinema, and his techniques, at least, must be the \uline{point of departure} for the revolutionary film-and-sound-track. These documentaries must improve on Vertov, however, by emphasizing the antagonism of the classes. Then, the showing of socialist documentaries must replace the showing of old novelistic cinemas. \uline{(Condition C.)}

The whole production and of course distribution of the film tends toward a high productivity of culture per worker. Vertov himself had a network of cameramen-correspondents throughout the Soviet Union who supplied him with footage. He and his assistants then put it together in his Moscow workshop. Of course, this procedure is standard for the newsreel. 

\framebox[1.1\width]{
The amount that U.S. Communists can do now to implement all this
is of course very limited. But when our movement is far enough 
along that we have to have cultural activities, we must make our
own films, as the British Left did. They will be documentaries,
making the points specified by the leadership, through the selection
and organization of material. Note that when we publish photographs,
they are always of real events. We never publish photographs 
of scenes staged with actors. Whatever our reason is, it is 
the argument against the novelistic cinema. Further, for us to fool
with the novelistic cinema, with elaborate sets, actors, and cos-tumes
would not only be escapist, but an absurdly wasteful expense.
Then, if the bankers circulate a movie or play so fascist 
that we have to demonstrate against it, we must publicize our own
films in the demonstration.}

\subsection*[(\enquote{\textsc{Theatre}})]
The Legitimate Theatre, Broadway and off-Broadway, is a culture of the luxury-consuming \'{e}lite. It is a luxury analogous to handwoven clothes. As soon as Legitimate theatrical productions are a success, they are filmed. The film is far more efficient than the Legitimate Theatre. Even though the Theatre is more original and intimate than the \uline{theatrical} cinema, it will tend to be replaced by films of itself, because of their greater efficiency. The Legitimate Theatre survives to the extent that a luxury-consuming \'{e}lite supports it. 

Vaudeville has reappeared, but on television, on videotape. Even the comic nightclub monologue, which is not being filmed, is being recorded. 

As for the feudal and \enquote{folk} theatres, Noh, the Japanese puppet theatre, the circus, the Brazilian \uline{festa}, they are already dying out, 

Thus, the Legitimate Theatre is a highly inefficient luxury com-pased to the film. It is revolutionary leadership to do everything with the film; to replace the Theatre, including performances of old Dramas, with the socialist film. (\uline{Condition A}: higher labor pro-ductivity; \uline{Condition B}: against \'{e}litist luxuries.) 

\subsection*{(\enquote{VISUAL ARTS})}
Before the Invention of photography, much Art involved representation, from Chinese silk painting to the French Academy to Indian sculpture. As photography was developed in bourgeois society in the 19th century, and rapidly achieved widespread popularity, Serious Artists progressively lost Interest In representation. 

The decline of representational Art resulting from the invention of photography is an accomplished fact. The acceptance of this irreversible technical advance is obligatory for revolutionary leadership. (\uline{Condition A.}) As for calligraphy, it is replaced by the Roman alphabet and typography; and as for decoration in the \enquote{applied arts}, I have already dealt with it, 

Bourgeois Artists have turned to producing non-documentational Art Objects, which glorify the bourgeoisie by symbolizing methods. These Art Objects are luxuries, which survive to the extent that a luxury-consuming \'{e}lite supports them. In general, the symbolizing Art Object, and the Art Galleries, are appropriately replaced by the film. \uline{(Conditions B, A, C.)}

It is very significant that during the same high tide of the documentary film I have described: \enquote{At that time the Leftist papers and periodicals carried articles on painting and its expected demise as an art.} The author of those words was Alexander Dovzhenko, a man who had made painting his career, then became so worried by the cultural trend he described that after a sleepless night in his 32nd year, he left his painting studio for good and went into film, 

To conclude, the film's general superiority for culture, in the era in which the masses take over the stage of history, is so great that the theatre and the visual arts must move into its wake. 

The only variety of drawing which symbolizes rather than documents, which the oppressed have shown they want, is the socialist political cartoon. But the political cartoon is an element of journalism; it doesn't have leading cultural significance as does the film. 

\subsection*{\textsc{Fiction}}
The best-seller novel is a \uline{middle-class} art. Its \enquote{equipment}, the ability to read, pen and paper, the prose fictional narrative, is simple. Mass paperback publication and monopolistic distribution of the best-seller make it much more efficient, much less of a luxury then the theatre. However, the film is more immediate, more accessable to the oppressed. 

Many bourgeois intellectuals are becoming more interested in film than in the novel. Even so, the middle classes may continue to demand the best-seller. If they do, the best-seller will present a dilemma. The whole experience of the best-seller shows that it can only have flair, \enquote{freedom}, \enquote{life} when it is irresponsible to the correct political line, to Communist virtuousness. That is, when it is cynical about revolution, scatological, or the like. The novel that approximates the Party line, Marxist historiography, the Socialist Realist novel, can only be stodgy, inept, and dull---and is not a \enquote{best-seller}. There is no getting around it: Dostoevski is more \enquote{alive} than any Soviet novelist; the political cynics, Hemingway, Mailer, Baldwin, and their kind, make the most \enquote{alive} novelists, The reason is precisely that the best-seller violates Condition C. The flair of the best-seller, even the \enquote{true-to-life} best-seller, comes from its fictionality, its fantasy, its escapism, 

No doubt the fiction of Proust, Joyco, Samuel Beckett will die out. (Condition B.) But otherwise there maybe no \enquote{best possibility} for fiction, The politically irresponsible best-seller may have to be permitted indefinitely, if the middle classes demand it. (But not Socialist Realism---we can't back an escapist art that the masses don't even want.)

At best, if the workers' pressure for a revolutionary political line in the best-seller becomes great enough, it may be possible to divert them from the best-seller to the socialist journalistic book and film. (By the \enquote{journalistic book} I mean the book-length account of current events by a writer-participant, such as John Reed's \uline{Ten Days.}) 

\framebox[1.1\width]{ To implement this, U.S. Communists should encourage filmmakers, not novelists; because we can get so much more out of the film in the short run, as well as the long run. It is wisest for us to leave the writing of the best-sellers to the cynics, who do it so well.}

Can as much established culture as I have said really be exhausted? It is extremely significant that during the same early Soviet period I have referred to, all of the famous cultural specialists, even Eisenstein, even Mayakovsky, believed that Art would ultimately disappear; as every honest history of the period admits. In general, this period was the most revolutionary yet in cultural history; there must be some truth in the unanimous belief of its celebrities, 

\subsection*{\textsc{Theory of Culture}}
Two topics particularly need study. The most important is: Culture as the instrument of an economic class. The overwhelming evidence that culture is a class instrument must be assembled. The other topic is the finances of culture, of the multibillion-dollar cultural institutions: patronage and commissions; Opera boxes; book sales; ticket sales; the Art market; record sales; the banks and the cinema monopolies; the financing of Broadway plays; sponsorship of TV dramas; private endowment of Orchestras, Museums, Concert Halls; state-monopoly capitalist support, which is increasingly important. 

Top published studies of the finances of culture are Francis Haskell's \uline{Patrons and Painters: A Study in the Relations Between Italian Art and Society In the Age of the Baroque}; investment banker Richard H. Rush's \uline{Art As An Investment}; and bourgeois critic Harold Rosenberg's \enquote{The Art Establishment}, in Esquire, January 1965; because they elaborate with Machiavellian frankness on the relation of culture to the ruling class, for a large area of culture. These revealing studies need only be placed in context to have revolutionary implications. Unpublished material of mine which I am collecting under the title \uline{Class and the Development of Culture} will provide an adequate general theory of culture. 

\section{5}

In many cases a reference to one of Conditions A--C has been a sufficient argument for revolutionary leadership. For example, a reference to Condition A is a sufficient argument for prefabricated architecture. 

Only the applications of Condition B need to be explained: To show that certain \enquote{folk art}, Ballet, Grand Opera, exploiting-class Music (\enquote{Classical} or \enquote{Serious} Music), Poetry, the Legitimate Theatre, bourgeois Modern Art---and European popular music---\uline{promote the formation of a labor aristocracy, the stratification of the workers by nationality, and the formation of bureaucracies in [he proletarian dictatorships.}

Economic causes have greatly stratified the world proletariat. The bourgeoisie has used part of the profits from one section of the workers to bribe another section. The low labor productivity of proletarian dictatorships in backward countries is another cause. At the top of the world proletariat is the white labor aristocracy, and the revisionist managers in the proletarian dictatorships; at the bottom are the \enquote{street Negroes}, the \enquote{field Negroes}. The upper layer of the proletariat seeks to consolidate its privileges, and uses methods pioneered by the aristocracies of earlier eras. 

One of the methods pioneered by the aristocracies was through Art. Most of the culture that Condition B excludes originated on commission from exploiters of one kind or another, back when Artists were virtually part of a patrician's stable of servants. The exploiters demanded of their Artists culture virtually prepared to consolidate and enhance the exploiters' position, to differentiate the exploiters as much as possible from the less privileged, the plebeians battering at the doors of privilege. This culture is intentionally wasteful. Where there was national oppression, the culture was prepared to differentiate the exploiters from the oppressed. Today, even though the European bourgeois Modern Artist is no longer a servant, his patrons are the aristocracy of business and finance, The major function of his Art is still to differentiate and consolidate the aristocracy, and the aristocracy of the aristocracy, as Rosenberg's \enquote{The Art Establishment} so brilliantly proves. Yes, this bourgeois critic understands his own class much bettor than the \enquote{Marxist} critics do.

As a consequence, the social climbers, the newly rich and the middle classes, insistently demand Art Appreciation courses to initiate them into the mysteries of snob culture. From this social layer come the \enquote{Marxist} critics, clutching their programs of Music of the Masters, their pamphlets explaining the Old Master Paintings. 

Catholic Music and Art were prepared to awe the serfs, thus consolidating and enhancing the hierarchy's position. They succeeded so well that even the \enquote{Marxist} critics don't recommend then to the workers, A materialist can never agree that the function of Raphael's \uline{Madonnas} was to glorify God. Their major function was to serve their aristocratic patrons. Francis Haskell's \uline{Patrons and Painters} describes both how Italian Baroque painting served the patricians, such as Barberini Pope Urban VIII and Mario de Medici; and how it helped the newly rich, such as Girolamo Manfrin, to climb in society. Then, the whole historical experience of Opera in New York has been that each successive wave of the newly rich, the Astors, Cuttings, Beekmans and Livingstons, then the Vanderbilts, Goulds, Harrimans, Rockefellers and Huntingtons, now the Goldings, Chalks, Grahams (Elizabeth Arden), Lauders, and Cummings', has found Grand Opera---ownership of an Opera Box---indispensable to consolidate its position at the zenith of society. And this consolidation is the historic social function of Grand Opera. Wagner's interest in exclusiveness led him to restrict the performance of his Operas to a specially constructed temple, the \uline{Festspielhaus}; and to encourage his devotees to fast before the performances. Since then his Operas have given unequaled service to the German bourgeoisie. As for the provincial merchants on the make and other lower orders of social climbers, the \uline{Reader's Digest} and Book-of-the-Month Club provide them with their Art Appreciation courses.

Thus, when this culture is introduced among the workers, it promotes their stratification. The labor aristocracy, the revisionist managers, the reactionary intellectuals rally to it. When Art Appreciation is forced on the masses, I am convinced that the most oppressed or class-conscious workers can never embrace the Art in question, the \uline{Turandots} and Bruckner Symphonies, as their own, because it is too alien to their condition. They rightly see it as snobbery. The revolutionary intellectuals, too, instinctively see it as conservative. But it encourages other workers to climb socially, 

Then, snob culture today is dominated by European bourgeois culture, a domination secured by the guns of the imperialists. Its whole \enquote{idea} is that it is high and pure, while street-Negro music is low and \enquote{racial}. This white-supremacist conception will persist as long as snob culture predominates. Snob culture divides the workers into white and black, and consolidates the white workers' position above the black workers. 

An event which illustrates the role of snob culture in the proletarian dictatorships was the appearance of La Scala Opera of Milan at the Bolshoi Theatre in 1964. When it was announced that La Scala would open the 1964--5 Moscow season, the series was sold out almost instantly, Tickets were appropriated by Government Ministries---and their wives' dressmakers. Even the Soviet ticket scalping system (Blat) was helpless as two million requests for seals poured in. La Scala performed \uline{Turandot}, \uline{Trovatore}, \uline{Lucia di Lammermoor}. The first-night audience included almost every member of the Soviet \'{e}lite from Yuri Gagarin to Ulanova. Premier Khrushchev attended the second \uline{Turandot}, and came the next night with his whole family, along with Mikoyan and Marshal Zhukov. On the second night, the 95 brilliant chandeliers of the 134-year-old Bolshoi trembled with the applause from the six tiers of red plush and gold-trimmed balconies. Now the craving of those two million people for tickets, to see (and be seen at) an aristocratic frivolity like \uline{Turandot}, which is precisely what the whole cultural policy of the Soviet Union for decades has been contrived to produce, is a craving to climb into the managerial \'{e}lite. Yes, this culture finds its most appropriate, natural, and indispensable place in the aggressive consolidation of the bureaucracy. It is also significant that the top Soviet performing Artists, Ulanova (the \emph{\enquote{tsarina}} of Dance), Rikhter, Oistrakh, Rostropovich, and the stars of the Moscow Art Theatre, the Ballerinas who after a performance leave the Bolshoi in their fur coats and ride off in their chauffeur-driven private limousines, live in an ease second only to that of the top Government leaders. Sooner or later, the Interpretation of patrician Art requires patrician performers. In People's China too, orchestra players receive twice the salary of steelworkers. 

European popular music is the exception in the culture Condition B excludes, in that it was a revolutionary symbol in 1789. But now that the European and European-American workers have come to have a position in the world proletariat above the colored workers, the \enquote{idea} of European popular music is that it is higher and purer than low, \enquote{racial} street-Negro music. This white-supremacist conception will persist as long as \enquote{white} music does; European popular music now symbolizes white supremacy, Further, the whole idea of \enquote{folk music} and \enquote{folk ballet} is \enquote{cleaned-up n------r music and dances}, European popular music and \enquote{folk music} consolidate white supremacy among the workers. 

\section*{6}
A general remark on implementation of the best possibility for theatre and film. In the early stages of the replacement of the Legitimate Theatre and the novelistic cinema, some more plays and some neo-novelistic cinemas may be produced. These dramas and cinemas present a dilemma exactly analogous to that of the best-seller. They can have "life" only when they are irresponsible to the correct political line. The more they approximate the Party line, the more they will fall flat. LeRoi Jones against the Chinese "proletarian drama": that is the dilemma. The politically irresponsible drama and novelistic cinema may have to be permitted for a while. But a more hopeful possibility is that workers' dissatisfaction with their politics will help to displace them. 

\section*{7}
I have so far limited myself to the present period, but I would like to make one remark with respect to the period after the defeat of the U.S.-European bourgeoisies. Music-dancing will then be an amusement rather than a class-struggle symbol. In the present period, street-Negro music is permeated through and through with \uline{competition}. There is competition between the singers and combos. There is competition on the dance floor. The conceptions of one's superiority to another, one's getting ahead and leaving another behind, or of one's mediocrity or inferiority, are involved. Thus, disappointment and failure are possible in this music-dancing. Now competition is inimical to amusement, I think the people of the future will turn against this competition. They will find a way, in music-dancing and other amusements, for everybody to be "best". 

\section*{8}
If the bourgeoisie could keep control indefinitely of the world's most productive economies, the U.S. and West Europe, then all cultural leadership, no matter how revolutionary, would come to nothing, 
