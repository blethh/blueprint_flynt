\chapter{My New Concept of General Acognitive Culture (1962)}

{\itshape [This essay was written c. May 1962 and published in \journaltitle{d\'{e}collage No. 3.} This transcription serves to correct the typographical errors. Footnotes are written in 1992.]}

Of the adult (human) activities I discredit explicitly, consider pure mathematics (and structure art and games of intellectual skill), and Serious Culture\slash all art\slash literary culture\slash science fiction\slash music. I show that these activities (as such) should be repudiated. Now humans are likely in any case to resist this radical idea of repudiating these major institutionalized activities; but especially if nothing were to take their place, if the idea were negative only. Even when the activities' Serious Cultural pretensions have been discredited and repudiated, and their obvious confusions of purpose have been noted,\footnote{cf. \essaytitle{Concept Art} on music} humans are likely to be interested in them still, to like them in at least one respect: for their entertainment, recreational value; for their value as \enquote{ends,} in themselves. (And are thus likely to fear that to repudiate these activities without anything's taking their place would be to give up all recreation, doing things \enquote{just for fun,} doing things just liked.) Now this chapter will be first, an analysis of the concept of entertainment, recreation, of doing things just liked, which will criticize the activities even as just entertainment. (And will discredit my own initial notion of \term{acognitive culture,} as not going far enough.)

I discredit these activities, show they should be repudiated, for \enquote{everybody,} adult humans and creeps. Now since I am a creep, my primary constructive concern is to point out something rather than these activities, for creeps: my new concept of \term{creep acognitive culture.} However, I am going to \enquote{do adult humans a favor} in the hope that it will keep them from just changing the discredited activities into something no less wrong and confused, and will encourage them to repudiate the activities. \enquote{Creep acognitive culture} is, to speak generally, a concept of \enquote{recreation} (resulting from analysis of the concept of recreation) for conscious organisms. Part of it is applicable for adult humans (as well as creeps), in replacing the discredited activities for them. I am going to give that general part here, in this book\footnote{This essay was a chapter in a book in early 1962; that book must have become From Culture to Veramusement.}---my new concept of \term{general acognitive culture.} (The specialization for creeps I will give in Creep.) The specialization of this concept for adult humans I will leave to them, since that is their concern. Incidentally, even though generally applicable, the characteristics of general acognitive culture may be reminiscent of creepiness, but they will not in any case embarrass mature adults, which is where I draw the line between the adult human and the really creep.

To give a better idea of the major area of life, \enquote{recreation,} I am concerned with here, let me mention, along with the activities mentioned above: games, possibly athletics \enquote{for fun,} conventional entertainment and recreation, and children's play. Or \term{acognitive culture} in my initial sense. Further, let me suggest the area with respect to its place in (adult) human life today. Naively, a worker has a job, job hours, an occupation, does work (which produces material wealth), to obtain his means of consumption. His job is a \enquote{means}; even though he may like it he is pretty much forced to do it. This can be extended to apply to the whole area of his responsibilities to society. Then he has after-hours, time when he doesn't have to do anything, and does what he does more as an end, in itself, \enquote{for fun,} because he likes it: here is where recreation is included. This is when workers listen to music, read science fiction, play games, and the rest. A thing is more purely recreational the more it is done just \enquote{for fun,} the more is it is not an extension of the job, a means. This can be extended to apply to the whole area of what he does just because he likes it; and the area can now be conceived as existing (presumably as a matter of course) side-by-side what he does \enquote{for society.} All this can be said about recreation today.

To arrive at the preliminaries of my concept of general acognitive culture, a certain concept of \enquote{recreation} applicable for any conscious organisms (my initial notion of \term{acognitive culture}), let me give some characteristics which the activities I have listed, in their recreational aspect, have in common, which would apply for any conscious organisms. No one of the activities is biologically necessary (or biologically harmful) to the organism. Probably no one is necessary for society, co-operation among the organisms. They are not technology (although they may use it). As (\enquote{mere}) recreation, they are not supposed to have cognitive value (and in particular are without associated cognitive pretensions, so that they cannot be Serious Culture). (They may use believings, especially wrong ones, as \enquote{experiences,} but these are not claimed to have cognitive value in any way.) They do not involve anything, in particular sensuous indulgence, which has sophistication-proving significance. And of course, they are entertainment, recreation, are things just liked. These characteristics are the preliminary, initial determination of the parts of life, of any conscious organism, which I am selecting out to consider as one area, a unity, that of acognitive culture.

Having located and initially determined the area of life I am concerned with, I will now analyze, explicate the concept of pure entertainment, recreation, doing things just liked (with respect to the individual); and at the same time elaborate my new concept of general acognitive culture. Consider, for contrast, work, or the cognitive. With respect to these, there are \enquote{objective} or \enquote{intersubjective standards of value,} for ex., whether a table top is level, or whether many people like a thing. One may well make a contribution to these areas even if one doesn't like the areas, or one's contribution; one can make a level table even if one dislikes the table and finds making it tedious. It makes sense to specially exert oneself to contribute to these areas, to drive oneself to work in them even though one would just as soon do something else. Now 'recreation' connotes, \term{general acognitive culture} is defined to be, exactly the opposite. One does the latter because one likes it (now), for no other reason. It doesn't make sense to try to do acognitive culture as objectively valuable, in conformity with objective standards. If one doesn't like what one does, it can't be acognitive culture. One can't create acognitive culture as a profession.

It is obvious, then, that Serious Cultural institutionalized activities, doing things in Serious Cultural institutional Forms, such as the Fugue, cannot be recreation, acognitive culture. What is not obvious, a point of this analysis, is that the whole institution of society's providing Forms (for the individual to do things in) supposedly for his recreation and self-expression, such as Science Fiction and Pole-Vaulting (or my Linact\footnote{
Linguistic acognitive cultural activity. Extant examples are my \essaytitle{Poem 1} and \essaytitle{Poem 4.}}), is absurd. The notion that the Forms are the real right ones, represent the real right thing to do, are objectively valuable, inevitably grows up around them. As an example, consider the Form of \enquote{Composition,} as any writing of specifications of activities (supposedly) for others to do as recreation. Compositions are primarily the writings, as opposed to doing the activities specified; their existence begins when the writings are completed. They are for \enquote{others} to do (and may never be done by anyone), showing that they are thought to be objectively valuable. The tendency is to turn out and store up quantities of them no matter whether the composer or anyone else likes them. Recreation, acognitive culture, cannot include Composition. Then there is the notion that given a Form, such as I am considering, one should do things in it whether one likes to or not, until one \enquote{understands} the Form, because one will like to then; and that the Form is objectively good if this happens. This has no place in recreation, acognitive culture. People who do things in these Forms all do so largely because they have acquired the notion that the Forms are the real right ones, are objectively more valuable than just anything. A proof of this is that the Forms are so extremely \enquote{objective,} common, impersonal. This is why one can be unable to tell anything about the people themselves from what they do in the Forms. People have no idea of the extreme extent to which they are socialized even in what they do for recreation, self-expression. Even being a writer of any kind, a maker of objects, a creator of works, in the traditional, established, and common sense, is already extremely objective, impersonal, and indicates that one is extremely socialized. (This is what was wrong with my initial notion of \term{acognitive culture}.) The reader may ask, if these Forms are so impersonal, what a personal Form will be like, how personal one can get. The answer will be given below.

Thus, an excellent determining principle is that it's pure recreation, acognitive culture only if it's what one would have done, would do, are doing, \enquote{anyway}; \enquote{prior} (to being \enquote{advanced} enough) to \enquote{know} the real, right, objective, the impersonal things to do, not from trying to contribute to an established real, right Form. Acognitive culture is not created by special exertion. One does it \enquote{anyway} \enquote{first,} and \enquote{then} it turns out to be in the category of \term{acognitive culture.} In fact, the concept of acognitive culture is only used applying retroactively. One doesn't set out to produce so many units of acognitive culture; one realizes that what one did which one would have done anyway was acognitive culture. What, then, is the reason for making the analysis, having the concept at all? Conscientious persons who have suspected the impersonality, of established Forms supposedly for their recreation and self-expression, have had great difficulty in repudiating the Forms, in not being ashamed of not contributing to them, not feeling that they have stopped doing anything. The reason for making the analysis, having the concept, is to help these persons with this difficult step, and to show those who are to give up the discredited activities what replaces them: to show that in giving them up they have not given up doing things just liked. So that they will \enquote{take seriously,} pride themselves on what they do just \enquote{for fun,} doing what they like, would do anyway; rather than being ashamed because they do not contribute to the discredited activities. The analysis, concept, is to make possible an attitude so one can thoroughly, consciously do things just liked.

Since acognitive culture is what one would do anyway, does entirely because one likes it, is for one's liking, it excludes entertaining others, conforming to another's likes---which are an intersubjective standard, making entertaining work. Further, on analysis, being entertained by another, another's creation, becomes questionable. Can the \enquote{creation} of another be liked by oneself, be for one's liking, represent oneself, as well as the \enquote{creation} of oneself? One may admire work by another, with respect to an objective standard, as being better than one's work with respect to that standard, but all that is irrelevant to acognitive culture. If it fits oneself who's doing the liking, if one allows oneself one's likings, then oneself is the source of value and, it would seem, will as a matter of course like one's creations best. Does it make sense for me to appreciate \enquote{great} chess players, poets, pole-vaulters (if their activities are to be regarded as recreation)? My point here is quite radical, but would seem entirely plausible. To go back, analysis of the concept of entertainment shows that separation of entertainer from entertained is incompatible with a thorough-going concept of pure entertainment; entertaining as work is discredited. This does not exclude every kind of involvement of others in one's recreation.

All this leads to the idea of (one's) acognitive culture as a part of oneself---as within oneself, at least so far as specifications are concerned. This would seem to be the opposite of contributions to impersonal Forms. Acognitive culture (being what one would do anyway) would not, it would seem, consist of artifacts built up outside of, separate from, oneself, to be gone back to (for ex. recordings, writings); or specifications one would have to be concerned about remembering. If one is wanting \enquote{what one likes, would do anyway,} one will have it; one shouldn't have to be concerned about retaining it.

The reader may have been asking, 'But may not merely what one would do anyway be less interesting than the pseudo-recreation which is created by special exertion, such as Flynt's \essaytitle{Reproduction of the Memory of an Energy Cube Organism}?' Strictly speaking, this question doesn't make sense: how could anything be more interesting to oneself, likable, than what one just likes, than what one would do anyway \enquote{prior} to \enquote{knowing} the real, right thing to to? However, I will give a heuristic answer to the question. Asking the question shows that one has as yet no idea of what specific doings would be included by the category of \enquote{acognitive culture} as I have defined it. They may well be so different from the discredited activities, the traditional, established, common real right Forms supposedly for recreation and self-expression, as to be irrelevant to them, so to speak. They are going to be indefinitely\footnote{incalculably?} more \enquote{new,} \enquote{different,} interesting, just as individuality is more so than anonymity. It is a matter of one's realizing that what fulfills the supposed function of the discredited activities are things one would not have thought of as replacements for them. All this will become obvious, when one considers what specific doings of oneself meet all of the specifications, are included by the category of \term{acognitive culture} as I have defined it. It may further be asked whether doing just what one would do anyway won't lead to a nihilism of acognitive culture's becoming indistinct, being absorbed in undistinguished personality, life, leaving only \enquote{nature}; or a nihilism that if acognitive culture needs to happen it will just happen, a nihilism of not doing anything. Well, something disappears, namely trying to do things just liked as a real right objectively valuable Form, a profession, by special exertion. However, acognitive culture doesn't disappear, because conscious organisms in any case just do anyway things just liked, which are distinguished, and which are \enquote{then} included by the category of \term{acognitive culture,} \enquote{people have their recreation}---the category of \term{acognitive culture} represents a selecting out of things which presumably the life of any conscious organism will include, for which there will presumably be a place in any life.

As I have mentioned the possibility that the reader may as yet have no idea of what specific doings would be included by the category of acognitive culture as I have defined it, it might seem in order for me to describe some examples of such specific doings. Actually, however, it is just not in the spirit of acognitive culture to try to describe such examples. Real acognitive culture is not likely to lend itself to reduction to words. And trying to describe examples of acognitive culture cannot but be a tendency to make them into works; actually, there is no reason why one's acognitive culture should mean anything to another, or even to oneself at another time. Thus, although I might informally describe examples in conversation, I am not going to try to write any up. The reader who does not yet understand what specific doings are included by the category will just have to study the specifications of acognitive culture some more, and then consider what specific doings meet all of them. When the reader does understand, then he can discover the parts of what he does anyway, already does, that are included by the category of acognitive culture: they are his acognitive culture.

This completes the elaboration of the concept of general acognitive culture. My proposal can now be seen to be plausible, that one give up the discredited activities, all established real right activities which would otherwise be retained as quasi-recreation; and have in their place \enquote{nothing,} except one's acognitive culture, or rather recognition of it. Now this chapter is relatively short, and the ideas in it are intrinsically simple. At the same time, it is of major scope; and it is socially radical, counter to major entrenched interests, institutionalized chess, institutionalized art, Olympic games, and the rest. In the past, there has been a tendency for people to read, but not \enquote{notice,} such writings. I want last to say something to counter any such tendency with respect to this chapter. This chapter may be short and simple, but it is what I have been led to, my complete conclusions, after years of contributing to art and post-artistic activities and thinking about aesthetics and post-aesthetic fields (in an attempt not to waste time as a result of taking the wrong things for granted). Further, it will be outrageous if this chapter is ignored, just bypassed, merely because it discredits major entrenched interests while being short and simple.

