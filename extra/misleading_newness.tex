\chapter[The Supererogatory, Misleading Notion of \enquote{Newness} (1960, 1975)][The Supererogatory, Misleading Notion of \enquote{Newness}]{The Supererogatory, Misleading Notion of \enquote{Newness} (1960, 1975)}

\signoff{\uline{From ``Culture'' to Brend}, Addition, Chapter 4.}

\vskip 2em

Quite apart from Serious Culture, metaphysics[,
Serious Cultural Neoism]; in \enquote{culture} a production is sometimes 
said to ha \enquote{new.} A production is sometimes said to be (positively)
valuable because it is \enquote{new.} There are controversies over whether 
productions are \enquote{new;}. and over what \enquote{real newness} is. There 
are controversies over whether \enquote{newness} is good or bad. In general,
there is the notion of \enquote{newness,} not limited to \enquote{culture}: things 
are said to be \enquote{new;} things are said to be valuable because \enquote{new} 
--- here is the vague, general, valuational notion of \enquote{newness.} 

A few \enquote{culture} producers, taking this existing vague
valuational notion of \enquote{newness} for granted, try to produce \enquote{culture}
(which is (for the present, to be appreciated now; all 
right, but) valuable entirely because it is \enquote{new};) which is
primarily \enquote{new,} is \enquote{new,} \enquote{different} \uline{as such}; without any
thought of other value, irrespective of its other characteristics.
In their attempt, one thing they do is the intellectualistic,
consciously experimental rearrangement of the elements of productions
or an activity just to obtain a \enquote{different} production.
One can play this little game indefinitely. Of course, what has
enabled artists to believe in rearrangement as much as they have
is that the results do have a little curiousness, surprise value.
The classic example is the projectors in \uline{Gulliver's Travels} who 
were trying to develop an ointment which would remove the wool
from sheep, and to propagate the breed of naked sheep throughout
the kingdom. The music concert without performers, the audience
without a concert, painting a brush with a canvas, and so forth
to infinity. Note the similarity to the central Dadaist techniques,
which are relevant because the Dadaist technique of satire (Dada's
principal purpose) is to change a thing so it appears to have its
original purpose, but can't possibly fulfill it. Then, thinking
about \enquote{newness} without regard for other value has led by several
paths (for ex., from taking \enquote{newness} as next in a tradition to
identifying anything as such a next thing) to the conclusion that
anything is new. Attempts to do \enquote{anything} naturally tend to take
the form of doing free-floating, purposeless, trite, simple things.
An example was my own rolling a tall across the floor, supposedly 
in the context of no activity or purpose. Then, they try to think
up arbitrary new purposes, new activities. An example was my 
attempt, when I first concieved it, to develop a percussion-sounds
ritual which would magically make a toy car roll across a desk. 
Finally, those who are a little more sophisticated theorize that
the appearance of newness has something to do with complexity and 
real purposiveness, and, although still merely trying to do 
something \enquote{new,} try to make their productions \uline{appear} to have these characteristics, giving a quasi-aesthetic experience of surface newness.

The notions of principal interest, the most problematic 
notions, the principal notions to be analyzed are the existing
vague valuational notion of \enquote{newness,} and the notion of \enquote{newness}
\uline{as such} (irresoective of other characteristics). (Incidentally, 
such \enquote{newness} cannot be identified with the exciting, the shocking
as \enquote{new} sometimes seems to be used to refer to; certainly the
most exciting, shocking things are not \enquote{new} in any sense, but
are as old as humanity and well-known to it --- religion, obscenity,
violence). The key point is that valuational \enquote{newness} is, "newness'
\uline{as such} \uline{as a value} must be, valuational notions. In the
non-valuational senses, everything can be considered \enquote{new}; but
the connotation of the notions of principal interest here is that
only selected things \enquote{really} deserve to be said to be \enquote{new} ---
one speaks of \enquote{real newness.} The best explication for the term
\enquote{(really) new} here is that one applies \enquote{new} approvingly to a 
thing \uline{one is encountering for the first time}, which one finds 
\uline{has some major value} quite irrespective of \enquote{newness,} quite
irrespective of whether it is \enquote{new.} The \enquote{newness} of interest here 
is best explicated as not a \enquote{primary} value or characterisic of
a thing, but rather an extra, \enquote{accidental,} \enquote{secondary} characteristic
a thing, which has some major value quite irrespective
of \enquote{newness,} can have; the characteristic of being encountered
for the first time. My conclusion readily gives the solutions
to all the problems about \enquote{newness.} The notion of a thing having
just \enquote{newness,} \enquote{newness} \uline{as such} irrespective of its other
characteristics or value) as a characteristic, as its value,
is absurd, inconsistent; represents taking a \enquote{secondary} characteristic
as a \enquote{primary} value, represents a confusion of the
formal and the substantive. The case of \enquote{newness} \uline{as such} is like
the case of \enquote{ability} \uline{as such} or \enquote{freedom} \uline{as such} or competition
\uline{as such}; it represents taking a formal matter, a matter of context,
as a substantive matter. (As usual, the mere formal matter
isn't worth making an issue of, thinking about.) 

In fact, it can be concluded that it is better to omit
the issue of \enquote{newness} in determining whether a thing is valuable.
The thing is \enquote{new} only if it is independently valuable, can't
be known to be \enquote{new} before it is known to be valuable (and anyway,
even if it is valuable, its \enquote{newness} is only a matter of when
you happen to encounter it). And, thus raising the issue of 
\enquote{newness} does lead to the notion of \enquote{newness} as an independent,
primary value, and to resultant confusion. Further, \enquote{new,} in
the neutral uses I listed, can easily be eliminated, by replacing
it with the underlined equivalents I gave for it. Thus I see no 
case where the term is uniquely useful: the notion of \enquote{newness}
is supererogatory. As the term \uline{is misleading}, I suggest that it
be consigned to oblivion, at least as a term for rigorous discourse.
\vskip 1em
\signoff{[\enquote{Newness is dropped.}]}


