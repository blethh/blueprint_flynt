\chapter{Personhood II: Attachment's Turbulent Causation (May 1981, rev. 1991)}

CONTENTS

    A. Preliminary: Ordinary Personhood

    B. Object-Perception and Personal Identity

    C. Longitudinal Identity and Modes of Existence

    D. Language and Human Self-Image

    E. Language and Thought

    F. Propositional Knowledge and Personal Identity

    G. Attachment to an Experience-World

    H. Mental Stability and Biographic Identity

    I. Other People and Self-Objectification

    J. Culture as a Phase Discriminated in the Person-World

    K. Community; Society as a Grandiose Other

    L. The Ostensible World as a Delusion

    M. Imminent Character as an Invariant to Psychedelics

    N. Thematic Personal Identity

    O. Fixation to a Cumulating Social Role

    P. Attachment's Turbulent Causation

    Q. The Determination of Personal Fate

    Afterward: An Orientation for Personhood II

[I presented my first "paradigm of personhood" in a manuscript of December 5, 1980. I followed it with a series of critical manuscripts; then I embodied critique and elaborations in a second paradigm, Personhood II (May 1981).

In this revision of Personhood II, I proceed at once to the exposition. Context for the venture is supplied by my other writings on personhood, especially "Personhood IV" (1984; 1991).]

\section{Preliminary: Ordinary Personhood}

"Ordinary personhood" is the realm of functioning which encompasses the following.

There is a bonding of my direct awareness (including feelings, urges, moods) to "objectivities." I interact with objectivities fragmentarily and sequentially while conceiving them as persisting wholes. I seek logico-perceptual coherence of the objectivities, sorting them out by identifications, distinctions, memories, expectations, appellations, etc. (It may be helpful to note that objectivities have a circumstantial and hearsay character.) A definite logico-perceptual collation of objectivities is called a perception-world or perception-reality.

I can act, producing change or expending effort. (Mental action, somatic action, action upon exterior objectivities are all included.)

    a. I can realize a preference in action: implemented choice or willful action.

    b. I may act contrary to my preference: "loss of self-control."

    c. There is a spectrum of actions between those which are acutely willful and those which are acutely unwanted: habit, being enthralled, lassitude, etc. 

I can be self-conscious: direct awareness.

I can fantasize, etc.: imagistic mentation, etc.

More fundamental than the above: "the totality" is polarized as self and non-self (or world). The features characterizing self are centered activation, presence, drive. These features can be attenuated in a fever and in some other modes of existence.

*

B. Object-Perception and Personal Identity

1. The perception-world is polarized into well-behaved perceptions on the one hand, and perceptions which are shared or replicable but which are segregated as misbehaved, on the other. The latter perceptions are called illusions, multistable figures, intermediate-zone perceptions (a phrase I coined for e.g. ringing in the ears), etc. It is normal to experience

    the waterfall illusion,

    the crossed-fingers tactile illusion,

    the half-immersed dowel which seems bent,

    contact at the tip in tapping with a stick,

    double image of a dowel held vertically in the visual field,

    perspective-reversal of the Necker cube,

    the Necker cube with concurrent incompatible orientations,

    ringing as the after-effect of a bang,

and many other examples with which I assume familiarity. These perceptions are segregated and stigmatized as perversities. In effect, the segregated perceptions carry a label which says

    YES THIS HAPPENS BUT IT DOESN'T COUNT

Why? Because the illusions are already incongruities in the mandated reality. They belie tenets such as the following.

    a. detachment of object from awareness (cf. the Necker cube);

    b. intersensory unity of the object (cf. the half-immersed dowel);

    c. contentlessness of inconsistency (cf. the waterfall illusion, the paradoxical Necker cube).

[1991. The dichotomy of veridical and illusory perceptions is required in the intensive analysis to follow. But I may note that this dichotomy understates the problematicity of perception. The straightforward perceptions are achieved by tendentious selectivity and mental reversal of the sense-evidence. Veridical perceptions are something like habitual paranoid imputations to sense-contents. You continually seize on obscure cues in the apparition to mentally twist the apparition into your pre-selected theory of the substantial world. "An object in front of a wall is really a shadow on the wall if it doesn't move relative to the wall when you move." All perception involves cues which you learn to spot, a pre-selected theory of the substantial world, and a twisting of the apparitions into the theory.

The dowel touched to the tips of crossed fingers vs. the dowel placed between the tips of uncrossed fingers. In the latter case, you "truly" sense one dowel; but this is false to the sensation, which is of two finger-contacts. (Hold a two-stick sheaf between uncrossed fingers: now each finger contacts a different stick, and you falsely perceive one stick.)

In looking at a dowel held vertically in the center of your visual field, there is an alternative: your gaze falls "in focus" or "in the distance." Parity of the right and the wrong.

And consider the cartoon of the climbing bear. You imaginatively adduce an entire bear from outline paws on a outline trunk. What is more, a different reading of the image is possible.

(figure)

2. These considerations pose the question of where personhood theory is positioned (showing the answer of the 1980 theory to have been inadequate). The analysis has not yet ventured beyond my "mind." Yet I sense the presence of culture: which dictates to me which perceptions are well-behaved, and which designates some shared, replicable, normal perceptions as misbehaved (even as threats to reason and logic--as with the Necker cube and paradoxical Necker cube). The discussion hasn't even arrived at verbalization, I am still talking about non-appellative sight, touch, etc., and already I am faced with culture's determined segregation of certain normal perceptions because they threaten reason and logic.

It may be that this discussion pertains to one culture more than others--namely modern rationalism--which sees these illusions as threats which have to be segregated. So this discussion may pertain to modern rationalist personhood; and other cultures may have treated the person-world constituents differently. [Well, this may hold as a generality; but facile relativism is not helpful. Other cultures are just as pragmatic and as stern as ours in insisting that perception find the substantial object. Does China or India want you to believe that the half-immersed dowel is bent? I will leave it open whether I am speaking of culture in general or of the present culture.--1991]

3. I now begin the intensive exposé. The standard illusions are supposed to be replicable; but my personal researches have found that a number of "unimpaired" individuals do not experience them. These individuals happen to be involved in natural science as a career. There is an obvious speculation: that the vested interest of these individuals in reason and logic is so great that they have to block normal perceptions which mock reason and logic.

(I don't know if a psychology experiment would confirm my findings. I met the scientists while they were off their institutional platforms, and I challenged their vocations. They knew that if they admitted seeing the illusions, they would lose arguments with me about their lives. Nobody would consent to having their life shown up like this as an experiment.)

There is a species of perceptions stigmatized as undesirable by the culture which are not interpersonally replicable. I refer to hallucination, and to fantasy so intense as to verge on hallucination. (Cultural psychiatry finds that half the population has at some time had a hallucination of a deceased relative.) This area allows an observation which complements my observation about the blocking of normal illusions by scientists. Evidently there are, or were, identifiable groups in the population for whom it is culturally more acceptable or normative to have non-replicable illegitimate perceptions.

In any case, my contacts with scientists show that I want to involve this analysis in respects in which individuals differ. I don't want to be limited to my unique self/world relationship, or to a universal self. In order to lay open simple, non-appellative perception, I have to acknowledge groups of people, and to acknowledge that the present culture, specifically, mandates a slight specialization in reality as between groups. (Appropriate perceptions for scientists as opposed to housewives.) Of course mandated norms and group behavior need not be borne out by every individual.

I began, in (A), with the sense-of-self, being an "I," as fundamental. Does the culture mandate that the sense-of-self should have different degrees for different groups? In speculating about social groups, I don't want to descend to facile social psychology. Moreover, the mentioned groups might differ at a level less elemental than selfhood. The differences I have noted could be culture-correlated character differences. All this will be developed below. Selfhood and character need to be distinguished; but they could also overlap.

Let me return to the scientists who do not even experience normal illusions. The circumstance that there are shared perceptions which are approved by (the) culture seems to require no special justification. But what of the circumstance that there are perceptions which are disapproved by the culture, but which nevertheless are very widely shared, but which however fail to be experienced by a handful of zealots of rationalism? If normal interaction with the Necker cube is acquired only through being taught--or is an imposed deformation of the psyche--then the implication is that the culture vigorously instills perceptions it doesn't want people to have. As for the zealots who don't have these perceptions, are they manifesting deficiency, or repression? Considerations in (C) below suggest that they are manifesting repression.

[1991. I can no longer postpone the question of where this investigation is positioned: the methodological equivocations are crushing. I adopt the standpoint of my self/world relationship, yet periodically I shuttle to depositions about cultures (including those to which I do not belong), and about other people's subjectivities. In the preceding paragraph, I went so far as to discern a conflict in the culture's mandates (as if I were ascribing conflicting wants to a grammar of people's behavior). Moreover: who is my reader?

In this vintage view of Personhood II, the answer is that at first I allow myself to speak of culture and of other minds as externalities with which I am unaccountably conversant. Later it turns out that I am making discriminations within my self/world relationship. (Example: I do not pronounce "sure" the way I spell it. That is my behavior, but I don't take the credit for it.) Then I explain mandated (fantasized) objectivities via these observations. It is also crucial that the investigation does not have to yield an affirmative creed. (I am unfolding the incoherence for instrumental purposes.) So there is a rotation from conventional reality-assumptions to the person-world in the course of the essay--as I speak about language, other minds, culture, etc. The definitive explanation is in "Personhood IV," and I should emphasize that it is not simple. My principles of astute hypocracy and of levels of credulity are involved. I have chosen not to sort out this version of Personhood II because that would eliminate its vintage value; moreover, the essay would become too counter-intuitive for the uninitiated reader.

As a sidelight, I may mention common sense as a vernacular world-model. Common sense

    a. privileges those perceptual gestalts which are held to be material realities;

    b. tries to abstract the world from any idiosyncratic standpoint; and

    c. declares my mind, and other minds, to be limited entities in the world.

Natural science presupposes the common-sense world-model operationally. At the same time, common sense is rationally indefensible, and is known to be so.]

*

C. Longitudinal Identity and Modes of Existence

There is a conceptual partition of my existence into

    waking,

    dreaming,

    hypnagogic hallucination,

    "morning amnesia" (a phrase from my "Critical Notes on Personhood"), fever,

    psychedelic episodes,

etc. The culture mandates this partition. It demands that I keep track of whether I am awake or dreaming. But in a given state, I may not assign it to its classification; and at times I am not capable of assigning the immediate state to its classification (e.g. in the dreaming/waking dichotomy). In order to keep score, I have to view my existence in retrospect, and classify entire episodes and modes of existence differently from the way I classified them while they occurred. The circumstance that this distinction is demanded is of the greatest importance. Even if I still limit the discussion to the logico-perceptual collation of objectivities (to the way I glean substantial objects in perception, discern the insubstantiality of shadows, etc.), this collation manifests asymmetrical variation in tandem with dream/waking alteration. With respect to waking states, there is a consistency of collation from one state to the next. The waking collation is stipulated by the culture to be standard and to be desirable: my visual vantage-point never moves outside my body; shadows never detach and become objects; etc.

Collations in dreams, hypnagogic hallucinations, psychedelic episodes, etc., are variable and idiosyncratic. The culture construes them as threats to reason and logic and demands that they be segregated. The segregated states are assigned a label which says

    YES THIS HAPPENS BUT IT DOESN'T COUNT

Again, these considerations pose the question of where personhood theory is positioned, and show the 1980 theory to have been premature. That theory pretended that my existence could be a continuum of waking states only, so that it was only required to portray the waking, or standard, or desirable logico-perceptual collation of objectivities. But that was a misrepresentation; and the theory is now faced with the following choice. If the paradigm is a paradigm of the immediate moment, then collations and indeed the entire "synthesis of a world" are so different as between dreaming and waking existence that more than one paradigm is required just for one ordinary person. But if all of one ordinary person's existence is to be covered by a single unified paradigm, then that paradigm must allow for variations in "the system of synthesis of a world" in short-term personal history. But then, even though the discussion is focused on the immediate moment, I am already confronted with personal history and retrospection or memory. I claim (or am directed by the culture to claim) an extrusion "behind" the immediate moment which is all "me" and only "me" even though it incorporates drastic variations in "the system of synthesis of a world." (And that is not even to mention that in dreams my identity as myself can be compromised or confused even in the present moment; and that in fever and morning amnesia my selfhood can be thinned out or shut down.)

That is not all. I am required to make the retrospective judgment that a mode or episode of existence was a dream even though I judged the state to be a waking state during its occurrence. I am required to make conceptual, judgmental connections of my present with my extrusion behind the present (longitudinality), and to characterize whole phases of my existence as something different from what they were as they occurred. Here we have the phenomenon of delusion in the conventional sense. The culture requires me to confess that my whole existence can be a delusion (as often as once every night). But mightn't my whole existence at any present moment then be a delusion? The culture assigns this question a label which says

    YES THIS IS UNANSWERABLE BUT IT DOESN'T COUNT 

Viewed from another angle, I can make an observation similar to one I made about illusions. The culture vigorously instills in me the capacity to ask a question it doesn't want me to ask. And there is a further parallel with illusions. A handful of individuals remember no dreams. Once again, some of these individuals are scientists, and there is an obvious inference that they have to block phases of their whole existence which threaten reason and logic (which are equated with the waking "system of synthesis of a world"). Perhaps other individuals who never remember dreams are uncomfortable with sexuality; etc. I propose that scientists who don't experience illusions or remember dreams are manifesting repression rather than deficiency. But this is extraordinary. It means that scientists mutilate their basic perception: that they perform a feat as remarkable as cancelling all shoes out of reality or cancelling all eating out of reality. As for the majority of people, they are, if anything, in a stranger condition: the culture forces them to undergo phases of existence which it doesn't want them to have or notice in some respect.

*

D. Language and Human Self-Image

1. Let me focus on language as one of the objectivities. Language is a peculiarly configured heterogeneous phenomenon, of quite a different order from the objectivities I have already discussed. At one level, language consists of "physical events" whose important feature is that they can be duplicated--the tokens. At another level, these tokens occur or are produced in patterns (cf. moves in chess). At yet another level, "comprehension of a message" requires the addressee routinely to associate ideation to the token-pattern with which he or she is confronted. A fourth feature is that changes in pattern correspond not only to differences in ideation but to different "modes of address": statement, question, command, definition. And one mode of address, the statement, has two alternative functions. It can picture or portray (narrative fiction). It can also claim or avow.

If I write on a chalkboard "There is a chalkboard in this room," a curious circuit is established: from patterned smudges on a chalkboard, to associated ideation on the part of whoever reads, back to the chalkboard or the reader's perception of the chalkboard (establishing that the proposition is authentically descriptive or "true," according to traditional wisdom). Of course this account is simplified, one-sided, and unfashionable; but it focuses some important peculiarities of language.

Here I may have to affirm that I am talking about the person-world in modern rationalist culture. Modern rationalist culture is comfortable with things or objects, and with "social" phenomena as thing-to-thing relationships (e.g. a command to close the door has been understood if the addressee closes the door). Modern rationalist culture is phobic toward subjectivity, thought, mind. Thus, accounts of language relentlessly seek to exclude ideation from the linguistic process--and to exclude name/referent connections also. Language is conceived solely in terms of its thingist extrusions; images are provided of language as tokens, token-patterns, and behavior. Of course these rationalist images of language are misrepresentations. For example, the circumstance that an addressee does not obey a command does not at all prove that he or she has not understood the command.[1] And if language were no more than token-patterns, it would not be capable of describing token-patterns. (The rules of chess cannot be formulated in chess moves.) But what is most instructive is that so misrepresentative an image of language could have achieved any plausibility and acceptance at all. Natural-language use is a remarkable species of activity which connects subjective mentation, perception, and subjective intentions (cf. lying) to physical events, and their patterns, and overt behavior. The point is that the physical events are sufficiently separable from the subjective ideation that scientific linguistics can pretend that there isn't any subjective ideation, and not be ridiculed into oblivion.

[1991. This essay does not treat language, as common-sensically believed to exist, exhaustively. On one hand, language is a phenomenon of consciousness. It involves ideation of meanings and the speaker's wants. Indeed, with respect to conscious understanding, language is comparable to despair or romantic affection in being a "generic subjectivity." On the other hand, no individual's mental contents account for language as common-sensically believed to exist. In that perspective, the individual merely "borrows" natural language, which has a grammatical agenda that remains opaque to native speakers. But to pursue the alienness of language to the speaker in the obvious way would, again, stray from the person-world orientation. This alienness has to be treated as I will later treat culture. A further consideration--the existence of natural languages which I know of but don't know--exposes the person-world orientation as highly counter-intuitive. Again, "Personhood IV" is devoted to confronting these junctures vigorously.]

2. There is an arcane aspect to the speaking person's involvement with ideation which I wish to discuss here. The scientific linguist says "I want to talk about token-patterns but not about subjective ideation." But how can this demarcation, this non-interdependency or non-interpenetration, be contrived?

When I count a row of objects silently, then token-patterns are subjective ideations. The conscious "observation" of a pattern in a congeries of simultaneously present, persistent things is subjective ideation.

Indeed, at some point I will have to recognize that patterns in things are assertionally imputed. Cf. the six-bar image in which any of three patterns can be seen, and which the initiated can see without pattern.

figure

Then, mere patterns cannot make claims about--i.e. describe--patterns.

And when the linguist identifies the printed number series

1, 2, 3, 4, 5

with the vocalization of that sequence or the silent reading of that sequence, he or she declares language to be a phenomenon in which a manifestation of simultaneously present, persistent things is the same as a succession, of discrete sounds or subjective mental events, which appear and disappear in time. In the scientist's self-interpretation, this "knowledge" that "they're the same" must subsist and be validated without subjective thought. We are supposed to know that a manifestation of simultaneously present, persistent things is a succession of subjective mental events which appear and disappear in time--without any involvement of subjective thought in this uniting of incomparable phenomena. But the lesson is that the claim of language as a thingist structure presupposes wildly imaginary reality-types. Let me anticipate and mention the case of different selves claiming the pronoun "I." The full meaning of this locutory protocol cannot be explicated by scientism.

The culture's replies to the observations in this subsection are

    YES THIS HAPPENS BUT IT DOESN'T COUNT

and

    YES THIS IS UNANSWERABLE BUT IT DOESN'T COUNT 

3. The challenge for person-world analysis, then, is that language connects physical events to subjective thoughts in a way which lends credence to the denial of that connection. It is not enough, to support the culture, that there should be a a medium of communication (and avowal). All recognized media of communication (including music as well as speech) must be capable of being pictured as physical events independent of subjective thought. The culture requires communication and doctrinal loyalty without thought and mind. If the medium of communication did not have a separable thingist facet, then it would continually, blindingly belie the thingist ideology of the culture.

The culture cannot subsist on the basis of means which straightforwardly and candidly perform their functions. It must have means contorted to seem so different from what they are that we finally, impatiently, say "they are required to be not what they are."

Can the tortuous conformisms which are being elucidated be supposed to subsist without stress, force, or fear?

*

E. Language and Thought

I turn now to the involvement of language in my immediate existence and state-of-action. Through language, I name phenomena, formulate expectations, etc. Via language, I make factual judgments (or espouse beliefs) about objectivities--thereby further determining the objectivities. I may conceive my beliefs to be guesses; or I may conceive my beliefs to be assured facts (yet I can find them to be refuted on their own terms by subsequent occurrences).

1. Consider the future, the next moment--the future toward which urge and action are directed. I have saved the topic for now because avowed expectations, and thereby a future conceived as a future, are inseparable from linguistic expression. (Do animals have avowed expectations, do they make express predictions?--as when a cat crouches beside a mouse hole?)

An avowed expectation is not arisen when one merely enacts a future in fantasy non-verbally: what the latter lacks is assertion (and the capacity therefor).

If discomfort impels action, without verbal thought, a representation of a future has not arisen. Experiential memory can be taken as assertive, but that is because the past is taken as being already decided. So the experiential or non-verbal memory is conceived as an echo relative to an assured actuality, an episode lived through. As for the future, and non-verbal anticipation or projection, they are not taken as having the relation of a decided event to an echo. The notion of remembering the future--i.e. of rigorous pre-cognition--is a minority notion, not found in the consensus. The mainstream, as I know it, conceives of non-verbal anticipation or projection of a future as being fantasy and nothing more, until it has been implemented and can be represented as a past.

2. Consider the claim that language solves the problem of intersubjectivity, that it guarantees that observations are communicated. Suppose I and another person stand before a house. Suppose the other person says "I see a house." The culture's preferred interpretation is that this interpersonal corroboration proves the objective reality of the house. But the culture also gives me the capacity to speculate that the other person is lying to please me. Or to speculate that the other person sees what I would call an elephant and calls it a house.[2] Pragmatically, we curb such miscommunication by prolonged cross-checking. But in the moment--or in principle--the alternatives are indistinguishable. The culture assigns this observation a label which says

    YES THIS IS UNANSWERABLE BUT IT DOESN'T COUNT

Again, the culture has instilled in me the capacity to see a gap which it doesn't want me to see.

3. As another aspect of language, let me focus on the relation of conceptualization to perception in connection with the significance ascribed to logical consistency . For a concept x to be well-established, there must be a decision program which splits the world or the multiplicity of picturable (possible, intensional) entities in two, and attaches x exactly to one section and non-x exactly to the other. Non-x is the exact reverse, in a partition of everything, in which x is the obverse. To say this is to proclaim that x and non-x do not apply simultaneously to any entity which may be under consideration. Such is the basis of the culture's tenet that x is not non-x or that "x-and-non-x" is an empty concept, a null concept.

Language has an arbitrary, stipulative character in the sense that the word "prestidigitation" is expendable relative to the word "legerdemain," etc. At another level, however, language is not "optional" at all. Conventional thought requires stable, agreed-upon distinctions. The community indoctrinates its new members in language as an integral part of enculturing them generally. Thus, there emerges a close correlation between linguistic categories and the individual's ingrained interpretations of sensation--what is called perception. In the process of enculturation, perceptual distinction becomes deeply correlated with linguistic distinction. Recognizing this correlation, linguistic distinctions can no longer be considered "mere" stipulations. [Again, I am unaccountably making social psychological declarations. The discussion would have to be derived differently to become uniform with the person-world standpoint.--1991]

Relative to a given concept x, we have a conventional, ingrained, perceptual and linguistic program to attach x, or else to withhold x and attach non-x, to everything we may encounter, every picturable entity. But in certain cases, we are confronted with a picturable entity which, our decision program tells us, we must simultaneously call an x and a non-x. That is, our ingrained perceptual routines tell us we must simultaneously call it an x and a non-x. The choice of labels here is not optional or whimsical; it is as mandatory as appellative judgments can be.

A case of such a picture, or visual image, is the waterfall illusion. (And we are back to misbehaved perceptions and B.1.c.) One's perceptual routines are disoriented; one's capacity to use concepts at all and to keep reality tidy begins to crumble. The experience of a logical impossibility in the waterfall illusion cannot, again, be dismissed as a mere effect of whimsical appellative stipulations. While any given word is arbitrary relative to the existence of foreign languages, etc., in practice our capacity to use concepts at all results from a community consensus which is linked to our most ingrained perceptual routines. To repeat, inculcation of logic connects with the designation of illusions as misbehaved perceptions, as treated in (B).

*

F. Propositional Knowledge and Personal Identity

1. Aside from the action of illusions as conceptual-logical anomalies, natural-language conceptualization is pervasively inconsistent in the sense that inferential extrapolation of any concept at the level of assertional discourse, relative to culturally mandated doctrine as a whole, will yield contradictions. In the present account of concrete, substantive features of the person-world, we continually encounter incoherencies--whose propositional expressions comprise paradoxes. At the level of doctrine as a separate activity, common sense (the culturally mandated conceptual medium of ordinary apprehension of the world and ordinary interpersonal interaction) can be codified. Numerous contradictions in common sense are listed in my "Paradoxes of Common Sense" (1988).

I find the culturally mandated conceptual medium to be a disguised biased inconsistent system: naive inferential extrapolation will yield inconsistencies everywhere. Some of these inconsistencies are welcomed or at least tolerated, while others are suppressed; but the self-image of the medium is that it is not inconsistent, or that it can be repaired and made consistent.

Thus, it is indispensable to the culture to deploy a conceptual system which quite literally violates its own foremost logical principle that x should not be non-x.

It was very difficult for me to admit how the bias of the inconsistent system was maintained, that is, how wanted contradictions were divided from unwanted contradictions. The rationalist culture's mystique suggested that the "grading process" would have to be described by hundreds of pages of intricate symbolic logic. What the culture's orientation did not encourage me to recognize was that the mystery of grading is a matter of barefaced lying defended by naked interpersonal coercion--yet that now seems to be the main secret of grading. As this account proceeds, I will mention numerous methods by which culturally mandated incoherencies are sustained.

2. One abstract incoherency in particular repeatedly comes up. Space and time are identified although they are qualitatively incomparable. Already this identification was invoked when I said that my past "extended behind" me. Perhaps movement can be offered as a basis from which these correlations which equate space and time can be derived. But in any case, the identification of qualitatively incomparable aspects of experience is not something that can ever be "proved to be true." (It goes without saying that many other aspects of time, such as the procession of the present moment, are occasions of paradox.)

3. Let me briefly consider abstract knowledge, as an aspect of language and conceptualization. I take mathematics as the case in point. How is it established that it is proper to use the same numbers to count qualitatively different species of entities? In particular, what is the relation of the "real-world" enumeration which metamathematics must use to count token-occurrences in mathematical expressions to "real-world" enumeration generally?

Then, to count a row of objects, orally or silently, is to pair a manifestation of simultaneously present, persistent things with a succession of events which appear and disappear in time. How is it established that the result of this procedure is meaningful?

Or consider 1/0, 0/0, and 00. An individual who pursues mathematics

in seclusion, naively performing the indicated operations on the basis of an initial understanding of the symbols, probably will not arrive at the solutions to 1/0, 0/0, and 00 mandated today by the mathematics profession. Authority is required to stipulate which answers are institutionally significant.

4. From the beginning, when I was only talking about non-appellative perception, I said that one strives for logico-perceptual coherence of the objectivities. By now, it is clear that such coherence is never remotely achieved. Yet one typically does not judge oneself to be insane; and one typically feels that there is an adequate degree of coherence of the objectivities; because the gibberish one espouses is culturally approved gibberish. One who is under the action of interpersonal approval can live so amicably with barefaced lies as to crazily experience them as constituting coherence. We are already confronted with extreme phases of knowing self-deception in tandem with the culture.

*

G. Attachment to an Experience-World

1. A situation in which objectivities are conjoined with my feelings, urges, expectations, and anticipations fixates me on a system of factual judgments and a system of actions. In more detail, the situation fixates me on a logico-perceptual collation of objectivities, on a system of factual judgments, on a method of ascertaining facts, and on an action-system or praxis (including skills, judgments of feasibility, etc.). I can be fixated by anticipation (involving discomfort, fear, or hope), by emotional dependence on other people, etc. This situation may be called an attached state of consciousness (as distinguished from detachment). Attachment does not have to be all-encompassing; I can be attached in part and aloof or contemplative otherwise. Attachment must not be thought of as the outcome of a pragmatically calculated choice; it is altogether involuntary while it occurs.

Roughly, an attachment-content consists of a logico-perceptual collation of objectivities, a system of factual judgments, a method of ascertaining facts, and skills and judgments of feasibility. Attachment-contents would vary from person to person. At the same time, there would have to be considerable interpersonal congruence--otherwise there would be no human mutuality. As I say throughout, enculturation's aim is congruence. [Here again I unaccountably speak of selves other than mine.]

In regard to the "factual aspect of perception," to say that I am fixated on a logico-perceptual collation of objectivities is to say that I habitually "impute contexts of objectivity" to my sensations. When I see a parked automobile, I automatically assume that it is an object, with a reverse visual side, with tactile solidity, etc. Indeed, by mentally denoting the apparition as an automobile, I use language to express just these assumptions. [And here I have finally come to the problematicity of straightforward perceptions as mentioned in B.1; and to the example of the "climbing bear."]

2. Although it is normal in the waking state to be attached, it is also normal not to be attached, so that the option of suspending a familiar, habitual belief is available if I want it. Indeed, when confronted with a dowel partly submerged in water which appears to bend at the water's surface, I have been warned that I will find an exception to the prescribed belief that "sight and touch will correspond and thereby prove that my senses apprehend a single objectivity."

When I have the option of suspending conventional beliefs, then I can modify the conventional determination of reality by exercising this option. But in dreams and some other states, I completely lose this option of suspending belief.

I call the difference between a state in which attachment is partial, and a state in which it is all-embracing, a difference in "cognitive morale."

3. In an attached state, I am impelled to realize preferences in action, based on anticipation and on the factual judgments and the praxis to which I am fixated. I am imminently forced to formulate preferences on the basis of guesses and to realize these preferences in action. While a condition of "operating on automatic pilot" is possible for me (habit?), I am here forced off automatic pilot.

The ideology of determinism which says that my preferred action is predetermined by prior objectivities is meaningless and useless to my praxis here. This is where so-called empirical freedom of the will comes in--but my characterization is a great improvement on "freedom of the will."

I can rehearse for a future situation in fantasy.

*

H. Mental Stability and Biographic Identity

1. Not only can I observe and denote; I can "observe" and denote my functioning.

I can pass judgment on myself with respect to my performance and my level of satisfaction.

a. I can judge whether my actions are effective, and can judge what achievements are feasible for me. (Pragmatic self-judgment.)

b. I can judge my level of fulfillment, especially according to the standards provided by the beliefs I am fixated on. I can judge e.g. whether I act out of intimidation, or out of loyalty and affirmation. (Cathetic self-judgment.) But I must add that the discussion has shown that there is a whole layer of intimidation which is palpable but which is nevertheless overlooked or "not counted." An illustration is the scientist whose dreams, and experiences of illusions, are amputated from his life.

c. I judge whether I am "sane." Here "sanity" is used in a vernacular sense as pertaining to my composure regarding my cognitions. "Insanity" is a feeling of cognitive insecurity and panic, possibly accompanied by disordered desires and moods and conative futility. (So I am not concerned with the man who smugly believes he is Napoleon and rationalizes all external evidence to the contrary.)

My judgment of my sanity involves whether I am maintaining the pretense of logico-perceptual coherence of objectivities; whether my desires and moods are sufficiently coherent; and whether I can act effectively more often than not. It can also involve sophisticated issues of my relation to the community which will be stated in L.5. It is not usual, incoherent thinking which makes one feel insane. Rather, having usual thinking exposed as incoherent may make one feel insane. Yet such an exposé will not discompose most people because they don't take matters of principle seriously--a topic I will address intensively in later sections.

Tampering with personhood immediately places sanity at risk.

2. I have a biographic identity: a remembered personal history, and a projection of my future. But we have already encountered the issue of personal identity relative to personal history and memory; and we saw that the persisting personal identity was complicated by such modes as fever, morning amnesia, and above all dreaming. What is now the lesson? That the 1980 personhood theory corresponded to the the personal identity which the culture mandates for me: the fiction that "my life" can be a continuum of waking states only, so that "the system of synthesis of a world" which is consistent through my waking states is the only personal identity I have to cope with. As for the phases of my existence which clash with this solution, the culture assigns them a label which says

    YES THIS HAPPENS BUT IT DOESN'T COUNT

3. Out of context, the phrase "knowing self-deception" seems paradoxical. But paradoxical or not, by now it is beginning to seem the most frequent feature of personhood. So far, I have discussed culturally mandated self-deception; but there is also knowing self-deception as a personal variation or idiosyncratic adaptation.

One important and extreme procedure of knowing self-deception is to obtain gratification from mental play-acting. When a personally motivated representation of your identity is concerned, you engage in a representation of yourself which you know is untrue to the present (and to past and future as well). (A Walter Mitty fantasy.) But also, a group of people can engage in mental play-acting with respect to impersonal doctrine. My phrase for this [as of 1991] is a shared pretense or consenting sham. Culturally supplied doctrines seem pervasively to have the character of consenting shams. Where these culturally approved fantasies are concerned, the mere circumstance that a doctrine is a manifest lie or that an activity or enterprise is a manifest fraud is not an objection to it. The manifest lie is accepted as a source of gratification. The individual lives amicably with the lie. The lie is sustained by non-"cognitive" motives.

Other procedures of knowing self-deception are presented as personal adaptations; but certainly they can be systematically encouraged in groups of people.

a. I can suppress painful self-consciousness by frantically affirming what I doubt or disbelieve, or by repressing what I suspect.

b. Instead of acting upon objectivities to get what I want, I can withdraw to gratification in fantasy, or imaginary gratification.

c. I can becloud the imperative of implemented choice-making, in order to dull its risks and loneliness. (One of the most vivid risks is that of subsequent self-condemnation if my choice proves to be regrettable.) Affirmation and denial of choice-making are both intensified by repetition, by habituation.

*

I. Other People and Self-Objectification

1. It has taken a long discussion to reach the stage where it is appropriate to focus on other people. I can identify other people as objectivities. One implication is that I interact with other people fragmentarily and sequentially, and only thus; nevertheless, I sort them out as persisting wholes.

Given other people as objectivities, I can additionally conceive them as conscious, willful counterparts of myself. Other people have consciousnesses which are counterparts to my own but to which I have access only by the actions of their bodies (e.g. speech). It is implausible to deny other consciousnesses, because I identify so many of my thoughts as penetrations of myself by the others' consciousnesses. (This is the issue of culture, which I will focus on below.) But then other people have mental lives in which I do not participate and whose conclusions can be withheld from me. Other people, whom I encounter as elements within "my" world, have "worlds" of their own which meet mine but are in another respect wholly outside mine. I conceive of other people as objectivities among other objectivities, and yet as being persons to themselves, and as "locations" of minds which can meet my mind. (I may further conceive of other people's implemented choices as being determined in advance by objectivities--so that they, who are counterparts to me, are in a deterministic process while I, in the moment of realizing choice, am not.) These notions are massively incoherent. And they force me into even more extreme incoherencies in order to apprehend another person's death.

A case which further complicates the conception of other people is any encounter I may have with the mentally deficient. In this case, it is no longer "obvious to common sense" that other people are counterpart sentiences to myself. Here is a zone in which the culturally mandated interpretations are less confident. But observe further that any other person manifests variations in sentience: when in stupor, shock, illness, sleep, etc. That I can conceive the other person as sustained in spite of these variations in sentience is notable too.

2. Through reflection, I may conceive of myself as a counterpart to other people. I become an objectivity to myself. (And I can imagine that my implemented choices are determined in advance by objectivities--even though this notion is useless at the moment of realizing choice and is a denial of that moment.)

Thus, I am supposed in the end to conceive of myself as one object in a "world" of objects. This notion is violently incoherent at the most elemental level. Only the pressure of the culture keeps me from feeling insane (in the sense of H.l.c above) as I espouse this notion.

3. I may be in overt conflict with another person.

I can be emotionally dependent on another person; and on his or her conscious reaction to me. Thus, I may judge myself by comparison with him or her, or by acceding to his or her judgments of me.

I can obscure my choice-making by becoming another person's vassal.

I have one degree or another of emotional sensitization or capacity, which must be attributed to my interaction with other people.

I may seek gratification through solidarity and intimacy with another person.

I can knowingly deceive another person.

I judge whether another person is knowingly deceiving me.

I judge whether another person's self-deception is misleading me.

Past involvement with another person which was emotionally stressful can have involuntary echoes in the present. Here is an example. I may have a public quarrel with another person, and I may cope with that quarrel by pursuing it in my fantasy as well as in public action. Then that person may die or otherwise become irrelevant as an "actual" antagonist. But I may continue the fantasy-quarrels--even though common sense says I will never again see this person outside of fantasy.

My sincere opinion of another person's qualities can be sharply different from his or her presumably sincere opinion of his or her qualities.

*

J. Culture as a Phase Discriminated in the Person-World

1. In the foregoing discussion, I have repeatedly related constituents of personhood to "culture" as their source--a source which is often antagonistic or coercive. It is now necessary to focus on culture, both to clarify a paramount influence in the person-world and also to extend and complete the account of the interpersonal arena.

"Culture" is comprised by those phenomena which are in one respect my "skills," but which I want to credit to other people and through which I meet other people in consciousness. [1991. Here, then, is the rotation to the person-world promised in B.3. Culture is drawn in to my self/world relationship.] A favorite example is English orthography. I know English orthography fairly well, and conform to it to make my written communications less irritating, but I consider it perverse, and do not wish to suppose that I am the origin of it.

Similarly with the natural language generally. I hear other people talk, and it is a "skill" of mine to supply thoughts to their utterances; I credit the thought-content to the other people and not myself. [1991. Well, as I mentioned in D.1, natural language as common-sensically believed to exist has aspects which no individual authors. Here I abstract from that complication.]

Similarly with practical mathematics. Similarly with creeds and ideologies which I may reject, but pretend to espouse in casual encounters because other people expect them. I "know" to remove a hat when entering a building; and how to tie a Windsor knot. I know that I must obtain "the" answer to 1/0, 0/0, or 00 from an authoritative textbook or professional mathematician, rather than trying to compute it on the basis of definitions learned in elementary school. I see a man wearing a clerical collar, and it is one of my "skills" to recognize that the collar makes a statement which is not purely idiosyncratic to its wearer.

Indeed, culture has a meaning which generalizes beyond the specific people I encounter. When I hear people speak Chinese, which I do not understand, it is nevertheless a skill of mine to recognize that they are conversing, and that their language is Chinese. The significance of this foreign language extends beyond the specific people I encounter as its users. There are Chinese books, newspapers, etc. [1991. Again, the existence of languages I don't know is a sharp challenge to the person-world orientation. Not until "Personhood IV" do I address that challenge head-on.]

2. By emphasizing that aspect of culture which is alien or hostile to me, I emphasize that I cannot afford to deny that this phase of my person-world is special. "Culture" is comprised by my skills, by constituents of myself; but they are skills I don't wish to claim the balance of responsibility for.

On the other hand, it would be a dangerous oversimplification to leave the impression that culture is merely regrettable. All of my skills have a cultural aspect. My self-expression and self-assertion may consist in evolving personal variations and recombinations of the culturally supplied procedures. But even my personal variations are influenced by my assessment of the community's situation; and my personal variations are contrived with the intent that they will feed back into the culture and the interpersonal arena. I am getting way ahead of the immediate topic, but it is urgent to establish a viable attitude toward value judgments of culture. Culture is like "people," or "life," or "the world." I may experience culture as hostile; it may be self-protecting and nurturing of me to define it as regrettable. But it is unavoidable as a raw material; and the quality of my self-assertion is a matter of what I make of this raw material, of the astuteness of my selectivity toward it and my compromises with it.

Take, for example, my natural-language skills. Of course the natural language is extremely traumatized and deleterious; but merely to proclaim that it is "bad" is too easy and changes nothing. The natural language is a medium in which we find ourselves willy-nilly. To understand that it is deleterious is self-protecting and nurturing. But a mere proclamation that it is "bad" does nothing to release us from the natural language's consequences. An effective struggle against the natural language can come only from the dedication to make sophisticated compromises with it, to employ it to struggle against the "mode of life" supporting it. We can be delivered from the debasing culture only by acceding to an enchanted community/higher civilization. But poseur extremism is not to be identified with that accession. Only dedication will gain results of substance. The issue is not whether to compromise, whether to turn inherited media against themselves; it is rather the astuteness, the sophistication, the placement of the compromises. The issue is to understand the difference between an extreme proclamation without action, and a drastic action which nevertheless can succeed because its risks are mostly matters of phantom barriers erected by cultural indoctrination.

*

K. Community; Society as a Grandiose Other

1. Resuming with ordinary personhood, other people and culture are palpable to me. Other people and culture jointly constitute the interpersonal arena--or community.

2. I now arrive at the issue of "society." Society is the aggregation which is hypothesized as subtending the (palpable) community. Society is the kingdom, the race, the nation. It is an abstraction, a matter of faith, to which allegiance is demanded by palpable specific people.

Indeed, in mentioning society I am for the first time mentioning a "grandiose Other." Each of the grandiose Others is supposed to be the ultimate source of meaning, the ultimate source of my emotional gratification and judgmental self-consciousness; but is at the same time primarily speculative and outside "my ostensible world."

The universe of physics must be mentioned in this connection as hypothetical, inferential, derived, and assuredly grandiose. However, it is emphatically not claimed to be a source of meaning. (That is the locus of a screaming incoherency in the modern rationalist outlook. Scientists make officious, sententious, unctuous moral pronouncements; but their intellectual stance provides no basis whatsoever for them to do so.)

In modern rationalist culture, society has priority among the grandiose objectivities. (Codified in theory by figures such as Comte, Marx, Spencer, Pareto, and Sumner.) Society's claim on us as persons (even when we are treated as pawns) is far broader and more important than the physical universe's claim on us. Attachment makes society more urgent than the physical universe. (Except that for a diabetic, for example, belief in insulin is more important than belief in the United States.) Grandiose Others of the past included

    the personal Deity or Heavenly King (Judaism),

    supra-mundane worlds (polytheism; neo-Platonism),

    cosmic consciousness (Vedanta).

Here again, context already narrows the discussion. The latter Others have already lost credibility for modern rationalism. Person-world analysis would agree with the rationalist debunkers here, if nowhere else. Conceptions of supra-mundane worlds are unavoidably extrapolations of conceptions of community and society. I may "renounce Man for God," but not without encountering Man first. Also, the deity who has served the actual function of religion has been a heavenly father and king. Even the god of the philosophers had to be a person who made laws. (And we come upon science's embarrassing secret, perhaps best codified by Leibniz: that without God's guarantee of sufficient reason, there is no physics, not one physical occurrence anywhere. Secretly, modern rationalism has not advanced beyond God; it is desperately wedded to God. But I never said that modern rationalism was cogent. Meanwhile, the avowed loyalties in modern culture are to society.)

3. Extending from one's emotional involvement with other people, society becomes an object of one's passionate belief. The hypothesized abstraction seems to be a living presence--as when people march off to war for the Nation. That is attachment. Because society is an object of passionate belief, because it becomes a hallucinatory living presence, it cannot be sharply distinguished from community (which is palpable), even though it remains impalpable (a hypothesized abstraction). So society has a close and compelling connection to the palpable phenomena of other people and culture. For the only time in this investigation, let me derive the palpable from the hypothesized--and say that culture is that palpable aspect of society which is interior to me and at the same time is an externality broader than other people as individuals. Recognizing how close society is to community in belief, I propose to be flexible with regard to whether the person is conceived in a communal or a social context.

4. The community confronts me with symbols and offices which imply an organized collective, legitimation, manifestations of a group will, etc.

One cultural phase of community life includes the community's "tradition," symbolism, ritual, etc.--all of which are emotionally charged. This phase must be considered one source of my emotional sensitization or capacity.

The community may force upon me a significance, and an assortment of privileges and disadvantages--so much so that I am forced to carry out this "imposed social role" or to grapple with it. The role may place me in competition or conflict with other people. I can also be gratified by the celebration in ritual of my imposed status (although I do not earn this gratification).

I have a greater or lesser degree of autonomy, relative to the community, in respect to being supplied with pursuits and goals, and in respect to making judgments of every sort.

I can obscure my choice-making by becoming a vassal of "society"--of a legitimated organization or institution.

I may engage in a pursuit which I suspect to be dishonest or otherwise contemptible because the community approves of it. Of course I do so to gain tangible rewards, in analogy with knowingly deceiving another person to benefit myself. But something beyond my craftiness is involved here. I maintain a knowing self-deception and vassalage in which legitimacy means more to me than sincerity.

5. My judgment of whether I am "sane" (H.1.c) can involve two aspects of my relation to the community.

a. Culture amounts to a system of ideology-saturated interpersonal, even social, relationships in my mind. Normally it is a matter of habit for me to maintain the (sham) coherence of that system. But if this sham coherence is subverted by an "indigestible" experience or idea, then I undergo a "personality-death" and the entire society undergoes a "personality-death" inside my mind. My sanity is placed in doubt.

b. Certain sorts of condemnations by the community of my inclinations, my urges, the "possible personality" which represents my penchant and loyalty can challenge my sanity.

6. The interpersonal arena is a source of meanings to me. My connections to the interpersonal arena in regard to praxis, emotional sensitization, indoctrination, etc. have an effect on my sense of sanity, my personal identity, my level of fulfillment, etc. Thus, the interpersonal arena can be a source of skills worthy to be sustained and regenerated. It can also be a source of acute dilemmas and destructiveness impinging upon me. In either case, the interpersonal arena is a source of problems and missions.

Moreover, the problems and missions can appear in my consciousness as consequences of my skills. Having been indoctrinated with little choice in the matter, that indoctrination now surfaces in the guise of my skills, for one thing. (Examples at the level of the present discussion are language use, mathematics, music, profit maximization.) If I do not consciously review my indoctrination, then I will carry it with me by default. Moreover, my private and idiosyncratic dilemmas with the language, with mathematics, with art, with profit maximization, etc.--and my private and idiosyncratic ventures in these fields (e.g. I might seek to prove that 1+1 != 2 or that intrinsic pricing is a delusion)--can represent vital dilemmas and ventures for the interpersonal arena.

But the community's destructiveness or bankruptcy may consist precisely in its inability to embark upon vital ventures--and in its fostering of individual pursuits which disregard and exacerbate its dilemmas.

I can undertake a vital venture or address a vital problem; or I can avoid doing so. And I can belong to a community which wants such a task addressed; or to a community which discourages attention to such a task. The possible ramifications of the community attitude, for my judgment of myself, are complicated. Inner pride or lack of it can run counter to express community approval or contempt.

*

L. The Ostensible World as a Delusion

The ordinary ostensible world is a delusion in the following precise sense. My ostensible world--that is, the perceptions and beliefs which inform it--are palpably affected and sustained by emotions of anticipation, by emotional dependence on other people, by morale, by esteem, by realized choices (volition), by knowing self-deception, etc. In this sense, one's ostensible world is the resultant of deformations. One unwittingly undergoes a deformation of perception and attitude (which one does not feel as illness). Nevertheless, as I have done here, the deformations can be analytically exposed and introspectively recovered.

*

M. Imminent Character as an Invariant to Psychedelics

Let me return to the individualized or idiosyncratic constituents of personhood. But I shall now treat constituents which presuppose explicit consideration of culture and community in order to be comprehensible. Perhaps I am not introducing new constituents here. I may be making a different selection from constituents already described. I may be selecting certain tendencies in the person-world because they involve issues of pressing interest.

Let me borrow the word "character" as a word for a facet of personality, and give it a new definition specific to this discussion. Imminent character consists of certain inclinations which inform my realized choices. (I am stretching "imminent" to mean in-the-moment.) One way of delimiting character is to mention that it is a constituent of the person-world which the psychedelic experience is unable to affect (unlike sense-of-self and perception-world).

The following questionnaire elucidates some alternative inclinations.

a. Referring back to G.1, am I able to differentiate my sensations from my habits of imputing contexts of objectivity to them--from the assumptions which I impose by appellation? [Can I realize that the cartoon does not show a climbing bear?] Further, can I refrain from carrying assumptions imposed by labelling over to previously unclassified sensations (such as the psychedelic experience of twinkling air, or such as any hypnopompic hallucination)?

b. Do I assert autonomous cognitive norms? Or do I equate truth with what convinces the group, with group beliefs?

c. Do I manifest a capacity for painstaking effort?

d. Do I seek to asset my sincerity and concern in the interpersonal arena--even to force my sincerity into the interpersonal arena which it is not especially welcome? Do I refrain from asserting my sincerity in the interpersonal arena, out of fear or weariness? Am I willing for all my interaction with other people to consist of play-acting dictated by them, or to consist of disengaged manipulation? Do I manifest my sincerity only in fantasy, only in imaginary gratification? Do I pay attention to other people in a way that goes beyond disengaged manipulation (do I "grant other people's right to exist")?

e. Do I make a distinction between what the community wants and what it needs? If so, how do I proceed as a consequence?

f. Can I function self-satisfactorily in the face of community norms which are hostile to my penchant and loyalty? [1991. Well, this trait is common to a genius and a sociopath. I am not providing sufficient conditions for worthiness here. (Language, after all, is a crude medium of determination.) I am indicating traits which under certain interpretations are crucial to my undertaking.]

g. Do I already have a hunch "that the 'synthesis of a world' can be effected with a different system," that is, that the totality could be appropriated according to a different principle? Or do I already have a glimpse of a gratification which everyday existence, or the established compartmentation of faculties, denies to us? And what is the state of my emotional sensitization?

h. Am I capable of admitting, if I should find myself in danger--in conjunction with having an illusion shattered--that I was partly responsible for creating the illusion and the danger? Can I admit a mistake in judgment while not equating that mistake with my whole self? Or do I have to believe that the only reason I ever find myself in danger is because other people betray me?

i. Am I able to plough through disillusionment to an outcome which is absurd and extreme by conventional standards; and then to review that outcome repeatedly until I can extract a new cogency from it?

j. Do I express the attitude that "nothing makes any difference" and that "I don't care about anything?"

*

N. Thematic Personal Identity

There is a level of self-consciousness at which my whole, thematic identity is at issue. Earlier, I explained some conventional sources for the concoction of this identity--and I say "concoction" advisedly. My life is a venture, a sojurn, which has a meaning or outcome to me, or fails to have one. I have a personal history toward which I feel regret or satisfaction, and a future of which something must be made.

1. I have said that (in an attached state) choice-making is forced upon me. I am impelled to realize preferences in action, based on anticipation and on the factual judgments and the praxis to which I am fixated. I am imminently forced to formulate preferences on the basis of guesses and to realize these preferences in action. In the moment of realizing choice, my choice may be cued by perceived "external" conditions of the moment. But my realized choice cannot be reduced to, that is, derived from a perceived external condition. Realized choice and external condition are alongside each other; they are equal constituents of a single "world."

2. My choice-making can explicitly pertain to my whole, thematic identity. But choice-making at this level is occasional, not continual. Choice-making is usually frivolous or pragmatic. (The existentialist notion that all choice-making is implicitly the actualization of a whole, thematic identity is an example of a vogue making an easy, empty universal out of a phenomenon which is significant only when it is specific and explicit.)

Already my capacity to admit a past mistake in judgment without equating that mistake with my whole self--and my capacity to plough through disillusionment--implied that I have a whole, thematic identity.

3. As far as choice pertaining to my thematic identity is concerned, one avenue of choice concerns how I conceive the arena of action, and thus how I shape my loyalties (how I reconceive and redirect my loyalties). A concomitant avenue of choice concerns how I conceive effectiveness and gratification, and then pursue those conceptions.

In more detail, my past can manifest distinguishable thematic identities which may be incongruous. In other words, different identity-themes can be possible for me. I am forced to formulate a preference for one identity-theme as opposed to another and to realize that preference in action. An identity-theme may represent intimidation from without; or it may represent my penchant and loyalty. For convenience, let me call the latter my authentic identity-theme. My authentic identity-theme can be already disclosed; yet I can repress it because I assume that other people will censure me for it. I can also disregard it, or at least fail to uphold it, just because it takes special effort or painstaking effort to uphold it.

Whether to uphold my authentic identity-theme is a dilemma in which one or another realized choice must occur. To pursue my authentic identity-theme means that I must be vigorously willful in a context of uncertainties. After all, it is also an option to drift. And, subjectively I often have to gamble--notwithstanding that my purpose remains fixed.

*

O. Fixation to a Cumulating Social Role

I come to another topic which requires me to be unaccountably conversant with more than my self-world relationship. It is commonplace for a person's whole thematic identity to be a matter of attachment to one's social identity as it has cumulated in the past. One is overwhelmed by the significance society thrusts upon one. One is overwhelmed by the pursuits, goals, and cues for one's judgments which society thrusts upon one. I have already said that culture is a part of yourself for which you are not exclusively responsible. But the dynamic balance of attachment can be such that your self is submerged by parts which come from society and for which you are not exclusively responsible--by the assortment of privileges and disadvantages which society has thrust upon you. Your self is submerged by what has been done to you by your intimate associates and by the more impersonal community--and the assessments of the "venture of living" which you have formed therefrom.

This submergence of the person by a cumulating social role is one alternative in which the person is guaranteed to be traumatized, stigmatized, impaired, truncated. It is a specific feature of personhood theory that it demands this conclusion. Certainly, in some cultures or communities, socially acclaimed and validated roles can also allow intrinsic splendor. (Even so, we must not allow the doctoring of history to obscure the fact that these socially approved achievements had great difficulty coming to the surface in the first place--and subsequently were dishonored by deteriorating communities.) But to exist in fixation to a cumulating social role is always a depersonalized, mythified existence--even when it is producing useful output. Of course, being submerged in a social role is only one of a number of ways in which existence can be depersonalized and mythified. "Mortification of the self to please God" is another way--which, however, has already been sidelined by modern rationalism.

Imminent character, for the person submerged by a social role, is necessarily a zone of desperation and impairment. The dynamic balance of this desperation and impairment can be analyzed, but it is not assured that such analysis can change anything. In the cases I am aware of, the desperation and impairment consequent on being submerged in a social role are destructively self-reinforcing.

In the remarks to follow, I want to bear witness to my experiences with academics and other "aware" people. I want to characterize them as I found them; neither idealizing them nor scorning them. To begin with, as I said, scientists amputate their perceptions to protect their belief-system. Unfortunately, this condition is self-reinforcing: just the perceptions which would give the lie to their belief-system have been expunged.

A published account of a personal crisis was Zdenek Mlynar's Nightfrost in Prague (New York, 1980). Mlynar was a Czech Communist official who, after the invasion in 1968, was removed to Moscow with his colleagues. During the invasion, he had realized that he could be shot at any time. In Moscow he was subjected to stress negotiations to force him to authorize a permanent Soviet occupation of Czechoslovakia. He began to have visions, and underwent an instantaneous conversion from Communism without logic or analytical thinking. The outcome of his crisis was that, after seven years in limbo, he escaped to Vienna and (evidently) became a NATO intelligence officer. Today, world public opinion would laud his defection. I cannot quarrel with his disillusionment. Yet his salvation was as banal as the Readers' Digest. An experience verging on the supernatural propelled Mlynar to become an officer of techno-capitalism. That capitalism is the older and continuing basis of the dilemmas which my undertakings address.

My view of Mlynar and of the generality of "aware" people I have known is that analysis is not likely to uncover escape hatches in the dynamic balance of their desperation and impairment. Even an experience verging on the supernatural--one far beyond any pressure I could bring to bear--only taught Mlynar that salvation means being an officer of global capitalism.

*

P. Attachment's Turbulent Causation

The preceding section poses a problem of explanation, a problem which has been implicit throughout this manuscript. A social role's submergence of a person is cited by personhood theory as a cause of

    amputated perceptions,

    emotional numbness,

    mutilation of faculties (the science/poetry and science/occultism dichotomies in modern rationalist culture),

    "success"-directed striving (in the American sense),

etc. More accurately, the social role is said to fixate the individual to mutilated perception, for example. But I then say that the social role is a sort of ideology and skill which the individual is fixated to. These formulations give social role--or thematic identity--or imminent character--the guise of a self-caused cause or looped cause.

Let me try to resolve this. The circuit of attachment through the person-world is not a linear causal phenomenon; it is a phenomenon of scrambled or turbulent causation. It is a dynamically balanced confined turbulence. What is awful about being submerged by a social role in the cases known to me is precisely that such submergence is self-reinforcing. Shame can live with itself only by glorying in shame. To expand on this conclusion, it is appropriate to mention some of the "ruinations" of social identity.

1. The culture may mutilate a child's faculties (again science/poetry and science/occultism) and inculcate him or her with debasement (secularism's world consisting exclusively of things)--and yet not push the child to the point where he or she demands escape as a right or becomes a precocious social critic.

(Yet we encounter again the social doctoring of history, this time biographic history. Children do cry out, they do demand escape as a right, they are precocious social critics until they are subjugated. Higher and higher tolerances for anguish, or compensating rewards, have to be developed.) In due course, the child begins to perpetuate the stigmas in him or herself. At the least, he or she acquiesces; at the most, he or she may become a well rewarded advocate of the community.

By the time one becomes an adult, though, one must have experienced enough diversity and enough responsibility to begin to know manifest degradation and mutilation for what they are. An undercurrent of shame appears; one begins to suspect oneself, if not to despise oneself. If one's abasement is then suddenly spotlighted by somebody else's acts of courage and splendor, one's shame may be magnified and become a matter of crisis. But there is another possibility: that because of the mutilation of one's faculties, acts of courage and splendor will be invisible to one. In a crisis of shame, the person submerged in the social role of e.g. the renowned and well-paid professor of economics can only flee his shame by affirming and advocating debasement and mutilation.

Moreover. It is commonplace for all academic personnel to be exposed while they are graduate students to the idea that their discipline is a hoax; and for them to react, after a period of faltering, by redoubling their efforts to rise in the profession which they now know is a fragile, protected swindle. (A good satire on this is found in Joel Kovel, The Age of Desire, Ch. 1).

2. Consider American middle class elements in the Seventies who flocked to cults and to entertainment which ritualized degradation ("punk"). The normal course of life of these middle class elements led them into occupations (or, more generally, into a culture) consisting of hoaxes, silliness, and impoverishment. At the same time, they were too intimidated and unimaginative to attempt a genuine escape. Consequently, we must conceive them as despising themselves. They sought stupefaction, and they sought rituals of shame and mortification. In this way, manifest hoaxes, silliness, and desensitization became sought-after-experiences for the urban middle class in the Seventies. (I don't want to demean this manuscript by naming names; I assume a reader familiar with the history.) The individual was encouraged to become a swindler (a cult member, for example); or to conduct rituals of shame. Ultimately, people ritually abased themselves because they were rewarded for doing so, and in order to express their shame.--And they knew that they were sordid because they occupied and sustained themselves by ritually abasing themselves. Ritual abasement became a preferred experience; and people knew that they were sordid because they preferred ritual abasement.

Social history is superficially changeable. The American Establishment launched a campaign to win back the middle class in the Eighties; and the latest thing was to be a Yuppie. Then, with the recession at the end of the Eighties, the Yuppie role became tarnished. These ebbs and flows are not the level I should address. Let me summarize what should be understood here. First, the cults and the ritualized degradation were only the most graphic symptoms of the lasting trend of techno-capitalist civilization. Secondly, we should ponder the cults and the ritualized degradation whether they are in vogue or not. Negatively, they revealed personhood in an exposed manner. (If I defile myself in a public ritual, what am I that I can do this?)

*

Q. The Determination of Personal Fate

1. Let me refer back to (N), and to my choice-making regarding my whole, thematic identity. There is a case which evidently is extremely rare; but which deserves mention because it is hopeful. In retrospect, I may judge that my thematic identity is far more vivid and well-organized now than anything I could have imagined or even understood earlier in life. Furthermore, it may be that my thematic identity has no overall outside cue, paradigm, or promoter--so that I am unlike people who accomplish something they did not expect because they are pressganged by the consumers of their talent. I may then claim that the thematic identity which represents my penchant and loyalty comes from the future: from a systematic and coherent but incomparably novel future. This notion is supported if the thematic identity is tied to a pressing task (confronting an acute dilemma) which is an implication of my skills (an implication of the culture) but which has no community acknowledgement or sponsorship.

In the rare case that one's authentic identity-theme comes from the future, guiding oneself toward it remains a matter of pronounced willfulness in a context of uncertainties. It is possible to drift rather than to push toward the distant identity-theme. And subjectively I often have to gamble--even if my purpose remains fixed. The dilemma of whether to uphold an authentic identity-theme coming from the future is a crisis which compresses one's future into one's present--a moment in which future and present touch each other. The crisis gives one some choice over the way one's future shapes one's present. (By upholding or relinquishing the identity-theme from the future, one guarantees or nullifies it as a future?)

[1991. I include this speculation because I am looking for escape hatches. It's rather romantic--and never before articulated. It may exceed the restriction to palpable phenomena which I have endeavored to uphold in person-world analysis. Retroactive signification. Unprecedented fate.]

2. Let me now resume my discussion of the person submerged by a social role. That person emerges as a person who is "done to." In contrast, the person who e.g. upholds an authentic identity-theme coming from the future emerges as a person who "does" or "does to." But why is a given person one way or the other?--and can he or she be switched from one type to the other?--and does a person who is always one type nevertheless have a potential for the other type?

This question requires thoughtful distinctions. Medieval serfs were illiterate and never saw money in their entire lives. Today their descendents in Western Europe all read, possess money, and spend money every day. The reason why serfs did not learn to read or to allocate money was that (in effect) they were not recruited and given cultivation to these ends.

There is a view which would say that the serfs, as a multitude which had been assigned the same fate, became aware that they were being taken advantage of in a common way, and fought for the cultivation (schools, etc.) which they subsequently received. This is not false (the French Revolution); but it is misleading. It does not take into account that the descendents of the serfs remained outside the controlling class--that the "toilers" have never commanded the system. The collapse of the workers' paradises makes this observation all the more decisive.

It is more realistic to say that advanced capitalism continually revolutionizes technology and continually erases and replaces social relationships. (Capitalism also spurs developments such as the dissolution of the nuclear family, and feminism, which the Establishment did not calculate.) As a result, the achievements and satisfactions which are possible to people come to be seen as results of how much cultivation the Establishment gives them.

But in personhood theory, the question of why people are what they are focuses in a different way. The topic was anticipated in (O) and (P). One reason why I turned to personhood theory was that presumably clear and blunt presentations of the invalidity of the scientific outlook, and of elements of post-scientific culture, were simply shrugged off by the "aware" people (the cognoscenti, the intellegentsia, academia, bohemia), and remained invisible even after campaigns to publicize them.

What, then, was the nature of the "aware" people's adherence to the intellectual status quo which made them impenetrable to whatever I (or Hennix) had to say? Attempts were made to transfer sociological analysis (such as the one above about serfs) to this question. It was said that people were not "geniuses" because they had been deprived in childhood--they had not been given sufficient cultivation--their "genius" had been suppressed. People were waiting to explode with "genius" once the right button was pushed.

As I said at the end of (O), all of my experience in the matter runs counter to this. It would be unwise of me to assume that the reason why the "aware" people shrug me off is that the Establishment did not give them enough cultivation. Somehow, the sociological perspective, which is tacitly tied to a doctrine of underprivilege and socially engineered redemption, misses the point. Let me present a shock-question to clarify the issue.

    Would a Nobel-prizewinning physicist agree

    that he believes physics because his naiveté

    was exploited by malicous elders, because he

    was crushed by his elders, because his elders

    did not give him enough cultivation?

The sociological perspective--in the name of recognizing that the serf's backwardness was imposed from without--treats the serf's effects on other people as if they were imaginary or didn't matter. It treats the serf's choices and life as if they were tuberculosis--a fatal disease which a few pennies' worth of medication could have cured. We have been living with sociology ever since Comte, and we don't realize how odd it is. Capitalist technology and centralization have created the possiblity of imposing changed fates on entire populations. A member of the administrative class can regard all the choices and lives of a population as a reversible condition. Then people really are what the administrator chooses to make them by pushing this or that button. People are so thrilled by the prospect of human manipulation on this level--or by the prospect that the Establishment is due to give them cultivation--that they overlook that the sociological perspective makes all their choices and their lives chimerical (or revocable). "You did it because you were programmed improperly." How do you choose and act if you believe that your choices and actions have the ontological type of a disease, an error in past programming? And who says that the serf's life was "bad" or unnecessary? And yet people have learned to think in these terms--to want to be told that what their betters permit them is what they are.

The ambition to transfer social engineering to seriousness and originality, by vaccinating people with seriousness and originality, is an ill-conceived ambition. Seriousness and originality are not "done to"; they "do (to)." They are not implanted. They appear unpredictably. (Of course, my attempt to assert my sincerity and to make the intepersonal arena conducive to it may reawaken seriousness and originality in another person.) I included the speculation about the authentic identity-theme which comes from the future to show that one need not assume the social engineers' cause-and-effect. We do not even have to believe that "solutions" are fabricated from past to present.

Having speculated in Q.1 about such a thing as unprecedented fates, could average people be said to have routine fates? My considered answer is no. That is because to say that a person fulfills a routine fate cannot be distinguished from saying that that person is determined by the past, by circumstances.

I entertained the notion of an unprecendented fate because a novelty arises in how we conceive or apprehend, understand or appreciate.

According to Hennix, the "aware" people (the Ph.D. physics candidates) are not really inert or smug. They are torn by deep inner conflict and terror (or inadequacy). We don't see that, because they unstintingly conceal it. They resolve their anguish by doing the understandable and presentable thing at any given moment, whatever that may be. Hennix, then, does not see stagnancy tied to sheer lack of "genius." And yet one would expect Hennix to insist on the role of genius.

3. Let me resume the definitive pronouncements of personhood theory. Personhood theory refuses to acknowlege people as objectivities in a deterministic process. (Except to acknowledge that this conception itself is one of the characteristic nonsensical fantasies.) One who adopts the person-world outlook cannot consider his or her choices and life as a reversible mishap. Personhood theory cannot consider palpable choices and lives as chimeras or as revocable.

The demand for a calculus of society is, in the light of personhood theory, an ill-conceived demand. Again, I included the speculation about the authentic identity-theme coming from the future to show that one need not assume the social engineers' cause-and-effect. We do not even have to believe that "solutions" are fabricated from past to present.

Let me make some last comments about the question of seriousness and originality.

a. Stigmatization is typically self-reinforcing. The conformist opportunist has to be displaced to a whole different environment even to be able to acknowledge his or her shame. And then he or she may be destroyed by his or her self-visibility.

b. Seriousness and originality cannot be thrust upon any given person by outside manipulation. Metaphorically, escape hatches are opened by the future, as coherent novelty, in conjunction with moments in which choice is forced--moments in which the arena of action might be reconceived, loyalty might be shifted, effectiveness and gratification might be reconceived, etc. 