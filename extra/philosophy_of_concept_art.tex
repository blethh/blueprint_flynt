\newcommand{\action}[1]{[\textit{#1}]}

\newcommand{\speaker}[1]{\vskip 0.2em \textsc{#1}: }
\newcommand{\speakermod}[2]{\vskip 0.2em \textsc{#1} \textit{(#2)}: }

\chapter{Philosophy of Concept Art (1987)}

{ \centering \itshape
An interview with Henry Flynt \\
by Christer Hennix \\
Dec. 6, 1987 \par }


\speaker{FLYNT} I'm going to give a summary of how I originated Concept Art 
in order to bring it up to the point where it's understandable why I 
speak of you (Catherine Christer Hennix) as my only successor in the genre. 
Summarizing briefly, I see two things coming together. One of them 
was my involvement with the modern music community of the time---Stockhausen, 
Cage, LaMonte Young---and the other aspect was that I 
had been a mathematics major at Harvard and already knew that I 
thought of myself primarily as a philosopher---that my intention had 
been when I was very young, when I didn't understand the situation 
that I was in---my intention had been to become a philosopher with 
nevertheless a specialization in mathematics. Of course, many people 
actually did that. 

So, having said that, one of the things that I began to notice about 
the modern music of that time was this extremely strong pseudo- 
intellectual dimension in Stockhausen---Stockhausen's theoretical 
journal \journaltitle{die Reihe}---the impression that they were doing science 
actually---for example Stockhausen had a long essay on how the 
duration of the notes had to correspond to the twelve pitches of the 
chromatic scale \ldots

\speaker{HENNIX} "\ldots\ how time passes\ldots"\footnote{\journaltitle{die Reihe 3}}

\speaker{FLYNT} Yes, and what is more, the other rhythms had to correspond to 
the overtone structure above those frequencies as fundamentals. 

\speaker{HENNIX} Yes, I'm quite familiar with that. 

\speaker{FLYNT} Yes, I would expect you would be. I remember Bo 
Nilson---you will like this---in 1958 at the same time I saw Stockhausen's 
score---he went even one step further than Stockhausen because he 
used fractional amplitude specifications---so this is even more than 
Stockhausen, and so forth and so on. 

Cage took a considerable step further in the sense that in Cage this 
kind of play with structure is carried to the point where there is an 
extreme dissociation between what the composer sees and what the 
performer sees in terms of the structure of the piece and what the 
audience knows. They are completely divorced from one another. Cage 
would compose a piece on a graph in which the time that a note begins 
is on one axis and the length of the note is on another axis. What he 
would do was to superimpose that on some picture like from a star 
catalogue--- 


\speaker{HENNIX} \opustitle{Atlas Eclipticalis}--- 


\speaker{FLYNT} Yeah, well, that's the particular piece. I'm making up a 
composite of his compositional techniques but the result is that when you 
break up a sequential event in that way, it's not like a pitch-time graph 
where there's an intuitive recognition of the way the process unfolds. 
He would have one structure for beginnings and another structure for 
durations. Well at any rate, already in Cage's music there was a kind of 
ritual aspect to performing classical music. I mean in Cage's piece, 
which is actually all silence---the only thing the pianist does is open and 
close the lid of the piano or something like that. 

Then LaMonte Young comes along. His word pieces were the first 
that I ever saw, composed in mid-1960. I saw them in December 
1960.\footnote{Other composers have earlier dates, but for me, 
Young crystallized the genre. [H.F., note added]}
It was a very different kind of structural game. It was no longer like 
twelve-tone organization and so forth but rather it was like playing 
with paradoxes---it was nearer to making a paradox than making some 
kind of complicated network. 

And I felt that matters had reached the point where there was 
some kind of inauthenticity here because the point of the work of art 
had become some kind of structural or conceptual play, and yet it was 
being realized under the guise of music so that the audience had no 
chance of really seeing what was supposed to be the point of the 
piece---the audience was actually prevented from seeing. Certainly 
Cage's methods had exactly that effect. The audience receives an 
experience which simply sounds like chaos but in fact what they are 
hearing is not chaos but a hidden structure which is so hidden that it 
cannot be reconstructed from the performed sound. It's so hidden that 
it can't be reconstructed but nevertheless Cage knows what it is. So I 
felt that the confusion between whether they were doing music or 
whether they were doing something else had reached a point where I 
found that disturbing or unacceptable. 

At the same time at that period there was a great fascination in sort 
of taking the Stockhausen attitude and looking back at the history of 
music from that point of view. Stockhausen's analysis in \journaltitle{die Reihe 2} of 
Webern's \opustitle{String Quartet [Op. 28]} tried to show that Webern was 
composing total serial music and not just twelve tone music. That was 
the attitude, they were rewriting the history of music, trying to show 
that all previous important figures were essentially preoccupied with 
structure, that they had been complete structuralists. 


\speaker{HENNIX} Really? I thought it was only Webern that was given that 
treatment. 


\speaker{FLYNT} Well, they were digging up all these composers from the 
Middle Ages, the isorhythmic motet and everything like that---they 
were sort of dredging that up because that was the previous 
period---the medieval scores in the form of a circle and the use of insertion 
syncopation,\footnote{My term for the rhythmic feature common to Magister Zacharias' \opustitle{Sumite Karissimi} and \opustitle{Klavierst\"{u}uck XI}. See Willi Apel, \booktitle{The Notation of Polyphonic Music} (4th ed.), p. 432 for \opustitle{Sumite Karissimi}. [H.F., note added]}
it appears with the red notes ina medieval score and then 
it reappears in Stockhausen's \opustitle{Klavierst\"{u}ck XI}. They were just jumping, 
they were dismissing what we would call the baroque, classical and 
romantic periods periods as completely worthless. In other words, the 
last music before Stockhausen was in the 14th century, this is the way 
the history of music was being rewritten. And LaMonte was getting 
into Leonin and Perotin and all that kind of stuff. Well, anyway, that's 
quite an excursion. 
At any rate there is in music, there is this preoccupation with---it 
may be a kind of quasi-Pythagoreanism, I don't know\ldots

\speaker{HENNIX} The way I looked at it was that they saw in Webern, first of 
all the harmony was going away. And they saw in Webern a way of 
determining the note more and more precisely, in terms of all of its 
parameters, pitch, duration, timbre and all that. What was left was that 
timbre was not serialized yet. And that, as far I see it, was what the 
Darmstadt school did---they added--- 

\speaker{FLYNT} Stockhausen's \opustitle{Kontra-Punkte}--- 

\speaker{HENNIX} Yeah. And they all considered Webern the god of the new 
music--- 

\speaker{FLYNT} Yes--- 

\speaker{HENNIX} ---and also a little bit Messiaen--- 

\speaker{FLYNT} Yes. 

\speaker{HENNIX} It was Webern and Messaien that determined the entire 
fifties in Darmstadt. In other words, they were saying that Cage was no 
good. He was just looking in \booktitle{I Ching}---it was a random thing. And you 
cannot recover the structure, it's hidden, as you said. The problem was 
that Stockhausen, when he played his \opustitle{Klavierst\"{u}ck XI}, you couldnt 
recover the structure either. It was so complex now. So the complexity 
of the serialist music became exactly the complexity of Cage. Cage 
looked his numbers up in random number tables; the others were 
sitting calculating rows of numbers. But in addition to that they also 
had to fake it. Because---you find that yourself when you do serial 
music---the music moves too slowly. So you change the numbers to get 
the music up a little bit. 

\speaker{FLYNT} Yes. We're taking longer on this than I meant to\ldots

\speaker{HENNIX} But I wanted to say this. The completely deterministic com- 
position technique and the completely random, aleatoric technique, 
gave exactly the same results. And that was the complete breakdown of 
the Darmstadt school. That's when they started to improvise in Darm- 
stadt. Not before that was there improvisation in Darmstadt. 

\speaker{FLYNT} When they first tried to serialize duration, they tried to pick a 
fundamental unit and use multiples of it; in other words, that's not the 
way you serialize pitch. You don't take one cycle per second and then 
use two cycles per second, up to twelve. That's not what you do. But 
that's what they did with duration. And that's what produced the 
Boulez pieces that move so slowly. In other words if you treat rhythm as 
multiples of like a whole note then it was moving too slowly for them. 

But Cage was for them what was wrong with America or something. 
I mean, the center of what Stockhausen was doing was the 
concept of scientificity. In other words at that time I fantasized the 
composer appearing as performer, on the stage in a lab coat carrying a 
slide rule---there were no electronic calculators at that time, it would 
have to have been a slide rule---but that seemed completely approp- 
riate. In other words, a composition was a laboratory experiment. I 
mean they viewed Cage as a typical American---coming in a vacuum--- 
American superficiality---a vacuum with no scientificity. But Cage was 
actually not using a random number table, he was flipping coins, he 
was using the \booktitle{I Ching}. Yet it was not even that---what Cage was doing 
was much more whimsical than using a random number book. He 
would just copy a leaf---in the \opustitle{Concert for Piano and Orchestra} he just 
put the staff over a leaf and then the main points defining the shape of 
the leaf he just copied them on and he ended up with a circle or not a 
circle, but a group of notes in cyclic shape, and so the pianist was 
supposed to play around the circle. This was completely whimsical 
actually and yes, I remember very well these debates that they had, the 
one and the other\footnote{Serial vs. chance.}---I didn't have any idea that I was going to spend 
this much time competing with the music critic of the \journaltitle{New York Times}
about who remembers the 1950s the best. 

At any rate\ldots\  There is of course a larger tradition in art which has 
a kind of quasi-scientific involvement in structure that does go very 
much to the Renaissance, for example. Althought I was not so conscious 
of that---I looked that up much later. But it was certainly there. 

So, on the one hand concept art came from the idea of lifting 
structure off and makinga separate art form out of it. The structure or 
conceptual aspect, and making a separate art form out of it. The other 
thing that was coming---the development of my philosophical thinking 
---I have to explain first that the version of mathematics that I received 
at Harvard in the 1950s in which Quine was the head of the department 
and editor of the \journaltitle{Journal of Symbolic Logic} and so forth and the 
hottest thing in philosophy was considered to be Quine's debate with 
Carnap. And I was a schoolmate of Kripke, Solovay, Goodman \etc\ 
\etc, \etc. I'm just mentioning that to locate the period of time. Actually 
my conversations with them were insignificant as far as the philosophy 
of mathematics was concerned, there was no discussion between me 
and them on any of that but it will locate the time frame that I'm talking 
about.* 

\footnote{I'm being too diffident. I had quite significant discussions with Kripke and Goodman in 1961. [H.F,, note added]}

But observing what was going on at that time, I picked up the idea 
that the most plausible explanation of what mathematics is, is that it is 
an activity analogous to chess, or in other words that chess captures the 
characteristic features of mathematics, even though, as I have told you 
privately many times, everybody knew who Brouwer was and what the 
intutionist school was, but nobody studied it, and from my point of 
view looking at it and knowing what it was, I felt no inclination to 
pursue it further. 

The reason why this chess game explanation of mathematics 
seemed so plausible---you know, at the end of the nineteenth century 
they found themselves with three geometries---this is not Henry Flynt 
saying this, this is the canard, the story in the text books. There were 
three geometries; one of them fit the real world. They thought it was 
Euclidean, but it might not be. It might be one of the others like elliptic, 
for example; nevertheless, all three were consistent. Now what was the 
epistemological status of the two out of the three geometries that were 
true without having any correspondence to the real world, while one of 
them did have a correspondence to the real world and was also true? 
But what of the other two---the ones that were called true even thought 
they had nothing to with the world? You know presumably Hilbert 
wrote \essaytitle{Foundations of Geometry} as the original answer to that 
question. 

Although---I can't pursue this here, it is much too technical---this 
is now an open question for me. It has never been an open question in 
the past. I just accepted what I was told---that Hilbert solved this by 
seeing that a system of mathematics that has no relation to the real 
world---in what does its truth consist? Its consistency as an uninterpreted 
calculus as they would say---axioms, proofs, formation rules, 
transformation rules. Certainly it was clear in the early twentieth 
century that the concept of an abstract space was established. This was 
what geometry was about. Geometry did not attempt---in Kant's time it 
was assumed that when you were talking about geometry you were 
talking about the geometry of the real world. That's the only geometry 
that there was. The idea that there was a different agenda for geometry 
other than the real world---how Kant could have moved geometry into 
the constitutive subject and said that it was congenital to the mind---Euclidean geometry. 
In hindsight that seems to be one of the biggest 
mistakes he made, tremendously embarrassing, because by the mid-twentieth 
century it was completely taken for granted that the job of the 
mathematician was to study structures which do not have any reality. 
And that from time to time you will give an interpretation to one or the 
other of these structures, like a physical interpretation, and then it may 
be found to be true or false in reality or not. Meanwhile, you have 
another sense of the word "interpretation" which has to do with relative 
consistency proofs by something having a model. 

This is now a completely open question for me, what they thought 
they were doing. In other words what Hilbert thought that he was 
doing---he interpreted one or another non-Euclidean geometry---what 
was the interpretation that he used? It was a denumerable domain of 
algebraic numbers.\footnote{Foundations of Geometry, pp. 27--30}

\speaker{HENNIX} I think his ideas go back to Klein's models---which are 
Euclidean in the center of the circle and then at the periphery they have 
turned non-Euclidean (in the complex plane). 

\speaker{FLYNT} You had to have an explanation of how mathematics could be 
true in any sense whatsoever even though any claim of a connection 
with the real world had been completely severed, and it was being 
pursued in some kind of vacuum. What does mathematics mean in that 
case? And the answer that Hilbert gave was that it does not have to 
mean anything. 

That's the answer. So it's a chess game. And the only difference 
between mathematics and a chess game is that there are additional 
complications created in mathematics by the fact that it deals with 
infinitary games. By the way, I completely overlooked that aspect at 
that time. You know, I can only see it now, kind of like two superimposed 
pictures, because I see what I know now and compare it with what I knew then. 

\speaker{HENNIX} Yeah, the same for myself. I didn't know that this idea of 
Hilbert's was forced by Frege until later. Frege was the one who said 
that either the parallel axiom is true, or it's not. Which way do you want 
it? And so he caused the big stir in the foundations of geometry in the 
end of the nineteenth century and that's why he became enemies with 
Hilbert. They were life enemies. 

\speaker{FLYNT} The reason I see it like two superimposed transparencies--- 

\speaker{HENNIX} But even today this debate with Frege---you have to go to a 
single volume in Frege's posthumous writings---it is not mentioned in 
any textbook---no lecture mentions it, and, so far, nobody has 
explained it properly.\footnote{\booktitle{Nachgelassene Schriften und Wissenschaftlicher Briefwechsel}, vol. 2, Felix Meiner, Hamburg: 1976. (Gottlob Frege, The Philosophical and Mathematical Correspondence, University of Chicago Press: 1980)}

\speaker{FLYNT} Yes, yes, yes. You're talking about an obscure origin of something 
and what I'm talking about is a kind of consensus that had grown 
up, since everybody agreed that mathematics should study unreal structures. 

\speaker{HENNIX} But that consensus was forced on us, that that was what we 
were supposed to do. 

\speaker{FLYNT} The problem then---I thought mathematics was like chess. 
What I understand now is that even a good formalist would not agree 
with that. A good formalist would say that when you have a finite game 
like chess, the problems of validity and soundness become transparent 
or intuitively ascertainable, therefore a finite game is too trivial to be a
proxy for mathematics. At that time I did not understand that distinction. 
I've read in many books since then that mathematics is the science 
of infinity---that is the way mathematics is defined now in half of the 
books that I look at. But at that point I did not understand. I thought 
the finite game was already, I mistakenly thought, a complex enough 
problem to stand for mathematics. Or that the reliability of a finite 
game was sufficiently complicated to stand for mathematics so I basically 
focused just on a finite game. 

\speaker{HENNIX} By the way, this was exactly the late Wittgenstein's view of 
the philosophy of mathematics---it's not a complete misunderstanding, 
that is to say, other people thought of it that way too. 

\speaker{FLYNT} The question then arose of even the soundness, the reliability, 
the consistency of a finite game---this then is the problem for example 
whether it is possible to follow a very simple rule correctly or not. The 
other thing that was feeding into everything that was going on was that 
Wittgenstein's \essaytitle{Remarks on The Foundations of Mathematics} was in 
the Harvard Bookstore when I walked in as a freshman my very first 
day there---so in other words I was looking at Wittgenstein's Remarks 
on The Foundations of Mathematics from 1957--- 

\speaker{HENNIX} Ten years before me--- 

\speaker{FLYNT} ---but very cursorily. Because I had a philosophical 
agenda---I passed over this material in a very cursory way because I had a 
philosophical agenda. I was not involved in the distinction between a 
finite and an infinite structure. I was not involved in that. 

\speaker{HENNIX} You thought there was no such distinction? 

\speaker{FLYNT} Well no, I thought that---it didn't seem that there was very 
much point in worrying about that when there were much more 
extreme problems to be worried about. But Wittgenstein wrote a lot 
about the possibility of following very simple rules. And I assumed that 
if there were epistemological questions for mathematics that this game 
interpretation---this chess interpretation---had displaced the question 
of the soundness and reliability of the mathematics to the possibility of 
understanding a very simple rule like writing the series "plus 2". 

And having gathered that this was the way that I should picture 
mathematics---I mean we understood very well that there were other 
pictures of mathematics, but we thought they were philosophically 
obsolete. In other words the person who believed that mathematics was 
a description of a real supra-terrestrial structure, and certainly there 
were people like that--- 

\speaker{HENNIX} Still today. 

\speaker{FLYNT} ---we thought that this was a philosophy that had been 
exposed as superstitious by Positivism and possibly even by Ockham 
several centuries earlier. So it was not that we didn't know about that. I 
drew a personal conclusion that that position could not be defended by 
any arguments that are acceptable by modern standards. What I really 
meant was by Carnap's standards. That's what modern standards 
meant to me. 

In my philosophy I was not concerned with the specifics of 
mathematics; I was concerned with the problem of how I knowa world 
beyond my immediate sensations. That was actually the question that I 
began with---the question of propositions of material fact, like "it is raining" 
or "the \textsc{Empire State Building} is at Fifth Avenue and 34th Street." 

I had read a very simplified exposition---it was actually some 
lectures that Carnap gave in England in the 1930s on what Positivism 
was.\footnote{R. Carnap, \booktitle{Philosophy and Logical Syntax} (1935).} 
They were very simple lectures and very different from his actual 
published books with all this supposed apparatus and symbols and so 
forth but a very simple exposition of what it is for a proposition to be 
meaningful---that it must be empirically testable and so forth and so on 
and the solution of questions of metaphysics that make assertions that 
are not testable are therefore meaningless---the possibility of solving 
questions of what is real by declaring if there is no way of deciding them 
they are therefore meaningless. That seemed to me to be, at the time, a 
stunning contribution. Because I come out of a background---I was in 
high school reading Kant and so forth and so on. And Carnap's 
solution was much more attractive to me than trying to participate with 
Kant, to experience his question and try to take one side or the other 
when he already said it's not really answerable; I solve it by simply 
having faith or something like that, which is what he said about the 
famous God freedom and immortality---I found it immensely attractive 
when Carnap came along and said that there is no way of answering 
these questions; therefore, words are being used nonsensically. 

I went through a process of thinking about that without ever 
having seen Carnap's \booktitle{The Logical Structural of The World}. When I 
was in Israel Scheffler's philosophy of science class, I tried to write a 
text which in effect gave my own empiricist constructions of what it 
means to say that A causes B and so forth, to give empiricist constructive 
definitions of those---which is, I suppose, in the spirit of Carnap's 
program, even though I hadn't actually seen what he had written, and if 
I had it would have confused me---no, I wouldn't say "confused"; I 
would say it would have discredited him completely. I wouldn't say 
"confused" because that's too modest. 

\speaker{HENNIX} No, I wouldn't think "confused," I would think it would 
have upset you\ldots

\speaker{FLYNT} No, I wouldn't say "confused." I would say he had been 
discredited. 

I very quickly passed to the position that the propositions of 
natural science were meaningless metaphysics. 

\speaker{HENNIX} On what basis? Can you pin that down? A little bit, only. 

\speaker{FLYNT} This is something I want to compress---it says a little bit about 
this in \booktitle{Blueprint for a Higher Civilization}\footnote{H. Flynt, Blueprint for a Higher Civilization (Milan, 1975). Recently reissued and an expanded and corrected edition by \textsc{Salitter Workings}}---like 
the proposition, "this key is made of iron" or something like that, I comment on that in the 
essay \essaytitle{Philosophical Aspects of Walking Through Walls}.

\speaker{HENNIX} I didn't recall the example actually. 

\speakermod{FLYNT}{reading} "The natural sciences must certainly be dismantled. 
In this connection it is appropriate to make a criticism about the logic 
of science as Carnap rationalized it. Carnap considered a proposition 
meaningful if it had any empirically verifiable proposition as an 
implication. But consider an appropriate ensemble of scientific propositions 
in good standing, and conceive of it as a conjunction of an infinite 
number of propositions about single events (what Carnap called 
protocol-sentences). Only a very small number of the latter propositions 
are indeed subject to verification. If we sever them from the entire 
conjunction, what remains is as effectively blocked from verification as 
the propositions which Carnap rejected as meaningless. This criticism 
of science is not a mere technical exercise. A scientific proposition is a 
fabrication which amalgamates a few trivially-testable meanings with 
an infinite number of untestable meanings and inveigles us to accept the 
whole conglomeration at once. It is apparent at the very beginning of 
\booktitle{Philosophy and Logical Syntax} that Carnap recognized this quite 
clearly; but it did not occur to him to do anything about it." 

The only point that I'm trying to make here is that I began to move 
very quickly when I was still very young towards a position of extreme 
disillusionment and cognitive extremism. I moved very quickly. This 
was not a slow process. I just immediately took Carnap's critique of 
metaphysics, decided that it applied directly to natural science---you 
dismiss natural science as meaningless. The problem: is there an object 
that is beyond my experience, is there a glass which is beyond what they 
would call the "scopic" glass, the "tactile" glass \action{gestures toward the 
glass from which he has been drinking}---is there a glass other than 
those glasses---when you first think about it, that question seems to 
have exactly the status of the propositions about God, freedom, and 
immortality that Kant said are unanswerable and that Carnap said are 
meaningless. However, there is one additional step for people who are 
interested in the history of philosophy. Kant, in the second edition of 
\booktitle{Critique of Pure Reason}, added this notorious refutation of idealism to 
prove the existence of the real world independent of my sense 
impressions---you may not know about this---this was the basis of 
Husserl's phenomenology---Husserl's phenomenology was invented in 
this passage and it also tremendously preoccupied Heidigger. It was 
one of the sources which causes Heidigger to say that the essence of 
Being is Time. Kant said that essentially it is the passage of time which 
proves that there must be an external world. This is notorious in the 
history of philosophy. Because on the one hand it is so deeply 
influential for later thinkers; and on the other hand, for example, 
Schopenhauer said it was a complete disgrace---it was such an obvious sophistry 
that it was just disgusting---that it had the effect of ruining the 
\booktitle{Critique of Pure Reason}. 

Actually this refutation of idealism is distributed throughout the 
\booktitle{Critique of Pure Reason}, it's not in any one place---a foot note here, 
a preface there, another passage somewhere else. In one of the footnotes 
Kant makes the same point. In order to ask the question whether 
there is a glass beyond my sense impression of it---I cannot ask that 
question\ldots

\speaker{HENNIX} Oh you mean the \term{ding an sich} question. 

\speaker{FLYNT} Well that's what Kant would have been talking about but I 
don't want to fit that narrowly into Kant's controlling the terms of the 
discussion. I'm trying to ask it as someone who has embraced 
Logical Positivism and is now turning around to question Logical 
Positivism---you see the point that I was just making there---when 
you say that this key is made of iron, which is Carnap's favorite 
example---and then a protocol sentence, for example 
"if I hold a magnet near this key, the key will be attracted to the magnet"---it 
is not clear where Carnap stands on 
the question whether only my sense impressions are real---just talking 
about this situation---only my sense impressions are real---or is there 
supposed to be a substantial key? 

By the way, I don't know Carnap's work that well. I passed over 
these people in a very offhand way, so much so that many times I've 
talked to people and they've concluded in their own mind that I dont 
really know philosophy because I seem to have just glanced at these 
people---picked up one or two points---the reason for that is that I was 
moving so quickly to my own terminus---I only needed to see the 
slightest symptom from these people to know that they were spending 
all their time worrying about something that it was a waste of time to 
worry about since it could only be a secondary issue. Here is Carnap 
with this key made of iron---while I'm trying to ask is there a key other 
than the scopic key, the tactile key \emph{now}---since the past and the future 
are beyond immediate experience. I mean they cannot be cited as 
evidence---or whether they are evidence or not, is the same problem. 
Should I believe in the past and the future even though they are not 
immediates? Should I believe in the glass, even though what I 
presumably have is a scopic glass---at this very moment, a visual glass 
apparition, from that should I conclude a glass? 

The first reaction to that question for somebody who is coming 
from Kant and Carnap and who does not mind how extreme his 
answer is---that's the key thing. In other words, if I came to a 
conclusion that was completely untenable as far as social circumstances---that 
didn't bother me at all. At first the question whether there is a real glass 
beyond the apparition would seem to be an unanswerable 
question---one of Kant's metaphysical questions---but then you think---that if you 
know what the question means, then there must be a realm beyond 
experience, because otherwise it is unclear how the question could be 
understandable. 

From my point of view---if you want to make an issue out of 
semantics---this is the profound issue. What the mathematical philosophers 
and philosophers of mathematics were doing, talking about 
semantics, interpreting geometry as an algebra and algebra as a 
geometry---really for the purposes of relative-consistency proofs or 
because they found they could solve problems by using a machinery 
developed in another branch of mathematics by seeing these structural 
similarities---but to confuse that with what I thought the bona fide 
semantic question is: how would I understand the question whether 
there is a substantial glass other than the scopic glass---you know the 
conclusion---I can't tell you the exact breakdown---but I am talking 
now about the 1961 manuscript, \essaytitle{Philosophy Proper}\footnote{Published in \booktitle{Blueprint for a Higher Civilization}. This book.}
---I may have 
already come to the conclusion at that time---that the question itself 
forces a yes answer. This does not mean that a proof of the existence of 
the external world has been given. It meant that the proposition of the 
existence of the external world would verify itself even if it were false! 

\speaker{HENNIX} I find this extremely interesting and rewarding, what you are 
saying now, because I never heard you say it this way before. I just want 
to ask you one question before you go on: namely, I see something for 
the first time which I hadn't seen before---but before you go on I just 
want to ask you one leading question: the simple existential statement, 
"there is a glass on the table." You include that also in what will be 
doubtable here. In other words not just "there is a glass on the table" 
but "there exists a glass," the existential statement. I guess I wasn't very 
clear now. 

\speaker{FLYNT} No, the thing is, the approach that I'm taking doesn't break it 
down the way that you're talking about. Let me tell you. You may not 
be \emph{sympatico} with empiricism. When you are trying to deal with 
philosophy at all---you have to make some allowance for the 
fact---you have to understand that the philosopher may be carving up 
problems in a way that is temperamentally alien to you. 

\speaker{HENNIX} Yeah\ldots

\speaker{FLYNT} You have to understand that. This is why somebody like 
Carnap would read Hegel and say it's not saying anything. Actually, 
Hegel is saying something. In fact, you might go so far as to make a case 
that Hegel is actually rebutting Carnap, becaue if you understand what 
Hegel is doing you realize even more than one would realize anyway 
that Carnap has an untenable position---that he's sort of---that he 
wants what he cannot have. He has made a set of rules that does not 
allow him to have the thing that he demands to have. Hegel would have 
seen that immediately. Carnap thinks that the problem of a logic of 
consistency is an easy problem and a solved problem. In effect, Hegel 
was saying there is something very misleading in thinking that that is a 
solved problem. I'm trying to give you a sense of misunderstandings 
between philosophers that are the results of temperamental incompatibilities. 


\speaker{HENNIX} What you are giving me is a two-step way to skepticism. You 
ask a certain question---is there something beyond this perception of 
the glass? And you say the answer "yes" is forced on me, but then you 
realize this was a meaningless question. 

\speaker{FLYNT} No, it's the other way around. 

\speaker{HENNIX} Oh, okay, but here's where you have to explain in detail 
because here's where I miss you. 

\speaker{FLYNT} Let me go through the series of steps again. The series of steps 
was\ldots\  I'll have to doit all at the same time. You have to understand---I 
don't think that you even understand what an empiricist is. It's a 
peculiar attitude. And one of the reasons why you have very little 
training in this attitude is because people who claim to be 
empiricists---it's always a fraud. All people who appear in public and say they are 
empiricists, they are all lying all of the time. The reason that they're 
lying is that they have this doctrine of the construction of the world 
from sense impressions. That is their doctrine. But they do not stay with 
that doctrine. And the reason why they do not stay with that doctrine is 
because in addition to having the doctrine of the construction of the 
world from sense impressions, they also want to have things like 
science--- 

\speaker{HENNIX} Ethics\ldots

\speaker{FLYNT} No, not ethics---one of the characteristics of the twentieth- 
century philosopher was the appearance of the tough-guy philosopher 
who rejects all of ethics as meaningless, which Carnap certainly did and 
people who are close to him like A.J. Ayer---no, they did not want 
ethics. But they wanted science. And the problem with wanting the 
construction of the world from sense impressions on the one hand and 
wanting science on the other is that the two finally have nothing to do 
with each other at all---and when they said that the two were the same 
thing as Carnap did---he was lying---I made a hero out of Carnap---I
derived some kind of positive impulse from him or something like that 
without---I never actually read---my serious reading of Carnap was like 
three or four pages of excerpts in a paperback popularization. I owned, 
I had in my library Carnap's so-called real books, like 
\booktitle{Logical Foundations of Probability} and \booktitle{Meaning and Necessity} and all the rest of them 
and I never read them.\footnote{Again I'm being too diffident. I thoroughly studied portions of the 
Carnap books I owned---beginning with \booktitle{The Logical Structure of Language},
which I bought while in high school [H.F., note added].}
And in hindsight that was good, because I took 
his slogan seriously and assumed that he meant what he said and drew 
the necessary consequences of it. If I had actually read his books I 
would have been thrust into this massive hypocrisy, and I must say 
stupidity, because the man did not realize that his answers were not 
adequate, did not realize how preposterous his constructions of the 
world were--- 

\speaker{HENNIX} I would say vulgar. 

\speaker{FLYNT} Yes, yes. And\ldots\ what is even worse about empiricism is, in the 
case of somebody like Mach, not only does he want to have his sense 
impressions and does he want to have his science, but he wants to have 
science explain sense impressions! And nevertheless it was supposed to 
be the sense impressions that were primary, not the science. Mach is 
seriously telling you, I will tell you why you see a blue book---because 
the frequency of blue light is---and then he gives some uncountable 
number, I mean some number that is pragmatically infinite, or something 
like that. And how do you know that blue light is exactly 
$3.2794835\mathrm{e}{15}$ and not one more or less---? Well, 
certainly not by just looking, I'll guarantee you that! You have to go 
into a laboratory with a few million dollars' worth of equipment or 
something. But that's what it is to see that the book is blue. 

I'm trying to give you the sense of what it would be to be an 
authentic empiricist. You ask does a glass exist; an authentic empiricist 
would have to say that he already has a problem with that---that he has 
to regard that as an undefined question or statement. It's undefined, 
because if you are asking me if at this moment I quote unquote 
have---interesting word there, "have"---that is what our ordinary 
language gives us as the idiom for this. 

\speaker{HENNIX} Or "suffer!" 

\speaker{FLYNT} Yes, "have" or "suffer," that's right. I have or I suffer a scopic 
glass or visual glass apparition---then that is identically true. That is 
identically true. If you express any surprise at that, we have a problem 
here. I have a scopic glass. If I say I have an apparitional glass, would 
that be okay?---I mean from this point of view the sense impression is 
not open to dispute. It's meaningless to dispute it. It's an impression, an 
apparition---the sense impression is that for which seeming and being 
are identical. For the empiricist the phase of the world or range of the 
world for which seeming and being are identical is the sense impression. 
If that seems strange to you then maybe I can make it less strange by 
pointing out to you to make this as clear as possible---for the empiricist 
to say that I have an apparitional glass is to say nothing about Reality 
with a capital R at all! This is the so-called subjective psychological 
moment---although an empiricist would never say that---the reason an 
empiricist would never say that is that even to call it subjective is 
already much too strong because that implies that you can guarantee 
an objectivity to compare it to. And a bona fide empiricist would not 
agree that my sense impression is subjective---subjective in comparison 
to \emph{what}? 

\speaker{HENNIX} So an empiricist would be a person who would not doubt 
whether he had a toothache or not. In other words, if he had a 
toothache\ldots

\speaker{FLYNT} You would regard it as being a mistake to do what? I'm not 
sure about the word "toothache"---if you mean that he would not 
doubt whether he had a toothache sensation. Whether there is an 
organic---in the language of medicine---whether there is an organic 
substrate for the toothache impression---this in a medical sense is a 
question of what is called hysteria or something like that\ldots

\speaker{HENNIX} Suppose I have a toothache. But now I'm an empiricist so I 
say I'm doubting this impression. I probably don't have a toothache. 

\speaker{FLYNT} No, no\ldots

\speaker{HENNIX} I have to accept the toothache? 

\speaker{FLYNT} No, you don't have--- 

\speaker{HENNIX} The glass you said was---I couldn't doubt the perception of 
the glass. You said that was beyond doubt, in some sense, for the 
empiricist. 

\speaker{FLYNT} It would be some kind of logical mistake to think that there 
was anything there to be doubted. 

\speaker{HENNIX} Okay. And the same with the toothache. 

\speaker{FLYNT} Yes, yes. I mean the point is not so much that we have come 
into an area in which the empiricist is prepared to have faith---that 
would be completely missing the point. No faith is required---that's the 
point. The point is that it would be some kind of logical error. Once you 
understand what a sense impression is, the terminology of doubt does 
not apply to that level. 

\speaker{HENNIX} I see. Just that was my question. 

\speaker{FLYNT} The terminology of doubt does not apply to apparitions. It 
doesn't make sense to doubt subjective apparitions. The empiricist is 
already nervous when you ask does a glass exist. If you are asking 
whether I have a "scopic" glass, it's identically true. Wait, wait. There 
are already problems there. I'll come back to them. But when you 
say---it sounds like what you're asking me is whether the fact that I see a 
glass is sufficient to prove an objective glass---that sounds like \ldots

\speaker{HENNIX} No, no, that's not what---

\speaker{FLYNT} Well, ok. Most people when they say: 
"do you concede that there is a glass on the table---I'm sitting here looking at it," what they 
mean is: "do you concede that from your visual glass apparition you should conclude an objective glass, a substantial glass?" I'm taking it for 
granted that you know enough about philosophy to have a sense of the 
full weight those two words "substantial" and "objective" have in 
philosophy. 

\speaker{HENNIX} Yes. 

\speaker{FLYNT} That at great length is my reaction to your question about 
doubting "there is a glass on the table" versus doubting "there exists a glass." 
A bona fide empiricist would say, "Why are you asking me this?" 
The scopic glass is simply here for me. As far as concluding that an 
objective glass exists from the existence of that apparition---the traditional 
problem of concluding whether the apparition is a symptom of 
some transcendent world---I think the word "transcendent" is sometimes 
used in that sense in philosophy---the world beyond any sense 
impression--- 

\speaker{HENNIX} This is why I used the example of the pain---because it 
would be senseless for me to claim that \emph{I} can have \emph{your} toothache! 

\speaker{FLYNT} Now just a minute. An empiricist---what you're really getting 
at what you're sort of squeezing out of me here---I'm glad to have it 
squeezed out of me---I have no embarrassment about this---is that with 
empiricism either you must be prepared immediately to depart 
absolutely from the conventional world view, or else you will just plunge 
yourself into a quicksand of hypocrisy. When you're asking me, can l 
have your toothache\ldots\  A good empiricist would say, 
\textquote{I have not established so-called other people except the other-people apparitions 
that occur for me from time to time in waking life \emph{as they do in my 
dreams!} And are you now going to ask me can I have the toothache of a 
person who appears to me in a dream?} Then the spotlight would be 
turned on you---what kind of an issue are you trying to make there? 
What do you believe is the reality status of the furniture in my dreams? 
For the empiricist, nothing remotely like that question has arisen yet, 
because I haven't got outside of my own quote unquote head yet. 

Maybe you're just squeezing more and more. Either the empiricist 
must be a "madman" or else he must be insincere. I took the alternative 
of the madman. This is important not for me but for the general public 
to be told---something which the general public has never been 
told---and I know why they have never been told---maybe it is necessary to 
complete this point. The point is that empiricism was contrived to 
paper over a kind of---I mean there was sort of this 
epistemological---Science epistemologically was resting on some sort of very shaky 
foundation---they saw that. They brought in this empiricism in the 
hope that it would solve a problem, that it would substantiate science 
while at the same time it would cut away the common-sense notion of 
causality as being unnecessary to science. Empiricism was going to give 
you a more sophisticated science that did not need the traditional 
metaphysical or common-sense notion of causality. It told you how to 
get along without that, but at the same time it validated everything that 
the scientist needed. And, at the same time, empiricism was supposed to 
be---in the case of Neurath---he wanted to make some kind of unification 
of empiricism with Marxism and make it like a complete demythified view of society. 

\speaker{HENNIX} There was even an attempt to bring ethics into it. 

\speaker{FLYNT} Well, in Neurath's case, yes. 

\speaker{HENNIX} Schlick too, I think---Schlick, I recall, did something in 
ethics.\bootnote{\booktitle{Fragen der Ethik}, Vienna, 1930.}

\speaker{FLYNT} I was talking about why empiricism is not portrayed honestly 
in the general picture that exists of philosophy---the public picture of 
philosophy---it was brought in to solve the problem of what is a base 
for science---namely, sense impressions are going to be taken as 
elemental. Science is going to arise from sense impressions by construction. 
Nevertheless it is required that both scientific knowledge and the 
common-sense social world be produced by this approach--- 

\speaker{HENNIX} Neurath, you mean. 

\speaker{FLYNT} No, no. Well, Carnap did not deny the existence of other 
people. All of the positivists\ldots

\speaker{HENNIX} Rather, he had nothing to say about it. 

\speaker{FLYNT} I didn't say ethics---I said the common-sense social world. I 
wasn't talking about anything ethical\ldots


\speaker{HENNIX} The existence of tables and cars and--- 

\speaker{FLYNT} Well, what I'm saying is that the existence of other people is on 
the same level as the existence of tables and automobiles. And what is 
even worse than that is that the ones who were scientists in fact wanted 
to see perception itself as the product of the abstract and quantified 
sequence that the biophysicist or the psychophysicist sees---the light, 
the lens, the retina, the optic nerve, the visual cortex, and so forth and 
so on---they wanted to have that as prior to the sense impression but at 
the same time they wanted to have all that constructed up from the 
sense impressions. Why would this remain in place? Because it was a 
more palatable---it's just like why would formalism remain in place? 
Everybody learns that formalism died with Godel's incompleteness 
theorems---it certainly didn't die for me; it isn't even clear what the 
incompleteness theorems are supposed to have done or not to have 
done---the fact remains that if you don't explain mathematics as an 
uninterpreted calculus, then for us there was nothing left but 
superstition. Those are the choices that you are given. If you don't explain that 
science is constructed up froma ground of sense impressions, then how 
do you want it to be constructed, down from God? You see, we don't 
take that \emph{seriously} anymore. 

As a matter of fact Hume wrote two philosophical works and in 
the first work\footnote{\booktitle{Treatise on Human Nature}}
there is the notorious passage in which he himself 
understands what it means to be a genuine empiricist.\footnote{Book I, Part IV, VII "Conclusion"}
He says, \textquote{I feel that I am an outcast from the human race,} and so forth in this famous 
passage---he says, 
"I do not know if the glass continues to exist after I've looked away from it." 
That line in Hume should have told you 
whatever you wanted to know about the existence of the glass. You 
should be able to ascertain the appropriate answer to your question. 
Hume says: "I do not know if the glass exists when I look away from it." 

Hume's second book\footnote{\booktitle{An Enquiry Concerning Human Understanding}}, 
when he was trying to vindicate himself, 
when he had dropped the whole business of being a madman, it was 
much nearer to what empiricism means today: an attempt to construct 
science from a more meager inventory of elements, namely sense 
impressions. And that is where Hume presents his doctrine that science 
does not need and should not invoke metaphysical causation, that it 
should replace the old-fashioned causation with some sort of construction 
which is more flat or more network-like. 

Well, at any rate, I'm going into this long thing---this is why it's 
never dealt with in public in a sincere way---the only time it was was by 
the guy who invented it, Hume, in the book that he wrote when he was 
twenty-three years old. That's the only honest version of it and everything 
after that is a fraud. 

The way it goes is this: I ask the question whether there is a 
substantial glass, an objective glass, a material glass, something that is 
over and above the visual glass of the moment. When first considered 
this seems to be a question which I have no method of answering. That 
would seem to place it like a Kantian metaphysical question which 
doesn't have a provable solution, though interestingly enough Kant 
thought that the existence of the external world in general could be 
proved but only in the second edition. And in that second edition in 
those little passages, Kant did really get into the existence of this 
individual thing like a unicornand how that would or would not fit into 
the general proof of the existence of the world and also the question of 
how dreams would affect the validity of the proof. He touches on all of 
those in a way which is just awful. It's a disgraceful performance. But he 
had the issue there, actually. 

Well, your first reaction is, "I have no way of answering this." Your 
second reaction is, that \emph{if I understand the question}, then there must be 
an external world. So it would seem that I have actually proved the 
external world---that's what Kant actually said. Or he came very near 
to saying something like that. The third step is the realization that the 
statement would validate itself not only if it's true---but if it's false it 
validates itself equally well! 

\speaker{HENNIX} Given this method of understanding the question. And the 
method remained unspecified so far---as far as I know nobody has been 
able to do very well at specifying it. 

\speaker{FLYNT} What? Do you mean if somebody asks whether there is an 
external world---my last remark is a comment about semantics---the 
genuine semantic issue, as I said, and it's very different from the sort of 
thing that Tarski is going on about which I think is just ridiculous. 

Maybe I'd better stop and tell you why I think it's ridiculous. It's 
because I'm now talking about things which are exactly the fundamen- 
tal issues. If Tarski thinks that he can talk about the theory of chess 
before the question of whether the universe exists or not has been 
answered---they are deliberately creating specialized problems which in 
their minds do have answers and then they are proceeding to answer 
them. The larger question of whether the work has any meaning at 
all---it's like somebody spending his whole life working on the King's 
Indian defense in chess or something like that, and thinking that 
somehow that makes it unnecessary to answer such questions as does 
the chess board exist or is it only apparitional? If it's only apparitional 
then there is no guarantee of the continuity of the position of the pieces 
in the absence of moves. What happens is that people treat those basic 
questions as if they are so basic that it's sort of preposterous to make an 
issue of them. Kripke said very clearly in his book on Wittgenstein that 
once the question, "Does language exist?" has been asked, not to give 
an affirmative answer is "insane and intolerable."\footnote{S. Kripke, 
\booktitle{Wittgenstein on Rules and Private Language}, p.60} 
It's the same reaction as there is to solipsism---that solipsism is the philosophy of the 
man in the lunatic asylum. 

The thing that may come before all the discussion so far is the 
question of \emph{what is my position on being classified as insane} is the 
beginning This of philosophy for me. 

\speaker{HENNIX} Well, this is the classical beginning of philosophy. 

\speaker{FLYNT} Because if you're not willing to face up to being classified as 
insane---if you want to avoid that confrontation---you can't be a 
philosopher. That confrontation is at the center of bona fide philosophy. 

\speaker{HENNIX} Or was\ldots

\speaker{FLYNT} Yes. At any rate, I had reached this point in something like 
1961. I had not yet done \essaytitlte{The "Is There Language?" Trap}. But I had reached 
the point of saying that to claim the existence of a world beyond 
experience is untenable. However I understood very well that it begins 
to create problems for me to say, \textquote{I have a visual glass apparition,}
because there is a lot of structure in that sentence. And it's not clear 
what is supporting that structure after the world has been cut away. 
Even the use of the idioms like "have" and "suffer." The use of the word 
"I"---after the objective world has been cut away it's unclear what is the 
basis for all of that. And this is the point I had reached in 1961 and this is 
the point when I did \essaytitle{Concept Art}.

On the one hand you have an art which is about structure and 
conceptual things. On the other hand this art is not going to \emph{affirm}
traditional doctrines of structuredness and conceptualization. It is 
deliberately in every case going to violate them. It is going to express the 
fact that there has been a philosophical discovery made. I would have 
said chess is not a sound game. It's not well founded. It can't be. The 
whole problem of Wittgenstein's famous question---what is the meaning 
of a rule? My answer would be it doesn't have one. When you look 
at it from the standpoint of Hume when he says I have become a 
monster, I am outside the human race---the standpoint of the person 
who chooses insanity as opposed to intellectual dishonesty! 

The person who chooses being a madman---even chess doesn't 
work. The whole question of its consistency. The point of Concept Art 
is on the one hand to transmit the tradition from the isorhythmic motet 
and the five Platonic solids, in Leonardo---and on the other it's to blow 
it up because each work of concept art must be a counter-example to 
that tradition. And at the same time to say that it is art means---when I 
passed to \essaytitle{Concept Art} I left behind many things that traditionally 
would have been considered crucial features of art, like sentiment, for 
example. Let me just leave it at that. 

When the Renaissance people did study geometry and art, they 
developed perspective to paint people, not to paint abstractions. And 
you know I have to admit quite bluntly, my Concept Art was already 
the product of the acceptance of an abstract art. And now, many years 
later I can see that that was an historical juncture, to consider it 
tolerable that art should break with sentiment and with the representation 
of people. It's like moving toward an Islamic view of art. And then 
saying, now however, in the future, instead of Mosque decoration we 
will do a piece that has the visual, sensuous delectation, but it's completely 
abstract. But whereas Islamic art was trying to express the 
\emph{truth} of a certain theorem in group theory, Concept Art must express 
that you can't have that---that that theorem fails. Now I'm formulating 
an unsolved problem---I never did a concept piece the purpose of which 
was to rebut the symmetry involved in a visual pattern, with that as the 
opponent to be hit. I mean I very well could and perhaps should. 

All of my pieces were uninterpreted calculi. Because I accepted 
that that was the only way of explaining what mathematics is: that it 
consists of a body of truth about a world that does \emph{not} exist, and 
explicitly so. And that all of the traditional explanations of mathematical 
content are now seen to be anachronistic superstitions. They are just 
indefensible in the modern world. Put those two things together and 
mathematics becomes a chess game, an uninterpreted calculus. 

All of my Concept pieces are using the terminology of Carnap's 
\booktitle{Logical Syntax of Language}---the formation rule, the transformation 
rule---but in each case they wish to express the violation, the failure of 
some traditional organizing principle of these uninterpreted calculi, 
For instance there is one where, among other things, the very notation 
itself has an undisplaced active interaction with the subjectivity of the 
quote unquote reader.\footnote{dated 6/19/61---later titled "Illusions."} 
And that determines the structure of the derivation, the proof. 
It was pointed out to me many years later that it's not 
just that you don't get this in schoolbook mathematics---this is what 
they are most concerned to exclude. 

I had another one, in which there was no general transformation 
rule.\footnote{\essaytitle{Transformations}, retitled \essaytitle{Implications} in the second edition.}
There were only completely nominalistic transformation rules, 
In other words, for each step you are told, for that step only and for this 
moment only, what the transformation rule is. And by the time you are 
ready to take the next step, that rule is forgotten and inoperative. 

\speaker{HENNIX} This is the \essaytitle{Energy Cube Organism}? 

\speaker{FLYNT} No, no. \essaytitle{The Energy Cube Organism} was not Concept Art at 
all. No, no. It was a different genre. That one was the piece called 
\essaytitle{Transformations}.

\essaytitle{The Energy Cube Organism} and the \essaytitle{Perception-Dissociator} in my 
own classification are not Concept Art. Only the pieces labeled 
"Concept Art" are Concept Art. And I only did four of them until 1987. 
Three of them are in \booktitle{An Anthology}, and the fourth was published in 
\journaltitle{dimension 14} (1963). \essaytitle{The Energy Cube Organism} and the 
\essaytitle{Perception-Dissociator} were in other genres. I drew these distinctions of genre 
rather narrowly, actually. 

This is the one \action{pointing to 6/19/61 in \booktitle{An Anthology}} where there 
is, in an uninterpreted calculus, interaction between notation and the 
subjectivity of the quote unquote reader. 

This is \essaytitle{Transformations.} You are just taking these objects, you 
are burning them, melting them, doing all sorts of things to them. The 
point of this is that each step in the proof---you have to think of it as a 
proof---you see it has the tree structure of a proof. This is my nominalistic 
transformation rules, because each rule is stipulated only at that 
step, and then it is thrown away. The point that I was trying to express 
was that's what they do in all of it---even in chess, when you move the 
pawn to King's Bishop 3, you think that you are conforming to a 
general rule written in Heaven. But in fact there isn't any general rule, 
and when you move the pawn to Bishop 3, you're just making up what 
you are doing right at that moment, and there isn't any general rule. 

\speaker{HENNIX} You would label this ad hoc? 

\speaker{FLYNT} That's right. That would be perhaps a better word for it. All 
transformation rules and probably even all formation rules are ad hoc, 
yes, yes. 

I said "nominalistic" because they are only there individually. 
They do not add up to any general--- 

\speaker{HENNIX} System of rules? 

\speaker{FLYNT} No---not that---they do not add up to any generality, to a 
general rule that covers all cases of a certain class. 

What is inadequate about this---and I realized very quickly that 
it's inadequate---is that this does not actually give some profound 
reason in concrete practice for questioning chess. That's what the 
inadequacy of the original Concept Art pieces is. That they don't really 
give you some kind of operative situation where you can see that 
following the chess rules is failing. I don't provide that. I only provide 
something that's ritualistic. Saying this is how you would behave if you 
realized that following any rule is ad hoc. 

A conventional mathematician would say, you have not proved 
that the world that this is designed for is the world that I have to live in. 
\textsc{That}'s the inadequacy. He would say that I am only ritualizing the 
world of impoverishment or disorganization. I'm not showing that 
that's the world that people in general have to live in because it's in 
force. That's the difference between then and now. The reason that I 
want meta-technology would be to give a situation where somebody 
can actually see that you \textsc{Can't} play a game of chess---or that you want 
to play one and that I, by putting it in the appropriate context, make it 
clear that the general rules on which playing it depends are not in fact 
available. 

But to show that in a serious way. From the prevailing point of 
view I would be talking about contriving a miracle. In other words, to 
actually substantiate any of these---what is interesting is not so much 
\essaytitle{Transformations}---but it would be some situation that would substantiate 
that the conventional view is actually unavailable. And to do 
that you have to violate what are considered today to be the soundest 
laws of science. I'd need a miracle to manifest that I'm right, so to 
speak. So by the time I get to meta-technology I'm in the job of 
constructing miracles, I mean constructing situations that are
absolutely physically impossible (or in some cases logically impossible) by 
currently accepted scientific and commonsense views of what is the real 
world. 

\essaytitle{Innperseqs} is the one that is visually sensuously the best. You are 
making a rainbow halo that you can get by breathing on your glasses 
and looking at a point light---you get a rainbow halo around the light. 
Eventually I will set it up so that you don't need glasses or anything so 
that the whole business of seeing the rainbow halo is moved out and 
does not require any special preparation by the spectator. The rainbow 
halo is the sensuous delectation. The derivation, the proof, the 
specification of propositions, is something that you do as the halo is fading. 
You have to quickly specify---I never analyzed exactly what was going 
on there but it was as if---you have a notation which is externally 
changing, and therefore the quote unquote reading of a mathematical 
system has to be a process that is taking place in experienced time. 

By acts of attention you have to choose sentences, to choose 
implications---it's a display. You are given an external display which is 
changing out there, not in your head. And you have to place a structure 
on it by specified rules. 

You know another point that can be made is, that \essaytitle{Innperseqs} is 
philosophically inconsistent with \essaytitle{Transformations}---that these pieces 
are mocking each other. 

At the time that I did this, I did not have the kind of maturity that I 
would have today to put it together in a strong way. These were 
gestures. And they are not even uniform ona question like whether a 
rule exists or not. Well actually, frequently I'm too hard on myself. I 
think that in the essay \essaytitle{Concept Art} I do say something like, objective 
language doesn't exist, but I'm still free to work with what you think the 
text says---I can use that in \emph{art}: this is \emph{art}! 

There are three ways that the art part comes in. One is the visual 
display, the delectation. The second way the art part comes in is---well, 
if LaMonte Young's Word Pieces are art, then this is art too. But the 
third thing is that this does not claim to have objective truth. It is a 
construction for the world-hallucination or the world-apparition or 
even a construction for the private world-apparition. 

\speaker{HENNIX} You are actually extending the world by new constructions. 

\speaker{FLYNT} But it's the world-apparition. In a sense if I believed that these 
rules were objectively established, then it would almost indicate that I 
had not learned the lesson of the very piece which sits beside it on the 
page!\footnote{\essaytitle{Innperseqs} versus \essaytitle{{Transformations,} second edition.} 
And what am I doing talking about a page and a text? So the 
answer is that I have abandoned the provision of truth as the purpose of 
this activity and I have moved to the provision of experiences where the 
possibility of these experiences is a surprise. 

\speaker{HENNIX} And you don't have to be an empiricist to be surprised. 

\speaker{FLYNT} Yes. Yes. But the truth claim that you would have from a 
Kripke or a Goodman has been dropped. The meaning of the text is the 
meaning that the reader associates to it. And the thing is, that in 
conventional intellectual work that's an unacceptable answer, because 
usually you are trying to get independent of the reader's 
distortion---that's the whole hope---that you can make something that is 
independent of the reader's distortion of it. This is a different game. This is not 
classical mathematics; it's not classical science. It's like giving a 
Rorschach blot. Then I don't mind if you have a unique subjective reaction. 
If my purpose is to make Rorschach blots, then I do not object, I have 
not failed, if you have a unique personal reaction. 

These pieces are designed for the individual reaction rather than in 
spite of it. 

The only other Concept Art piece---in \journaltitle{dimension 14}---\enquote{one just 
has to guess whether this piece exists and if it does what its definition 
is.} That was the piece. And that was a response to Cage's dissociation 
of what the composer sees, the performer sees, the audience sees. 
Starting from that, going through all the games that LaMonte had 
played with the idea of performance, where we were performing pieces 
first and composing them second, maybe many months later. So finally 
with the Concept Art piece, even whether the piece exists is completely 
indeterminate, but I meant for people to try to take that seriously. I was 
having a joke with the person who thinks that concepts form an 
objective world, which the individual who cognizes only discovers bit 
by bit. In effect, 1am giving him this: thank you for believing that there 
is a piece here---I'm leaving it to you to find it. I wash my hands of that 
Problem---\emph{you} find it! 

Well, there's a natural pause that comes here because I think that 
I've summarized perhaps fairly thoroughly where I was when I did the 
work published in 1963. The entire subsequent career of the label 
Concept Art, its misapplication to Word Pieces and all the rest of it, we 
have not begun with. After that, we can go on to the discussion of your 
visual pieces of the 70s and how they resume the genre of Concept Art. 

