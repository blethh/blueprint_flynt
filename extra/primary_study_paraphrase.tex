









\chapter{Primary Study: Informal Paraphrase (1979)}

Consider the assertion that there is (meaningful) language at some time and place---that is, something more than arbitrary marks on paper, something with objective rules and referents. The assertion that there is some language should be a descriptive assertion. (In fact, a descriptive assertion, about natural language, in a natural language.)\footnote{Logical positivists may object that the assertion that there is language is and should be analytic (or meaningless). But this objection cannot be sustained. The attempt to find a criterion which will exclude some grammatical English sentences as meaningless (or analytic) without excluding too many, and without being arbitrary, has always failed. Indeed, is it not so, that one brings out the concepts of "analytic" and "meaningless" arbitrarily, to cover up embarrassing problems? Existence claims, that is, descriptive assertions of the form `There is \uline{\qquad}', are unavoidable in cognition. And it is not plausible that any grammatical natural-language statement is true independently of all experiential or contingent considerations. Nor can an arbitrary, unsubstantiated condemnation of some grammatical statements in the natural language as meaningless be respected. And a retreat to artificial languages cannot evade the questions at issue above.}

The questions at issue here cannot be evaded by retreating to artificial languages, because artificial languages cannot be constructed, or explained, or taught, independently of the natural languages. (Which is why I don't use logical symbols in this study.)

The point is that the assertion that there is language should describe a very definite state of affairs. At birth, every human knows no language at all. For a human to first learn language, language must be related to non-language, or in general, to something outside itself. There is, then, language which is relatable, and related, to non-language. There is some (meaningful) language. There is language. The assertion that our language relates to non-language should be descriptive, just as I said at first.

Since the assertion that our language relates to non-language should be descriptive, it must be possible that it is false. It should be possible to doubt, or even to deny, that there is language, without contradicting oneself or otherwise nullifying oneself (just as it is possible to deny that there is a Taj Mahal without contradicting oneself).

But---

It is not possible to doubt that there is language, or to assert that there never is language, without falling into a fundamental contradiction---because this doubt and this assertion are themselves language.

The assertion that there is language must be true if it can be made at all, unlike a descriptive assertion. The assertion that there is language should be descriptive---but it cannot be.

This "trap" vitiates language. Primary Study (1964) was written to exhibit it, by a more formal approach.
