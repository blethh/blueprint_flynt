\chapter{The Radicalism of Unbelief (1982)}

If we are going to talk about enlightenment and deliverance, I do not see that 
enlightenment and deliverance can come from anything as straightforward as an 
individualistic search for happiness, or a mental hygiene of happiness. To me, the 
life I have now, which unfolds through ongoing interaction with other people---and 
within this definite culture---is the arena that matters. In other words, I am located 
in a \emt{shared} basis of life. To me this circumstance is of outstanding importance. 
While the medium of thought, the capabilities, the skills which are possible for me 
are interior to me, at the same time they engage me with other people in 
consciousness---and I must regard other people as their source in most cases. (In other 
words, I do not invent the English language, etc.) My consciousness and my 
capabilities are, by and large, a fragment of a culture. The most worthy capabilities 
in the culture become possible capabilities of mine. The most profound dilemmas 
or failures in the culture, in the interpersonal arena, become my personal dilemmas. 

What I have just said is not the same as the idolatry of "society." I do not accept 
the sociologists' notion of reality, or conformism as a goal, or the obligation to pay 
homage to societal abstractions like The Nation. Indeed, one of our culture's 
extreme dilemmas and failures is its idolatry of society, an idolatry which aggressively 
underestimates and devalues both the scope of the self and also the interpersonal 
arena. One of the most far-reaching questions posed by our contemporary 
era is whether inter-subjectivity (community) will evolve beyond "society" as it is 
defined by sociology (a sort of statistical mechanics applied to bodies). Here is an 
outstanding reason why I do not see how enlightenment and deliverance can come 
from an individualistic hygiene of happiness. The modalities necessary for enlightenment 
are novel and uncommon; and they are outside the scope of the ordinary 
person's struggle for happiness in everyday existence. The necessary modalities 
have to be achieved by dealing with dilemmas which arise from the culture as a 
totality: enlightenment requires a "rotation" (transformation) of the entire culture. 
Life is worthless unless I can inject whatever personal vision I have into the 
ostensible, interpersonal arena, and seek to influence that arena so that it becomes 
conducive to my sincerity and concern. 

In order to express whatever sincerity and concern I have in the ostensible, 
interpersonal arena, I must engage with the ostensible world; I must incur the risk 
of realized choices; and I must "grant other people's right to exist." 

\visbreak

\emt{What I seek is a transformation of the ostensible world and of the shared basis of life.}
This is to be accomplished on the basis of two enterprises which will eventually be 
fused: a theory of palpable interrelations of the entirety of immediate constituents 
of "my world" called "the personhood theory"; and a new instrumental modality 
called "meta-technology." In this introduction, I will focus on meta-technology 
without bringing in the dimensions added to it by the personhood theory---largely 
because the latter is as yet tentative. But I have another reason as well for 
underlining the contribution of meta-technology. \textit{There can be no genuine transformation 
of the shared basis of life as long as the community's technological means is restricted to the 
material technology we know today. The instrumental modality must come to embody the 
takeover of technology by the psyche, by personhood. There is no genuine transformation 
of the shared basis of life unless instrumental efficacy is at stake in that transformation, 
unless the challenge to the prevailing basis of life is carried into the domain of material 
technology.}

As of now, I have assembled many meta-technological elements or procedures. 
These elements, however, are isolated and limited. What I have accomplished is 
analogous to Becquerel's discovery that uranium fogs photographic film, My 
procedures are effective as curiosities. But they will not be any more than curiosities 
until they are subjected to an entire phase of extension and interconnection---an 
undertaking which requires collaborative effort on a wide scale. 

On the other hand, the analogy to Becquerel is misleading in that a meta-technological 
procedure is of an entirely different species from Becquerel's discovery. 
Radioactivity occurs in the exterior realm of things (objectivities): it is an 
effect of a thing on another thing. But generally speaking, a meta-technological 
procedure is based not on a relation between things, but on an interdependency 
between subjectivity and things. 

Because I am located in a shared basis of life, a culture, that culture is of 
overwhelming importance both as a source of possible capabilities and as a source 
of dilemmas and limitations. To respond to this state of affairs, the meta-technology 
must accumulate information which is of more than personal significance. It must 
address dilemmas which are shared and which are culture-wide. That is why I 
investigate mathematics, "real-world" logic, etc. It is also why my interest in 
dreamed experience relates to a proposal to modify the shared basis of life---rather 
than to the familiar purposes of divination and psychiatry. 

I disregard all claims of sorcery or miraculous feats which inherently come as 
reports by a second person about what a third person did (tall tales, fish stories, 
legends). lam not interested in miracles which are always performed by somebody 
else somewhere else. Indeed, my objections to occultism go much further than this. 
But the principle which I want to emphasize now is that every meta-technological 
procedure is required to be formulated as an instruction to be carried out first-hand. 

Below I will explain that a starting-point of meta-technology is an 
adversary attitude towards credulity. One aspect of this phased unravelling of credulity 
is a critical examination of claims of meaningfulness for reportage which intrinsically 
precludes first-hand testing. 

\visbreak

What then is my attitude to the immediate, overt, ostensible world? I have little 
use for the doctrine that the ostensible world is a sham which conceals another, 
perfect world behind it---a perfect world which can only be known by hypothesis. 
In other words, I do not treat the ostensible world as a facade for something lying 
behind it, as a front for another world which is unperceivable. And I have little use 
for the notion of a perfect world which is hypothetical and imaginary. \textit{This present 
life, which unfolds through ongoing interaction with other people, and w ithin this definite 
culture, is my arena of concern. Imaginary lives and gratification in fantasy are unimportant 
to me. I accept the ostensible world as the arena of my concern, and as one of the 
raw materials of enlightenment and deliverance.}

The attitude I have just expressed does not imply that I admire whatever 
ostensible world we inherit. Quite the opposite. \textit{Precisely because the ostensible world 
matters to me, the arrival of enlightenment or deliverance has ta be demonstrated by a 
transformation of the ostensible world and by a transformation of the shared basis of life.}
Further, while I do not view the ostensible world as an illusion standing between 
me and some perfect world which must be known by hypothesis, there is a sense in 
which I view the ostensible world as a delusion. It is a delusion in that the very 
perceptions which characterize it are palpably affected and sustained by emotions 
of anticipation, by emotional dependence on other people, by morale, by esteem, 
by knowing self-deception, etc etc. \textit{Everyday existence is the hallucination produced 
by the so-called socialization process. Morale, esteem, etc. are co-determinate with 
"perception."}

Thus, again, the arrival of enlightenment or deliverance has to be demonstrated 
by a transformation of the ostensible world and of the shared basis of life. But what 
I propose is not to strip off the ostensible world to reveal a unique perfect world 
behind it. Rather, I want the ability to consciously "mutate" or plasticize the 
ostensible world itself. 

\visbreak

Inasmuch as I demand that enlightenment and deliverance should be evinced 
by transformation of the ostensible world, I am a kind of secular revolutionary. 

To me, the means of enlightenment and deliverance must begin with an adversary 
attitude toward credulity and toward phenomena whose existence is solely a 
product of credulity. In the Seventies, there was a rash of novels in the U.S. about 
demonic possession. The protagonists in these novels were always Catholics. The 
novelists knew that Catholics were protagonists who could plausibly be liable to 
visitations by demons. If you do not want to see demons in your living room, 
all you have to do to escape them is to stay outside the subculture that believes 
in them. 

The lesson of this example, properly understood, is the starting point of enlightenment 
and deliverance for me. If it is obvious that a phenomenon can be 
abolished by unbelief, then the "reality" of that phenomenon is of a very low order. 
The phenomenon has only the reality of chimera or fantasy. (On the other hand, it 
is obvious that a lor of people enjoy their chimeras, and \textit{do not want} to escape 
everything that can be abolished by unbelief.) I make it a principle to disregard 
phenomena whose existence depends so obviously on credulity. The attitude 
which is encouraged by all kinds of superstition and propaganda is "How many lies 
can I (manage to) believe?" The question which I always ask is "How much of what 
I am expected to believe is a lie?" 

On the other hand, the tssue of whether the existence of a phenomenon is 
a product of credulity \textit{is not necessarily straightforward}. In the first place, 
\textit{we have to distinguish between getting rid of phenomena by unbelief and 
getting rid of them by suppression or censorship}. I often encounter situations 
in which scientists refuse the opportunity to experience an anomalous phenomenon.
Nobody denies that the phenomenon is "real," that is, accessible at first hand. The phenomenon is 
disregarded or suppressed because it is a nuisance, because it conflicts with the 
scientist's ideology. 

In the same vein, I am not asking anybody to deny his or her own experience just 
because it is abnormal, anomalous, or singular. But I am asking that such experiences 
not be misrepresented and inflated through knowing self-deception---especially 
in reporting them to others. I have long speculated that reports of 
so-called astral projection etc. might have an experiential basis in hypnagogic 
hallucinations etc. Unfortunately, the sort of person who relishes reporting such 
episodes is also prone to inflate them via culturally supplied hyperbole. Reports of 
abnormal experiences could have serious uses if the reportage did not surround the 
experiences with chimerical objectivities, and if it took a painstakingly critical 
attitude toward the "ontological" assumptions built into descriptive language. 

Another consideration is that \textit{a thorough and ruthless effort to repudiate all 
phenomena whose existence depends on credulity will begin to undermine phenomena which 
our culture defines as legitimate and plausible. Unbelief does not just dissolve 
superstitions and chimeras; it begins to affect phenomena which rational authority defines as 
valid.} At this point rational authority has to step in and disparage unbelief as a 
social blunder. Here the role of community intimidation in sustaining the ostensible 
world comes to the surface. But I do not shrink from this consequence of 
unbelief. \textit{Indeed, the radicalism of unbelief is a basis of my ability to obtain results which 
are novel and astonishing relative to the established culture.}

\visbreak

Let me give some examples of meta-technological investigations: 

\term{A priori neurocybernetics}\footnote{Neurocybernetics is an existing branch of neurophysiology which seeks to explain thought by investigating the brain as a "bionic computer."}
deals most directly with interdependencies between awareness and objectivity. As one example, it uses perceptually multistable figures\footnote{e.g. the Necker Cube. \cubeframe}
as logical notations. The result is to establish awareness-objectivity interdependencies in language which are tangible and inescapable and can be analyzed 
and potentiated. The technique can be applied to break the framework of scientific 
objectivism in many ways. As another example, I note that our "perception of 
objects" is actually a mental collation of visual and tactile apparitions. There are 
many cases in which the normal intersensory correlations are disrupted (the perceptual illusions)\footnote{
An example of an intersensory discorrelation is Aristotle's tactile illusian: touch the tips of crossed forefinger and middle finger at the left hand te a projecting dowel while also looking at the dowel. You see ane dowel and feel two. The perceptions the two fingers are not only disjoined. they are inverted. The subject attributes to the index finger what is touched by the middle finger and \textit{vice versa}, as can be shown by applying two distinct stimuli to the finger---a point and a ball, for example.
Even better: try the experiment first with eves closed, and then open the eyes. Sight captures touch, and the fingers are switched without any motion taking place. (Adapted from Merleau-Ponty, \booktitle{The Phenomenalagy of Perception, pg. 205.)}}. If we take the illusions as a paradigm and reinterpret "normal" 
phenomena in accord with that paradigm we are in a different reality, disjoined 
along the sight-touch frontier. Bode's Law that two material bodies cannot occupy 
the same position in space at the same time ceases to be usable, because the 
determination of what is a material bady is seen to involve a vicious circle. 

The \term{evaluational processing of experience} studies, as one example, the circumstance 
that different levels of reality are attributed to waking experience and dreamed 
experience even though both are equally vivid, equally palpable. What is at issue 
here is the fabrication of an "impersonal order of nature"; the inter-subjective 
character of reality; and the choice of rules for testing the objectivity of phenomena. 
Again, once these elements are understood consciously, they can be consciously 
altered. 

The \term{logic of contradictions} is a wide-ranging, umbrella discipline. The unifying 
theme of the discipline is the recognition that inconsistent conceptualizations, sa 
far from being vacuous mistakes which can be eliminated from thought, are 
pervasive and inescapable in thought as we know it. Conscious control of this state 
of affairs is an extremely powerful achievement. The investigation begins with the 
interdependency between traditional logic and perceptual habits in the real-world 
logic of consistency. It then considers perceptions or events which are faithfully 
described by inconsistent descriptions, such as illusions and dreams. I characterize 
these apparitions as contradictory because that is the characterization given them 
by shared language and paradigmatic real-world logic---as all the perceptual psy- 
chology textbooks agree. Then, I study contradictions which are cognitively im- 
plicit in our most authoritative or obligatory propositional thought. (Paradoxes of 
common sense; the meta-theoretic inconsistency of arithmetic and set theory.) 
Finally, I study how the communal milieu and its influence on esteem enables 
people to assent to openly inconsistent doctrine. (Mathematics' co-optation of its 
own inconsistencies; etc.) This research yields a very wide-ranging capacity to 
produce anomalies or uncanny world-states. 

My recent investigations into personhood have shown that meta-technology can 
be significantly widened and deepened by studying not only linkages of perception 
and descriptive language, but their co-determination by morale, esteem, etc. 
Studying the entire "vertical" organization of self or self-image could result in the 
realm of perception being transformed. 

\visbreak

Our civilization has long been characterized by the way it molds human faculties 
to produce a cleavage between scientific functioning, on the one hand, and poetic, 
emotional "human" functioning on the other. Meta-technology is beyond this 
cleavage of faculties. Also, it is worth repeating that meta-technology does not 
consist of the sort of magic tricks attributed to pre-scientific religious figures. What 
is a religious miracle like changing water into wine? It is---purportedly---an objectively 
consequential manipulation of the thing-world, a type of cause-and-effect 
technology. It takes place "out there," replacing a thing with another thing. 

Meta-technology does not appear as hearsay; and it does not make any special 
appeal to credulity. Rather the contrary. Its primitive procedures are given as 
instructions to be carried out at first-hand. Presupposing a conventionally 
indoctrinated individual, it achieves anomalies by a decrease of the conventional level of 
credulity. It is not centered on thing-to-thing relationships or causation "out there." 
It is centered on the interdependencies between subjectivity (awareness, self-image) and things. 

In addition, there is a third constituent important enough to be mentioned 
separately: \textit{the communal milieu, and especially its influence on esteem}---as when 
intimidation by community authorities maintains the legitimacy of ridiculous 
beliefs. It is at the juncture I have just sketched that "the world" is synthesized, that 
the determination of reality occurs. \textit{Meta-technology attacks the credulities which are 
elements of this juncture. It works with the linkages among "perception," descriptive 
language, and abstract cognition (logic, mathematics.)} Currently I am extending the 
research to include linkages to personhood---the high integrative level, the "vertical" 
organization of self or self-image. 

There is a big gap between the primitive meta-technological procedures which I 
have already formulated, and the communally implemented, culturally implemented 
meta-technology which I envision. The primitive procedures can be 
carried out by an isolated individual (and yield a sort of insight of sensibility); but at 
that level they are, in a sense, only curiosities. \textit{The whole point is that meta-technology 
acts on the cultural determination of reality as such. Unlike a miracle or magic trick, 
which wants to remain a one-shot event in an otherwise lawful everyday world, 
meta-technology must be extended through a community and a culture to realize its promise.} It 
is not a one-shot event but a "rotation" of an entire culture. 

What is more, to reach its full potential, meta-technology will probably have to be 
tied into existing natural science. But meta-technology would give a shock to 
natural science which must not be underestimated. Natural science would be conceptually 
shattered, and reorganized so drastically as to become unrecognizable. 

\visbreak

If meta-technology were implemented at the level of an entire community, 
that community would have the power to consciously modulate what is now 
thought of as the objective world. To speak of walking through walls would not be a 
mere joke, Both the physical universe and mental acts as its antithesis would 
disappear, in the sense of becoming inapplicable concepts. It would be possible to 
achieve sustained, composed uncanniness, to live in a state of consciously modulated 
enchantment. In this regard, the impulse underlying meta-technolgy is an 
impulse toward an ecstatic form of life. (It must be understood, however, that the 
rational mentality produced by modern Western civilization might experience the 
enchanted community as a nightmare.) 

When meta-technology shifts the focus from the thing-world to the interdepencies 
between subjectivity and things, it leads us to our whole humanness. It 
carries out a takeover of technology by the psyche or by personhood. For a 
community to attain a consciously modulated uncanniness would tend toward an 
ecstatic form of life---an achievement which the prevailing culture would classify as 
esthetic or spiritual, not scientific. That is what must be conveyed: acceding to 
one's whole humanness is neither science nor poetry because it is beyond both. 

\vfill

Postscript: The foregoing is not meant to promise a salvation which is blind 
to economics and politics. The present article is limited to giving a few rudiments 
of the meta-technology: my proposed extension or replacement for the physical 
and exact sciences. My views on the social context are at least as unusual as my 
views on science and form an entire line of argument in their own right. The 
transformation I speak of would clearly be in conflict with the capitalist formation. 
On the other hand, I hold that historical experience has obsolesced Marx's original 
timetable and game plan for the supersession of capitalism. 

Readers seeking more information or exchange of ideas are invited to write the 
author care of Ikon Magazine. 


