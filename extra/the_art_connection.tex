\newcommand{\topnote}[1]{{\vskip 1em \raggedleft \parbox{2.5in}{\raggedleft \textit{#1}} \vskip 1em}}
\newcommand{\Pb}{\plainbreak{2}}
\newcommand{\Db}{\fancybreak{{$\diamond$}}}
\newcommand{\arttitle}[1]{\textit{#1}}
\newcommand{\sidenote}[1]{\vskip 1em {\raggedleft \parbox{2.75in}{\raggedright\itshape #1}\vskip 1em}}
\newcommand{\asidenote}[1]{{\footnotesize{#1}}}

\newcommand{\lilsection}[2]{{\Large \textbf{{#1}.} \hskip 2em \textit{#2} \hspace\fill}}
\newcommand{\lilsubsection}[2]{\vskip 0.75em {\large\textbf{{#1}.} \hskip 2em \textit{#2} \hspace\fill} \vskip 0.75em}

\chapter[\textsc{The Art Connection:  My endeavor's intersections with art (2005)}][\textsc{The Art Connection}]{The Art Connection: My endeavor's intersections with art}

\topnote{26 October 2005}

If you ask me to review the art world or the institution of art, it's not an interesting question if I am asked to speak like an art critic.  On the other hand, I have a great deal to say about the art institution if I am allowed to view it as a piece of a civilization, and to practice censorious sociological aesthetics.  Beginning in 1963 especially, I made appraisals of fine art as:  an elite institution which distinguishes modern European civilization from other civilizations. 

But that critique of the status quo is less interesting to me right now than autobiography.

I can legitimately say that taking art as a the thematic axis for a chronicle of my work is not fair to the work.  In the first place, from the beginning I was interested in the correlation of arts.  I quickly graduated to \enquote{interdisciplinary projects} such as concept art, which had art as a precedent, but stemmed from my iconoclastic philosophy of 1960, and had outgrown art.  To force my projects back into the art mold made it impossible to understand them.

That bears directly on this recitation.  From the beginning, I was committed to correlation of the arts and of all culture.  I used compositional techniques from \enquote{new music} in visual works, and I used \enquote{abstract expressionist} drawings as pitch-time graphs, etc.  The early story cannot be fairly told unless unless I can move freely among the disciplines.  Concept art was assembled from works formerly called music and works formerly called mathematics.

As I turned to new ethnic music, my musical practice became more separable---and I will keep references to it to a minimum.  But that is still artificial, because various artists I have known have been musicians, Conrad, Hennix, and my exchanges with them are not compartmentalized into disciplines.

In the late Eighties, I revived concept art for tactical reasons.  Then I learned all over that one can't force the public to \enquote{get} concept art.  As long as its premise is unacceptable---outside the civilization---it will be received as \enquote{art junk.}

My engagement with art had many distinctly different periods.  Because I am unprecedented or iconoclastic, that's what I want to review.  I'm going to be autobiographical.

\Pb

During 1957--60, I fell in with an (American) crowd whose trailblazers made a mystique out of newness.  (Definable as \enquote{baffling \'{a} la Dada.})  I had probably dismissed opera, dance, and even fiction as corny in my teens.  I went through a rapid evolution in which I talked myself out of the branches into which culture was divided such as painting, sculpture, poetry, music.  I felt invited, by Young and his entourage, to charge ahead into the bafflingly unprecedented and the iconoclastic.  By late 1962, I had rejected art altogether, along with entertainment and competitive pastimes such as bridge and checkers.  In fact, in spring 1962 I called for \textbf{the civilization in one mind.}  By early 1963, I had convinced myself that Marx's social critique and social utopianism had to be integrated into the perspective of culture; the relatively apolitical postures of Cage and Young were incomplete at best and collaborationist at worst. 

\Pb

I took positions at the beginning of my twenties which I may still admire.  But I periodically have retreated from my severe positions for tactical reasons.  As to the Marxism, I disagree with what I formerly talked myself into assenting to.  However, it had great plausibility to be a Leftist in the early and mid-Sixties---I can't regret it.

At certain junctures in my life, it seemed that I was hopelessly isolated, that I was limited to mere survival while I piled up work that nobody knew or cared about.  I changed direction, sometimes retreating from my severe positions to do so. 

In some cases I'm comfortable with these retreats. 

In other cases, I feel that I was imposed on, that there was a social pressure which pigeon-holed me in ways that were unfair to my work.  However, there were redeeming features to the pigeon-holing.  My relationships with the impresarios, Maciunas and Harvey, were much richer than the usual artist-dealer relationship.  They were my support systems and confidants.   (Not that any \enquote{big} dealer ever recruited me or ever followed through after we had met.)

\Pb
 
\newcommand{\Ed}[1]{\textit{#1} --- } % "enumDate"
\begin{enumerate}[label=\Alph*]
\item \Ed{1958 or 1959 through 1960}  College and the first year out of college.  My modern art pilot projects, so to speak.  Correlation of the modern arts---formalism.

\item \Ed{1961} Approximate date of my first in-person meeting with La Monte Young.  The period of \essaytitle{Philosophy Proper} and my \enquote{iconoclastic interdisciplinary projects,} particularly concept art.

\item \Ed{1962}  Continuing the \enquote{iconoclastic interdisciplinary projects}; beginning of my anti-art crusade.  First printing of \booktitle{An Anthology}.

\item \Ed{1963--4}  Affiliation with Marxism.  Adding a censorious sociological aesthetics as preface to the anti-art theory.  Accusation of political imperialism in the way musics were ranked in high culture.

\item \Ed{1965--6}  Bearing down on a number of agendas simultaneously.  I reconfigured my critique of culture as a Communist program in culture.  The rock songs with Walter De Maria.  I begin aggressively to recast my \enquote{interdisciplinary projects.}

\item \Ed{1967--8}  Return to anti-art utopianism.  Down with participation.  The absolutization of subjectivity.  Publication of two more \enquote{interdisciplinary works} from 1961--2 (in \journaltitle{Ikon} magazine).

\item \Ed{1970--1984}  Inactivity in \enquote{visual art,} except for photographing the SAMO\scalebox{1.2}{\textcopyright} Graffiti and except for the \enquote{archeology} embodied in my 1982 Backworks show.

\item \Ed{1985--1989}  I begin to revive concept art for tactical reasons.

\item \Ed{1989--1993}  Having joined Emily Harvey's gallery, I become a career artist, defining what I do as concept art, modern art, and fantasy.

\item \Ed{1994--1999}  My art career on hold.  Occasional pieces in group shows. 

\item \Ed{2000--2005}  Development of my attack on \textbf{modern art} as a turn in European civilization which crystallized at the beginning of the twentieth century.  \enquote{Baffling without substance, cult of the lurid, impoverishment chic, making the collector pay to be scammed.}

\item \Ed{2005}  Commence making abstract cinemas as an extension of abstract painting:  pilot projects.  Only a small venture so far.
\end{enumerate}
 
\Db
 
\lilsection{A}{1958 or 1959 through 1960.  College and the first year out of college.  My modern art pilot projects, so to speak.  Correlation of the modern arts---formalism.}
\lilsubsection{A}{Starting my sophomore year in college and lasting through my first year out of college.}

Roughly, works surviving, or reconstructed, as \arttitle{Grey Planes}, \arttitle{Ugly Drawing}, \arttitle{Poems 1--4}, \arttitle{Aleatoric Painting}, \arttitle{Spirit-World Paintings} (originally a single cartoon with multiple titles).  \enquote{An abstract expressionist} drawings for music, namely November 1960 No. 2 \scalebox{0.9}{[survives].}

Aleatoric Painting was oils on canvas board.  The others were drawings on bond paper or notebook paper, physically diminutive.

Retrieval in this period depends on my memory, and remakes of various pieces from surviving fragments.  (a random number table for Grey Planes; the \enquote{English translation} of \arttitle{Poem 4}.)  A few works survive intact in the collections of others.  Ugly Drawing.  Musical scores from late 1960.

These works, whch were carefuly thought through before I executed them, had three sources.  My sincere preference for abstract painting, especially Pollock.  My willingness to entertain modern poetry, which fell short of actual love for it.  My affiliation with the so-called new music.  I had arrived at college as a Bartok fan, was told that that was passe.  To be respected by the people whose respect I wanted, I had to get on the \enquote{new music} bandwagon.  I was very enthusiastic about Pollock, but knew it only from reproductions.  (I did not stand in the presence of a Pollock until I was 65.)

{\vskip 1em \raggedleft \parbox{3in}{\itshape \textsc{Note}: The credibility of ethnic music for me, which means that I find the potential for a complete musical experience in musical languages learned by ear: \textit{I do not find an analogy in visual art.}  Genre painting, if that is what it is called, folky painting, is kitsch to me.  As a visual artist, the prospect of trying to \textit{be Southern} never interested me in the least.  I am a critic of modern art, but I am not even slightly an aesthetic conservative, a classicist.  I was never able to commit to visual art based on rendering.

To me, iconic painting amounts to a handmade photograph with elements of cartooning.  I could not commit to it as a goal.  Maybe just cartooning.

I respect rendering in ancient Egypt, but nowhere else.  I have no interest in duplicating that achievement which united their language and their mythology. It would be utterly anachronistic.

In fact, I think that abstract cinema is a better medium than painting because painting is a handcraft medium.  (The problem with abstract cinema is that its practitioners did not know what they ought to be doing.  Also abstract cinema has needed the \textsc{dvd} to come into its own.)}\vskip 1em}

      From the outset, say 1958, the division of art, culture, into categories which became separate professions meant nothing to me.  I simply disrregarded the compartmentation of culture, and assumed that I should pass freely among philosophy, exact science, linguistics, poetry, painting, music, whatever.  Using each to illuminate the others and transferring methods from one to the other.  My first \enquote{flat visual works} were precisely translations from serial music and so-called chance music.  My poems also.

      While enrolled at Harvard, I began familiarizing myself with jazz recordings.  I had begun a conversion away from modern music.

There had been an exchange about the latest jazz via Young's correspondence with Conrad.

      The direction of everything I did was given by investigations in \enquote{epistemology} or whatever you call it which were unique to me.  I espoused positivism in high school and while enrolled in college, but quickly moved far beyond it.  Chomsky told me in 1960 that Philosophy Proper, Version 1 (it must have been Version 1) was worthless.

 
\lilsubsection{B}{My attitude toward success, 1958--60.}

      Those I fraternized with at Harvard who were trying their luck at the creative pursuits were not mercenary.  Wilder, who introduced us to the new music, did not become a career composer.  Conrad showed no interest in specializing in music.  Christian Wolff was satisfied to be a succes d'estime and did not intend to specialize in music.

      At Harvard I proceeded as if art were pursued for its own sake as a branch of worthy human possibility.  I took it for granted that I should \enquote{be creative,} and I approached my encounters with $<$ the culture in play in my milieu $>$ $<$ as the early Marx assumes it will be in a pure Communist society>:  the exploration of human possibility for its own sake. 

      I was aware that there was grumbing that Universal Edition was manipulatively promoting International Style.  Obviously Stockhausen, a product of apprenticeships with famous composition teachers, had to be a master at securing funding.  But I didn't take any of that as a signal of what to do.

      All in my circle in e.g. 1959--61 assumed that the arts were a phase of worthy human possibility.  As far as I saw at the time, we were non-mercenary.  De Maria and Morris may have known that they would quickly begin to earn.  Young soon became a career composer but the affluence did not come until later.  Cale would become commercially successful by crossing to Pop.  I didn't see that.  I assumed that our efforts were labors of love.  We were knowledgeable and competitive but we were not earning.

      The commercially successful artists I was aware of were i) Johns, ii) the Pop Artists, who would be a smash in the early Sixties.  The Pop Artists meant nothing to me as role models.

      I left Harvard on probation for low grades to write my first philosophical monograph, Philosophy Proper Version 1.  The orientation for everything else I ever did.  The warrant for my iconoclasm toward cognitive claims for art, laws of art; the warrant for my transfer of \enquote{intederminacy} etc. into the exact sciences.

\Pb
 

      Probably 1960, I attended a John Tate lecture which began with algebraic varieties and passed to ringed space.  \textit{[Tate's specialty, class field theory]} It inspired me to write a one-page text as a parody.  I seem to have anticipated the proliferating diagrams of category theory, and to have used color and non-alphanumeric symbols.  Cf. Wette for color.

Bafflingly out there, and I could not substantiate its cogency.  Like double-talk.

Mathematicians would have said then, and now, that it was worthless, but that's too easy.  Composers were producing scores that amounted to notational double-talk.  Indeterminacy had become a feature of ordained structure in \enquote{serious music.}  It had to translate to the \enquote{exact sciences} (and the very label would become a misnomer).

Increasingly I would come to see a legitimate role for indeterminate mathematics.  Mathematics waiting for its content to be focused. 

\sidenote{[That actually happens in mathematics, Miles Tierney cleaned up topos theory. Then Hennix blurred it again.]}

${}^\circ$ It's the idea that a text can be indeterminate, can be ahead of its substantiation.  comes back in full strength in \essaytitle{1966 Mathematical Studies}, then in the proposal for \journaltitle{Journal of Indeterminate Mathematical Investigations.}\editornote{Referred to in \essaytitle{Creep}}

 

\lilsection{B}{\Ed{1961} Approximate date of my first in-person meeting with La Monte Young.  The period of Philosophy Proper and my \enquote{iconoclastic interdisciplinary projects,} particularly concept art.}

 I met Young in person and met his entourage (as I thought of them).

{ \vskip 1em
\begin{enumerate}[label=\alph*)]
\item Young challenged the \enquote{infinitely new and radical} claims of International Style and Cage.  his response:  word pieces, monotony (minimalism)

\item Young absolutized the mystique of the new.  New was the definition of best.  If I had listened, I would have realized that his associates did not subscribe to this.
\end{enumerate}
\vskip 1em}

 I interpreted all this as an exploration of human potential that had become engrossed in being enigmatic.  art was not an arena in which it was the greatest achievement to discover a new beauty.  The greatest achievement was to baffle and frustrate the audience.

I was told that the Dadaists were the greatest precedents because they had been the most extreme, and the most extreme always won.  Young's entourage made sure that I knew about \booktitle{The Dada Painters and Poets, ed. Robert Motherwell} (1951).  Young's greatest coups had to do with dismantling the art process logically and with baffling the audience. (E.g. \opustitle{Compositions 1961}, identical pieces performed before they were composed.)

\Pb

Young and I rejected opera and dance as intrinsically corny.
Young was also the first jazz musician with whom I truly fraternized socially.  December 1960.  That confirmed the conversion I had been undergoing. Young also confirmed my already estabished attraction to Hindustani music.  He was already knowledgeable about the vocal music, and introduced me to many phases of Hindustani music.  (Conrad had recordings of some prominent Carnatic musicians.)

After being presented Young's word pieces in December 1960, I did numerous pieces directly prompted by them.

\Pb

\begin{itemize}
\item My first formalist system (mathematical logic), January 1961, had what amounted to an abstract color drawing as the \enquote{system's base,} with plastic overlays a la Cage.
\item Möbius strip piece.  \opustitle{Piece No. 2 (2/3/61)}.  \enquote{The instructions for this composition are on the other side of this strip}   

the fact I did these pieces is now proved by surviving works and their references to no longer existing pieces and by correspondence.\editornote{They're also viewable in MOMA's online collection.}
\item 1961.  Each point on this line is a composition.
\item Music of the future. \asidenote{[composition ordained that some music a thousand years from now was the composition's content]}
\item Rationale of concert held in my mind.
\end{itemize}

\Pb

At some point in the first half of 1961, I completely lost interest in the \enquote{art professions,} music, painting, sculpture, poetry.  My works at this time were \enquote{interdisciplinary projects} or out-of-category projects---which were shaped by my philosophical perspective, which I continued to refine throughout the spring.  When I mailed \essaytitle{Philosophy Proper, Version 3} to Carnap in 1961, he didn't reply.  \asidenote{[if not \essaytitle{Version 1} in 1960]}

At that time I saw a unity between intellectual innovations and artistic embodiment:  concept art.  But I would continuously re-examine my own premises for rationality.

\Pb

As I arrive at concept art, June 1961, I part company with the syllabus of art school---because everything I did stands or falls on the intellectual premises.  I'm not going to browbeat you about logic or philosophical linguistics.  You are the products of specializations which I happen to regard as derelict and irresponsible and retarded.

\Pb

People who are actually comfortable with this culture love the idea of the artist as a raging creature of instinct, who vomits emotion onto a canvas.  Who hops around like a jungle shaman.  As a matter of fact, post-war art music was not like that:  it was pseudo-scientific.  In fact, as everyone knows, many successful visual artists intimidated or baffled by being clinical and sterile.  However, their \enquote{blank wall} art was all posture---the equivalent of looking cool by wearing dark glasses---it made no intellectual contribution. 

The larger public still loves the mystique of artists as a caste which channels irrationalism and superstition and instinct.  Of course the instinctual artist is a fraud, since what they turn out is dictated far more by an evolving stylistic consensus. 

The assumption that infantile incompetence better qualifies one to \enquote{get feeling into the object} was a notion of European painting as classic painting disintegrated in the $19^{th}$ century.  Regression, insanity, etc.  Stockhausen's \opustitle{Aus den Sieben Tagen}.  Totally contemptible.

[[I accept that an artist would be intuitive and would draw on alternative consciousness.  I don't like the caste role assigned to the artist as a raging savage.  Artists supposed to be intellectually vacant.  Not to mention the art theory produced by critics and curators:  pseudo-intellectual trash. (\#)

\Pb

back to me:  I began to do \enquote{interdisciplinary} projects: \\
\begin{itemize}
\item \essaytitle{Concept Art}, which became \essaytitle{1966 Mathematical Studies}, \essaytitle{The Apprehension of Plurality}, etc.
\item \essaytitle{Mock Risk Games} \asidenote{[its publication history in \journaltitle{Ikon}]}\editornote{That is: published 1961 as \essaytitle{Exercise Awareness States}, disavowed and/or lost, re-created from memory as \essaytitle{Mock Risk Games}, then anthologized as \essaytitle{Exercise Awareness States} after all.}
\item \essaytitle{Energy Cube Organism}, which ended as \essaytitle{Choice Chronology Project}
\item \essaytitle{Perception-Dissociator} which became \essaytitle{Superseding} \asidenote{[the unfinished work of 1962 redone and published in 1968]}
\end{itemize}

\Pb
\textbullet WSTNOKWGO\\
another step in indeterminate mathematics (or specification of structure) \\
---instructions for performance in letter to Young, late 1961 \\
structural black box: \\
---extra-terrestrial broadcasts that earth theoretically capable of intercepting. \\
---actually existing verbalizations, texts, that are irrecoverably lost (aside from my destroyed work) \\

\Pb

When I met Maciunas in connection with appearing in his gallery in 1961, he made some favorable remarks about the Soviet Union.  in the same evening, Jackson MacLow was making caustic remarks along non-Communist Left lines.  (MacLow had been a rigorous anarchist-apacifist in the Forties.)  Dick Higgins seemed to be making gestures in the direction of the CPUSA.  I began to convert myself to Marxism.

\Pb

As of 1960 I was spoiled because so many people seemed to be admired for being iconoclastic and intransigent.  Cage and Ornette Coleman were just now famous.  In 1961 Young's entourage told me that Dada was the best because it had been the \enquote{most radical.}

The artists whom I met through Young did not seem to be full-time artists.  Only De Maria was already a \enquote{power artist}; I didn't register it because in person he was affable and generous and because I had signed off on the art machine almost before I knew what it was. Morris, a student of Lippold at Hunter, explicitly condemned wanting to get rich and famous in a letter to Young. In other words, Morris nominally rejected the actual purpose of major public art, which is professional success. (The golden paintbrush.)

\textbf{In the second half of 1961,} after half a year of active association with Young and his entourage, I began to imagine that I could gain recognition and earn my living by playing improvised music in clubs with Young.  I thought we could follow in the wake of Ornette Coleman.  (I did not register that Coleman got disgusted with club performance and quit.)

\Pb

That being said, I did nothing shrewd to pour myself into the mold of success.  I only cared about doing what had my enthusiasm.  I experienced the avant-garde as a crisis situation; I cared deeply about whether activity in the crisis situation was rational.  I was never shrewd about cultivating the exhibition system or giving the audience what it wanted.  I was happy to be naive and heedless.

I would write \enquote{The Exploitation of Cultural Revolutionaries in Present Society} in fall 1961.  I would take cases like Galois, Abel, and Van Gogh, and make an issue of them.  The cultural judges couldn't get it right because cultural revolution upset the standards of excellence themselves.  At the same time, there was a legitimate role for arcane work which could not be asked to support itself commercially.  I experienced this as an appalling trap:  for the cultural revolutionary, life was a death march.  The only real solution would be a Communist utopia as envisioned by the early Marx. \textbf{I converted to Communism out of self-interest, because of a \textit{rare} personal problem, before I began to worry about the masses and to endorse institutional Communism.}

\lilsection{C}{\Ed{1962} Continuing the \enquote{iconoclastic interdisciplinary projects}; beginning of my anti-art crusade.  \enquote{First printing} of \booktitle{An Anthology}.}

Begins 1961 but takes shape in 1962.  I realized that I was involved in a knock-down drag-out competition---the neo-Dada avant-garde---and I began to wonder if it was not like being a college student.  Being forced to compete on demeaning terms, being immersed in a demeaning subject-matter.

      I become more and more uncomfortable that artists were offering things that inrinsically weren't worth doing, whose only payoff was to leave the audience feeling baffled and frustrated.  They were competing for social approval on that basis.  They were making careers out of bluff, posture, hoodwinking the experts into giving approval for what nobody would do without the social context. 

      Already at the time of the loft appearances, Feb. 1961, when I had the ability to garner respect from the people whose opinion I most valued\slash whose approval mattered to me\slash I was bothered by the game we were playing. \\

the Harvard concert of 1961:  the reason it was \enquote{possibly Henry Flynt} and I didn't perform.

{\itshape \footnotesize
[Despite all the talk about new new new, the artistic fraternity could only deal with painting this, sculpture that.  their inovation consisted in brandishing postures at each other for social effect.  by definition they were intellectually vacant.

{\scriptsize [it's all posture, a game being played inside an elite institution, the self-important overpriviledged cognosenti, posture game or intimadation game.  this cognoscenti does not have anything to say about philosophy, science, economics, government that I respect in the least.] }

they were not seeking interdisciplinary or out-of-category innovation.  Thus, the works which I poured myself into developing went utterly over their heads.  drew a blank.

Substantial innovation, e.g. concept art, went over their heads.]}

I passed from the mystique of the avant-garde to the conviction that art had a flawed premise.  Cage had already said it, with a different rationale.  But he didn't mean it.  {\itshape \footnotesize [Later, Ben Vautier would deliberately use anti-art as a ploy, the collectors paid him to scam them.]}

\Pb 
 
All the while, the \enquote{first printing} of An Anthology, with Concept Art, was 1962.  Definitive printing 1963.

\Pb 
 
\begin{itemize}
\item My New Concept of General Acognitive Culture was first presented publicly with the creep lecture, Harvard, May 15, 1962

\item New York, June 5, 1962  acognitive culture talk.  Young was mainly responsible for assembling the audience.  Almost all were people who were or would be famous in the \enquote{creative world.}  Like the February 1963.  I was attacking art to an audience of fledgling famous artists.

\item lecture \essaytitle{Pure Recreation}, Harvard, August 7, 1962

\item publication in \journaltitle{decollage No. 3}
\end{itemize}

 \lilsection{D}{\Ed{1963--4}  Affiliation with Marxism.  Adding a censorious sociological aesthetics as preface to the anti-art theory.  Accusation of political imperialism in the way musics were ranked in high culture.}

In 1962--63 I became deeply interested in artistic culture as a phase of modern European civilization:  for the purpose of a destructive critique, a hostile or censorious sociological aesthetic.  I spoke to audiences of culture professionals who would go on to become icons.  I told them to destroy their work and renounce art. 

I converted myself to Marxism and began to require that everything I was doing be squared with a basic Marxist premise:  that market democracy was intolerable, it had to be opposed.  it was unacceptable to be at peace with it.  Mills' \booktitle{Listen Yankee}, apartheid in South Africa, what become the Vietnam war, the history of bourgeois-democratic intervention in primary producer nations.  It seemed that bourgeois democracy was incapable of supporting the bourgeois-democratic revolution and in fact saw every liberation as a threat.

How could I espouse a closed and \asidenote{[doctrinaire]} ideology like Marxism?  At that time, as far as I knew, the progressive forces had become embodied in vast institutions.  One was honor-bound to support them, as opposed to remaining on the sidelines. \asidenote{[There is not a single person in this audence who would recommend remaining on the sidelines during the Second World War, who would be neutral between the Allies and the Axis.]}

Marxism had several versions, and had already redefined itself several times, as the locus of insurgency moved.  (Not the Western proletariat, but the colonial subject, then the college student.)  It was the only tradition that touched the essential bases and that aligned itself institutionally with liberation struggles.  Namely:
\begin{enumerate}[label=\roman*)]
\item \textbf{[market democracy was wage slavery and ruthless imperialism]} \\
economic individualism said it was the greatest good for the greatest number \\
why wasn't it the greatest good for the least number?
\item \textbf{a just social order could and should be built, based on economic collectivism.} \\{}
[at that time, I assumed that Cuba was what its supporters said it was] \\
\end{enumerate}
I agreed with Marxism that there was no future for the reversion to the simple life.  (In the Seventies, there would in fact be rural comune experiments.  They all failed, of course.)  The new society would be planned, would cultivate and extend technology.  Its economic laws would be counter-intutive at first:  not entrepreneurs producing for profit.  The general population would have more disposible time.  It would not mean becoming indolent and taking longer vacations; it would mean the exploration of human possibility without regard for market signals.  The life-style I spent my life fighting to have, I wanted to make it possible for many people.

[\textbf{I admit to self-deception out of desperation.}  my interests isolated me so profoundly that the only thing I had to look forward to was mere physical survival, maximizing the length of life.  I had to find a way to break out.

\Pb

Precepts (i)--(ii), found in Marx, became deeply interwoven in my sociological aesthetic. 

my rejection of European art music for black and East Indian music.  \asidenote{[or for that matter, Rumanian gypsy music, the one music on the continent of Europe which had my enthusiasm]}

the official aesthetic dispensation, e.g. the musicology departments, which put the best music at the back of the bus---I tied this to colonialism.  the victorious empire could impose a culture which was intimidating, which incorporated technology and massed forces, but which was otherwise poison.

\Pb 

{\itshape\footnotesize[$\diamond$ I continued musical activity right through the anti-art period, even though I unfortunately destroyed my earliest \enquote{new ethnic music} recordings.

Why wasn't I attracted to Southern ethnic music when I lived there as a child?  As a child I did not have the confidence or \enquote{creativity} to imagine the world for myself.  I saw, accurately, that Southern ethnic music was a truncated cultural experience.  It was associated with a brawling life-style and with obscurantist Protestantism. 

the idea that there could be a refined Southern music which would engage the whole person.  I single-handedly invented that.  The role model I needed at 15:  I had to become that role model, and it did not crystallize until the Seventies at the earliest.]}

\Pb 
 
Another key issue was competition among artists.  If art was gratuitous self-expression, how could one artist be competitively superior to another?  My peers tried to define it, and gave answers such as new or the most extreme or the best ideas.

Meanwhile, I looked back on my sojurn in classical music as an episode of competiting in a demeaning contest.  The horror of classical music to the competitive student---how I felt is indicated by a July 26, 2005 news story:

Bryan O'Lone was scheduled to play at a competitive recital in Carnegie Hall.  He prepared Chopin.  When he arrived, they told him he must play something he hadn't prepared.  When he played the Chopin anyway, the music teacher who sponsored the recital walked onto the stage and shut the keyboard cover on his fingers.  He sued her for a million dollars.  This quirky story, to me, is emblematic of the entire experience of being a competitive classical music student.  The National Music Camp and Bloody Friday.

\Pb 

The demonstrations against Stockhausen in 1964.

\lilsection{E}{\Ed{1965--6} Finds me bearing down on a number of agendas simultaneously. I reconfigured my critique of culture as a Communist program in culture.  The rock songs with Walter De Maria. I begin aggressively to recast my \enquote{interdisciplinary projects.}}

Communism was an actual constellation of nations and insurgent organizations, stemming from the early Soviet Union, claiming an inspiration from the doctrine of Marx. 

I give my sociological aesthetic the guise of a cultural program for this Communism.  Perhaps the exercise sounds dully compliant and subservient, but it was nothing of the sort.  At that time we supported their social objectives, or thought we did, but I and Maciunas ripped the consensus Marxist cultural program to pieces.  Our argument was that the entire Marxist tradition had gotten culture all wrong.

I reconfigured \essaytitle{From Culture to Brend} (1963--4?) to an avowedly Communist program for what culture should be.  E.g. protest rock.  I give my sociological aesthetic the guise of a cultural program for Communism.  This was iconoclastic because it was a break with the official Soviet view of culture unheard-of on the hard Left.  I was the first person who said that rock ought to be politically radical.

The perspective on music was my most obvious innovation in Left cultural appraisal.  But we re-examined the entire range of Communist cultural policy---to some extent converging with early Soviet figures like Rodchenko, although I did not know of him at the time.

I was the first person in the Marxist Left to say that black music, and rock, were not American decadence, but rather musical languages [created] collectively by the very people our movement claimed to speak for.  The idea that Beethoven was wealth which transcended class, about which the Party ought to instruct the proletariat (like a Metropolitan Museum art appreciation course) was one of Communism's immense mistakes. 

I was also an innovator relative to rock practice, because rock had been configured as saleable entertainment, and had not been concerned to find a political interest against patriotism and the existing economic order. In 1966 I made my recommendations concrete by recording what was released forty years later as \opustitle{I Don't Wanna.}

{\itshape\footnotesize [I was more comfortable with music because I admired traditional musical languages and the possibility of renewing or broadening them through innovation.  Visual art:  I find traditional languages corny (after ancient Egypt).  I adhered to the modernist principle that visual art needs to create the language as it goes, not to work with a long-practiced language.]\par}

\Pb 

Even during the years of ostensible Marxism, I never surrendered my intellectual gains.  Primary Study was published in 1964. \essaytitle{Mathematical Studies} came in 1966.  As I saw it:  I found the necessary elements of a complete perspective embedded in various hidebound traditions or disciplines.  What I wanted was uniformity, to bring each necessary element up to the level of the other.  So that you didn't have to switch ideologies on different days of the week. 

At the same time, I was transferring concept art and the other \enquote{interdisciplinary projects} out of art altogether---and into exact science formats or foundations of science formats. \\
1966 Math. Studies \\
Perception-Dissociation of Physics \\

What began as concept art etc. now became insane extremism in mathematical logic and foundations of science.

\lilsection{F}{\Ed{1967--8}  Return to anti-art utopianism.  Down with participation.  The absolutization of subjectivity.  Publication of two more \enquote{interdisciplinary works} from 1961--2 (in \journaltitle{Ikon}).}

I left the organized Left because I was now willing to bet my life that the Communist regimes could not become the utopia of the future via \enquote{regeneration.}  I decided that the Stalinist regimes were a new exploiting class.  If they aren't, it doesn't vindicate Trotsky; it means that the regime he headed and unconditionally defended was a crackpot version of capitalism which gained its credibility from an \enquote{oppressed people} ideology. 

I became willing to forego \enquote{participation.}  I revived the utopianism of my cultural position.  What ought to be was so far from what was socially feasible that there was no bridge between them.  I chose to again emphasize what ought to be.  It was a drop-out stance which combined utopian social speculation with solitary self-realization.

My interest in technical economics began at this point, probably beginning 1968.  I began to realize how badly I had been served by the Left's economic illiteracy, or rather, denial.  It became one of my projects to substantiate the utopia in technical economcs.

I left the organized Left because I became convinced that Leninism could never deliver the utopia Marx had promised, it was as far from that utopia as capitalism was.  To characterize the Soviet Union is still an unsolved problem, but those who called it state capitalism were on the right track.  The prospect of being a cultural administrator for official socialism lost all interest for me. 

I resumed the pure rejection of art.

I returned to the [devotion to] subjectivity and individuality as the necessary  outcome of avant-garde positions.  I was prepared to forego the participatory dimension of culture.  In fact, I rarely enjoyed attending concerts.  I almost always preferred to listen alone to recordings.  At concerts, other members of the audience became the message.

\journaltitle{Journal of Indeterminate Mathematical Investigations}, November 1967.  Another reconfiguration of the interdisciplinary projects of 1961.

HF, \opustitle{Lecture on Brend}, was Film-Makers' Cinematheque, Feb. 14, 1968

I was prepared to forego the use of culture as symbols of a social tendency.

\Pb

\lilsection{G}{\Ed{1968--1984} Inactivity in \enquote{visual art,} except for photographing the SAMO\scalebox{1.2}{©} Graffiti and except for the \enquote{archeology} embodied in my 1982 Backworks show.}
\begin{itemize}
\item \Ed{1975} publication of \booktitle{Blueprint}

\item \Ed{1977} Hennix arrives in New York and we quarrel because Hennix chose to show the Algebras in an art gallery.  Hennix wants the art world to subsidize radical thought.  I object that it just demeans the work.

\item \Ed{1979} I photographed SAMO©

\item \Ed{1982} Backworks: \opustitle{Fragments of a Destroyed Oeuvre}
\end{itemize}

\lilsection{H.}{\Ed{1985--1989} I begin to revive concept art for tactical reasons.}

	I am asked to remake a 1961 concept art piece for a show in 1985.  1987, Avenue B Gallery, I make new work.  revival of concept art to publicize the joint work of myself and Hennix, to ask the belated reward for our pioneering, to unfold concept art to a much greater extent than I had in the beginning. \opustitle{Stroke Numeral}, \opustitle{Tritone Monochord}.  I kept a concept art journal.  I sought to unfold concept art and showcase it.  There was an entire series of concept art signs and only a few got made.
\begin{itemize}
\item Self-Validating Falsehood
\item Short Additive Semigroup
\item One True Sentence
\item Two Honest Texts
\end{itemize}

\lilsection{I}{\Ed{1989--1993} Having joined Emily Harvey's gallery, I become a career artist, defining what I do as \textbf{concept art, modern art, and fantasy.}}
\begin{itemize}
\item \Ed{1987} new concept art works.  $\diamond$ the art career, the attempt to become successful by asking to be rewarded for my pioneering.  I became an art soldier. $\diamond$

\item First one-person show 1989

\item  \opustitle{One True Sentence}.  the wallpaper.  a piercing self-reference anomaly.  muted yellows, kitchen wallpaper, a mild tone.  If somebody actually used it for the intended purpose, it would be like residing next to one of the civilization's intellectual fault-lines.

\item \opustitle{Logically Impossible Space}

\item last one-person show 1993

\item SAMO\scalebox{1.2}{©} at the Lyon Biennale.
\end{itemize}

it was not a waste of time, but I did not enjoy a \enquote{success.}   I had to submit to being pre-classified as a Fluxus artist although that didn't make any sense.

\Pb

\lilsection{J}{\Ed{1994--1999}  I let my life as a career artist lapse.  Occasional pieces in group shows.}

The joint appearance of Flynt and Hennix at MELA was 18 May 1997.

\lilsection{K}{\Ed{2000--2005} Development of the attack on modern art as a turn in European civilization which crystallized at the beginning of the twentieth century.}

\enquote{Baffling without substance, cult of the lurid, impoverishment chic, making the collector pay to be scammed.} 

$<$ All the while, my practice embraces one or two tenets of modern art, such as the move from rendering to abstraction. $>$

 
\lilsection{L}{\Ed{2005} Commence making abstract cinemas as an extension of abstract painting:  pilot projects.  Only a small venture so far.}

